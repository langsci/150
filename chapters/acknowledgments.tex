\addchap{Acknowledgments}
\begin{refsection}

%content goes here
This book has a long history and I would like to thank all the people that made it possible. It all started with my teacher of Russian language and literature in the mathematics class of Moscow's 57th school, Sergei Vladimirovich Volkov. My decision to study at the linguistics department is due to him and to the organizers of the traditional linguistics competition in Moscow.

During my first year of graduate studies I was lucky to attend a special course on formal semantics by Igor Yanovich that was complemented by a course on presuppositions given by Philipp Schlenker at the summer school in St.\ Petersburg, organized by John Bailyn. That was the time when I fell in love with formal semantics. Later I was lucky to attend lectures and seminars by Barbara Partee, who provided students in Moscow with knowledge that would not be accessible otherwise. I remember very well the great amount of attention, patience, and respect that always accompanied Barbara's classes. 

My passion for formal semantics is complemented by a love of logic, compact formalizations, formal languages and programming. In this respect I would like to specially thank Vladimir Andreevich Uspenskij, for his aim was not only to teach mathematics, but to enhance his students' common sense and cultural background. Apart from the invaluable personal interaction, I learned from him to take responsibility for the clarity of my explanations and to not be afraid of acknowledging my mistakes.

Shortly before finishing my master studies at Moscow State University I received an email from Barbara Partee advertising a PhD position at Heinrich Heine University in Düsseldorf under the supervision of Laura Kallmeyer. Thank you, Barbara, for playing a special role in my life and thank you, Laura, for taking the risk of accepting a student from Russia.

The last five and a half years that I have dedicated to this work were full of wonderful people. My advisers, Laura Kallmeyer and Hana Filip, guided me through the writing process with patience, encouragement, and valuable advice. I have been extremely lucky to collaborate with two people that are experts in computational and theoretical linguistics, as this work belongs to both domains. I am proud of not giving up either part and this is due to my advisers. 

I would also like to thank my colleagues at the University of Düsseldorf, especially Daniel Altshuler, Zsofia Gyarmathy, Timm Lichte, Rainer Osswald, Simon Petitjean, and Guillaume Thomas. I am also profoundly grateful to a number of people I met during these years at summer schools, conferences, and colloquiums, as well as to all the anonymous reviewers of the abstracts and papers this work builds upon. I would like to especially thank Lucas Champollion, Emmanuel Chemla, Judith Degen, Michael Frank, Michael Franke, Hans Kamp, Olga Kagan, Daniel Lassiter, Fred Landman, Fabienne Martin, Antje Rossdeutscher, Susan Rothstein, Philipp Schlenker, Gregory Scontras, and Matthijs Westera.

I separately thank my family and especially my son Maxim for their patience during the hard writing periods.

Last, but not least, I thank the German Research Foundation for funding SFB 991 where I have worked for more than four years. 
\printbibliography[heading=subbibliography]
\end{refsection}

