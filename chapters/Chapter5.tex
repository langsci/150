% Chapter 5
\chapter{Semantics of individual prefixes} % Write in your own chapter title
\label{Chapter5}
%\lhead{Chapter 4. \emph{Semantics of individual prefixes}} % Write in your own chapter title to set the page header
\section{Semantic approach to verbal prefixation}
The main things that we have discussed so far are an efficient way of collecting and verifying the data and the fact that these data cannot be fully accounted for by means of existing syntactic approaches to Russian prefixation. Let us now explore what has been done in the domain of prefix semantics.

Semantics-oriented studies of Russian prefixes can be divided in three groups: (i) studies following Russian tradition that investigate nuances of different prefix usages, (ii) studies following ``Western'' tradition that aim to find uniform semantics (or one function) for all the prefixes (not only in Russian, but in Slavic languages in general), and studies that try to bridge the gap between the first two types of approaches. Let me provide a bit more details about each of these directions of research.

The main question that is addressed in Russian tradition is nicely formulated by \citet[18]{Boguslawski:63}, who writes that ``the problem of defining all the meanings of ``the same prefixes'' is first of all a practical problem and is of a great importance for the lexicographic studies". The main purpose of the Grammar \citep{Grammar:52, Shvedova:82} and dictionaries \citep{Dict:50, Dict:57}, as well as of many other studies of Russian prefixes (\citealt{Avilova:64, Golovin:59, Lopatin:97, Tixonov:98}, among others) is to examine the data in great detail and provide a full picture of the different usages that a particular prefix may have. As a next step, the type of relation (polysemy of homonymy) between these usages is analysed \citep{Krongauz:97, Plungyan:01}. This work is necessary, but its focus is on descriptive adequacy and not on finding differences or similarities between different prefixes or explaining why a particular combination of stacked prefixes is available or not.

As for the ``Western'' approaches, the main idea they exploit is that Slavic verbal prefixes are markers of perfective aspect (\citealt[see, e.g.,][]{Binnick:91, Krifka:92, Zucchi:99}, among others). Perfective aspect itself then gets analysed in terms of quantization (first proposed in \citealt{Krifka:86, Krifka:92}, and later repeated by \citealt{Pinon:95}), from which it follows that the semantic function of verbal prefixes is to contribute quantization, defined by \citet{Krifka:86} as shown in \defref{def:quant}. 

\theoremstyle{definition}
\begin{definition}{Quantization}\label{def:quant}
\textit{QUA(P)} $\leftrightarrow \forall x,y[P(x) \wedge P(y) \rightarrow \neg y<x]$\\
A predicate \textit{P} is quantized iff, whenever it applies to \textit{x} and \textit{y}, \textit{y} cannot be a proper part of \textit{x}.
\end{definition}

However, \citet{Filip:92} noticed that matters are more complicated, as there are perfective verbs that fail to be quantized according to the \defref{def:quant}. \citet{Filip:92} raised a number of questions in this respect, and proposed that ``the semantic property of the Incremental Theme NPs that is determined by aspect should not be characterized in terms of the `cumulative/quantized' distinction, but rather in terms of the `bounded/unbounded' distinction, which characterizes aspect' \citep[][147]{Filip:92}.

As a next step, \citet{Filip:92} shifted the focus to the contribution of Slavic linguistic tradition \citep{Wierzbicka:67, Rassudova:75, Merrill:85} and concluded that verbal prefixes must be associated with local quantificational effects\footnote{A-quantification in terms of \citealt{BachPartee:87, BachPartee:95}, which is typically expressed at the sentence level or at the level of VP with sentence adverbs, ``floated'' quantifiers (e.g., \textit{each}), verbal affixes, auxiliaries, and various argument-structure adjusters.} (among other meaning components). This got later reformulated by  \citet{Filip:99} as a proposal to analyse Slavic verbal prefixes as scalar expressions and became a departure point for the subsequent analyses \citep{Filip:00, Filip:03, Filip:05, FilipRothstein:05, Kagan:11, Kagan:12, Kagan:13, Kagan:book}. For example, \citet[183]{Filip:99} writes that the prefix \textit{na-} ``adds to a verb the meaning of a sufficient or large quantity, or a high degree measured with respect to a certain contextually determined scale and with respect to some standard or subjective expectation value.'' Later \citet{Filip:08} also formulated the general idea that prefixes (at least under certain usages) ``contribute to the specification of the ordering criterion on events'' and proposed to include them into the class of scale inducing expressions. This idea allowed \citeauthor{Kagan:12} (\citeyear{Kagan:12}, \citeyear{Kagan:book}) to further develop the semantic approach to prefixation under which ``the major semantic function of a prefix is to impose a certain relation between two degrees on a scale''. Various prefixes then differ with respect to the type of the scale they can apply to and the exact relation between the degrees  they establish. 

Following \citet{Filip:08}, the idea of scalar interpretation of verbal prefixes serves as a bridge between the two traditions: on the one hand, it reveals the common core of the prefixes, and on the other hand, it provides the space for explaining the distinctions between individual prefix usages. 

I propose to use a scalar approach to prefix semantics in order to account for another complex issue: prefix combinatorics. \citet{Tatevosov:09} correctly notices that descriptive approaches and structuralist theories of semantics of Russian prefixes, such as \citet{Avilova:64}, \citet{Golovin:59}, \citet{Lopatin:97}, and \citet{Tixonov:98}, did not bring us closer to the understanding of how complex verb formation functions. On this basis \citet{Tatevosov:09} concluded that a semantic approach is not helpful for predicting the existence and properties of complex verbs. This conclusion is, however, not a valid one: an inspiring counterexample is the work by \citet{Filip:03}, who uses the \textit{one delimitation per event} constraint to motivate the exclusion of some prefix-verb combinations on semantic grounds. This constraint is formulated by \citet[79]{Tenny:94} as ``[t]he event described by a verb may only have one measuring-out and be delimited only once''. It is grounded in the independent restrictions that come from the grammar of measurement in natural languages and it operates across both nominal and verbal domains. 

Taking this as a point of departure, I propose to analyse certain restrictions on the formation of complex verbs as semantic restrictions. As I have shown in Chapter~\ref{Chapter2} and Chapter~\ref{Chapter4}, a significant amount of data cannot be treated adequately in the syntactic approaches: biaspectual verbs, stacking of prefixes, formation of the secondary imperfective verbs. I propose to look at these processes from a different angle, taking into account the semantics of verbal prefixes. I will show that scalar semantic approach can be successfully used to motivate stacking of prefixes (as well as the existence of biaspectual verbs and certain restrictions on the formation of secondary imperfective verbs) if such research question is posed and a formalism that allows to restrict derivations on the basis of semantic constraints is used. 

%My intuition here is similar to that of \citet{Kagan:book}: prefixes are related to scales and introduce some delimitation of events. Due to these delimitations events may become telic and perfective. 

%This intuition is also related to other ways of thinking about prefixation presented in the literature, including syntactic accounts. For example \citet{Tatevosov:09} constraints prefixes in the selectionally limited group in that they cannot be attached to a formally perfective verb. In the account I propose, the impossibility of such an attachment is explained via the incompatibility of the semantic restrictions associated with the verb and the prefix. This explains those cases that are exceptional from the syntactic point of view, as it turns out that semantic restrictions are compatible. This is, for example, the case of combining two prefixes with the delimitative function: the delimitative prefix \textit{po-} cannot be attached to a perfective verb, unless this verb is already prefixed with another delimitative prefix. We will discuss such examples in detail later in this chapter.
 
%The crucial difference between the previous semantic accounts of Russian verbal prefixation and this work is the different formulation of the research question and the formalization of the semantics of prefixes, which will be provided in Chapter~\ref{Chapter7}. An important property of the formal part of the account is its capability to grasp by means of the semantic representation not only the semantics the prefix contributes to the derived verb, but also the semantic restrictions on its attachment. In this chapter I will prepare the ground for this formalization, providing informal descriptions of the semantics and attachment restrictions of individual prefixes after carefully investigating their properties. 

The goal of this chapter is to motivate intuitions about the behaviour of individual prefixes and provide informal semantic analyses of the discussed prefixes in such a way that their combinatorial properties fall out naturally from their semantic properties. 
This discussion provides the basis for the formalization of prefix semantics that will follow in Chapter~\ref{Chapter7}. The prefixes that we are going to look at are the following: \textit{za-} (inchoative usage), \textit{na-} (accumulative usage), \textit{po-} (delimitative and distributive usages), \textit{pere-} (iterative, distributive, and excessive usages), and \textit{do-} (completive usage). I will occasionally mention some of the extra usages of the discussed prefixes, but analysing them, as well as other prefixes, is beyond the scope of this thesis.

For each prefix, the structure of the respective subsection is the same, covering three main issues and followed by a summary:
\begin{enumerate}
\item semantic contribution;
\item restrictions on the attachment: (in)compatibility of lexical semantics of verbal stems with the prefix semantics;
\item subsequent imperfectivization of a verb with the discussed prefix;
\item summary.
\end{enumerate}

%In the sections \ref{subsection:semantics:za}-\ref{subsection:semantics:do} I discuss contributions of individual prefixes on the pre-formal level. This discussion provides the basis for the formalization of prefix semantics. The goal of the formalization I propose is not only to adequately capture the semantics of the discussed prefixes, to be able to predict whether verbs containing certain combinations of a verbal root and prefixes exist and which aspect and semantics such verbs have. 

Before we proceed, I would like to note that moving the focus from the syntactic restrictions to the semantic ones in the domain of prefix stacking does not mean that no syntactic theory of verbal structure is needed. There still remain constraints that are better formulated in (morpho-)syntactic terms. An example of such constraint is the unavailability of multiple imperfective suffixes in Russian. 

Another module that is involved in regulating complex verb formation in Russian is pragmatics. I propose some preliminary pragmatic explanations for the non-existence of certain verbs in this chapter and provide some more details in Chapter~\ref{Chapter6}.

As scales are crucial for the analysis of the prefixes, let me provide a brief overview of the concept before discussing the properties of individual prefixes.

\section{Scales}

%Studies that look at the relation between telicity and either measure (Tenny 1994), or scale (Filip 93/99, 2004, Hay, Kennedy, Levin 99, Beavers 06, 07, Wehsler 05), or incremental theme (Dowty 91, Krifka 89, 92,98), or quantity criterion (Filip 2005), or ordering criterion (Filip-Rothstein 06)

%Rappaport Hovav: three kinds of scales recognized in the literature: property scales, path scales, volume/extent scales
The primary area of application of scales in linguistic is the domain of gradable adjectives. As has been suggested by \citet{Kennedy:99}, gradable adjectives (e.g., \textit{wide, tall, expensive}) denote properties that for different individuals hold to different degrees. This means that they are analysed as denoting a certain relation between an individual-type and a degree argument. One formalization of this idea is that an adjective lexicalizes a \textit{scale} and maps its argument to a certain degree on that scale \citep{Kennedy:01, KennedyLevin:02}. An alternative formalization \citep[e.g.,][]{Heim:00} represents such adjectives as taking a degree as an argument and providing as its output the set of the individuals for which the lexicalized property holds up to this degree.

 A \textit{scale} is defined as a set of points (degrees, values), %\footnote{The term \textit{degree} is used to refer to either points or intervals on the scale (see, e.g., \citealt{KennedyMcNally:05}), so we will refer to the either points or values on the scale to avoid confusion.}
 totally ordered along some \textit{dimension} (e.g., length, quantity, volume, duration). If the scale has the maximal and the minimal elements, it is a \textit{totally closed} scale (often called just \textit{closed} scale). If the scale has neither maximal nor minimal element, it is a \textit{totally open} (or just \textit{open}) scale. Scales that have a minimal and lack the maximal element are \textit{lower closed} and scales that lack the minimal and have the maximal element are \textit{upper closed}. These properties play an important role in accounting for the adjectival semantics \citep[see, e.g.,][]{KennedyMcNally:05, RotsteinWinter:04, KaganAlexeyenko:10}.
 
Another central notions in this domain are that of the \textit{comparison class} and the \textit{standard of comparison}. The relevant comparison class \citep[see, e.g.][]{Klein:80, KennedyMcNally:05, Kennedy:07} is constituted of objects similar to the individual argument in the relevant respects. The comparison class then provides the standard of comparison and the sentence like \ref{ex:expensive} is interpreted as asserting that the price of the house is higher then the the standard price of a house from the comparison class (houses with similar parameters in the same area).
 
 \ex.\label{ex:expensive}This house is expensive.

Comparative adjectives, such as in \ref{ex:expensive:more}, differ in that they overtly specify the comparison class and, thus, the standard of comparison. 

 \ex.\label{ex:expensive:more}This house is more expensive than the one we saw yesterday.
 
Differential degrees \citep{Kennedy:01} (also called difference values in \citealt{KennedyLevin:02}) and the operation of degree addition \citep{KennedyLevin:02} allow to represent the semantics of such sentences as \ref{ex:expensive:more:1000} by explicitly stating how the relevant degrees of the individuals are related.
 
 \ex.\label{ex:expensive:more:1000}This house is five thousand dollars more expensive than the one we saw yesterday.

Scalar approach to the semantics of event predicates has proven to have a lot of explanatory power and has been advocated in numerous works on the event semantics \citep[see, e.g.,][]{Ramchand:97, Hay:99, KennedyLevin:02, CaudalNicolas:05, FilipRothstein:05, Kearns:07, KennedyLevin:08, Filip:08, Pinon:08, Rappaport:08, Rappaport:11, McNally:11}. Let me provide a very brief overview of the works that adopt a scalar approach in order to account for the aspectual properties of event predicates (for a more detailed observation and extra references see \citealt{Arsenijevic:13}).

The first class of verbs that has been explored from the scalar semantics perspective is the class of degree achievements, such as \textit{cool}, \textit{grow,} or \textit{widen}. The crucial difference between adjectives and degree achievement verbs is that while the former map individuals to degrees, the latter denote a change of degrees: the degree to which the argument possesses the property at the end of the event is higher than at the beginning, so a temporal argument has to be introduced \citep{Hay:99, KennedyLevin:02}.

%\ex.\label{ex:Kennedy2}\a.Kim walked from the bank to the store in/??for an hour.
%\b. Kim walked for/??in an hour.\\
%= (4.4) in \citealt{Kennedy:12}, p. 104

As a next step, scalar approaches to degree achievements were integrated with earlier approaches to the aspectual composition. The theory of aspectual composition has been developed based on the observations about the behaviour of \textit{incremental theme} verbs. Such verbs are characterized by referring to eventualities that involve an incremental change that is related to the internal argument \citep[see][]{Garey:57, Wierzbicka:67, Verkuyl:72, Krifka:86, Krifka:92, Filip:92, Filip:99}. An example of a verb of an incremental creation is provided in \ref{ex:Kennedy1}. An important observation behind this example is that when the incremental theme has some specified quantity, the predicate is telic; when there is no such specification, the predicate is atelic.

%verbs of directed motion (as \textit{ascend}) and  were included in the class of verbs that receive scalar interpretation. Verbs of motion, according to this view, are associated with an increase of a degree on the \textit{path} scale. Incremental theme verbs 

\ex.\label{ex:Kennedy1}\a. Lee wrote a poem in/??for an hour. \hfill Telic
\b. Lee wrote poetry for/??in an hour. \hfill Atelic\\
\hbox{}\hfill\hbox{= ex.~(4.1) in \citealt[103]{Kennedy:12}}


%The analysis of incremental theme verbs has been later generalized to capture other types of verbs that demonstrate similar behavior: motion verbs, as in \ref{ex:Kennedy2}, and 
%
%\ex.\label{ex:Kennedy3}\a. The canyon widened 30 kilometers in/??for one million years.
%\b. The canyon widened for/??in one million years.\\
%= (4.5) in \citealt{Kennedy:12}, p. 104

Later \citet{Filip:05} has shown that the basic meaning of an incremental theme verb in English does not introduce a scale. This approach has been adopted by \citet{Rappaport:08}, \citet{LevinRappaport:10}, \citet{Kennedy:12}, and \citet{Bochnak:13} who concluded that measure of change functions must be associated with the incremental theme arguments that supply some value that is used to select an appropriate portion of the scale that has to be covered in course of the event. 

Now let us describe additional kinds of scale types that will be relevant for the discussion that follows. First of all, I want distinguish two types of situations involving a change along a scale: for the first type, the absolute value on the scale matters, and for the second type, the absolute values are not important and we are only interested in the difference between the values at the beginning and at the end of the event. For example, if John heated the water up to 40 degrees Celsius, it is the absolute value that matters, and if John gained 2 kilos it is the difference that is relevant. We will say that the first event proceeds along the \textit{temperature} scale (I will call the class of such scales \textit{proper scales}) and the linguistic context supplies the \textit{maximum value} on that scale. In the second case we will say that the event proceeds along the \textit{measure of change scale} for \textit{weight} and the direct object provides the \textit{measure of change value}. 

I adopt the notion of the \textit{measure of change scale} from \citet{KennedyLevin:08} and \citet{Kennedy:12}. The measure of change scale for \textit{weight} is of course related to the proper \textit{weight} scale, whereby the zero point (which is also the minimum point in this case) on the measure of change scale corresponds to the value on the \textit{weight} scale at the beginning of the event. The point that is related to the end of the event may be not so straightforwardly related to the measure of change: in the basic case, it can be represented as a sum of the value on scale at the beginning of the event and the measure of change. If John gained 2 kilos and his weight before this event was 70\,kg, his weight at after the event of gaining weight is 72\,kg. This leads to an idea of keeping only the proper scale in the semantic representation and express changes in terms of the difference between the absolute values, as it is done by \citealt{Kennedy:01} and \citealt{KennedyLevin:02} by means of differential degrees and degree addition. However, there are cases when the connection is not so straightforward.

To illustrate the last point, let us consider a lexicalized example of measure of change/proper scale opposition: duration/time pair. Duration can be seen as, but it is not reducible to a difference between two time points. For example, the event denoted by \ref{ex:John:dance} can consist of ten weekly one-hours classes. In this case the duration is a sum of the (approximate) durations of individual events, but not the difference between the time the first class started and the last class ended. 

\ex.\label{ex:John:dance}John took ten hours of dance classes.

\ex.\label{ex:Mary:bike}Mary did two hours of biking on Sunday. 

One can argue that such a case is special as multiple subevents are envolved. Indeed, in case of ten hours of dance classes we can represent the whole event as consisting of a series of events. This solution is not so obvious in case subevents do not naturally form a series: if \ref{ex:Mary:bike} is true, it could have been that Mary did two hours of continuous biking, or that she did one hour in the morning and one hour in the evening, or her whole day was full of small trips that resulted in a cumulative biking time of 2 hours (probably calculated by a fitness-tracker that also counted very short trips). I think that the semantic representation of the sentence should be neutral with respect to these scenarios, so I propose to keep distinct representations of time and duration as well as other proper and measure of change related attributes. This allows to leave the relation between the proper scale and the measure of change scale underspecified.

As it may seem that the discussion above is only relevant to the duration/time pair and not to the other types of scales, let me provide one more example. A hiking guidebook usually provides information about the elevation gain on the route. If one looks at the description of the circular route, the elevation gain will be positive (theoretically it can be also 0, but it is very improbable). At the same time, the difference between the elevation level at the start and at the end of the event is 0. In such situations, we are dealing with three different scales: a proper elevation scale that has heights as its points, the elevation measure of change scale, that represents the difference between the elevations of the start and the end points of the path, and the elevation gain scale that represents cumulative elevation gain on the route. From the example \ref{ex:elevation} we can conclude that English does not distinguish between the last two situations, as \ref{ex:elevation} can be interpreted as either the net elevation gain or the cumulative elevation gain of 1000 meters took place. 

\ex.\label{ex:elevation}The group of tourists went thousand meters up today.

\exg.\label{ex:elevation:rus}20 aprelja my podnjalis' na tysja\v{c}u pjat'sot metrov.\\
20 april we ascend.\glb{pst.pl.refl} on thousand five.hundred metres\\
\trans `On April 20 we made a 1500 meters ascend/reached the 1500 meters elevation.' (translation without context)\\
`By April 20 we had risen to an average level of 1,500 meters.' (English original)~~\hbox{}\hfill\hbox{\textit{Twenty thousand leagues under the sea}, Jules Verne, 1870}

As for Russian, some expressions can be interpreted using all the three scales: sentence \ref{ex:elevation:rus} is most naturally interpreted with respect to one of the measure of change scales, although it is a translation of the English sentence that refers to reaching the 1500 meters depth (by raising). On the basis of such observations, I would like to have the means for both the underspecification of the scale and the co-existence of various types of scales without hard connections between their points. For example, semantic representation of \ref{ex:elevation:rus} should only contain the information that the maximum point of the scale of the type \textit{elevation} is equal to 1500 meters without specifying whether this is a proper or a measure of change scale. If more information is available, as in \ref{ex:elevation:top}, both the measure of change (400 meters) and the elevation scale (with a marked point on 1917\,m) should be visible in the semantic representation.

\ex.\label{ex:elevation:top}We gained another 400 meters and reached the top of Mount Washington.

In sum, the crucial difference between the measure of change and the proper scale types is that only the latter type is directly bound to some parameters of the world, whereas for each measure of change scale there exist multiple proper scales it can correspond to. I claim that some of Russian prefixes are sensitive to this property, so in my analysis I will distinguish not only between open\slash closed\slash upper-closed and various dimensions of the scale, but also between proper scales (my term) and measure of change scales (term borrowed from \citealt{KennedyLevin:08}). It is also possible to relocate this property from the scale level to the level of the event: in this case a \textit{proper scale} event would be an event for which each degree on the scale is mapped to a unique time point, and a \textit{measure of change} event would only require the extreme points to be mapped to different time points. In the proposal presented here I leave the proper/measure of change feature on the level of scale properties, but in the future work I would like to investigate whether the event level would be conceptually more appropriate.

%Another important observation is that there is no unique mapping between the portions of event and the non-extreme points on the measure of change scale. Consider the sentence \ref{ex:Kennedy1} again. It conveys the following information: at some moment of time the poem did not exist and one hour later it was fully written. It does not tell us how the process was organized in between. There is some freedom in the selection of the dimension: it can be the length of the poem in symbols, words, or lines (a spatial extent scale), it can be a completeness scale that goes from 0$\%$ to 100$\%$, it can be a binary scale that represents non-existence as 0 and existence as 1. Even if we choose a scale that is non-binary, there is no guarantee that the event proceeds through the scale really incrementally: in the coarse of writing a poem some lines can be written and then erased and the poem can become shorter, maybe even without becoming longer again. The completeness scale is the scale along which the writing proceeds incrementally, but these is no information about how intermediate points on that scale are mapped onto the subevents. This leads to the conclusion that from the point of view of the quantity of information such scales are equivalent to binary scales. 

%A similar observation can be made for degree achievements: when clothes are hang outside to dry in the evening and the drying begins, then it rains during the night but they still become dry in the morning, the whole event can be described by \ref{ex:dry}. 

%\ex.\label{ex:dry}The clothes dried in 12 hours. 

%So when we analyze the sentence that involves a direct object like \textit{twenty songs,} we can talk about scalar analysis of the verb already when we assume that at the beginning of the event no songs have been listened and in the end all the twenty songs are listened, but there is no ordering on the parts of the set of songs neither any requirements that subevents of listening to individual songs do not overlap: in the extreme case, all the twenty songs could have been listened simultaneously. This situation is similar to a situation of a closed two point scale. This means that the situation when the direct object only provides one value is similar to the situation of a two point scale: there is a transition between those two points, but as we do not have any information about how the event proceeds between those points, we can


%As throughout this chapter I am going to say things like `the direct object contributes a scale' I want to talk a bit about what does it mean. Although it seems to be very natural to think about peeling 5 kilos of potatoes as proceeding gradually along the \textit{amount} scale, the connection between the referent of the direct object and stages of the event is not transparent. On one hand, scales themselves are ordered (according to the definition), so for any two points on the scale that we will be using to talk about the event progress we know which of them precedes the other. On the other hand, direct objects do not provide scales, but only supply values that can be used to construct a scale. In the potato peeling example there is some amount of the object and it all has to be involved in the event, so we can say that the event proceeds along the \textit{measure of change} scale for \textit{amount}, starting with 0 and ending with the full amount. This does not imply that there is any ordering on the parts of the direct object such that some of them get involved in the event earlier than others. In other words, for any two potatoes among those in the 5 kilo package we do not know which of them will be peeled earlier. So \textit{amount} itself is a scale, but what is contributed by the direct object is only a value on that scale and no mapping of the intermediate points on the scale onto the parts of the direct object is provided. It is also not required that the events proceeds gradually through the parts of the direct object: if Mary listened to 20 songs she could, in principle, have listened them at once. 

%Not all the incremental theme verbs are as complex. Eating an apple, for example, is supposed to proceed incrementally along the volume scale. Even this is not so straightforward, though. If one bites an apple, the volume that is eaten should increase at that moment. But what is they take the bite out of the mouth and put it on the table and then actually chew and swallow after the rest of the apple is consumed? What I want to say with this is that there is no linguistic information that can provide a cue about the coarse of the event. It is only the start and the end of the event that we know something about. 


%The situation if different with \textit{path} and \textit{time}. They are scales themselves, not points that allow to select an appropriate portion on the measure of change scale. 

%from Beavers:12\\
%(56) a. Caesar wiped the table clean (in/?for an hour).
%b. Caesar wiped the table (cleaner and cleaner) (for/??in an hour).
%
%Furthermore, progress towards cleanliness here may allow backtracking and
%stopping — something may get cleaner and dirtier on its way to cleanliness,
%and may spend some time at various levels of cleanliness. Looping, though,
%is not allowed, since the scale is one dimensional (as per fn. 20). These
%interpretations further motivate an MR-type analysis.
%]]
%
%These data also suggest that result XPs like clean, a fierce red, and
%to a nice shine can supply endpoints on scales just as goal PPs supply
%endpoints on paths, i.e. result XPs and goal PPs are two manifestations of
%the same notion. Further evidence for this comes from the fact that goal
%and result modifiers have similar effects on durativity. Recall that simplex
%paths give rise to punctual motion predicates, while complex paths give
%rise to durative motion predicates, for an atomic theme. For change-ofstate
%predicates, some result XPs likewise determine punctuality, while
%others determine durativity. This is shown in (57), where shot the sheriff
%is punctual with the result XP dead but durative with to death.
%(57) a. Wyatt shot the sheriff dead in five minutes. (after)
%b. Wyatt shot the sheriff to death in five minutes. (after/during)
%
%--------------

%An obvious exception is time: if the direct object contains an attribute related to the time, the subevents are automatically ordered. For example, if John is going to run for twenty minutes, he will first run for the first of those minutes, then for the second, than for the third, etc. This means that in case of the property and extent scales 
%
%The \textit{path} scales are in between the \textit{time} scales and other types of sales. On one hand, there are multiple ways to from the bank to the store that are acceptable to refer to with \ref{ex:Kennedy2}. On the other hand, there is an order of the points of each path, so if a particular path is selected the situations turns out to be similar to that for the \textit{time} scale. Intuitively, if the event of going from the bank to the store described by \ref{ex:Kennedy2} is close to its end, we know that Kim not only covered most of the distance between the origin and the destination but is also located at some point that is close to the store and not to the bank. In case of peeling potatoes when the event is close to the termination point we know that the peeled area of potatoes is close to the total area to be peeled but the probability of a certain potato to be peeled at the moment is equal for all potatoes.   there is an internal order of points that can be captured by the following construct. Take the length of the shortest existing path (that goes through all the relevant intermediate destinations) between the bank and the store. Mark this length as a maximum point on the measure of change scale for \textit{distance}. Now any path between the bank and the store (even those that have a different length) can be mapped onto the marked chunk on the measure of change scale, providing equivalence classes for the different points in space\footnote{The procedure is trickier in case there are intermediate destinations, e.g. when Kim is the owner of the store and has to pick up a key that she has at home on the way from the bank to the store which results in going through some point $x, y$ (space coordinates) two times: one before Kim picks up the key and one after that. In this case we have to add a dimension (just a simple one that has a $+/- key$ feature only) so that we are able to distinguish the equivalence class of the $(x, y, -key)$ point from that of $x, y, +key$ point.}.
%
%It is not important whether a banana is eaten from the top to the bottom or from the bottom to the top, it can be still described by the same predicate. It is important whether the path from the bank to the store is passed in one direction or in the other, the sentence \ref{ex:Kennedy2} is true only in one case. On the other hand, if there is a distance description instead of the path, it behaves in the same way as the information about the quantity of the direct object: the sentence \ref{ex:ran} is true independently of the direction of running. In this case, five kilometers serve not as a scale, but as the maximum point on the measure of change scale
%
%\ex.\label{ex:ran}John ran 5 kilometers.
%
%The distinction between the situations when the direct object or the context provide a point that allows to identify the appropriate interval on the measure of change scale and the situations when the direct object supplies directly the scale along which the event proceeds will be 

%A general idea underlying the scalar approach to event semantics is that an open scale in the event semantic structure leads to atelic interpretation and a closed scale leads to a telic interpretation (see, e.g., \citealt{KennedyLevin:08} or \citealt{Kearns:07}). The correlation is very prominent when one studies the variable behavior of degree achievements.

%Similarly, Piñón (2005, 2008) and Caudal and Nicolas (2005) propose degree-based accounts of aspect, which they apply to different predicates. 

%Levin and Rappaport Hovav (2006, and subsequent work) argue that the notion of telicity can in general be associated with scalar change. Also in this volume, they make a distinction between scalar change associated with particular verbs (their result verbs), which are basically change of state verbs such as break or open, and non-scalar change (their manner verbs). In the latter case, however, a scale can be introduced by the internal argument (e.g. with incremental theme verbs, on which see also Kennedy 2012) or by a path phrase (with motion events). Again, if the scale is bounded, the eventuality is interpreted as telic. Beavers (2008) builds on this system and adds the important observation that scales can be simple (a transition between two states with no intermediate states, as in achievements) or complex (as in accomplishments).

%There are a lot of proposals concerning the treatment of scales in the literature. I will not go into many details and peculiarities and will not discuss different approaches to formalization of scalar structures. I will only introduce those notions and distinctions that are relevant for the discussion in this chapter.
%
%So what we are going to need: 
%\begin{enumerate}
%\item the concept of the \textit{scale} (definition);
%\item the notion of the \textit{dimension} of the scale;
%\item the distinction between \textit{open} and \textit{closed} scales;
%\item the notion of a \textit{measure of change} scale.
%\end{enumerate}


\section{\textit{za-}}\label{subsection:semantics:za}
\subsection{Semantic contribution}
There are three main uses of the prefix \textit{za-} as described in the dissertation by \citet{Braginsky:08}: spatial, resultative and inchoative. The resultative meaning is further subdivided into four categories that Braginsky calls \textit{accumulative, cover, damage} and \textit{get}. In his dissertation, Braginsky shows that different usages of \textit{za-} can and should be analyzed in a unified way. \citet{Braginsky:08} argues convincingly that these meanings are neither applied without restrictions to any verbs nor distributed among different verbs. So a particular verb does not have to be compatible with any meaning of \textit{za-} nor does it have to have at most one interpretation when prefixed with \textit{za-}.

I will, however, only look in detail at the inchoative\footnote{I follow \citet{Braginsky:08} and adopt the term \textit{inchoative,} that he takes from \citealt{Zemskaja:55} and \citealt{Zaliznjak:95}. There are alternative terms in the literature, referring to the same usage of \textit{za-}, such as \textit{inceptive} or \textit{ingressive}, see also the relevant discussion in \citealt{Maslov:65}.} use of \textit{za-}, that is considered superlexical. The analysis provided here is extendable to other uses of \textit{za-}. For example, \cite{ZinovaOsswald:paper} cover the case of the spatial interpretation of the prefix \textit{za-}. The extension to the resultative uses is also possible, but requires some more work in order to define the procedure of selecting a scale along which the event is measured. Some of the resultative usage cases are covered in \citet{Zinova:14}, a paper about the locative alternation in Russian and English. The approach presented there is concerned with `accumulative' and `cover' subclasses of the resultative meaning of \textit{za-}, but does not include the `damage' type of meaning (see \citealt{Braginsky:08} for more details about the classification of the resultative sub-meanings).

As for the description of the semantics of the inchoative \textit{za-}, \citet{Braginsky:08} writes \citep[also referring to the work of][]{Sheljakin:69} that ``the function of the inchoative ZA- is to ensure that a given process\slash state, denoted by an input verb, has passed from the state of non-existence into existence.'' Importantly, there are no restrictions imposed by \textit{za-} on the duration of the process or state that is initiated.

\subsection{Restrictions on the attachment}
There exist a lot of discussions on the types of verbs that serve as input for prefixation with the inchoative \textit{za-} \citep{Isachenko:60, Zemskaja:55, Sheljakin:69, Zaliznjak:95, Braginsky:08}. Most of the work focuses on listing different types of possible derivational bases, but as this list turns out to be too long and still unlikely to be complete, I will try to approach the problem from the other side and concentrate on listing the restrictions on the derivational bases.

When one thinks about the inchoative semantics of the prefix \textit{za-}, the obvious restriction on the derivational base that will be prefixed with it is the presence of a time scale in the verbal semantic structure. On one hand, it seems that all verbs are connected to a time scale. On the other hand, there are indeed verbs that cannot be combined with the inchoative \textit{za-} and such verbs seem to be not non-eventive predicates. Let us first explore the literature on this point.

\citet[275]{Braginsky:08}, based on the proposals by \citet{Sheljakin:69} and \citet{Paducheva:96}, formulates the following conditions that have to hold in order for the verb to be incompatible with any of the core meanings of \textit{za-}:
\begin{enumerate}
\item the verb is not compatible with expressing motion into some location;
\item the verb does not have theme arguments;
\item the verb is not localized in time or the event denoted by the verb holds for extra-long intervals.
\end{enumerate}

The first condition captures the verbs that are combined with \textit{za-} in its spatial meaning and the second condition plays a role if one wants to attach the resultative \textit{za-} to the derivational base. What is interesting for us here is the third condition, as it refers to the inchoative usage of the prefix \textit{za-}. There are, according to \citet{Paducheva:96}, three classes of verbs the meaning of which is not compatible with the meaning of the initiation:

\begin{enumerate}
\item State verbs that denote properties and relations that are atemporal, i.e., cannot be localized at specific time moment or interval: \textit{stoit'}$^{\IPF}$ `to cost', \textit{vesit'}$^{\IPF}$ `to weigh', \textit{zna\v{c}it'}$^{\IPF}$ `to mean', \textit{imet'}$^{\IPF}$ `to have'.
\item State verbs denoting situations that are steady,  i.e., hold for extra long temporal intervals: \textit{golodat'}$^{\IPF}$ `to hunger', \textit{ljubit'}$^{\IPF}$ `to love', \textit{gorditsja'}$^{\IPF}$ `to feel proud', \textit{znat'}$^{\IPF}$ `to know'.
\item Activity verbs denoting occupation and behavior: \textit{\v{z}it'}$^{\IPF}$ `to live', \textit{pravit'}$^{\IPF}$ `to rule', \textit{u\v{c}itel'stvovat'}$^{\IPF}$ `to work as a teacher', \textit{filosovstvovat'}$^{\IPF}$ `to philosophize'.
\end{enumerate}

\citet{Paducheva:96} also writes that verbs denoting atemporal properties do not occur with punctual time or duration modifiers (e.g., \textit{sej\v{c}as} `now', \textit{vsegda} `always', \textit{X dnej} `for X days'). This seems reasonable if there is no time scale made available for these verbs, but it turns out to be an invalid observation: examples in \ref{ex:atemporal:adv} illustrate successful combinations of verbs listed above with such modifiers.


\ex.\label{ex:atemporal:adv}\ag.Moloko sej\v{c}as stoit 60 rublej za litr.\\
milk now cost.\glb{pres.sg.3} 60 rubles for liter\\
\trans `Milk costs 60 rubles per liter now.'
\bg.Takaja formulirovka vsegda zna\v{c}it otkaz.\\
such formulation always mean.\glb{pres.sg.3} rejection\\
\trans `Such formulation always means a rejection.'
\bg.On vesil 100 kilogramm 5 let.\\
he weigh.\glb{pst.sg.m} 100 kilos 5 years\\
\trans `He weighed 100 kilos for 5 years.'

Similar problem occurs with the observations made by \citet{Paducheva:96} about the verbs denoting steady states. \citet{Paducheva:96} writes that they are incompatible with punctual (as \textit{v X \v{c}asov} `at X hours'), frequency (as \textit{dva\v{z}dy} `twice', \textit{inogda} `sometimes') and intensive duration  (as \textit{ves' den'} `all the day long') modifiers. Examples in \ref{ex:steady:adv} illustrate that at least some of the verbs belonging to that class are compatible with some of those modifiers.

\ex.\label{ex:steady:adv}\ag.On ljubil dva\v{z}dy: v 18 i v 35.\\
he love.\glb{pst.sg.m} twice: in 18 and in 35\\
\trans `He loved twice: when he was 18 and when he was 35.'
\bg.On gordilsja synom ves' den', poka ve\v{c}erom oni ne porugalis'.\\
he feel.proud.\glb{pst.sg.m} son.\glb{instr} whole day, until evening they not argued\\
\trans `He felt proud of his son for the whole day, until they had an argument in the evening.'

Another observation is that if verbs like \textit{stoit'}$^{\IPF}$ `to cost' or \textit{zna\v{c}it'}$^{\IPF}$ `to mean' were atemporal and verbs like \textit{ljubit'$^{\IPF}$} `to love' were not semantically compatible with time descriptions, then the sentences in \ref{ex:steady:time} would not be acceptable.

\ex.\label{ex:steady:time}\ag.No vposledstvii my uvidim, kak i pod kakimi vlijanijami \`{e}tot obraz u nego razvilsja i \v{c}to stal zna\v{c}it'.\\
but later we will.see, how and under which influence this image of he.\glb{gen} develop.\glb{pst.sg.m}.refl and what become.\glb{pst.sg.m} mean.\glb{inf}\\
\trans `But we will see later how and under which influence this image of his developed and which meaning did it acquire.'\\\hbox{}\hfill\hbox{V. F. Xodasevi\v{c}. \textit{Esenin} (1926)}
\bg.Cement-to voob\v{s}\v{c}e be\v{s}enye den'gi stal stoit'!\\
cement-somehow {at all} mad money become.\glb{pst.sg.m} cost.\glb{inf}\\
\trans `Moreover, cement somehow started to cost a crazy amount of\linebreak money!'\hbox{}\hfill\hbox{Roman Sen\v{c}in. \textit{Elty\v{s}evy} (2008)}
\bg.Lida zdravo ob'jasnjala, \v{c}to tak ne byvaet, \v{c}toby v\v{c}era ljubil, a segodnja zabyl.\\
Lida soundly explained, that so not be.imp.\glb{pres.sg.3}, that yesterday loved, but today forget.\glb{pst.sg.m}\\
\trans `Lida explained soundly that it cannot be that today he forgot the person he loved yesterday.'\\\hbox{}\hfill\hbox{Nina Gorlanova. \textit{Filologi\v{c}eskij amur} (1980)}

In sum, verbs of these three classes are special in the sense of the relation to the time scale, but not ``atemporal'': they are compatible with time specifications. \citet[132]{Paducheva:96} herself notes that ``[m]nogie glagoly javljajutsja ili ne javljajutsja atemporal'nymi v zavisimosti ot tipa subjekta'' (many verbs are or are not atemporal dependent on the type of the subject). As an example she points to the verb \textit{stojat'} `to stand' that is, according to her, atemporal\footnote{According to \citet{Paducheva:96} the incompatibility with the adverbial \textit{sej\v{c}as} `now' is diagnostic of atemporality.} only when used with non-animated subjects, as in \ref{ex:stojat':xram} and not with animated subjects, as in \ref{ex:stojat':Vasja}.

\ex.\label{ex:stojat':xram}\ag.Xram stoit na xolme.\\
church stand.\glb{pres.sg.3} on hill\\
\trans `The church stands on the hill.'
\bg.\label{ex:stojat':xram2}$^?$Xram sej\v{c}as stoit na xolme.\\
church now stand.\glb{pres.sg.3} on hill\\
\trans `The church now stands on the hill.'

\ex.\label{ex:stojat':Vasja}\ag.Vasja stoit na xolme.\\
Vasja stand.\glb{pres.sg.3} on hill\\
\trans `Vasja stands on the hill.'
\bg.Vasja sej\v{c}as stoit na xolme.\\
Vasja now stand.\glb{pres.sg.3} on hill\\
\trans `At the moment, Vasja stands on the hill.'

In fact, the verb \textit{stojat'} `to stand' exhibits some atemporality (or, better, it is not compatible with the adverbial \textit{sej\v{c}as} `now') only when it is uttered with some of the subjects. Consider the noun \textit{kniga} `book'. Example \ref{ex:stojat':kniga} illustrates that the combination of the verb \textit{stojat'} `to stand' with the non-animate subject \textit{kniga} `book' and an adverbial \textit{sej\v{c}as} `now' is possible.  In my view, this is a clear evidence that ``atemporality'' is not a property of a verb, but part of the world knowledge: it is hard to imagine the church moving around in the normal world, so it does not make sense to utter \ref{ex:stojat':xram2}. The sentence becomes fine if uttered in a world where buildings can disappear and appear again at a nearby location. There are also cases when similar sentences can be uttered to describe a situation in our world: for example, there are some famous houses in Moscow that were moved to allow to widen the road. Another possibility is a change in the landscape: a small island may result being a hill if the water level drops. 

Note that if the word order (and, thus, the information structure) is changed in such a way that the hill becomes the focus of the sentence, as in \ref{ex:stojat':xram3}, the initial sentence \ref{ex:stojat':xram2} becomes unmarked also if uttered in the real world in non-exceptional situations. This favors the hypothesis that the problem with the sentence \ref{ex:stojat':xram2}, noticed by \citet{Paducheva:96}, is not due to the semantic properties of the verb \textit{stojat'} `to stand'. It also seems reasonable to suggest that the same applies to similar verbs in other languages. 

\ex.\label{ex:stojat':kniga}\ag.Kniga stoit na polke.\\
church stand.\glb{pres.sg.3} on shelf\\
\trans `The book is on the shelf.'
\bg.Kniga sej\v{c}as stoit na polke.\\
book now stand.\glb{pres.sg.3} on shelf\\
\trans `The book is on the shelf.'

\exg.\label{ex:stojat':xram3}Na xolme sej\v{c}as stoit xram.\\
on hill now stand.\glb{pres.sg.3} church\\
\trans `On the hill there is now a church.'

Let us now examine closer the incompatibility of the inchoative prefix \textit{za-} with verbs denoting atemporal/steady situations or occupations. At the first glance, verbs like \textit{*zastoit'} (\textit{za+stoit'} `za + to cost'), \textit{*zavesit'} (\textit{za+vesit'} `za + to weigh'), \textit{*zazna\v{c}it'} (\textit{za+zna\v{c}it'} `za + to mean'), \textit{*zagordit'sja} (\textit{za+gordit'sja} `za + to feel proud'), \textit{*zau\v{c}itel'stvovat'} (\textit{za+u\v{c}itel'strvovat'} `za + to work as a teacher') seem to be non-existent. However, after a careful consideration it becomes clear that there is no semantic reason why the core meaning of such verbs cannot be combined with that of the inchoative \textit{za-}. It turns out that these (and similar) verbs can be divided in the following three categories:

\begin{enumerate}
\item Verbs that can be prefixed with the inchoative \textit{za-}, as \textit{u\v{c}itel'stvovat'} `to work as a teacher'. The derived inchoative verbs are not frequent and thus seem odd out of the context, but native speakers do occasionally use them, as illustrated by \ref{ex:zateach}.
\item Verbs that can be combined with the resultative \textit{za-}, as \textit{zagordit'sja} `to become stuck-up', \textit{zavesit'} `to weigh something' (colloquial).
\item Verbs that do not exist in combination with the prefix \textit{za-}, as \textit{*zastoit', *zazna\v{c}it'}.
\end{enumerate}

\exg.\label{ex:zateach}Malen'kij Ilja \'eto soobra\v{z}al, a bol\v{s}oj vyros -- zava\v{z}ni\v{c}al, zau\v{c}itel'stvoval, nu i polu\v{c}il spolna, \v{c}to zarabotal!\\
little Ilja this understand.\glb{pst.sg.m}, but big grow.\glb{pst.sg.m} -- za.showboat.\glb{pst.sg.m}, za.teach.\glb{pst.sg.m}, {} and receive.\glb{pst.sg.m} full, that earn.\glb{pst.sg.m}\\
\trans `When he was little, Ilja understood this, but when he grew up, he started to showboat, to teach others, and got everything he deserved!'\\\hbox{}\hfill\hbox{
\url{positive-lit.ru/novels/gde-konchajutsa-relsy/224}}

The difference between the first group of verbs and the other two that one may see when looking at the lists above (except for the verb \textit{zagordit'sja} `to become stuck-up') is that verbs like \textit{u\v{c}itel'stvovat'} `to work as a teacher' are intransitive.\footnote{The verb \textit{zagordit'sja} `to become stuck-up', it is a reflexive verb, so in some sense the direct object is ``integrated'' in the verb, so we will leave it aside.}

Let us explore this connection. Note that there are verbs that can be combined both with the resultative and the inchoative \textit{za-}. In such cases one can notice that the verb with the inchoative \textit{za-}, as in \ref{ex:zagovorit:inch}, is intransitive, whereas the verb with the resultative \textit{za-}, as in \ref{ex:zagovorit:res}, is transitive.

\ex.\ag.\label{ex:zagovorit:inch}On zagovoril.\\
he za.talk.\glb{pst.sg.m}\\
\trans `He started talking.'
\bg.\label{ex:zagovorit:res}On zagovoril menja.\\
he za.talk.\glb{pst.sg.m} me\\
\trans `He made me forget about something by his talking.'
 
An evident exception to this observation are motion verbs. With motion verbs, transitiveness does not prevent the attachement of the inchoative \textit{za-}, as illustrated by \ref{ex:zacarry:indet}. At the same time, the resultative \textit{za-} cannot be attached to the motion verbs. What can be attached is the spatial \textit{za-}, but it requires the \textit{path} scale to be presented in the structure of the verb and the path itself has to be provided (more details in \citealt{ZinovaOsswald:paper}). As we have discussed in Section~\ref{subsection:perf:motion}, prefixes acquire spatial interpretations only with the determinate motion verbs. The derived prefixed verbs (see example \ref{ex:zacarry:indet}) may, in turn, look identical to the corresponding indeterminate motion verbs that are prefixed with the same prefix (see \ref{ex:zacarry:det:pf}) and then imperfectivized (see \ref{ex:zacarry:det:ipf} and compare the examples \ref{ex:zacarry:indet} and \ref{ex:zacarry:det:ipf}).
 \ex.\label{ex:zacarry}\ag.\label{ex:zacarry:indet}Ma\v{s}a zanosila$^{\PF}_{\text{\INDET}}$ posylki.\\
 Ma\v{s}a za.carry.\glb{pst.sg.f} parcel.\glb{pl.acc}\\
 \trans `Masha started carrying parcels.'
\bg.\label{ex:zacarry:det:pf}Ma\v{s}a zanesla$^{\PF}_{\text{\DET}}$ posylku Kate.\\
 Ma\v{s}a za.carry.\glb{pst.sg.f} parcel.\glb{sg.acc} Katja.\glb{dat}\\
 \trans `Masha brought Katja the parcel.'
\bg.\label{ex:zacarry:det:ipf}Ma\v{s}a zanosila$^{\IPF}_{\text{\DET}}$ posylku Kate.\\
 Ma\v{s}a za.carry.\glb{pst.sg.f} parcel.\glb{sg.acc} Katja.\glb{dat}\\
 \trans `Masha was carrying the parcel to Katja.'
 
As is pointed out by \citet[227]{Braginsky:08}, some transitive non-motion verbs can be prefixed with the inchoative \textit{za-} if the direct object is a bare plural noun (no measure phrases or numeral expressions).

\exg.Ivan za\v{c}ital$^{\PF}$ (*vse) / (*tri) / (*kak minimum tri) knigi.\\
Ivan za.read.\glb{pst.pl.m} all / three / at least three books\\
\trans `Ivan started reading books (in general).'\\\hbox{}\hfill\hbox{= example (17) in \citealt[227]{Braginsky:08}}

The verb \textit{\v{c}itat'} `read' can also be combined with the resultative \textit{za-}. The output is the verb \textit{za\v{c}itat'} `to damage as a result of prolonged reading' \ref{ex:zachital}. In this case the direct object must be definite, so also a bare plural noun is interpreted as a definite description.

\exg.\label{ex:zachital}Ivan za\v{c}ital$^{\PF}$ vse knigi.\\
Ivan za.read.\glb{pst.sg.m} all books\\
`Ivan damaged all the books by his reading.'\\\hbox{}\hfill\hbox{= example (37a) in \citealt[246]{Braginsky:08}}

The unifying property of all the examples we have just considered is that in cases when the attachement of the inchoative \textit{za-} is not possible, some scale, except for the time scale, is available either due to the verbal semantic structure or due to the direct object. In parallel, when the inchoative \textit{za-} can be attached, time scale is the only scale available. On the basis of this observation I agree with \citet{Paducheva:96} that the relation to the time scale is the crucial property for the attachment of the inchoative \textit{za-}, but I want to propose a different explanation for this fact. I claim that what prevents these verbs that have been categorized as holding for extra-long intervals of time by \citet{Paducheva:96} from being prefixed with the inchoative \textit{za-} is that they lexicalize some specific scale: the event of weighing is by default measured in some weight units, not in terms of time, as an event of jumping, for example. Time specification is still available for such verbs, but it is not the default domain, which prevents them from being combined with the inchoative \textit{za-}. This is related to the other pattern we will discuss later in this chapter: verbs that do not lexicalize any other scale, except for the time scale, are usually capable of serving as a source for prefixation with the delimitative prefix \textit{po-} (applied to the time scale). 

The proposed explanation does not cover the case of is the verb \textit{ljubit'} `to love', as it seems to be no other scale except for the \textit{time} in the semantic structure of this verb. I do not have an answer why the verb \textit{ljubit'} `to love' cannot be prefixed by the inchoative \textit{za-}, but I would like to note that is can acquire inchoative interpretation when it is prefixed with \textit{po-}. The result of the prefixation is the verb \textit{poljubit'} `to fall in love with'. If the verb \textit{ljubit'} were atemporal, the derivation of a verb with an inceptive interpretation from it would not be possible with any prefix, yet it is possible and also unusual, as the prefix \textit{po-} is (except in this case) only interpreted inchoatively when attached to determinate motion verbs. So it seems that the verb \textit{ljubit'} `to love' is special and deserves an investigation from the historical linguistics perspective. 

Let us now discuss another example, the verb \textit{za\v{z}eltet'} `to become seen as yellow', mentioned by \citet{Braginsky:08} as a verb that contains the inchoative \textit{za-}. The verb \textit{\v{z}eltet'} has two interpretations: `to become yellow' and `to have yellow color and be seen'. These two interpretations are connected to different internal scales: the first one is about color intensity, whereas the second one is about the visibility while the color is constant (yellow). The two interpretations also lead to different prefix contributions when \textit{za-} is attached: resultative semantics of the derived verb in case of `to become yellow' meaning of the derivational base, as illustrated by \ref{ex:zazeltet1}, and inchoative interpretation in case the derivational base denotes a  situation in which the object that has yellow color becomes visible, as in \ref{ex:zazeltet2}.

%In previous work (1991, 2006, and Rappaport Hovav and Levin 2010) they argue for a particular constraint on what a verb root can lexicalize, which has come to be known as manner/result complementarity. In particular, they propose that a single verb root can lexicalize manner (non-scalar change) or result (scalar change), but not both at the same time. In the contribution to this volume, the authors underline that this complementarity is a constraint rather than a tendency, and they discuss two cases, which have been brought forwards as counterexamples to the manner/result complementarity, namely cut and climb. 

\ex.\label{ex:zazeltet}\ag.\label{ex:zazeltet1}On podros i sdelalsja neprijatno zubastym, glaz za\v{z}eltel, zra\v{c}ki priobreli demoni\v{c}eskuju vertikal'nuju formu.\\
he pod.grow.\glb{pst.sg.m} and s.make.\glb{pst.sg.m.refl} unpleasantly toothy, eye za.become.yellow.\glb{pst.sg.m}, pupils pri.get.\glb{pst.pl} demonic vertical form\\
`He grew up and became unpleasantly toothy, his eye became yellow-colored and pupils acquired demonic vertical form.'\\\hbox{}\hfill\hbox{\url{https://books.google.com/books?isbn=5457040119}}
\bg.\label{ex:zazeltet2}\v{C}erez neskol'ko minut na gorizonte za\v{z}eltel svet far.\\
across several minutes on horizon za.seen.as.yellow.\glb{pst.sg.m} light headlight.\glb{pl.gen}\\
`In several minutes yellow headlights appeared on the horizon.'\\\hbox{}\hfill\hbox{\url{https://books.google.com/books?isbn=5457264963}}

It is sometimes very hard to distinguish between the resultative and the inchoative interpretations of the prefix \textit{za-}. To do this, the first idea is to use a part of the test traditionally used as a test for telicity (see Section~\ref{section:new:telicity}): try to modify the verbal phrase with a time measure phrase like \textit{za 3 \v{c}asa} `in 3 hours'. If this is not possible, then the verb can only have inceptive intepretation. Unfortunately, there is no implication in the other direction: if the event described by the inchoative verb has a non-instantaneous preparatory phase, such a verb is also compatible with the \textit{za 3 \v{c}asa} `in 3 hours' measure phrase. In order to distinguish such verbs from \textit{za}-prefixed verbs that have resultative interpretation, I propose to use the context schematically represented in \ref{context:za}.

\exg.\label{context:za}On Y-al, Y-al, i za-Y-al.\\
he verb.\glb{pst.sg.m} verb.\glb{pst.sg.m} and za.verb.\glb{pst.sg.m}\\
\trans `He was Y-ing, Y-ing, and finally Y-ed.'

Such contexts can be embedded directly into the original sentence in order to check the interpretation of the given verb in the given context. If the structure \ref{context:za} can be successfully embedded in the sentence, the usage of the verb prefixed with \textit{za-} is resultative. If the sentence does not make sense after the insertion of the context \ref{context:za} in it, the prefix \textit{za-} has inchoative semantics.

Let us run the test with the sentences in \ref{ex:zazeltet} in order to illustrate how it works. We substitute the verb \textit{za\v{z}eltel} `became yellow/seen as yellow' with the phrase \textit{\v{z}eltel, \v{z}eltel, i za\v{z}eltel} that under the `to become yellow' interpretation of the verb \textit{\v{z}eltet'} means `was becoming and becoming more yellow and then became yellow'. The same phrase under the `to have yellow color and be seen' interpretation of the verb \textit{\v{z}eltet'} can be translated as `it was yellow and was seen and seen and then it appeared and it was yellow'. It is obvious that the second interpretation of this phrase does not make sense, so the whole sentence \ref{ex:zazeltet:test2} can not be interpreted. The sentence \ref{ex:zazeltet:test1} is a perfect Russian sentence (although its English translation is not natural).

\ex.\label{ex:zazeltet:test}\ag.\label{ex:zazeltet:test1}On podros i sdelalsja neprijatno zubastym, glaz \v{z}eltel, \v{z}eltel, i za\v{z}eltel, zra\v{c}ki priobreli demoni\v{c}eskuju vertikal'nuju formu.\\
he pod.grow.\glb{pst.sg.m} and s.make.\glb{pst.sg.m.refl} unpleasantly toothy, eye become.yellow.\glb{pst.sg.m}, become.yellow.\glb{pst.sg.m}, and za.become.yellow.\glb{pst.sg.m}, pupils pri.get.\glb{pst.pl} demonic vertical form\\
\trans `He grew up and became unpleasantly toothy, his eye became more and more yellow and finally it turned completely yellow, and his pupils acquired demonic vertical form.'\\\hbox{}\hfill\hbox{\url{https://books.google.com/books?isbn=5457040119}}
\bg.$^\#$\v{C}erez neskol'ko minut na gorizonte \v{z}eltel, \v{z}eltel, i za\v{z}eltel svet far.\label{ex:zazeltet:test2}\\
across several minutes on horizon seen.as.yellow.\glb{pst.sg.m}, seen.as.yellow.\glb{pst.sg.m}, and za.seen.as.yellow.\glb{pst.sg.m} light headlight.\glb{pl.gen}\\
\trans $^\#$`After several minutes the yellow light was seen and seen and then appeared on the horizon.'

What these examples show is that in case the verb \textit{za\v{z}eltel} `to become yellow/to be yellow and become seen' has the color intensity scale in its structure (when interpreted as `to become yellow'), it acquires resultative meaning after being prefixed with \textit{za-}. If there is no other scale in the structure of the verb (for the second interpretation, `to be yellow and become seen' only the time scale is available), the attachment of the prefix \textit{za-} leads to the inchoative interpretation of the derived verb.

Similarly, obligatory transitive verbs are usually not compatible with the inchoative interpretation of the prefix \textit{za-}, as for these verbs the obligatory direct objects provide scales associated with them: the event of reading three books is measured in the cumulative length or quantity of the books that are read. As for the motion verbs, \textit{katat' tri tele\v{z}ki} `to roll three carts' is not measured by the number of carts rolled, as the action denoted by this phrase is perceived as happening simultaneously with all the three carts. So for indeterminate motion verbs the time scale is the only scale available. It is different in case of determinate motion verbs: the phrase \textit{katit' tri tele\v{z}ki} `to push three carts' describes rolling three carts along some path, so the attachment of the prefix \textit{za-} leads to the spatial interpretation.

There are also other cases, apart from indeterminate motion verbs, when the direct object does not contribute a scale to the verb and thus the attachment of the inchoative \textit{za-} is possible. This is, for example, the case of the verb \textit{xotet'} `to desire', mentioned by \citet{Braginsky:08}. As desiring three ice creams is not an event progressing along the \textit{quantity} scale but is only related to time, the prefix \textit{za-} has inchoative interpretation when attached to the verb \textit{xotet'} `to desire'.

\exg.\label{ex:zaxotet}Ivan zaxotel$^{\PF}$ tri moro\v{z}ennyx srazu.\\
Ivan ZA-wanted three ice-creams {at once}\\
\trans `Ivan began to want three ice-creams at once.'
\\\Source{= example (47b) in \citealt[254]{Braginsky:08}}

The explanation I offer for the (non-)availability of the inchoative interpretation of the prefix \textit{za-} with particular verbs is in some respect similar to the explanation of \citet{Braginsky:08}, who proposes that inchoative interpretations occur in cases where resultative interpretations are blocked. The absence of any other scale except for the time scale garantees that the resultative interpretation is not available. The advantage of the approach advocated here is that there is no need for a separate explanation for the cases when both resultative and inchoative interpretations are not possible, which is a part missing in the account of \citet{Braginsky:08}.

Now that we came closer to the understanding of the semantic properties that are required for the attachment of the inchoative prefix \textit{za-}, let us consider another type of restriction associated with this prefix. \citet{Tatevosov:09} categorizes \textit{za-} as a selectionally limited prefix, namely, a prefix that can be attached only to imperfective verbs. Judging from the available data and introspection, this generalization seems to be correct. A question one may ask is whether there is some deeper motivation for such a restriction. I claim that the answer to this question is positive and it again lies in the semantic domain. 

Let us consider the semantic structure of a perfective verb and the semantic contribution of the inchoative prefix \textit{za-}. A perfective verb normally (not always) denotes an event that is maximal with respect to some scale (i.e., the end point of that scale is reached). As we have just discussed, in order for the inchoative prefix \textit{za-} to be attached, the time scale should be the only available scale in the verbal semantic structure. This rules out the possibility to attach the inchoative \textit{za-} to any perfective verb prefixed with a prefix that selects not the time scale. What is left are those verbs that are measured with respect to the time scale (it can happen in case of perfective verbs with prefixes \textit{po-} and \textit{pere-}). The problem is that such events are associated with an endpoint at which the activity (denoted by the derivational base verb) stops. 

On the other hand, the inchoative \textit{za-} contributes the information that at the end of the event described by the derived verb the activity denoted by the derivational base is being performed. These two pieces of semantic information are incompatible and thus the attachement of \textit{za-} is not possible. There is one case when the explanation provided above would not be valid: this is the case when \textit{po-} has inceptive semantics. However, the inceptive semantics of \textit{po-} arises as a result of its attachment to a directed motion verb and is associated with an initial segment of the \textit{path} scale. There is one exception to this pattern, as we have seen above: the verb \textit{poljubit'} `to fall in love' contains the prefix \textit{po-} with inceptive semantics and it is not a motion verb. Indeed (and to my personal surprise), the verb \textit{zapoljubit'} `to start loving' is used by some native speakers, as illustrated by the example \ref{ex:zapoljubit}. The semantics of this verb is intensified inception, which is not a very clear concept, so I personally would not use it, but the number of the examples in the web evidencing this verb is such that its existence (at least in the colloquial language) is beyond doubt.

\exg.\label{ex:zapoljubit}ili \v{z}e, naoborot, igral s det'mi, \v{c}to o\v{c}en' v poslednee vremja zapoljubil\\
or again conversely played with children that very in last time za.po.love.\glb{pst.sg.m}\\
\trans `or, on the contrary, he played with children, which he suddenly started to love in the last time'\Source{\url{www.poezia.ru}}

From this it follows that the restriction on the aspect of the derivational base will follow out naturally from the semantic representations of the verbs and prefixes plus a principle that tells that two verbs belonging to a derivational chain cannot have exactly the same semantics, which is another way of saying that additional morphological complexity has to be avoided if the semantics is not enriched. As \citet{Braginsky:08} formulates it, ``the economy principle of the word-formation does not allow grammar to form new words with the
exact lexical meanings as the existing ones.'' This principle will be used often in the analysis I propose in this thesis.

%TODO: Perhaps this principle should be put into the context of the monotonicity hypothesis advocated by Koontz-Garboden (Fabienne); Horn constraint?

\subsection{Secondary imperfective}
It has been noticed that suffixing an inchoative \textit{za-}prefixed verb with the imperfective suffix is not always possible. The question when it is possible and when not is discussed in the literature, but the conclusions different authors arrive to are vague. For example, \citet[230]{Svenonius:04b} writes that ``inceptive \textit{za-} almost never forms secondary imperfectives in Russian'' and \citet[220]{Braginsky:08} states that ``some inchoative ZA-prefixed forms allow secondary imperfectivization.'' \citet[231]{Braginsky:08} also claims that ``[t]hose inchoative forms that do undergo secondary imperfectivization acquire a habitual reading of imperfective aspect, rather than a progressive one.'' In addition, he notes that this may be due to the fact that ``inchoative ZA-prefixed verbs are achievements'', but acknowledges that ``[t]he problem is, however, that most inchoatives block even a habitual secondary imperfectivization.'' On the other hand, in the account provided by \citet{Tatevosov:09} the inchoative prefix \textit{za-} is only associated with a restriction on its attachment site, but not with a restriction on the subsequent imperfectivization. With this in mind, let us look at the data. 

As we have already seen in Section~\ref{section:new:imperfectivization}, there are in fact cases when the imperfective verb derived from the \textit{za-}prefixed inchoative verb receives ongoing interpretation. One example, which we have already seen, is repeated under \ref{ex:zakurival:prog:rep}, another is given under \ref{ex:zakurival:prog:new}.

\exg.\label{ex:zakurival:prog:rep}Arkadij Sergeevi\v{c} kak raz zakurival, po\`{e}tomu ne zametil, kak na poslednej fraze Olafson po\v{c}emu-to vorovato strel'nul glazami.\\
Arkadij Sergeevich as time za.smoke.\glb{imp.pst.sg.m}, {that is why} not notice.\glb{pst.sg.m}, as on last phrase Olafson {because of something} thievishly shoot.sem.\glb{pst.sg.m} eye.\glb{pl.inst}\\
\trans `Arkadij Sergeevich was just lightning the cigarette, so he didn't notice Olafson's thievish glance during the last phrase.'\Source{= example \ref{ex:zakurivat2} here}

\exg.\label{ex:zakurival:prog:new}Ja dal emu sigaretu i, kogda on zakurival, ja zametil, \v{c}to u nego dro\v{z}at ruki.\\
I give.\glb{pst.sg.m} he.\glb{dat} cigarette and, when he za.smoke.imp.\glb{pst.sg.m}, I notice.\glb{pst.sg.m}, that near he.\glb{gen} tremble.\glb{inf} hand.\glb{pl.nom}\\
\trans `I gave him a cigarette and, when he was lightning it, I noticed, that his hands tremble.'\Source{Charles Bukowski, \textit{Jug bez priznakov severa}}\\\Source{[South of no north] (Russian translation)}

For many other verbs, however, the progressive interpretation is indeed impossible. \citet{Braginsky:08} provides the following examples of usages of perfective and imperfective verbs that contain the inchoative prefix \textit{za-}:

\ex.\label{ex:za:imp:Brag}\ag.\label{ex:za:imp:Brag1}Ivan zagovoril$^{\PF}$ / zagovarival$^{\IPF}$ s proxo\v{z}imi.\\
Ivan ZA-talked / {used to ZA-talk} with passers-by\\
\trans `Ivan started talking / used to start talking with the passers-by.'
\bg.\label{ex:za:imp:Brag2}Ivan zapel$^{\PF}$ / zapeval$^{\IPF}$ pesnju.\\
Ivan ZA-sang / {used to ZA-sang} song\\
\trans `Ivan started singing / used to start singing a song.'\\\Source{= ex.~(7) in \citealt[221]{Braginsky:08}}

Imperfective verbs in the examples \ref{ex:za:imp:Brag1} and \ref{ex:za:imp:Brag2} do not receive progressive interpretation. (At least, searching for the progressive usages of these verbs does not provide any result.) I claim that the difference between them and the verbs that allow progressive interpretation, as \textit{zakurivat'} `to start smoking' in the examples \ref{ex:zakurival:prog:rep} and \ref{ex:zakurival:prog:new}, is in the absence of a preparatory phase. 

The rule we can imply from this is the following: whenever a secondary imperfective is derived from the \textit{za-} prefixed verb with inchoative semantics, it can acquire progressive interpretation if the event denoted by the verb has a preparatory phase with a non-zero time span. In \ref{ex:zakurival:prog:new} the trembling happens in the period of lightning the cigarette, the end of which can be referred to by the perfective verb \textit{zakurit'} `to start smoking'\footnote{While English translation is ambiguous, Russian verb refers to the preparatory phase and not to the smoking event itself.}. 

So the idea of \citet{Braginsky:08} seems to be on the right track: many inceptive \textit{za-}prefixed verbs do not receive a progressive interpretation when imperfectivized because they denote achievements: inception events that are instantaneous and usually lack a preparatory phase. What \citet{Braginsky:08} has not described is the possibility of a progressive interpretation in case the event denoted by the verb can be coerced into an event with a preparatory phase. The preparatory phase here is understood as something that is unambiguously identified as preceding the start of the process/activity described by the derivational base verb. E.g., for the verb \textit{zaprygat'} `to start jumping' is it hard to imagine some phase that is unambiguously identified as preparation for jumping and is not a part of the jumping event. In case of \textit{zakurit'} `to start smoking' lightning a cigarette is, one one hand, an obvious preparation for smoking, but is also, on the other hand, not smoking per se. 

The situation with achievements in English is, in a way, similar: the progressive of some verbs denoting achievements is more acceptable than of some others (see examples \ref{ex:achiev:1} and \ref{ex:achiev:2}). As \citet{Rothstein:04} proposes, there is a possibility to coerce some achievements into accomplishments by adding a preparatory phase (for further discussion on this topic, see \citealt{Gyarmathy:15}).

\ex.\a.\label{ex:achiev:1}The train was arriving at the station.
\b.*John was finding his phone.\label{ex:achiev:2}

So the difference between the resultative and the inchoative interpretations of \textit{za-} can be formulated in the following way. Verbs prefixed with the resultative \textit{za-} focus on the culmination point (and may refer to this point plus a period that precedes it) achieved as a result of performing the action denoted by the derivational base, as revealed by the context \ref{context:za}. Verbs prefixed with the inchoative \textit{za-} focus on the point after which the action denoted by the derivational base is performed (and, again, may also refer to the preceding period), so they fit in the context \ref{context:za:inch}.

\exg.\label{context:za:inch}On za-Y-al i Y-al 10 minut.\\
he za.verb.\glb{pst.sg.m} and verb.\glb{pst.sg.m} 10 minutes\\
\trans `He started to Y and Y-ed for 10 minutes.'

Note also, that if a time measure phrase can be added to a verbal phrase headed by a \textit{za-}prefixed verb with the inchoative interpretation, this time phrase refers to the duration of the preparatory phase, but not to the duration of the initiated event. This is illustrated by \ref{ex:zarabotat:rep}. The implication to be derived is that such inchoative \textit{za-}prefixed verbs that allow progressive interpretation of the imperfective derived from them also should allow modification by the time measure phrase headed with the preposition \textit{za}. (There is no implication in the other direction as the completed preparatory phase can be identified via the initiated process, while an incomplete one requires other non-linguistic cues.)

\exg.\label{ex:zarabotat:rep}Kompjuter zarabotal za \v{c}etyre \v{c}asa.\\
computer za.work.\glb{pst.sg.m} behind four.\glb{acc} hour.\glb{sg.gen}\\
\trans `The computer started to work in four hours.'

Now we will explore the second point that has been noticed by \citet{Svenonius:04b} and \citet{Braginsky:08}, but not taken into account by \citet{Tatevosov:09}: the absence of the secondary imperfectives from many inchoative \textit{za-}prefixed verbs. 

The first class of such verbs consists of the verbs that in general do not form secondary imperfectives after being prefixed, such as \textit{\v{z}eltet'} `to become yellow/to be yellow and become visible'. As it is not possible to construct any secondary imperfective form of this verb, the restriction may be a phonological one or related to the fact that the verb is derived from a colour name. In this case the impossibility of the secondary imperfectivization seems to be associated with the verbal stem and not with the inchoative semantics of the prefix. 

The second class of verbs is more interesting: these are verbs that have secondary imperfectives, but not when prefixed with the inchoative \textit{za-}. For example, \textit{zatalkivat'} is an imperfective verb formed from \textit{zatolkat'} `to push inside/to start pushing', but it only means `to push/be pushing inside', not `to start/be starting pushing'. A similar behavior is observed for the verb \textit{zana\v{s}ivat'} that means `to wear/be wearing until the thing is damaged', but not `to start/be starting wearing', although the perfective verb \textit{zanosit'} can mean both `to wear until the thing is damaged' and `to start wearing'.

As for this class, the explanation I can offer is the following. On one hand, the resultative meaning of such verbs when they are prefixed with \textit{za-} is much more common than the inchoative meaning. So when the secondary imperfective verb is analyzed, the more frequent meaning is processed as a candidate meaning for the source perfective verb. And, as we have discussed above, resultative and inchoative interpretations are produced on the basis of different interpretations of the derivational base (one involving only the time scale, another including some other scale), so there is no possibility of an easy shift between these interpretations. On the other hand, there is an alternative lexical way to express the inchoative meaning: one has to use the combination of the non-prefixed verb together with the verb \textit{na\v{c}at'} `to start'. If the imperfective is needed, the `auxiliary' verb \textit{na\v{c}at'} `to start' can be imperfectivized. No comparable standard solution can be offered for the resultative interpretation of \textit{za-}. These two facts together may have lead to the current state in which \textit{za-}prefixed verbs that tend to be interpreted resultatively form the secondary imperfective only from this interpretation. This explanation is tentative and leaves space for further research.

The third class consists of verbs that seem to have no secondary imperfectives, but can form them, if needed. As an example, consider the verb \textit{zaigrat'} `to start playing'. Out of context, the verb \textit{zaigryvat'} is interpreted as `to flirt', but it can also mean `to start/be starting playing', if a supporting context is provided. This is the case of the example \ref{ex:zaigryvat}. 

\exg.\label{ex:zaigryvat}$\ldots$[n]o on smejalsja, zeval, preryval e\"{e} vostor\v{z}ennye me\v{c}tanija pros'boju zakazat' k zavtra\v{s}nemu obedu pobol'\v{s}e vet\v{c}iny ili, sosku\v{c}iv\v{s}is' slu\v{s}at' neponjatnye dlja nego zvuki, zaigryval na svoj lad pesenku, kotoraja vozmu\v{s}\v{c}ala vs\"{e} su\v{s}\v{c}estvovanie bednoj Ol'gi.\\
$\ldots$but he laughed, yawned, intervened her enthusiastic dreams request order to tomorrow dinner more ham or, {become.bored} listen {not understandable} for him sounds, za.play.imp.\glb{pst.sg.m} on his mood song.\glb{sg.acc}, that perturbed all existence poor Olga\\
\trans `$\ldots$[b]ut he laughed, yawned, interrupted her enthusiastic dreams with a request to order more ham for the dinner tomorrow or, bored from listening to the sounds he could not understand, was starting to play a song in his own way, that perturbed the whole existence of poor Olga.'\\\Source{E.\,A. Gan. \textit{Ideal} (1837)}

Another example is the verb \textit{zasmejat'sja} which can be interpreted both inchoatively (`to start laughing') and resultatively (`to laugh until reaching some state'), but the resultative interpretation is very uncommon. When this verb is suffixed with the imperfective suffix \textit{-iva-}, the resulting verb, \textit{zasmeivat'sja}, receives two interpretations: habitual interpretation `to regularly start laughing' that stems from the inchoative meaning of \textit{zasmejat'sja} `to start laughing', as in \ref{ex:za:laugh1}, and habitual interpretation `to regularly laugh until reaching some state' that is based on the resultative meaning of \textit{zasmejat'sja} `to laugh until reaching some state', as in \ref{ex:za:laugh2}. This is supportive evidence for the tentative explanation of the behavior of the verbs in the second class: the frequency of different interpretations seems to play a role in the possibility of getting a secondary imperfective with a particular interpretation.

\ex.\label{ex:za:laugh}\ag.\label{ex:za:laugh1}Priam vs\"{e} zasmeivalsja s bol'\v{s}im azartom\\
Priam all za.laugh.\glb{pst.sg.m}.refl with bigger rage\\
\trans `Priam started laughing again and again, every time with bigger\linebreak rage.'\Source{\url{https://ficbook.net}}
\bg.\label{ex:za:laugh2}$\ldots$postojanno do sl\"{e}z zasmeivalsja zaklju\v{c}\"{e}nnymi$\ldots$\\
$\ldots$constantly until tears za.laugh.\glb{pst.sg.m}.refl prizoner.\glb{pl.inst}$\ldots$\\
\trans `$\ldots$he always laughed at prisoners until tears$\ldots$'\Source{\url{mobooka.ru}}


\subsection{Summary}
In sum, the formal representation of the inchoative \textit{za-} should have the following properties: 
\begin{enumerate}
\item the inchoative interpretation of the prefix is only possible when the derivational base does not have any explicit scales except for the time scale in its semantic representation (and the derived verb can only be used in contexts that do not contribute a scale);
\item when the prefix is attached, it relates the starting point of the event to the state of the absence and the end point of the event to the state of the presence of the activity denoted by the derivational base.
\end{enumerate}

Other properties that we have discussed should be reflected in the representation of the verbs and the secondary imperfective suffix: e.g., verbs that denote events with an extended preparatory phase should have information about it in their semantic structure. In turn, the progressive interpretation of the secondary imperfective and the time measure phrase should be capable of modifying the preparatory phase of the event in case the event itself does not have any duration. The lexical entries of verbs that do not allow the attachment of the imperfective suffix under any circumstances should be marked as such.

What is not possible to formalize within the framework I adopt for the current analysis are the restrictions on the attachment of the imperfective suffix that are associated with the frequency (or probability) of a particular interpretation of the given verb. If a probabilistic approach to semantics is integrated in the system, this should become possible, provided the explanation offered above is on the right track.

\section{\textit{na-}}\label{subsection:semantics:na}
\subsection{Semantic contribution}
First let us have a look at the different usages available for the prefix \textit{na-}. For this, we consult the grammar by \citet[360]{Shvedova:82}, where the following six types of verbs that are obtained as a result of the prefixation with \textit{na-} are listed:
\begin{enumerate}
\item to direct the action denoted by the derivational base on some surface, to place on or come across something (productive type): \textit{nakleit'} `to paste';
\item to accumulate something by performing the action denoted by the derivational base (productive type): \textit{navarit'} `to cook a lot of';
\item to perform the action denoted by the derivational base intensively (productive type): \textit{nagladit'} `to iron thoroughly' (colloquial);
\item to perform the action denoted by the derivational base weakly, lightly, on the go (non productive type): \textit{naigrat'} `to strum' (colloquial);
\item to learn something or acquire some skill by performing the action denoted by the derivational base (productive type): \textit{natrenirovat'} `to train until\linebreak some level', \textit{nabegat'} `to train to run' (only in professional slang);
\item to perform the action denoted by the derivational base until the result (productive type): \textit{nagret'} `to heat up', \textit{namo\v{c}it'} `to make wet', \textit{napoit'} `to give something to drink'.
\end{enumerate}

The cumulative usage we are going to discuss in this section appears under (2) in the above list by \citet{Shvedova:82}. Note that other productive usages of the prefix \textit{na-} are not considered superlexical by those linguists who adopt the distinction. At the same time the representation I provide for the prefix \textit{na-} in Chapter~\ref{Chapter7} covers not only the second usage, but also the usages listed under three, five, and six.

The cumulative prefix \textit{na-} and the prefix \textit{po-} (in the delimitative meaning) that we are going to discuss in Section~\ref{subsection:semantics:po}, share some properties. Both prefixes are claimed to denote a vague measure function \citep{Filip:00, Souchkova:04}. \citet{Souchkova:04} formulates two differences between these prefixes: the direction of the relation and the dimensions of the scales they select for.

There are two main usages of the cumulative prefix \textit{na-} in Russian: transitive and reflexive. Transitive usage is exemplified by \ref{ex:na:trans}, where the prefix measures the quantity of the direct object (potatoes) that has been cleaned. Reflexive usage is exemplified by \ref{ex:na:refl}; here, the prefix \textit{na-} measures the degree to which the subject (Katja) is full after eating potatoes. The case of the reflexive usage will not be discussed in this thesis, for analyses see \citet{KaganPereltsvaig:11a,KaganPereltsvaig:11b,Souchkova:04,Filip:00,Filip:05}. (In fact, the analysis of \textit{na-} would remain the same, what is needed for this case is the interpretation of the postfix \textit{-sja} that would provide the appropriate scale.)

\ex.\ag.\label{ex:na:trans}Katja na\v{c}istila karto\v{s}ki.\\
Katja na.clean.\glb{pst.sg.f} potato.\glb{gen}\\
\trans `Katja peeled a lot of potatoes.'
\bg.\label{ex:na:refl}Katja naelas' karto\v{s}ki.\\
Katja na.eat.\glb{pst.sg.f}.refl potato.\glb{gen}\\
\trans `Katja became full by eating potatoes.'

There is another usage of \textit{na-} (listed under (6) above) that is closely related to the cumulative usage exemplified by \ref{ex:na:trans}. The verb \textit{namo\v{c}il} `made wet' in \ref{ex:namochit} denotes an event of making something wet that is non-cumulative in every respect: a single actor made a single object wet with a single move. Another difference with respect to the verbs such as \textit{na\v{c}istit'} `to peel a lot of' is the source of the scale: in \ref{ex:na:trans} the event is measured along the quantity scale provided by the direct object, while in case of \ref{ex:namochit} the relevant wetness scale is encoded by the verb.

\exg.\label{ex:namochit}Petja namo\v{c}il kisto\v{c}ku v stakane vody.\\
Petja na.wet.\glb{pst.sg.m} brush.\glb{sg.acc} in glass.\glb{sg.prp} water.\glb{sg.gen}\\
\trans `Petja made the brush wet by putting it into a glass with water.'

%As the cumulative usage of \textit{na-} is supposed to be productive, let us consider a verb that appeared in the language recently, i.e., \textit{guglit'$^{\IPF}$} `to google.' The cumulative \textit{na-} can be attached to this verb, producing the derived verb \textit{naguglit'$^{\PF}$} `to google a lot of/to find out by googling.' An example of the utterance is given in \ref{ex:naguglit'}. At the same time, the same verb can be used in the context like \ref{ex:naguglit':film} where no cumulative reading is possible, as in \ref{ex:naguglit':film}.
%
%\exg.\label{ex:naguglit'}Ona naguglila informaciju po \'{e}toj teme\\
%she google.\glb{pst.sg.f} information on this topic\\
%\vspace{0.5em}
%`She googled information about this topic.'
%
%\exg.\label{ex:naguglit':film}K tomu \v{z}e, poka ja guglila pro nix, ja naguglila fil'm ``Ximera'' ...\\
%to that also, whole I google.\glb{pst.sg.f} about them, I na.google.\glb{pst.sg.f} film ``Chimera''\\
%\vspace{0.5em}
%`In addition, while I googled information about them, I found the film ``Chimera''. '
%\begin{flushright}
%\vspace{-0.5em}
%http://liekenlee.diary.ru/
%\end{flushright}

To account for this, one can either accept the polysemy among the productive usages of the prefix \textit{na-} or try to unify them. If one considers the list of \textit{na-}prefixed verbs that do have clear cumulative semantics, one can notice that for verbs in this list there is another way to express the completion of the event denoted by the derivational base. For example, instead of \ref{ex:na:trans} the speaker could have uttered \ref{ex:na:po:gen} which would be neutral with respect to the quantity of the potatoes peeled or \ref{ex:na:po:acc} that would mean that Katja peeled all of the potatoes. The same happens in the pair of sentences \ref{ex:navarit} and \ref{ex:svarit}. The sentence with the verb prefixed with \textit{na-} refers to an event of cooking involving some quantity of the soup that exceeds the standard amount. The sentence with the \textit{s-}prefixed verb does not carry any information about the quantity of soup produced.

\ex.\label{ex:na:po}\ag.\label{ex:na:po:gen}Katja po\v{c}istila karto\v{s}ki.\\
Katja po.clean.\glb{pst.sg.f} potato.\glb{gen}\\
\trans `Katja peeled some potatoes.'
\bg.\label{ex:na:po:acc}Katja po\v{c}istila karto\v{s}ku.\\
Katja po.clean.\glb{pst.sg.f} potato.\glb{acc}\\
\trans `Katja peeled the potatoes.'

\ex.\ag.\label{ex:navarit}Liza navarila supa.\\
Liza na.cook.\glb{pst.sg.f} soup.\glb{gen}\\
\trans `Liza cooked a lot of soup.'
\bg.\label{ex:svarit}Liza svarila sup.\\
Liza s.cook.\glb{pst.sg.f} soup.\glb{acc}\\
\trans `Liza cooked soup.'

On the basis of these observations I can offer the following potential explanation of what is happening with the prefix \textit{na-}: the core meaning of the cumulative prefix \textit{na-} is `performing an action until the validation point is reached'. Validation point is, in different cases, either some standard quantity of the direct object or some degree on the scale such that when it is reached the action denoted by the derivational base counts as being performed. For example, the verb \textit{gret'}$^{\IPF}$ means `to warm' and the verb \textit{nagret'}$^{\PF}$ `to heat up' denotes warming until the warm state of the object is reached. Such an approach would unify the second, the third, the fifth, and the sixth usages in the list by \citet{Shvedova:82}, so that the only other productive usage not covered here is one associated with the spatial scale (first usage in the list above). 

This description is very close to that of \citet{Kagan:book}, who offers the semantic representation of the prefix \textit{na-} that is shown in \ref{Kagan:na}. \citet[55]{Kagan:book} proposes that ``\textit{na}- looks for a verbal predicate that takes a degree, an individual and an event argument and imposes the `$\geqslant$' relation between the degree argument and the contextually provided expectation value d$_c$. As a result, the degree of change is entailed to be no lower than the standard.''
\ex.\label{Kagan:na}$\llbracket na- \rrbracket = \lambda$P$\lambda$d$\lambda$x$\lambda$e.[P(d)(x)(e) $\wedge$ d $\geqslant$ d$_c$]\\
where d = degree of change \citep{KennedyLevin:02}\\\Source{= (17) in \citealt[55]{Kagan:book}}

The semantic representation proposed by \citet{Kagan:book} allows to capture the semantics of the cumulative and the resultative usages of the prefix \textit{na-}. What is left unclear is when exactly is the cumulative interpretation obtained. For example, for the verb \textit{nagret'} `to heat up' one does not want to derive the interpretation like `heat more than expected', as this would be the meaning of the verb \textit{peregret'} `to overheat'. A possible solution will be to simplify the semantics of \textit{na-} by restricting it to achieving the standard/expected degree on the scale and derive the additional component of exceeding the expectations in some cases in the pragmatic module. For this, one has to look at the competition between different perfective verbs derived from the same derivational base. If there is an alternative competing verb that is neutral with respect to the quantity of the direct object, uttering the verb prefixed with \textit{na-} implies a higher degree on the scale than the standard. Similar pragmatic reasoning is not uncommon in the literature: for example, \citet[21]{KennedyLevin:08} use pragmatic reasoning to explain some preferences in the domain of degree achievements. I will provide more details in this respect in Chapter~\ref{Chapter6}.

\subsection{Restrictions on the attachment}
As we have discussed in the previous chapter, the cumulative prefix \textit{na-} is usually attached to imperfective verbs. There are, however, exceptions to this generalization. At least two verbs formed by prefixation of perfective verbs with the cumulative \textit{na-} are accepted by all native speakers of Russian. These are \textit{nakupit'}$^{\PF}$ `to buy a lot of something' and \textit{napustit'}$^{\PF}$ `to fill with a lot of something'. In addition, \citet{Tatevosov:13a} notes that there is a group of speakers, seemingly from an older generation (and representing an outdated norm of the language) that accept a larger class of verbs derived by the \textit{na-}prefixation of perfective verbs, such as \textit{$^?$napridumat'}$^{\PF}$ `to come up with a lot of something', \textit{$^?$narasskazat'}$^{\PF}$ `to tell a lot of something', and \textit{$^?$naso\v{c}init'}$^{\PF}$ `to write/compose a lot of something'.

Starting with the information about the outdated norm of the language, let us take the diachronical perspective in order to explain the behavior of the cumulative \textit{na-}. Suppose some time ago the attachment of the cumulative \textit{na-} to a perfective verb was the norm of the language (for whatever reason). This does not mean that \textit{na-} was attached only to perfective verbs, but just the absence of the restriction (as is suggested by \citet{Tatevosov:13a} for those speakers who nowadays produce verbs such as \textit{narasskazat'}$^{\PF}$ `to tell a lot of something'). Then in such pairs as \textit{$^?$napridumat' -- napridumyvat'} `to come up with a lot of something', \textit{$^?$naotkryt' -- naotkryvat'} `to open a lot of', \textit{nakupit' -- napokupat'} `to buy a lot of' both verbs were acceptable. As the first members of these pairs are morphologically less complex, they might have been preferred over the second members of the pairs.\footnote{This can be explained by a pragmatic principle related to the one we have already discussed: if there are two forms with identical semantics, the less complex form is preferred. In this case forms of different complexity do not belong to one derivational chain, so this principle is only about the preference, not about the exclusion of one of the verbs.} 

Note that the difference in morphological complexity of the two members of the pair can vary. The morphological complexity difference between the competing verbs \textit{naotkryt'} `to open a lot of' and \textit{naotkryvat'} `to open a lot of' is only one morpheme: the imperfective suffix, as is clear from the derivational chains \ref{chain:naotkryt} and \ref{chain:naotkryvat}. In the pair  \textit{nakupit'} `to buy a lot of' and  \textit{napokupat'} `to buy a lot of' this difference is two morphemes: in order to derive a cumulative verb from an imperfective verb, a prefix should be added and a suffix should be changed, as illustrated by the derivational chains \ref{chain:nakupit} and \ref{chain:napokupat}.

\ex.\ag.\label{chain:naotkryt}ot-kr-y-t'$^{\PF}$ $\rightarrow$ na-ot-kr-y-t'$^{\PF}$\\
{to open} $\rightarrow$ {to open a lot of}\\
\bg.\label{chain:naotkryvat}ot-kr-y-t'$^{\PF}$ $\rightarrow$ ot-kr-y-va-t'$^{\IPF}$ $\rightarrow$ na-ot-kr-y-t'$^{\PF}$\\
{to open} $\rightarrow$ {to open/be opening} $\rightarrow$ {to open a lot}\\

\ex.\ag.\label{chain:nakupit}kup-i-t'$^{\PF}$ $\rightarrow$ na-kup-i-t'$^{\PF}$\\
{to buy} $\rightarrow$ {to buy a lot}\\
\bg.\label{chain:napokupat}kup-i-t'$^{\PF}$ $\rightarrow$ po-kup-a-t'$^{\IPF}$ $\rightarrow$ na-po-kup-a-t'$^{\PF}$\\
{to buy} $\rightarrow$ {to buy/be buying} $\rightarrow$ {to buy a lot}\\

To provide some evidence in favour of the theory of competition sketched above, let us consider some cases where the perfective verb is equally or more morphologically complex than the corresponding imperfective verb. In the first pair of verbs, \textit{o\v{s}\v{c}ut-i-t'}$^{\PF}$\slash\textit{o\v{s}\v{c}u\v{s}\v{c}-a-t'}$^{\IPF}$ `to feel', the imperfective verb is as complex as the perfective one, as the two verbs include the same number of morphemes. In the second pair, \textit{vz-j-a-t'}$^{\PF}$\slash\textit{br-a-t'}$^{\IPF}$ `to take', the perfective verb is morphologically more complex than the corresponding imperfective verb. It turns out that in both pairs the cumulative prefix \textit{na-} can only be attached to the imperfective verb for all the speakers of Russian (see chains in \ref{chain:naoschu} and \ref{chain:nabr} and examples \ref{ex:naoschutit} and \ref{ex:nabrat}). 

\ex.\label{chain:naoschu}\ag.o\v{s}\v{c}ut-i-t'$^{\PF}$ $\nrightarrow$ $^*$na-o\v{s}\v{c}ut-i-t'$^{\PF}$\label{chain:oschutit}\\
{to feel} {} {}\\
\bg.\label{chain:oschuschat}o\v{s}\v{c}u\v{s}\v{c}-a-t'$^{\IPF}$ $\rightarrow$ na-o\v{s}\v{c}u\v{s}\v{c}-a-t'$^{\PF}$\\
{to feel/be feeling} $\rightarrow$ {to feel a lot}\\

\ex.\label{chain:nabr}\ag.vz-j-a-t'$^{\PF}$ $\nrightarrow$ $^*$na-vz-j-a-t'$^{\PF}$\label{chain:navzjat}\\
{to take} {} {}\\
\bg.\label{chain:nabrat}br-a-t'$^{\IPF}$ $\rightarrow$ na-br-a-t'$^{\PF}$\\
{to take/be taking} $\rightarrow$ {to take a lot}\\

\exg.\label{ex:naoschutit}Instinkt \v{z}izni diktuet nao\v{s}\v{c}u\v{s}\v{c}at' kak mo\v{z}no bol'\v{s}e za \v{z}izn'.\\
instinct life.\glb{sg.gen} dictates na.feel.\glb{inf} as possible more for life\\
\trans `The instinct of life dictates to feel as much as possible during your life.'\\\Source{Mixail Veller. \textit{Belyj oslik} (2001)}

\exg.\label{ex:nabrat}On nabral celoe o\v{z}erel'e raku\v{s}ek [$\ldots$]\\
he na.take.\glb{pst.sg.m} whole necklace shell.\glb{pl.gen}\\
\trans `He gathered shells for a whole necklace [$\ldots$]'\\\Source{Aleksandr Dorofeev. \textit{\`{E}le-Fantik} (2003)}

Taking this into account, we can modify the assumption about the absence of a restriction on the attachment of the cumulative \textit{na-}, saying that the attachment to the imperfective verbs was still slightly preferred over the attachment to the perfective verb. Together with the pragmatic principle that penalizes morphologically more complex verbs we then obtain a system that corresponds to the outdated norm. 

Now that we have discussed the competition between different verbs in the situation when the cumulative \textit{na-} can be attached to both imperfective and perfective verbs, let us see what happens when the norm shifts and the attachment of the cumulative \textit{na-} to a perfective verb becomes significantly dispreferred. At this moment the rules of the competition change: increasing the morphological complexity of the verb by one morpheme becomes better than violating the aspectual restriction. And in such pairs as \textit{napridumat'} vs. \textit{napridumyvat'} `to come up with a lot of something' the second member becomes preferred over the first. If, however, increasing the morphological complexity by two is still penalized more than violating the aspectual restriction, verbal pairs with greater difference between in morphological complexity between the perfective and the corresponding imperfective would still allow the attachment of the cumulative prefix \textit{na-} to the perfective derivational base. And this is exactly what we observe in case of \textit{kupit' -- pokupat'} `to buy'.

Another exception is the verb \textit{napustit'}$^{\PF}$ `to fill with a lot of something' that is derived from the perfective verb \textit{pustit'}$^{\PF}$ `to let'. It is not clear what exactly happens with this particular verb, but it is exceptional not only with respect to the combination with the cumulative \textit{na-}. First of all, a whole range of prefixed verbs that seem to be formed via prefixation of the derivational base \textit{puskat'}$^{\IPF}$ `to let' turn out to be imperfective: \textit{otpuskat'}$^{\IPF}$ `to let leave', \textit{zapuskat'}$^{\IPF}$ `to start something', \textit{napuskat'}$^{\IPF}$ `to fill with a lot of something', \textit{spuskat'}$^{\IPF}$ `to let out', etc. If we assume that these verbs are indeed derived from the imperfective verb \textit{puskat'}$^{\IPF}$ `to let', as shown in \ref{chain:puskat}, we have to postulate non-perfectivizing usages for a number of prefixes. This is an argument in favor of the alternative hypothesis: an assumption that the last step in the derivation of these verbs is imperfectivization, as shown in \ref{chain:pustit}. Such explanation is not complete as it just reduces the problem to the puzzle about a concrete verb, not about the prefixation system, but I have no solution for this new puzzle at the moment. I believe that the answer lies in the historical linguistics perspective and may have similar roots as the answer to the puzzle of the motion verbs. I leave this question open for future research.

\exg.puskat'$^{\IPF}$ $\rightarrow$ zapuskat'$^{\IPF}$ / napuskat'$^{\IPF}$ \label{chain:puskat}\\
{to let} $\rightarrow$ {to (be) starting something} / {to (be) fill(ing) with a lot of}\\

\exg.pustit'$^{\PF}$ $\rightarrow$ zapustit'$^{\PF}$ / napustit'$^{\PF}$ $\rightarrow$ zapuskat'$^{\IPF}$/ napuskat'$^{\IPF}$ \label{chain:pustit}\\
{to let} $\rightarrow$ {to start something} / {to fill with a lot of} $\rightarrow$ {to (be) starting something/} {to (be) fill(ing) with a lot of}\\

\subsection{Subsequent imperfectivization}
The attachment of the imperfective suffix to verbs prefixed with \textit{na-} is treated in the literature similarly to the case of the inchoative prefix \textit{za-}: \citet[230]{Svenonius:04b} classifies the cumulative \textit{na-} as a prefix that sometimes allows the formation of the secondary imperfective, whereas \citet{Tatevosov:09} does not pose any specific restrictions (if fact, such restrictions are absent in his account at all).

An illustrative example is provided by \citet[233]{Svenonius:04b} and repeated here as \ref{ex:na:Sven}. In \ref{ex:na:Sven:1} we see a perfective verb with a literal interpretation of the derivational base, whereas in \ref{ex:na:Sven:2} and \ref{ex:na:Sven:3} we observe that the secondary imperfective can not be interpreted literally. \citet[233]{Svenonius:04b} attributes this assymetry of the secondary imperfective formation to the difference in the structural positions. I claim that the verb \textit{nakalyvat'}$^{\IPF}$ `to pin/be pinning/to cheat/be cheating' is usually not interpreted as `to crack/be cracking a lot' not because of the position of the prefix in the structure of the verb \textit{nakolot'}$^{\PF}$ `to crack a lot', but because the latter verb also has the other meaning `to pin', derived from the spatial interpretation of the prefix \textit{na-}. 

\ex.\label{ex:na:Sven}\ag.\label{ex:na:Sven:1}On na-kolol orexov.\\
he cmlt-cracked$^P$ nuts\\
\trans `He cracked a sufficiently large quantity of nuts'
\bg.\label{ex:na:Sven:2}*On na-kalyval orexov.\\
he cmlt-cracked$^I$ nuts\\
\vspace{0.5em}
(`He was cracking a sufficiently large quantity of nuts')
\bg.\label{ex:na:Sven:3}On na-kalyval klijentov.\\
he on-cracked$^I$ clients\\
\trans `He was cheating the clients'\\\Source{= example (63) in \citealt[230]{Svenonius:04b}}

So the situation turns out to be similar to that of the inchoative prefix \textit{za-}: when a \textit{na-}prefixed verb has two interpretations, one (more frequent) of them involving spatial and the other involving cumulative meaning, the secondary imperfective of this verb will be normally interpreted as formed on the basis of the spatial interpretation. The reason is also similar: there is a regular lexical way to express the meaning that a secondary imperfective verb with the cumulative interpretation of the prefix \textit{na-} would have (use the non-prefixed imperfective and the adverb \textit{mnogo} `a lot'). For the lexical meaning of the prefix, no such regular replacement of the secondary imperfective is available. Indeed, if we search for the examples of the usage of the verb \textit{nakalyvat'}, we mostly find sentences like \ref{ex:nakalyvat}, involving the spatial usage of the prefix \textit{na-}. 

\ex.\label{ex:nakalyvat}\ag.Izvestny slu\v{c}ai, kogda e\v{z}i podbirali i nakalyvali na svoi igly okurki ili pytalis' ``vyvaljat'sja'' v kofejnyx zernax.\\
known cases when hedgehogs pod.take.imp.\glb{pst.pl} and na.prick.imp.\glb{pst.pl} on their needles {cigarette stubs} or try.\glb{pst.pl} vy.waalow.imp.\glb{inf.refl} in coffee beans\\
\trans `We know about cases when hedgehogs picked up and pined on their needles cigarette stubs or tried to roll in and get covered with the coffee beans.'\Source{\url{http://www.ogoniok.com}}
\bg.O\v{c}i\v{s}\v{c}ennye orexi nu\v{z}no nakolot', ja nakalyvala vilkoj - tak bystree, \v{c}em zubo\v{c}istkoj.\\
peeled nuts necessary na.pin.\glb{inf}, I na.pin.imp.\glb{pst.sg.f} fork - so faster, then toothpick\\
\trans `You have to make holes in the peeled nuts, I pierced them with the fork, this is faster than when using a toothpick.'\\\Source{\url{www.carina-forum.com}}

At the same time if we consult the dictionary, it turns out that the first interpretation provided for the verb \textit{nakalyvat'} is `to crack something in some (normally big) quantity' \citep{Efremova:00}, which is exactly the interpretation of the secondary imperfective verb derived from the verb \textit{nakolot'} `to crack a lot of', that, according to \citet{Svenonius:04b} does not exist and, according to the internet data, is at least very uncommon, if used at all. As dictionaries tend to represent an outdated norm, this phenomenon can be related to the norm shift we have discussed above.

I want to emphasize that the imperfectivization of verbs prefixed with the cumulative \textit{na-} is available in a larger number of cases than it seems at the first sight. I have sketched a possible explanation why its formation is dispreferred in case a spatial interpretation of the derivational base is available, but this explanation is about the preference, not the complete unavailability and uses information about the relative frequency of different interpretations. Consider the verb \textit{navarit'}$^{\PF}$ `cook a lot/to weld to something'. For the perfective verb, the cumulative interpretation is the default one, but the spatial interpretation is accessible in the relevant context. After the attachment of the imperfective suffix, the spatial interpretation (see example \ref{ex:navarivat1}) is the default. The cumulative interpretation is dispreferred, but possible and easy to find, as illustrated by \ref{ex:navarivat2}.

\ex.\ag.\label{ex:navarivat2}Ona navarivala sebe bol'\v{s}ie kastrjuli kompotu i s''edala ego s serym xlebom, v odino\v{c}ku.\\
she na.cook.imp.\glb{pst.sg.f} yourself big pots compot and s.eat.imp.\glb{pst.sg.f} him with grey bread, in singleton\\
\trans `She regularly cooked herself large pots of compote and ate it on her own together with grey bread.'\Source{\url{http://gatchina3000.ru/}}
\bg.\label{ex:navarivat1}V ob\v{s}\v{c}em, vse vyxodnye brigada mestnyx svar\v{s}\v{c}ikov latala im nos, navarivala listy ob\v{s}ivki prjamo poverx izmjatyx.\\
in general, all weekends team local welders patch.\glb{pst.sg.f} them bow, na.weld.imp.\glb{pst.sg.f} sheet.\glb{pl.acc} sheathing directly {on top} wrinkled\\
\trans `In sum, the whole weekend the team of local welders patched their bow, welding the sheathing sheets directly on top of the wrinkled ones.'\Source{\url{http://kamafleetforum.ru/}}

It turns out that the formation of secondary imperfective verbs from the verbs prefixed with the cumulative \textit{na-} is in general available, although the derived imperfective verbs may not sound acceptable without a context. To provide another example, let us try to imperfectivize the verb \textit{naguglit'} `to find something by googling'. The derived verb \textit{naguglivat'} `to find something by googling occasionally' is used, as evidenced by the examples one can find in the internet, such as \ref{ex:naguglivat}. This verb is interpreted exclusively habitually which can be explained by using the principle based on the Horn's division of labour (see \citealt{Horn:84}): if there are two verbs that express the same meaning, the simpler one should be used. Indeed, the potential progressive interpretation of the verb \textit{naguglivat'} is `to google something', exactly the same as the interpretation of the verb \textit{guglit'} `to google' when it is used transitively. As for the habitual interpretation, there is a clear difference between the semantics of the basic imperfective verb \textit{guglit'} `to google' and the semantics of the derived secondary imperfective verb \textit{naguglivat'} `to find something by googling occasionally', as the latter includes the resultative component for every event of googling. 

\exg.\label{ex:naguglivat}Spaseniem dejstvitel'no byli sovremennye stat'i, blogi, sajty, kotorye ja naguglivala na plan\v{s}ete, v kotorom \v{z}e borolas' so ``Star\v{s}ej \`{E}ddoj''.\\
salvation really were contemporary articles, blogs, pages, that.\glb{pl.nom} I na.google.imp.\glb{pst.sg.f} on tablet, in that.\glb{m.sg.prp} again fought with ``older Edda''\\
\trans `My salvation was in contemporary articles, blogs and web pages that I googled on my tablet, that I also used to fight with ``Older Edda''.'\\\Source{\url{http://www.livelib.ru/review/259836}}

Based on what we have observed so far, one can hypothesize that the progressive interpretation of the secondary imperfective verbs that include the cumulative prefix \textit{na-} should be possible in cases when the derivational base is interpreted not just resultatively, but also carries the `a lot' component (which happens due to the competition with other verbs). This is confirmed by the data. As an example, consider the verb \textit{nagotovit'} `to cook/prepare a lot'.\footnote{I consider it instead of the verb \textit{navarit'} `to cook' here, as there are no other interpretations involving spacial \textit{na-} available for it and thus the secondary imperfective is in general easily accessible. The neutral perfective derived from the verb \textit{gotovit'} `to prepare/be preparing' is the verb \textit{prigotovit'} `to cook/prepare smth'.}  The derived secondary imperfective verb \textit{nagotavlivat'} `to prepare/be preparing a lot' can be interpreted progressively \ref{ex:nagotovit1} as well as habitually \ref{ex:nagotovit2}.

\ex.\label{ex:nagotovit}\ag.\label{ex:nagotovit1}s 5 \v{c}asov u\v{z}e ne spitsja, nagotavlivaju detjam\\
from 5 hours already not sleep.\glb{pres.sg.3.refl}, na.prepare.imp.\glb{pres.sg.1} child.\glb{pl.dat}\\
\trans `I can't sleep since 5 a.m., so I am preparing food for the children'\\\Source{\url{www.plastic-club.ru}}
\bg.\label{ex:nagotovit2}Vprok nikogda ne nagotavlivaju, ljubim vse sve\v{z}ee.\\
{in store} never not na.prepare.imp.\glb{pres.sg.1}, love.\glb{pres.pl.1} all fresh\\
\trans `I never cook food for the next several days, we prefer to eat all fresh.'\\\Source{\url{forum.bel.ru}}

\subsection{Summary} 
In sum, the formal representation of the cumulative prefix \textit{na-} should have the following properties: 
\begin{enumerate}
\item the prefix requires an open scale that is provided by the verb and is a parameter of the object;
\item when the prefix is attached, it specifies the starting point of the event being at the starting point of the scale and the end of the event being at (or, possibly, at or above, see the discussion in the beginning of the section) the standard degree on the same scale.
\end{enumerate}

Similarly to the analysis of \textit{za-}, I am not going to restrict the attachment of the secondary imperfective to the verbs prefixed with the cumulative \textit{na-} in the semantic module.

\section{\textit{po-}}\label{subsection:semantics:po}
\subsection{Semantic contribution} For the start, let us again look at the Russian grammar by \citet{Shvedova:82}, who provides a list of possible usages of the prefix \textit{po-} and their productivity. \citet[364--365]{Shvedova:82} names the following five types of situations the verbs prefixed with \textit{po-} can refer to:
\begin{enumerate}
\item to do the action that is denoted by the derivational base with low intensity, sometimes also gradually: \textit{poprivyknut'} `to get somehow used', \textit{po\-izno\-sit'sja} `to get somewhat worn out', \textit{pomaslit'} `to put some butter on something'  (productive, especially in spoken language);
\item to do the action that is denoted by the derivational base repeatedly, with many or all of the objects or by many or all of the subjects: \textit{povyvezti} `to take out many/all of something' (productive, especially in spoken language);
\item to do the action that is denoted by the derivational base for some (often short) time: \textit{pobesedovat'} `to spend some time talking' (productive);
\item to start the action that is denoted by the derivational base: \textit{pobe\v{z}at'} `to start running' (productive);
\item to complete the action denoted by the derivational base: \textit{poblagodarit'} `to thank' (productive).
\end{enumerate}

We are going to look at the usages of the prefix \textit{po-} that are traditionally called delimitative and distributive. The delimitative usage covers both the first and the third classes of the \textit{po-}prefixed verbs listed by \citet{Shvedova:82}, and the distributive usage corresponds to the second type of the outcome in the list above. The fourth usage (inceptive) is encountered when the prefix \textit{po-} is attached to a motion verb; this usage is discussed in \citealt{ZinovaOsswald:paper}. As for the last usage from the list by \citet{Shvedova:82}, I will show that it can be unified with the delimitative usage of \textit{po-}. In sum, I will provide a unified underspecified semantics for the prefix \textit{po-}.

\subsubsection{Delimitative \textit{po-}}
Traditionally, the delimitative meaning of \textit{po-} is associated with some characteristic of an event being lower than the expected value: for example, an event lasting for a short period of time, a small quantity of the theme consumed, etc. This usage of \textit{po-} is also called attenuative by some authors \citep[e.g.][]{Svenonius:04b}. According to \citet[47--48]{Filip:00}, who compares it with accumulative \textit{na-}, ``[t]he prefix \textit{po-} contributes to the verb the opposite meaning of a small quantity or a low degree relative to some expectation value, which is comparable to vague quantifiers like \textit{a little, a few} and vague measure expressions like \textit{a (relatively) small quantity / piece / extent of}.''

\citet[183]{Braginsky:08} applies a neat test in order to show the difference between the verbs prefixed with the resultative \textit{za-} and the verbs prefixed with \textit{po-}. The idea of this test is to continue the given sentence with `but it is hard to call it X' where X is the result state corresponding to the derivational base. Such a continuation is only possible if there is no restriction on the degree reached on the relevant scale by the end of the event. \citet[183]{Braginsky:08} provides two examples repeated under \ref{ex:Brag:pogustet} and \ref{ex:Brag:porzhavet} here. What these examples show is that, indeed, when sentences are headed by the \textit{po-}prefixed verb, the result state must not be reached, which is not the case with the \textit{za-}prefixed resultative verbs.

\ex.\label{ex:Brag:pogustet}\ag.Varen'je pogustelo$^{\PF}$, no ego e\v{s}\v{c}e trudno nazvat' gustym.\\
Jam PO-thickened but it yet hard {to call} thick\\
\trans `The jam thickened a bit, but it is hard to define it as thick yet.'
\bg.*Varen'je zagustelo$^{\PF}$, no ego e\v{s}\v{c}e trudno nazvat' gustym.\\
Jam ZA-thickened but it yet hard {to call} thick\\\Source{= example (49) in \citealt[183]{Braginsky:08}}

\ex.\label{ex:Brag:porzhavet}\ag.Gvozd' por\v{z}avel$^{\PF}$, no ego e\v{s}\v{c}e trudno nazvat' r\v{z}avym.\\
Nail {PO-became rusty} but it yet hard {to call} rusty\\
\trans `The nail became a bit rusty, but it is hard to define it as rusty yet.'
\bg.*Gvozd' zar\v{z}avel$^{\PF}$, no ego e\v{s}\v{c}e trudno nazvat' r\v{z}avym.\\
Nail {ZA-became rusty} but it yet hard {to call} rusty\\
\Source{= example (50) in \citealt[183]{Braginsky:08}}

\citet{Souchkova:04}, analysing Czech prefixes, shows that \textit{po-} can quantify over different dimensions: duration, distance, degree of the property attained by the internal argument. \citeauthor{Souchkova:04} argues that despite different domains of quantification there is one single delimitative \textit{po-} and its meaning contribution is sensitive to the content of the VP. This is true also for Russian and allows us to unify the first and the third usages that are listed by \citet{Shvedova:82}: the unified semantic representation later combines with a scale provided either by the verb or by the direct object, leading to different relevant interpretations.

Examples of the delimitative usage of the prefix \textit{po-} include such sentences as \ref{ex:po:delim}, taken from \citet{Filip:00} and \citet{Souchkova:04} and also used by \citet{Kagan:book}, whereby the sentence \ref{ex:po:delim1} is taken to mean that the walk around the city was short, and \ref{ex:po:delim2} -- that the quantity of the apples eaten was relatively small.

\ex.\label{ex:po:delim}\ag.\label{ex:po:delim1}Ivan poguljal po gorodu.\\
Ivan po.walk.\glb{pst.sg.m} around town\\
\trans `Ivan took a (short) walk around the town.'\\\Source{= example (9c) in \citealt{Filip:00}}
\bg.\label{ex:po:delim2}Ivan poel jablok.\\
Ivan po.eat.\glb{pst.sg.m} apple.\glb{pl.gen}\\
\trans `Ivan ate some (not many) apples.'
\Source{= example (3) in \citealt[46]{Kagan:book}}

%\ex.\label{ex:po}\ag.On nemnogo po-razmy\v{s}lyal ob \'{e}tom.\\
%He {a little bit} {ATTN-thought} about this\\
%`He spent a little bit of time thinking about this.
%\bg.\label{ex:po2}Marina {za vremya} bolezni po-xudela.\\
%Marina during sickness {PO-lost weight}\\
%`Marina lost weight during the sickness.

Although the observations about the low degree on some scale, associated with the discussed usage of the prefix \textit{po-}, are commonly accepted and seem to be well established, the assumption that this degree has to be always low prevents us from accounting for some of the prefix usage cases one can find. As an illustration, let me provide some examples from the corpora.

\ex.\label{ex:po:alot}\ag.\label{ex:po:alot1}Znat', mnogo po svetu pobrodil, vsjakogo raznogo uspel {naslu\v{s}at'sja-} {nasmotret'sja.}\\
know {a lot} on world po.wander.\glb{pst.sg.m} all different {have time} na.hear.\glb{inf}.refl na.look.\glb{inf}.refl\\
\trans `You know, he wandered a lot around the world, he had time to see and hear all kinds of different things.'\\\Source{Marija Semenova. \textit{Volkodav: Znamenie puti} (2003)}
\bg.\label{ex:po:alot2}Kogda do stolicy ostavalos' tridcat' kilometrov, na\v{s}\"{e}l stolovuju i o\v{c}en' plotno poel, poskol'ku do sleduju\v{s}\v{c}ego pri\"{e}ma pi\v{s}\v{c}i neizvestno skol'ko vremeni.\\
when before capital stay.\glb{pst.sg.n.}refl thirty kilometers found canteen and very tight po.eat.\glb{pst.sg.m} because before next reception food unknown {how much} time\\
\trans `When I was about 30 km away from the capital, I found a canteen and had a very square meal, as I didn't know how long it would take until my next chance to eat something.'\\\Source{Anatolij Azol'skij. \textit{Lopu\v{s}ok} (1998)}

In \ref{ex:po:alot1} the verb \textit{pobrodil} `wandered', that presumably contains the delimitative prefix \textit{po-}, refers to a lot of wandering, and in \ref{ex:po:alot2} the verb \textit{poel} `ate' refers to a situation of eating a lot. If the semantics of the delimitative prefix \textit{po-} would include the semantic component `the degree is lower than the expected value', such sentences would be unacceptable or would trigger an additional pragmatic inference, i.e., be interpreted sarcastically. This is not the case: both \ref{ex:po:alot1} and \ref{ex:po:alot2} are unmarked. What is also important is that some verbs can be also used in combination with the adverbials denoting small quantity (such as \textit{nemnogo} `a bit'), as in the examples \ref{ex:po:abit}.

\ex.\label{ex:po:abit}\ag.On pobrodit nemnogo i sej\v{c}as \v{z}e ujdet.\\
he po.wander.\glb{pres.sg.3} {a bit} and now same u.go.\glb{pres.sg.3}\\
\trans `He will wonder around a little bit and immediately leave.'\\\Source{Anna Berseneva. \textit{Vozrast tret'ej ljubvi} (2005)}
\bg.My kupim pti\v{c}kam kormu i sami poedim nemnogo.\\
we buy.\glb{pres.pl.1} birds food and ourselves po.eat.\glb{pres.pl.1} {a bit}\\
\trans `We will buy food for the birds and eat something small ourselves.'\\\Source{V. P. Kataev. \textit{Bezdel'nik \`{E}duard} (1920)}

A possible solution would be to say that we are dealing with two different usages of \textit{po-}: a delimitative in the examples \ref{ex:po:delim1} and \ref{ex:po:delim2} and some other in the examples \ref{ex:po:alot1} and \ref{ex:po:alot2}, probably corresponding to the last, resultative, usage of \textit{po-} in the list provided by \citet{Shvedova:82}. This solution does not seem right to me: the verb \textit{poel} `ate' in \ref{ex:po:delim2} and the verb \textit{poel} `ate' in \ref{ex:po:alot2} seem to have the same meaning. If one consults a dictionary, one will find just one meaning of the verb \textit{poest'} `to eat' that reflects the meaning of the verbs \textit{poel} `ate' in the examples \ref{ex:po:delim2} and \ref{ex:po:alot2}. This can be either `to eat not much' \citep{Ushakov:50} or `to eat' \citep{Efremova:00} meaning. Another evidence in favor of the single meaning is that the verbal phrase in the example \ref{ex:po:delim2} can also be modified with an adverbial denoting sufficient quantity, as evidenced by the example \ref{ex:po:vdovol}, that is taken from the corpora.

\exg.\label{ex:po:vdovol}Togda on poel jablok vdovol'.\\
then he po.eat.\glb{pst.sg.m} apple.\glb{pl.gen} enough\\
\trans `Then he ate apples to his heart's content.'\\\Source{Aleksandr Ili\v{c}evskij. \textit{Matiss} (2007)}

So again I propose to apply the same technique as in the case of the cumulative \textit{na-}. We can define the semantics of the delimitative usage\footnote{I will use the term \textit{delimitative} to refer to the discussed usage in order to differentiate it from the distributive and inchoative usages, but I will not imply attenuativity.} of \textit{po-} in such a way that the verb prefixed with it can either denote the unmarked completion of the event or include the semantic component `quantity/degree is lower than some expectation value'. 

\citet[48]{Kagan:book}, following the analyses proposed by \citet{Filip:00} and \citet{Souchkova:04}, proposes that ``\textit{po}- looks for a predicate that takes a degree, and individual and an event argument and imposes the ``$\leqslant$'' relation between the degree argument and the contextually provided expectation value d$_c$.''

\ex.\label{Kagan:po}$\llbracket po- \rrbracket = \lambda$P$\lambda$d$\lambda$x$\lambda$e.[P(d)(x)(e) $\wedge$ d $\leqslant$ d$_c$]\\
where d = degree of change \citep{KennedyLevin:02}

This approach captures the semantics of the prefix in the examples discussed here as it includes the possibility that $d = d_c$ and thus both the completion and delimitation can be expressed by the same prefix. What needs to be added here is some explanation of the conditions under which the verb prefixed with \textit{po-} tends to be interpreted delimitatively when used out of the context or in the neutral context.

Let me sketch how the pragmatic competition mechanism can be used in order to evoke such conditions. Consider the sentence \ref{ex:po:delim2}. For this sentence, there are alternative ways of denoting a completed eating event, such as \ref{ex:sjest}. So if the speaker wants to describe an event of eating all of the apples, they can utter \ref{ex:sjest}. The most appropriate description of the situation of eating the apples until becoming full is \ref{ex:najestsja3}. Due to such a competition when the sentence \ref{ex:po:delim2} (that literally means that some apples were eaten is uttered), it gets enriched with an additional inference that the quantity of the apples eaten is lower than the number of apples available and the amount of apples necessary for the actor to become full. I will provide some additional details on this kind of pragmatic competition in Chapter~\ref{Chapter6}.

\ex.\ag.\label{ex:sjest}Ivan s''el jabloki.\\
Ivan s.eat.\glb{pst.sg.m} apple.\glb{pl.acc}\\
\trans `Ivan ate the apples.'
\bg.\label{ex:najestsja3}Ivan naelsja jablok.\\
Ivan na.eat.\glb{pst.sg.m}.refl apple.\glb{pl.gen}\\
\trans `Ivan ate the apples until becoming full.'

From the proposed competition between different perfective verbs, it also follows that if \textit{po-} is not the first prefix that is attached to the verb, it often tends to be interpreted as referring to a partial event because it competes with the perfective verb without the prefix \textit{po-}.

%TODO: provide an explanation when and what happens. Preliminary: delimitative effect occurs if in some sense the scale cannot be changed. So if the direct object is singular (zapisat' disk), then `pozapisyvat' disk' won't mean `record the disk completely'. If the direct object is plural, po will have distributive meaning, so we don't consider this case. If the direct object is singular but the imperfectivization with habitual intrpretation allows to `get rid of this scale' (proletet' mimo okna princessy -- poproletat' mimo okna princessy)

%Note that in any case two components of the semantic contribution have to be separated: the end of the event has to be linked to achieving the some given value on the scale and this value can meet the standard or be below it. This has to be not confused with the situation when the end of the event is linked directly to some value on the scale that is either at the standard or below it, as this will mean that the event can be not a completed one (what we need is a completed event possibly with some limitation). To provide an example, let us consider again the sentence \ref{ex:po:delim2}. The desired semantics of the sentence is `Ivan ate some of the apples (and the quantity of the apples eaten is lower than expected),' not 

\subsubsection{Distributive \textit{po-}}
%NOTE from Filip: Participant-based individuation of subevents yields readings involving notionsl ike individually, each separatelv. Individuation of subevents based on separate running times results in adverbial temporal meanings of successfully, consecutively, one at a time ( e.g.,pozamykat' to lock X part by part, one (group) at a time, after another'). Individuation of subevents based on separate locations yields readings like here and there, all over. With base verbs describing some action of applying or attaching something onto something else or creating marks on something, po- generates the totality meaning of `to cover x with V-ing

Another usage of \textit{po-} we discuss in detail is the distributive (second meaning in the list taken from the grammar by \citealt{Shvedova:82}). The distributive interpretation of the prefix \textit{po-} seems to be the least studied prefix usage among all the prefix usages that are classified as superlexical by those linguists that adopt the distinction. \citet{Tatevosov:09}, for example, identifies it as a left periphery prefix (the only one in this category) and suggests the reader to look in the other paper of the same author for the discussion, but this paper is a 2009 manuscript and is not available in any form. In the book by \citet{Kagan:book} the distributive usage of \textit{po-} is not discussed either. 

What one can find are some descriptive notes in Russian studies of verbal prefixation. For example, \citet[289--290]{Isachenko:60} compares \textit{po-}prefixed and \textit{pere-}prefixed verbs with distributive semantics and concludes that distributive verbs containing the prefix \textit{po-} ``obozna\v{c}ajut distributivnost' dejstvija, no bez ottenka poo\v{c}erednosti otdel'nyx aktov, svojstvennogo glagolam na pere-... Semanti\v{c}eskaja raznica, odnako, o\v{c}en' tonkaja i ne\v{c}etkaja'' [denote the distributivity of the action, but without the semantics of the succession of the separate acts, that is characteristic for the verbs prefixed with \textit{pere-}... The difference in the semantics between the classes of verbs is, however, very slight and fuzzy].

So for the moment let us assume that the distributive usage of the prefix \textit{po-} can be characterized as `performing the action denoted by the derivational base with all of the objects or by all of the subjects specified in the sentence, without the individualization of the subevents.' We will compare the distributive usage of the prefix \textit{po-} with the distributive usage of the prefix \textit{pere-} in Section~\ref{subsection:semantics:pere}.

\subsection{Restrictions on the attachment} 
Let us start with considering the delimitative usage of the prefix \textit{po-}. \citet{Tatevosov:09} classifies the delimitative prefix \textit{po-} as a selectionally limited prefix. As we have already discussed in Section~\ref{section:Tat09} of the previous chapter, there are exceptions to this observation. For example, the verb \textit{popriotkryt'} `to open very slightly' in the sentence \ref{ex:popriotkryl:rep} is derived by prefixing the perfective verb \textit{priotkryt'} `to open slightly' with the delimitative prefix \textit{po-}.

\exg. \label{ex:popriotkryl:rep}A na e\v{s}elone on nemno\v{z}ko chut' popriotkryl oko\v{s}ko.\\
But at {flight level} he {a little bit} {slightly} po.pri.open.\glb{pst.sg.m} window.\glb{sg.acc}\\
\trans `And at the flight level he just a little bit opened the window.'\\\Source{= ex. \ref{ex:popriotkryl} in Chapter~\ref{Chapter4} here}

If one consults the list of the usages of the prefix \textit{po-} provided by \citet{Shvedova:82}, one will find that the list of examples for the first usage contains verbs with two prefixes and no imperfective suffix, such as \textit{poprivyknut'} `to get somehow used' and \textit{poiznosit'sja} `to get somewhat worn out'. 

%\exg. \label{ex:popriotkryval:rep}A na e\v{s}elone on nemno\v{z}ko chut' popriotkryval$^{\IPF}$ oko\v{s}ko.\\
%but at {flight level} he {a little bit} {slightly} po.pri.open.imp\glb{pst.sg.m} window.\glb{sg.acc}\\
%\vspace{0.5em}
%`And at the flight level he used to open the window just a little bit.'
%\begin{flushright}
%\vspace{-1em}
%= example \ref{ex:popriotkryval} here
%\end{flushright}

A possible informal explanation of the observed facts is the following: the delimitative prefix \textit{po-} normally cannot be attached to a perfective verb, because such a verb already denotes a completed\footnote{``Completed'' here means that the maximum point or the contextually determined standard point on the scale is reached. Punctual events can be considered a marginal case when the maximum and the minimum points are identical.} event. The semantic contribution of the prefix \textit{po-} is weaker than the semantic contribution of prefixes that demand the culmination of the event to correspond to the maximum on the scale or be higher than some expected value. Consequently, combining perfective verbs that contain such prefixes with the delimitative \textit{po-} will not enrich their semantics. The only possible change is removing the completeness (reaching the maximum point on the scale) component from the source event semantics, but it is not possible if one accepts the Monotonicity Hypothesis \citep{Kiparsky:82}.

Let us consider again the already mentioned example (\ref{ex:po:Tat2} in Chapter~\ref{Chapter4} here) that originally has been provided by \citet{Tatevosov:09}. The verb \textit{zapisat'} `to write down/to record' refers to a completed event of writing something down or recording. The relevant scale in this case is provided by the direct object, so the event is considered completed when the whole object is written down/recorded. If the verb \textit{zapisat'} `to write down/record' could be combined with the delimitative prefix \textit{po-}, the semantics of the derived verb would remain unchanged: the derivational base includes the information that the maximum point of the relevant scale has been reached whereas the prefix contributes the information that some point on the scale has been reached. In this case the attachement of the prefix violates the pragmatic principle introduced above, as it leads to the derivational chain in which two subsequent verbs have exactly the same semantics.\footnote{This is the case when semantic representations would be literally the same, as the information contributed by the prefix is already contained in the semantics of the derivational base.}

\exg.\label{ex:po:Tat2:rep}Po\`{e}tomu zapustil programmu, zapisyvaju\v{s}\v{c}uju dejstvija na \`{e}krane, otkryl PSP, i nemnogo $^\#$po-zapisal ($^{\textit{OK}}$po-zapisyval$^{\PF}$), \v{c}to i kak.\\
{because of it} za.let.\glb{pst.sg.m} program.\glb{sg.acc}, za.write.\glb{PAP.sg.f.acc} action.\glb{pl.acc} on screen.\glb{sg.prep} open.\glb{pst.sg.m} PSP and {a bit} $^\#$po.write.\glb{pst.sg.m} ($^{\textit{OK}}$po.write.imp\glb{pst.sg.m}),what and how\\
\trans `For this reason I ran the program that records the actions on the screen and recorded for some time, what and how (was happening).'\\\Source{= ex.~(63b) in \citealt{Tatevosov:09} and \ref{ex:po:Tat2} in Chapter~\ref{Chapter4} here} 

Why is the proposed preliminary semantic explanation more preferable than the syntactic one? Exactly because, according to it, there is no reason why the verb \textit{popriotkryt'} `to open very slightly' could not exist. The semantic explanation why \textit{po-} does not usually combine with perfective verb hinges on the fact that most perfective verbs denote events such that the end point of the event corresponds to one fixed point on the scale. If a perfective verb denotes an event such that its end point is not bounded to the maximum (or contextually determined standard) point on the scale, but can be any point from a range of points, then it should be possible to prefix it with the delimitative \textit{po-}. The meaning of the resulting verb would be the intensified (which in our case means further limitation) meaning of the derivational base. This is exactly the case of \ref{ex:popriotkryl:rep}.

Another example is provided in \ref{ex:popod-:rep}. As follows from the described intuition, the delimitative prefix \textit{po-} is redundant when it is attached to a perfective verb, as its semantic contribution is already present in the semantic representation of the derivational base. This explains why such verbs are awkward without a good context that motivates the need to emphasize the low degree on the relevant scale. In \ref{ex:popriotkryl:rep}, the usage of the verb is motivated by the speakers intention to report the actor's idea that a tiny opening cannot harm. In the other example, \ref{ex:popod-:rep}, that we have already discussed in Chapter~\ref{Chapter4}, it would be very harsh to use the frequent verb \textit{podsoxnut'} `to dry to some extent' with respect to one's brains, so the author of this comment chooses to soften the description by adding another delimitative prefix, \textit{po-}. 

\exg.\label{ex:popod-:rep}Za sorok let despotizma mozgi popodsoxli.\\
after forty year.\glb{pl.gen} despotism brain.\glb{nom} po.pod.dry.\glb{pst.pl}\\
\trans `During forty years of despotism his brain kind of dried a bit.'\\\Source{= ex.~\ref{ex:popod-} in Chapter~\ref{Chapter4} here}

Let us go back to the discussion of the example \ref{ex:popriotkryl:rep}. It turns out that a perfective verb \textit{popriotkryvat'}$^{\PF}$ `to slightly open multiple times', that is formed with an additional imperfectivization before the attachment of the prefix \textit{po-}, also exists. This verb denotes multiple events of opening within a short time period. 

Consider the examples \ref{ex:popriotkryval:pf1} and \ref{ex:popriotkryval:pf2}. In \ref{ex:popriotkryval:pf1} the verb \textit{popriotkryvala}$^{\PF}$ `slightly opened' denotes a short series of slight opening of the mouth, so the prefix \textit{po-} temporaly limits the series of openings. This series, in turn, is denoted by the derivational base \textit{priotkryvat'}$^{\IPF}$ `to open/be opening slightly'. In the example \ref{ex:popriotkryval:pf2} the verb  \textit{popriotkryval}$^{\PF}$ `slightly opened all of' also refers to a series of opening events. The difference between \ref{ex:popriotkryval:pf1} and \ref{ex:popriotkryval:pf2} is that in the latter case each opening event takes place with a different object (all the pots where there were no saplings to see), so according to the works on Russian prefixation this \textit{po-} is not delimitative, but distributive.\footnote{One can say that the verb \textit{popriotkryvala} `sligthly opened multiple times' is distributive as well, if distribution over time is allowed.}

\exg.\label{ex:popriotkryval:pf1}Poprobovali dat' im krevetku, Oskar ne otreagiroval, a Matil'da nemnogo rot popriotkryvala$^{\PF}$, no tak i ne poela.\\
po.try.\glb{pst.pl} give.\glb{inf} they.\glb{dat} shrimp.\glb{sg.acc} Oskar.\glb{nom} not ot.react.\glb{pst.sg.m} but Matilda {a bit} mouth po.pri.open.imp.\glb{pst.sg.f} but so and not po.eat.\glb{pst.sg.f}\\
\trans `We have tried to give them a shrimp, Oskar didn't react at all and Matilda slightly opened her mouth several times but didn't eat it.'\\\Source{\url{http://cherepahi.ru}}

\exg.\label{ex:popriotkryval:pf2}Daby izbe\v{z}at' podobnogo, slegka popriotkryval$^{\PF}$ vatu vo vsex gor\v{s}o\v{c}kax, gde net vsxodov.\\
for iz.run.\glb{imp} similar.\glb{sg.m.gen} slightly po.pri.open.imp.\glb{pst.sg.m} {cotton wool} in all.\glb{prep} pot.\glb{pl.prep} where no sapling.\glb{pl.gen}\\
\trans `To avoid a similar situation, I slightly opened the cotton wool coverage on all the pots where there were no saplings to see.'\\\Source{\url{http://ganja-forum.com}}

In some cases it is not clear which meaning does the prefix contribute. Even the number of the relevant noun does not always help. Consider the example \ref{ex:popod-?}. It can be interpreted as a statement about the generation as a whole growing up a little bit and it can also mean that each person from this generation grew up. This example is useful to illustrate the intuition of \citet{Isachenko:60} that there is no object-by-object iteration when the verb contains the distributive prefix \textit{po-}.

\exg.\label{ex:popod-?}...a nyn\v{c}e \v{z} – novoe pokolenie, kak-nikak, popodroslo, a ono \v{z}, \`{e}to pokolenie, -- ogo-go!\\
...but nowadays well - new generation.\glb{sg.nom}, {after all}, po.pod.grow.\glb{sg.pst.n}, but it.\glb{nom} {} this generation.\glb{sg.nom}, -- wow
\\
\trans `...but now, after all, the new generation grew up a bit, and it is quite a generation!'\Source{\url{http://ergos-paragogis.livejournal.com/37099.html}}
 
The conclusion one can arrive at after considering the examples above and in particular \ref{ex:popod-?} is that the delimitative and the distributive meanings of \textit{po-}, despite being very distinct at the first sight, are instances of the same underlying semantic representation. As we have seen, it is sometimes hard to determine which of the two usages of prefixes we are looking at in the given example. This is an argument if favour of abandoning the hypothesis of a strict boundary between the delimitative \textit{po-} and the distributive \textit{po-}.

It turns out that the scalar approach to prefixation allows to provide a single representations that can result in either interpretation depending on the type of the scale selected to measure the event progress. As we have seen, distributive interpretation occurs only in cases when there is a plural direct object that is interpreted definitely. This means that in the representation of this object there is an attribute such that its value can be used as the maximum point on the measure of change scale. (The minimum point on the measure of change scale is always 0.) The maximum and minimum points then become linked to the start and the end points of the event, respectively. This is interpreted as the event taking place until the action denoted by the verb has been applied to all of the members in the set denoted by the direct object. If the amount of the direct object is indefinite, no value that can serve as a maximum on the measure of change scale is available, so the end point of the event will correspond to an arbitrary point of this scale, leading (through an additional step of pragmatic strengthening) to the delimitative interpretation of the event. More details about the pragmatic level and the formal representation of the prefix will be provided together in Chapter~\ref{Chapter6} and Chapter~\ref{Chapter7}.

\subsection{Subsequent imperfectivization of a verb with the discussed prefix}
As the prefix \textit{po-} in its distributive usage does not have any puzzling restrictions on its attachment, the intriguing part turns out to be located in the imperfectivization domain. \citet[365]{Shvedova:82} notes that many of the verbs prefixed with the distributive \textit{po-} are derived from the perfective verbs (and at the same time they are colloquial) and are synonymous to the verbs that are motivated by the imperfective counterparts of the derivational bases (some of these verbs are also colloquial, but their percentage is much lower), as in the pair \textit{povybit'}$^{\PF}$ -- \textit{povybivat'}$^{\PF}$ `to knock out many/all of'.

For the account presented here, such data poses a certain challenge, i.e.\ it has to be explained why, e.g., in the pair \textit{povybit'}$^{\PF}$ -- \textit{povybivat'}$^{\PF}$ `to knock out many/all of' the second verb could not be derived from the first one or, if it could, why it is perfective despite the fact that adding the imperfective suffix is the last step of the derivation. I propose to take the first path and to explain why imperfectivization is not possible after the attachment of the distributive \textit{po-} (or, adjusting to the merge of the two usages proposed above, why in the situation when the attachment of the prefix \textit{po-} leads to the distributive interpretation of the derived verb, this verb is not compatible with further imperfectivization). It turns out that if the semantics of the imperfective suffix is added to the semantics of the verb prefixed with the distributive \textit{po-}, the semantics of the resultant verb is similar to that of an imperfective verb that is not prefixed with \textit{po-}. Due to this, the derivation of a more complex form to express the same meaning is blocked.

To provide more details, let us consider the pair of verbs \textit{povybe\v{z}at'}$^{\PF}$ -- \textit{po\-vy\-be\-gat'}$^{\PF}$ `to run out'. The sentence \ref{ex:povybegat} illustrates the usage of the second verb in this pair. The first verb, formed from the perfective derivational base \textit{vybe\v{z}at'} `to run out', can be also used in the same sentence (the verb itself is colloquial) which is illustrated by \ref{ex:povybezhat}.

\ex.\label{ex:povy}\ag.\label{ex:povybegat}I povybegali$^{\PF}$ na ulicu, i stali smotret' v zv\"{e}zdnoe nebo i slu\v{s}at' goluboj zvon.\\
and po.vy.run.\glb{pst.sg.m} on street, and become.\glb{pst.sg.m} look.\glb{inf} in starry sky and listen.\glb{inf} blue ringing\\
\trans `And they all ran out onto the street and started staring at the starry sky and listening to the blue ringing.'\\\Source{Sergej Kozlov. \textit{Pravda, my budem vsegda?}}
\bg.\label{ex:povybezhat}I povybe\v{z}ali$^{\PF}$ na ulicu, i stali smotret' v zv\"{e}zdnoe nebo i slu\v{s}at' goluboj zvon.\\
and po.vy.run.\glb{pst.sg.m} on street, and become.\glb{pst.sg.m} look.\glb{inf} in starry sky and listen.\glb{inf} blue ringing\\
\trans `And they all ran out onto the street and started staring at the starry sky and listening to the blue ringing.'

If the verb \textit{povybe\v{z}at'}$^{\PF}$ `to run out' could have been suffixed in order to produce an imperfective verb, this verb would have two interpretations: progressive and habitual. Progressive interpretation in the above context would mean that people are in the process of running out to the street. This meaning can be conveyed with the imperfective verb \textit{vybegat'}$^{\IPF}$ `to run/be running out', as exemplified by \ref{ex:vybegat} (the verb in the second clause has to be changed in order to satisfy the discourse restrictions on the aspect of the verbs in the narrative sequence, see Section~\ref{sec:tests:new} for more details). The second possible interpretation of a potential imperfective verb formed by suffixing the verb \textit{povybe\v{z}at'}$^{\PF}$ `to run out' is habitual: each time after a certain other event, people run out onto the street and stare at the sky. This interpretation is also a possible interpretation of the sentence \ref{ex:vybegat}. So if we accept that there is a competition between different verbs such that when the semantics of the two verbs is effectively the same,\footnote{As I provide a compositional account, it cannot be exactly the same in this case as the representation of the derivational base gets updated after the prefixation with \textit{po-}. The semantics being effectively the same means that when the formal representation is interpreted, there is no semantic difference between the two verbs.} only the verb that is morphologically simpler can be used, the absence of the secondary imperfective verbs derived from the \textit{po-}prefixed verbs with the distributive interpretation is expected.

\exg.\label{ex:vybegat}I vybegali$^{\IPF}$ na ulicu, i na\v{c}inali smotret' v zv\"{e}zdnoe nebo i slu\v{s}at' goluboj zvon.\\
and vy.run.\glb{pst.sg.m} on street and start.\glb{pst.sg.m} look.\glb{inf} in starry sky and listen.\glb{inf} blue ringing\\
\trans `And they were running out onto the street and starting to stare at the starry sky and to listen to the blue ringing.'

This explanation is valid in case the only meaning that is contributed by the prefix is distributive. Now let us explore what happens if there is a delimitative component in the semantic contribution of \textit{po-}. Consider the verb \textit{poest'}$^{\PF}$ `to eat/to eat up', that we have already discussed. It can be suffixed with the imperfective suffix and yield the imperfective verb \textit{poedat'}$^{\IPF}$ `to eat up/be eating up'. Examples \ref{ex:poedat1} and \ref{ex:poedat2} show how the habitual and the progressive interpretations of this verb can be uttered. Note that it is the submeaning `to eat up/destroy by eating' that is relevant in these contexts.

\ex.\ag.\label{ex:poedat1}V dikoj prirode tak u\v{z} zavedeno: milye i trogatel'nye zveru\v{s}ki poedajut drug druga.\\
in wild nature so well organized cute and touching beast.dim.\glb{pl.nom} po.eat.imp.\glb{pres.pl.3} friend.\glb{sg.nom} friend.\glb{sg.acc}\\
\trans `It is just like this in the wild nature: cute and touching animals eat each other up.'\Source{\url{mixstuff.ru}}
\bg.\label{ex:poedat2}Ja s\v{c}itaju, \v{c}to \v{c}inovniki -- \`{e}to takoe sugubo nadstroe\v{c}noe soslovie, kotoroe sej\v{c}as prosto poedaet stranu.\\
I count.\glb{pres.sg.1} that official.\glb{pl.nom} {} this such especially superstructural estate that now simply po.eat.imp.\glb{pres.sg.3} country.\glb{sg.acc}\\
\trans `I think that officials are just a superstructural estate, that now is simply eating up the country.'\\\Source{Elena Semenova. \textit{Oligarx bez galstuka} (2003)}

Let us try to see why in this case the formation of the imperfective is not blocked. Consider the sentences \ref{ex:est1} and \ref{ex:est2} that are obtained by replacing the verb \textit{poedat'}$^{\IPF}$ `to eat up/be eating up' with the verb \textit{est'}$^{\IPF}$ `to eat' in the sentences \ref{ex:poedat1} and \ref{ex:poedat2}, respectively.
 
\ex.\ag.\label{ex:est1}V dikoj prirode tak u\v{z} zavedeno: milye i trogatel'nye zveru\v{s}ki edjat drug druga.\\
in wild nature so well organized: cute and touching beast.dim.\glb{pl.nom} po.eat.imp.\glb{pres.pl.3} friend.\glb{sg.nom} friend.\glb{sg.acc}\\
\trans `It is just like this in the wild nature: cute and touching animals eat each other.'
\bg.$^?$Ja s\v{c}itaju, \v{c}to \v{c}inovniki -- \`{e}to takoe sugubo nadstroe\v{c}noe soslovie, kotoroe sej\v{c}as prosto est stranu.\label{ex:est2}\\
\hspaceThis{$^?$}I count.\glb{pres.sg.1} that official.\glb{pl.nom} -- this such especially superstructural estate, that now simply po.eat.imp.\glb{pres.sg.3} country.\glb{sg.acc}\\
\trans `I think that officials are just a superstructural estate, that now is simply eating the country.'

English translations of the sentence pairs \ref{ex:poedat1}/\ref{ex:est1} and \ref{ex:poedat2}/\ref{ex:est2} show that  the meaning changes when the verb \textit{poedat'} `to eat up/be eating up' is replaced by the verb \textit{est'} `to eat'. The sentence~\ref{ex:est1} lacks the destruction meaning component and is naturally interpreted as referring to a situation of two animals sitting and chewing each others' parts simultaneously. So the sentence \ref{ex:est1} can be uttered instead of \ref{ex:poedat1}, but it does not convey the same meaning.

The difference between the sentences \ref{ex:poedat2} and \ref{ex:est2} is even bigger: while the sentence \ref{ex:poedat2} has the meaning that the country is being destroyed and in the end will be destroyed (`eaten up') completely by the officials, the sentence \ref{ex:est2} sounds strange, as the verb \textit{est} `eats' lacks the figurative meaning of destroying and is interpreted literally as officials nourishing on the country. It also lacks the component of the intention to eat the whole country. In sum, the verb \textit{est'} `to eat' refers to a situation of eating literally, whereas the verb \textit{poest'} `to eat/to eat up' can have both the literal and the figurative meaning and the verb \textit{poedat'} `to eat up/be eating up' retains only the figurative part of the meaning. This is summarized in Table~\ref{table:eat}. For the discussion of the similar phenomenon in English and Italian see \citet{FolliHarley:05}.

\begin{table}
\caption{Distribution of literal and figurative meanings of \textit{est'} `to eat' and its derivatives \label{table:eat}}
\begin{tabular}{lll}
\lsptoprule
& literal & figurative \\ \midrule
IPF & est' & poedat' \\
PF & poest' & poest' \\ \lspbottomrule
\end{tabular}
\end{table}

The verb \textit{popriotkryvat'} `to open slightly' provides another illustration of the same phenomena. As we have discussed, it can have both distributive and delimitative interpretations. The derivational chains in \ref{chain:popriotkryvat} show two ways in which the verb \textit{popriotkryvat'} `to open slightly' can be derived, whereby each way leads to a different aspect and a different interpretation of the verb: if the prefix \textit{po-} is attached on the last step of the derivation (chain \ref{chain:popriotkryvat:1}), the derived verb denotes a series of opening events, each of which is a slight opening. If the imperfective suffix is attached on the last step of the derivation (chain \ref{chain:popriotkryvat:2}), the derived verb is imperfective and denotes a set of very slight opening events. 

\ex.\label{chain:popriotkryvat}\ag.\label{chain:popriotkryvat:1}otkryt'$^{\PF}$ $\rightarrow$ priotkryt'$^{\PF}$ $\rightarrow$ priotkryvat'$^{\IPF}$ $\rightarrow$ popriotkryvat'$^{\PF}$\\
{to open} {} {to open slightly} {} {to (be) slightly open(ing)} {} {to slightly open multiple times}\\
\bg.\label{chain:popriotkryvat:2}otkryt'$^{\PF}$ $\rightarrow$ priotkryt'$^{\PF}$ $\rightarrow$ popriotkryt'$^{\PF}$ $\rightarrow$ popriotkryvat'$^{\IPF}$\\
{to open} {} {to open slightly} {} {to open very slightly} {} {to (be) open(ing) very slightly}\\

The imperfective aspect of the verb \textit{popriotkryvat'} `to open slightly' may be hard to access, but it is attested, as evidenced by the example \ref{ex:popriotkryvat:ipf}. 

\exg.\label{ex:popriotkryvat:ipf}A e\v{s}\v{c}e pojavljaetsja prikol'naja, \v{c}isto pontovaja, vozmo\v{z}nost' poprikryvat' {$\backslash$} popriotkryvat' kry\v{s}ku v ljuboj moment.\\
but also po.apear.\glb{pres.sg.3}.refl neat pure {show off} possibility po.pri.close.\glb{inf} {$\backslash$} po.pri.open.\glb{inf} lid in any moment\\
`And you also get a neat, purely show off possibility to very slightly close and open the lid at any moment.'\Source{\url{www.chevrolet-cruze-club.ru}}

Let us now consider the example \ref{ex:popisyvat} where the imperfective verb \textit{popisyval} `wrote' seems to be interpreted distributively. This sentence means that the actor wrote his articles without devoting much time to it, non-seriously. So the prefix in this case delimits the time spent during each writing session, but not the amount of the article written: the sentence is interpreted in a way that the articles were probably completed and it is also possible that during each writing session a whole article was written. On the other hand, this does not have to be the case and can be explicitly denied, as is illustrated by \ref{ex:popisyvat:none}. The holistic implication is also lost if the direct object is singular \ref{ex:popisyvat:single}, as in this case occasional writing is only possible if the article is not completed. 

\exg.\label{ex:popisyvat}V svobodnoe vremja on popisyval statji.\\
in spare time he po.write.imp.\glb{pst.sg.m} article.\glb{pl.acc}\\
\trans `In his spare time he wrote articles.'

\exg.\label{ex:popisyvat:none}V svobodnoe vremja on popisyval staji, no ni odnu ne zakon\v{c}il.\\
in spare time he po.write.imp.\glb{pst.sg.m} article.\glb{pl.acc} bot nor one not za.complete.\glb{pst.sg.m}\\
\trans `In his spare time he wrote articles, but never finished any of them.'

\exg.\label{ex:popisyvat:single}V svobodnoe vremja on popisyval statju.\\
in spare time he po.write.imp.\glb{pst.sg.m} article.\glb{sg.acc}\\
\trans `In his spare time he was writing an article.'

\exg.\label{ex:pisatstatji}V svobodnoe vremja on pisal statji.\\
in spare time he write.\glb{pst.sg.m} article.\glb{pl.acc}\\
\trans `In his spare time he wrote articles.'

This serves as an evidence that the delimitative interpretation of the prefix \textit{po-} only arises when the event progress is not related to the scale contributed by the direct object. The plural object creates the distributivity effect, which is also present in case of the non-prefixed verb: the sentence \ref{ex:pisatstatji} lacks the component of `non-serious occupation that does not take much time', but still refers to the situation of multiple articles being written on multiple occasions. 
%This is also related to what has been said above about the difficulty of separating the delimitative and the distributive usages of \textit{po-}. 

\subsection{Summary}
I propose to provide a unified formal representation for the delimitative, resultative, and distributive usages of the prefix \textit{po-}, thereby covering all the interpretations provided by \citet{Shvedova:82}. The following observations are crucial for the construction of the desired semantic representation:

\begin{itemize}
\item \textit{po-} can be attached to different scales; in the default case, the scale is one of the verbal scales; if an event denoted by the derivational base is an iteration, a \textit{cardinality} scale provided by the direct object can be used as well;
\item if the scale selected by \textit{po-} is of type \textit{cardinality}, then the start point of the event gets linked to the minimum point on the scale and the end point of the event gets linked to the maximum point on the scale; if the scale is a verbal scale, an arbitrary point on (the open end of) the scale is linked to the respective endpoint of the event;
\item in case the endpoint of the event results being linked to an arbitrary point of the scale, pragmatic strengthening can take place if there are other verbs capable of denoting events corresponding to some definite portions of the scale (for more details see Chapter~\ref{Chapter6}).
\end{itemize}


%\begin{avm}
%      \[\asort{event}
%             \feat{dim\_spec} & \[\@x\]\\
%             \feat{measure} & \[
%             	\asort{scale $\wedge$ \@x}
%             	\feat{min} & \@y\\
%             	\feat{max} & \@z \]\\
%             \feat{startp} & \@y\\
%             \feat{endp} & \@z
%        \]
%\end{avm}\\

\section{\textit{pere-}}\label{subsection:semantics:pere}
\subsection{Semantic contribution}
The prefix \textit{pere-} is notoriously polysemous. To start, we will consult \citet{Shvedova:82}, who distinguishes the following ten meanings that the prefix may contribute to the semantics of the derived verb (pp. 363--364):
\begin{enumerate}
\item to direct the action denoted by the derivational base from one place to another through the space or over the other object: \textit{perenesti'} `to carry something over something', \textit{perebrosit'} `to throw over' (productive usage, some derivational bases are perfective); 
\item place something between the other objects or parts of the other object by performing an action denoted by the derivational base: \textit{peresypat'} `to pour something between something else' (non-productive); 
\item to perform the action denoted by the derivational base again or anew: \textit{peredelat'} `to redo', \textit{pereizbrat'} `to reelect', \textit{pereproektirovat'} `to redesign', \textit{pereoborudovat'} `to reequip' (productive usage, some derivational bases are perfective or biaspectual, some derived verbs are biaspectual);
\item to perform the action multiple times with different objects of the same kind or by different subjects: \textit{pereglotat'} `to swallow all of something one by one', \textit{perezarazit'} `to infect all of', \textit{pereranit'} `to wound all of' (productive usage, some derivational bases are perfective or biaspectual);
\item to perform the action denoted by the derivational base with too much intensity or for a too long time: \textit{peregret'} `to overheat' (productive); 
\item to perform the action denoted by the derivational base intensively: \textit{perepugat'} `to scare a lot' (non-productive); 
\item to overcome someone else, performing an action denoted by the derivational base: \textit{peresporit'} `to win the argument' (productive, derived verbs are obligatory transitive); 
\item to perform the action denoted by the derivational base for some predefined time: \textit{pere\v{z}dat'} `to pass the necessary time waiting' (productive in colloquial speech);
\item to stop the state, process or activity denoted by the derivational base after a long time of this action being performed: \textit{perebolet'} `to recover from illness' (productive); 
\item a short, non-intense action, performed in the pause in the other action: \textit{perekurit'} `to smoke, taking a brake' (non-productive).
\end{enumerate} 

This is a detailed list of \textit{pere-} usages, some of which can be merged. For example, \citet[119--125]{Kagan:book} provides a unified account covering the following five different meanings of \textit{pere-}: 
\begin{enumerate}
\item `to cross' (corresponds to the first usage in the list above, see example \ref{ex:pere:cross});
\item `to redo' (corresponds to the third usage in the list above, see example \ref{ex:pere:redo});
\item excess (corresponds to the fifth usage in the list above, see example \ref{ex:pere:excess});
\item comparison (corresponds to the seventh usage in the list above, see example \ref{ex:pere:comparison});
\item spending time (corresponds to the usages eight, nine, and ten in the list above, see example \ref{ex:pere:time});
\end{enumerate}

\ex.\label{ex:pere}\ag.\label{ex:pere:cross}Vasja pereplyl reku.\\
Vasja pere.swim.\glb{pst.sg.m} river.\glb{sg.acc}\\
\trans `Vasja swam to the other side of the river.'
\bg.\label{ex:pere:redo}Vasja perepisal examen.\\
Vasja pere.write.\glb{pst.sg.m} exam.\glb{sg.acc}\\
\trans `Vasja rewrote the exam.'
\bg.\label{ex:pere:excess}Vasja peregrel sup.\\
Vasja pere.warm.\glb{pst.sg.m} soup.\glb{sg.acc}\\
\trans `Vasja overheated the soup.'
\bg.\label{ex:pere:comparison}Vasja pereigral Ma\v{s}u.\\
Vasja pere.play.\glb{pst.sg.m} Masha.\glb{acc}\\
\trans `Vasja outplayed Masha.'
\bg.\label{ex:pere:time}Vasja pere\v{z}dal do\v{z}d'.\\
Vasja pere.wait.\glb{pst.sg.m} rain.\glb{sg.acc}\\
\trans `Vasja waited for the rain to stop.'

Let me show how \citet{Kagan:book} unifies different usages of the prefix \textit{pere-}. For the base meaning, \citet[120--121]{Kagan:book}, following \citet{Janda:88}, takes the spatial interpretation `to cross'. Here is the characterization that \citet[121]{Kagan:book} gives to the underlying meaning of \textit{pere-}: ``[t]here is a certain spatial location, and the individual that undergoes motion moves through this location, eventually getting to `the other side'.'' Based on this, \citet[122]{Kagan:book} proposes that the ``prefix imposes a relation of inclusion between two intervals on a scale''. This is formalized as shown in \ref{Kagan:pere} (d$_s$ refers to the contextually provided standard degree).

\ex.\label{Kagan:pere}$\llbracket pere- \rrbracket = \lambda$P$\lambda$d$_s\lambda$d$\lambda$x$\lambda$e.[P(d)(x)(e) $\wedge$ d$_s \subseteq _U$ d]\\
where d = degree of change \citep{KennedyLevin:02} and $\subseteq _U$ is defined as\\
$\forall$d$\forall$d' [d $\supset$ d' $\leftrightarrow$ (d $\supset$ d' $\wedge$ max \{p: p $\in$ d\} $>$ max \{p: p $\in$ d'\})]\\\Source{\citep[from][123]{Kagan:book}}

The formal semantics in \ref{Kagan:pere} gives rise to the spatial meaning of \textit{pere-} when applied to the \textit{path} scale. When the same is applied to the \textit{time} scale, the meaning `to spend some particular time' arises. So the event of swimming described by \ref{ex:pere:cross} is terminated when the path covered in course of swimming includes the width of the (deep part) of the river. As for the \ref{ex:pere:time}, the time of the waiting event is determined by the time of the rain: the waiting started when the rain started (or shortly after) and the waiting stopped when the rain was over (or became insignificant).

\subsubsection{Excessive and comparison usages}
In order to derive the excess and comparison meanings, \citet[133]{Kagan:book} additionally strengthens the representation in \ref{Kagan:pere} by replacing the upper inclusion ($\subseteq _U$) relation with the proper upper inclusion ($\subset _U$). This is motivated by the fact that a sentence such as \ref{ex:pere:excess} refers to a situation when Vasja heated the soup necessarily more than the soup should be heated. (Note that \ref{ex:pere:excess} cannot be uttered in a situation when Vasja heated (and thus immediately started to overheat) the soup that was already hot at the moment Vasja started to heat it.) Similarly, the sentence \ref{ex:pere:comparison} refers to a situation where Vasja played better or longer than Masha, not equally good or long.

The two meanings are related to two different sources of the scales. Consider the example \ref{ex:pere:comparison}. The only scale that is present in the semantic representation of the verb \textit{igrat'} `to play' is the time scale. If \textit{pere-} is attached to it, we find ourselves in the \textit{excess} situation: the verb \textit{pereigrat'} `to play for too long' refers to exceeding the time of playing appropriate for the subject. Again, the verb \textit{pereigrat'} `to play for too long' cannot refer to a situation where any time of playing would be too long (in other words, when the playing starts at the point that marks the appropriate time for the subject to play). Together with the verbs \textit{poigrat'} `to play for some time' and \textit{proigrat' (3 \v{c}asa)} `to play continuously (for 3 hours)' the verb \textit{pereigrat'} `to play for too long' covers the domain of possible time-related meanings the speaker may want to express with respect to the playing event.

To acquire the comparison meaning, the verb has to become transitive, as noted by \citet{Shvedova:82}. The reason for this is that when it becomes transitive, the direct object becomes another, external, source of scales. The process of obtaining a scale may be not straightforward, though. An individual (e.g., \textit{Masha} in the example \ref{ex:pere:comparison}) is not a scale. So, in order to interpret the sentence, the scale has to be constructed. I propose to describe the scale construction process as proceeding along the following steps. First, one of the scales that are relevant in the situation described by the verb is picked (this can be playing quality or playing length in our example); second, one point that corresponds to the performance of the individual that is denoted by the direct object (how well or how long has Masha played) is marked on this scale. When this is done, the situation is no longer different from that of playing too much, where a point that represents the appropriate time of playing for the subject is marked on the time scale.

Before we proceed, I would like to mention two observations that concern the comparison meaning and reveal some details about the structure of this meaning. First, note that the discussed comparison verbs (when they do not refer to the time scale) are only used in the situations where the initial stage of the event favours the patient, not the actor: so for the sentence \ref{ex:pere:comparison} to be true it has to not only be the case that Vasja ended up outplaying Masha, but also that when Vasja started to play he had a weaker position than Masha. If this is not the case and they simultaneously start to play and there are no expectations about who will be playing better, another verb, \textit{obygrat' X} `to win from X' will be used, as in the example \ref{ex:comparison:obygrat}. 

\exg.\label{ex:comparison:obygrat}Vasja obygral Ma\v{s}u.\\
Vasja ob.play.\glb{pst.sg.m} Masha.\glb{acc}\\
\trans `Vasja won from Masha.'

Another illustrative pair of examples is constituted by the sentences \ref{ex:comparison:peregnat} and \ref{ex:comparison:obognat}, where the verb prefixed with \textit{pere-} (\textit{peregnat'} `to overtake') is used in the situation when the actor was located behind the patient (in the literal or metaphorical sense) at the beginning of the event, whereas the verb prefixed with {\textit{ob-},} \textit{obognat'} `to overtake' lacks this requirement: the sentence \ref{ex:comparison:obognat} can be used in a situation when the height of the trunks has been exactly the same all the time. If we try to modify the sentence, replacing the verb \textit{obognat'} `to overtake' with the verb \textit{peregnat'} `to overtake', the resulting sentence in \ref{ex:peregnat} is suitable to use in a situation when the periods of the `height leadership' of one trunk are followed by the periods of the `height leadership' of the other.

\exg.\label{ex:comparison:peregnat}Dognal, kone\v{c}no, i peregnal, potom sbavil skorost' i poravnjalsja.\\
do.race.\glb{pst.sg.m} {of course} and pere.race.\glb{pst.sg.m} then reduce.\glb{pst.sg.m} speed and po.equal.\glb{pst.sg.m}.refl\\
\trans `I caught up, of course, and overtook, then reduced the speed and came alongside.'\Source{I. Grekova. \textit{Na ispytanijax} (1967)}

\exg.\label{ex:comparison:obognat}Ix korni s maloletstva splelis', ix stvoly tjanulis' vverx rjadom k svetu, starajas' obognat' drug druga.\\
their roots from childhood weave.\glb{pst.pl}.refl their trunks strech.\glb{pst.pl}.refl up near to light trying ob.race.\glb{inf} one another\\
\trans `Their roots got weaved from the childhood, their trunks were stretching to the sun, trying to overtake each other.'\\\Source{M. M. Pri\v{s}vin. \textit{Kladovaja solnca} (1945)}

\exg.\label{ex:peregnat}Ix korni s maloletstva splelis', ix stvoly tjanulis' vverx rjadom k svetu, starajas' peregnat' drug druga.\\
their roots from childhood weave.\glb{pst.pl}.refl their trunks strech.\glb{pst.pl}.refl up near to light trying pere.race.\glb{inf} one another\\
\trans `Their roots got weaved from the childhood, their trunks were stretching to the sun, trying to overtake each other.'

The second observation is concerned with cases where the time scale is used for the comparison. Let us consider an example provided by \citet[142]{Kagan:book} and repeated here under \ref{ex:Kagan:perezhit}. The sentence \ref{ex:Kagan:perezhit} refers to a situation when the lifespans of Dima and Masha overlap and there is an interval following Dima's death when Masha is still alive. This sentence can be uttered also in case Masha and Dima are siam twins and were born simultaneously, as is illustrated by the example \ref{ex:siam}.

\exg.\label{ex:Kagan:perezhit}Ma\v{s}a pere\v{z}ila Dimu.\\
Masha pere-lived Dima\\
\trans `Masha outlived Dima.'\Source{= example (50) in \citet{Kagan:book}}

\exg.\label{ex:siam}V Londone umerli razdelennye siamskie bliznecy: odna sestra pere\v{z}ila druguju na 4 nedeli.\\
in London die.\glb{pst.pl} separated siam twins: one sister pere.live.\glb{pst.sg.f} other on 4 weeks\\
\trans `Separated siam twins died in London: one sister outlived the other for 4 weeks.'\Source{\url{http://www.newsru.com/arch/world/26dec2008/twins.html}}

Examples \ref{ex:Kagan:perezhit} and \ref{ex:siam} show that the only point on the scale that is taken from the information about the direct object is the date and time of death. The time when Dima was born does not matter for the truth conditions of \ref{ex:Kagan:perezhit}. So only the point of Dima's death becomes the fixed point on the scale and the information conveyed by the sentence \ref{ex:Kagan:perezhit} is that Masha started to live at some time before the death of Dima, lived at the moment of the death of Dima, and stopped living at some time after the death of Dima. This is exactly what \citet{Kagan:book} considers this sentence to mean. 

The difference between the approach I offer and that of \citet{Kagan:book} is that \citet{Kagan:book} operates with a time interval (corresponding to Dima's lifespan in the discussed example),\footnote{\citet[143--144]{Kagan:book} has to deal with additional difficulties related to the elimination of the condition that Masha started to live not later than Dima. She proposes to use \textit{an upper part} of the time interval of Dima's life.} whereas I propose to use only one point (that of Dima's death). The value on the scale has to change from some value below this point to some value above it in the course of the event. As follows both from the explanations provided by \citet{Kagan:book} and from what we have just discussed, the information about the birth of Dima is of no importance for the interpretation of the sentence \ref{ex:Kagan:perezhit}. So the proposal of \citet{Kagan:book} can be simplified by replacing the interval with the relevant point, as is done here. I will show how this works in Chapter~\ref{Chapter7}.

\subsubsection{Repetitive usage}
Now let us discuss how the analysis proposed by \citet{Kagan:book} can be extended to the repetitive usage of the prefix \textit{pere-}, as this extention seems to be more tricky. \citet[149]{Kagan:book} provides a lot of valuable observations in this respect, arriving to the conclusion that ``repetitive \textit{pere-} is only possible with those predicates that contribute closed scales''
such that ``an increase along the same scale can be repeated''. She also emphasizes the importance of the event and its iteration being connected to each other. \citet[148]{Kagan:book} ends up with the following description of the important properties of the repetitive meaning of \textit{pere-} (conditions (2) and (3) come together in the original proposal): 
\begin{enumerate}
\item ``An event that falls under the denotation of the VP (or brings about the same kind of result state) is presupposed to have taken place before event time.'' 
\item ``The event predicate is interpreted as telic. Both the presupposed event and the entailed one are associated with a natural endpoint.'' 
\item ``In the course of the presupposed event, this point [the natural endpoint] has been reached.''
\item ``Typically, the entailed and the presupposed event are interrelated and can be conceptually unified.''
\end{enumerate}

I agree with the second point about the telicity of the events and also with the last point about the two events being interrelated. As for the first point, we will discuss it in detail in the next chapter (Chapter~\ref{Chapter6}). 

As for the third point, there seems to be some confusion with respect to the identification of natural endpoints. \citet{Kagan:book} provides the example \ref{ex:perestirat} to support her point. She notices that \ref{ex:perestirat} cannot be uttered in the situation when the dress was first washed, than worn, became dirty and was washed again. A possible scenario would be one where the dress was washed but did not become clean and thus it had to be washed again. In this case the first event of washing terminates but it does not reach the natural endpoint which corresponds to the clean state of the dress.

\exg.\label{ex:perestirat}Lena perestirala plat'e.\\
Lena pere-washed dress\\
\trans `Lena rewashed the dress.'\Source{= example (56) in \citet{Kagan:book}}

In fact it is even possible that the first washing was not complete: for example, the power could have gone out, the washing machine stopped without finishing its cycle and because of this the whole washing of the dress had to be redone. So it turns out that exactly the fact that the event did not reach the natural endpoint motivates why the whole process must be repeated.

Another example \ref{ex:peresdat} describes a situation where a girl did not have a chance to finish the exam (which is a natural endpoint of writing it) because she was expelled. Nevertheless, a new attempt to pass the same exam can be referred to by either the perfective verb \textit{peresdat'} `to retake' or the imperfective verb \textit{peresdavat'} `to retake/be retaking'. This situation is not compatible with one of the conclusions of \citet{Kagan:book}.

\exg.\label{ex:peresdat}Sud ne razre\v{s}il peresdat' EG\`{E} \v{s}kol'nice, kotoruju vygnali s \`{e}kzamena za spisyvanie.\\
court not allow.\glb{pst.sg.m} pere.s.give.\glb{inf} EGE schoolgirl.\glb{sg.dat}, that vy.chase.\glb{pst.pl} from exam for cheating\\
\trans `The court didn't allow the schoolgirl expelled from the EGE exam for cheating to retake it.'\Source{\url{http://www.newsmsk.com/}}

One more example to think about is provided under \ref{ex:perestelit}. The event of redoing the bed (changing the linens) does not require the bed to be done inappropriately. The sentence \ref{ex:perestelit} can be used in the situation when Katja did the bed, someone slept on it, it became dirty and she changed it. What I consider crucial here is that Katja had to undo the bed before doing it again. This is revealed in comparison with the sentence \ref{ex:postelit} where the verb prefixed with \textit{po-} denotes an event of doing the bed but does not require the bed to be undone as a preparation step for the main event. 

\exg.\label{ex:perestelit}Katja perestelila postel'.\\
Katja pere.lay.\glb{pst.sg.f} bed\\
\trans `Katja changed the linens.'

\exg.\label{ex:postelit}Katja postelila postel'.\\
Katja po.lay.\glb{pst.sg.f} bed\\
\trans `Katja made the bed.'

I think that the semantics of the \textit{pere-}prefixed verbs in the examples \ref{ex:perestirat}, \ref{ex:peresdat}, and \ref{ex:perestelit} can be unified by imposing a requirement for the preparatory phase of the event denoted by a \textit{pere-}prefixed verb. The preparatory phase has to include annulling of the result of the previous event. This can be represented as moving from the point on the scale that has been reached earlier back to the start point. In case of \ref{ex:perestirat} an event of washing a dress after it has been washed and became dirty again is excluded due to the result of the washing being already annulled by the wearing of the dress. In case of the exam, the result of the previous attempt is annulled when the new attempt begins. If we are talking about redoing the bed, it still has linens at the beginning of the redoing event and the fact that they are dirty does not affect their presence. Thus we obtain the desired asymmetry between the examples \ref{ex:perestirat} and \ref{ex:perestelit}. This approach also works in other cases discussed in \citealt{Kagan:book} with respect to the repetitive usage of the prefix \textit{pere-}.

In sum, I propose to weaken the condition formulated by \citet{Kagan:book} that the first event must reach the natural endpoint and make the last condition about the two events being interrelated more precise. This is done by introducing the preparatory phase that includes an event that proceeded along the same scale and had some final stage associated with a certain point on this scale. The transition from the preparatory phase to the main event then necessarily includes annuling the result of the preparatory event, as this corresponds to the transition to the minimum point of the scale (that is, in turn, the initial stage of the main event).

%From what I have just proposed it follows that the only requirement on the state of the world at the beginning of the preparatory phase of the event denoted by the verb prefixed with the repetitive \textit{pere-} is that this state corresponds to the non-zero point on the relevant scale. This means that it is not necessarily the case that the event itself has to be repeated. It proves to be the correct prediction due to the presence of the examples like \ref{ex:perekrasit}
%
%\exg.\label{ex:perekrasit}Ja pervyj raz perekrasila volosy v 14 let, do six por svoj cvet vosstanovit' ne mogu, a krasit'sja snova i snova - volosy \v{z}al'.\\
%I first time pere.color.\glb{pst.sg.f} hair in 14 years until these time my color regain not can but color.\glb{inf.refl} again and again {} hair sorry\\
%\vspace{0.5em}
%`When I dyed my hair for the first time I was 14; I still cannot regain my natural color and dying the hair again and again is a pity.'\\
%\vspace{-0.5em}
%\begin{flushright}
%chatic.net
%\end{flushright}
%
%
%In the example \ref{ex:perekrasit} the verb \textit{perekrasila} `recolored' refers to the first time the actor ever dyed her hair. Obviously, her hair had some color before the event of dying (which satisfies the requirement of the repetitive prefix \textit{pere-}) but no dying ever occurred before. 

There is a certain flexibility with respect to the scale selection that leads to various possible interpretations of the same repetitive verb. For example, the verb \textit{pere\v{s}it'} `to resew' often refers to changing the piece of clothes to fit the size of the other person without changing its kind, as in the example \ref{ex:pereshit:same}.

\ex.\label{ex:pereshit}\ag.\label{ex:pereshit:same}ona s udovol'stviem pere\v{s}ila na devo\v{c}ek svoi svetlye, v melkij cveto\v{c}ek, v veno\v{c}ek, v buketik plat'ja\\
she with pleasure pere.sew.\glb{pst.sg.f} on girls her light in little flowers.dim in wreath.dim in bouquet.of.flowers.dim dresses\\
\trans `she took pleasure in resewing her light dresses with prints of little flowers, wreathes and bouquets for girls'\\\Source{Ljudmila Ulickaja. \textit{Kazus Kukockogo} (2000)}
\bg.\label{ex:pereshit:other}A barin-to byl v pot\"{e}rtom pal'ti\v{s}ke, pere\v{s}itom iz soldatskoj \v{s}ineli\\
but barin-that was in shabby coat.dim pere.sew.\glb{part.pst.sg.m.prp} from soldier greatcoat\\
\trans `And the barin himself was in a shabby coat resewn from a greatcoat of a soldier'\Source{V. P. Kataev. \textit{Almaznyj moj venec} (1975--1977)}

It is also possible to utter the verb \textit{pere\v{s}it'} `to resew' when one piece of clothes is transformed into the other, as in the example \ref{ex:pereshit:other}, where the coat that comes into existence as a result of the resewing event is no longer a greatcoat it used to be. This points to the fact that the scale is not necessarily bound to the type of the object sewn in case of the verb \textit{\v{s}it'} `to sew'. In such cases, however, the mismatch has to be explicitly specified. E.g., it is not possible to understand the sentence \ref{ex:pereshit:same} as an event after which some other clothes, not dresses, come into existence. What has to be the same even if the type of the clothes sewn in the process of resewing is the material, so the scale of completeness associated with the sewn piece of clothes is also related to the material used in the sewing.

%It has to include information about the event X that follows some other event Y. If it is denied that the event X that follows the event Y took place, there is no denial nor assertion that the event Y took place. 

One more remark that I want to add before we proceed to the distributive usage of the prefix \textit{pere-} is that the repetitive usage is more frequent and flexible then it may seem. Even in some cases when the attachment of the repetitive \textit{pere-} seems impossible, as for the verb \textit{napisat'} `to write down', it is occasionally produced by native speakers when they are in need of expressing the relevant meaning, as illustrated by \ref{ex:perenapisat}. 

\exg.\label{ex:perenapisat}Mog by kto-to perenapisat' \`{e}tu programmu, no tol'ko v si?\\
can would someone pere.na.write.\glb{inf} this program but only in C\\
\trans `Could someone reprogram this in C?'\Source{\url{www.cyberforum.ru}}

Usually the verb \textit{perepisat'} `to copy/rewrite' can be used to refer to the rewriting, but it means either copying or rewriting and correcting something that already exists. The semantics of the verb \textit{perepisat'} `to copy/rewrite' includes bounding the activity denoted by the verb \textit{pisat'} `to write' and relating it to another writing event that proceeds along the same scale. Now if we consider the attachment of the \textit{pere-} prefix in its repetitive usage to the verb \textit{napisat'} `to write down', the derived verb would be able to denote not only copying and rewriting something that turned out to be not good enough (for this, there is a morphologically simpler alternative -- the verb \textit{perepisat'} `to copy/rewrite'), but also creating something written again. This meaning is derived from `to create something written' interpretation of the verb \textit{napisat'}. This interpretation cannot be obtained by simply bounding the activity denoted by the verb \textit{pisat'} `to write'. Thus the verb \textit{perepisat'} `to copy/rewrite' cannot be used in context like \ref{ex:perenapisat}, where not only the writing per se has to be performed, but also the thinking and creating the structure of the code has to be redone to make the program function in the other language.

One more aspect that is related to the repetitive usage of the prefix \textit{pere-} is the realization of the requirement for the presence of a closed scale in the event structure. If \textit{pere-} is attached to a perfective verb or to a secondary imperfective verb, this requirement is automatically satisfied. Complications occur when the derivational base is a basic imperfective verb, such as \textit{\v{c}itat'} `to read'. As long as the derivational base refers to an unbounded event, the mechanism of constructing the repetitive meaning, described above, cannot be applied: there is no result state that can be annulled to license the repetitive interpretation as neither the final nor the initial stage of the event is defined. A way out in this case is to allow coercion that will select a scale using the context (e.g., a scale associated with the direct object) and map the beginning of the event onto the minimum point of this scale and the end of the event onto some other point on the same scale. (Note that a possible way to do this is to leave the scale underspecified by using a variable to identify it and provide the mapping that will be supplied with values later when the semantic representations of the arguments of the verb become available.)

\subsubsection{Distributive usage}
The last usage of the prefix \textit{pere-} that we are going to explore is distributive. We have already discussed the distributive usage of the prefix \textit{po-} in Section~\ref{subsection:semantics:po}, so let us compare them, considering the examples \ref{ex:distr:pere} and \ref{ex:distr:po}.

\ex.\ag.\label{ex:distr:pere}Ira pere\v{c}itala vse knigi v biblioteke.\\
Ira pere.read.\glb{pst.sg.f} all books in library\\
\trans `Ira read all the books in the library.'
\bg.\label{ex:distr:po}Ira po\v{c}itala vse knigi v biblioteke.\\
Ira po.read.\glb{pst.sg.f} all books in library\\
\trans `Ira read from all the books in the library.'

Two main differences can be spotted between the situations that the sentences \ref{ex:distr:pere} and \ref{ex:distr:po} can refer to:
\begin{enumerate}
\item when the reading event is referred to by the verb \textit{pere\v{c}itat'} `to read all of', events of reading single books are clearly individualized;
\item \ref{ex:distr:pere} denotes an event of reading all the books through, whereas \ref{ex:distr:po} is compatible with the situation of reading only certain portions of every book.
\end{enumerate}

The first difference can be addressed by saying that the prefix \textit{pere-} requires a proper cardinality scale as an input, whereas the prefix \textit{po-} does not impose such a requirement. Let me explain this in more detail. A natural form of representation of plural individualized objects is a set. When we deal with a \textit{po-}prefixed verb, we describe the event as happening with all the objects in this set by starting the event when zero objects have been affected and ending it when all the objects have been affected. This is achieved by using the measure of change scale on which the cardinality of the set corresponds to the maximum point but there is no mapping between the subsets of the objects and the intermediate points on the scale.

If we choose to describe the event using the \textit{pere-}prefixed verb, such structure is not sufficient and a proper scale that fixes not only the extreme points, but also all the intermediate points on the scale, is needed. It is important that the subevents do not overlap when the situation is described with the \textit{pere-}prefixed verb. For example, if Misha had five balloons and made them burst one by one, both \ref{ex:balloon:pere} and \ref{ex:balloon:po} can be used. If he was jumping on the balloons and each landing made some balloons burst (e.g., with his first jump he destroyed two balloons, then one, and then another two), then only the description \ref{ex:balloon:po} is suitable.

\ex.\ag.\label{ex:balloon:pere}Mi\v{s}a perelopal vse \v{s}ary.\\
Mi\v{s}a pere.burst.\glb{pst.sg.m} all ballon.\glb{pl.acc}\\
\trans `Misha bursted all the ballons (one by one).'
\bg.\label{ex:balloon:po}Mi\v{s}a polopal vse \v{s}ary.\\
Mi\v{s}a po.burst.\glb{pst.sg.m} all balloon.\glb{pl.acc}\\
\trans `Misha bursted all the ballons.'

The difference in the requirements of the \textit{pere-} and \textit{po-}prefixed verbs is also revealed when the direct object is a mass noun: in such a case, only \textit{po-}prefixed verbs can be interpreted distributively, as \ref{ex:po:merz}, and \textit{pere-}prefixed verbs need to acquire some other interpretation, as in \ref{ex:pere:merz}, where the verb \textit{perem\"{e}rz} `to freeze' is interpreted excessively. I explain this by a lack of a mechanism that would extract a proper scale from a cumulative description.

\ex.\ag.\label{ex:po:merz}Pom\"{e}rzla karto\v{s}ka-to u nas none, vsja pom\"{e}rzla.\\
po.freeze.\glb{pst.sg.f} potato-that at our now all po.freeze.\glb{pst.sg.f}\\
\trans `Our potato plants got frozen now, all of them.'\\\Source{V. G. Korolenko. \textit{\v{C}udnaja} (1880)}
\bg.\label{ex:pere:merz}Minuv\v{s}aja zima byla o\v{c}en' surovoj, i u mnogix uro\v{z}aj perem\"{e}rz v ovo\v{s}\v{c}exranili\v{s}\v{c}ax.\\
last winter was very severe and at many harvest pere.freeze.\glb{pst.sg.m} in vegetable.store\\
\trans `Last winter was very severe and many people lost there harvest in the vegetable stores as it was frozen.'\Source{\url{www.molsib.info}}

Another condition that has to be observed in order to obtain the distributive interpretation is that performing the action denoted by the derivational base with all the objects that are ordered to form a scale is only possible if every subevent (performing the action with a particular object) is somehow limited. (This is similar to what we have discussed about the repetitive usage of the prefix \textit{pere-}.) In other words, in order to map the whole event denoted by the distributive \textit{pere-}prefixed verb onto the time scale and ensure that the subevents do not overlap, we need to know not only the order of the subevents (determined according to the order acquired when a proper scale is constructed), but also the duration of each subevent. I propose to use the coersion mechanism in this case to delimit individual subevents if the derivational base is a simplex imperfective verb.

Another point that has to be mentioned with respect to the distributive usage of the prefix \textit{pere-} is that it cannot arise when the prefix is attached to a perfective verb. This has been noticed by \citet{Tatevosov:09}, who identifies this usage of the prefix as selectionally limited. Indeed, when we try to attach the prefix \textit{pere-} to a perfective verb, we obtain the verb with the repetitive and not the distributive interpretation: prefixing the verb \textit{otkryt'} `to open' provides us with the verb \textit{pereotkryt'} `to open again', prefixing the verb \textit{zapisat'} `to write down/to record' leads to the verb \textit{perezapisat'} `to write down anew/rerecord', but not `to write down/record all of'. This is natural given how the semantic structure of the perfective verbs is organized according to the view I propose.

Let us consider the verb \textit{zapisat'} `to write down/to record'. In its semantic structure this verbs carries information that the start of the writing event is related to the minimum point of the scale contributed by the direct object. The end of the event is related to the maximum point on the same scale. It is a scale of the measure of change type and the maximum of this scale is either the length of the direct object, if it is singular, or the number of objects, if the direct object is plural. What it cannot be is the length of one object belonging to the set denoted by the plural direct object. And if the distributive \textit{pere-} were added to the verb, this is exactly what had to be denoted by the embedded event. This is easier to see by looking at the formal representations (see Chapter~\ref{Chapter6}).

Another approach is offered by \citet{Demjjanow:97} who suggests that the distributive interpretation of the prefix \textit{pere-} should share the prefix schema with the repetitive interpretation. This is motivated by the idea that verbs prefixed with the distributive \textit{pere-} trigger presuppositions (similarly to the verbs prefixed with the repetitive \textit{pere-}). As an example, \citet{Demjjanow:97} provides the sentence \ref{ex:demj} that she claims to mean that some of the candles were blown out.

\exg.\label{ex:demj}On ne peretu\v{s}il vse sve\v{c}i.\\
he not pere.blow.out.\glb{pst.sg.m} all candles\\
\trans `He did not blow out all the candles.'\\\Source{= example (153) in \citealt[120]{Demjjanow:97}}

As the presuppositional view on the repetitive usage of the prefix \textit{pere-} will be discussed in Chapter~\ref{Chapter6}, here I only want show that it is not required that any part of the action denoted by the distributive \textit{pere-}prefixed verb was performed if such verb is uttered under negation. Indeed, the most natural interpretation of \ref{ex:perelistat} is that the editor (Panferov) did not look through any part of the manuscript.

\exg.\label{ex:perelistat}Pridja v redakciju ``Oktjabrja'', Juz polo\v{z}il pered Panferovym tolstuju rukopis', i tot, da\v{z}e ne perelistav, napisal na nej: ``V nabor''.\\
come.\glb{part} in {editorial office} Oktjabr'.\glb{gen} Juz po.lay.\glb{pst.sg.m} {in front} Panferov.\glb{inst} thick manuscript and that even not pere.thumb.\glb{part.pst}, na.write.\glb{pst.sg.m} on she.\glb{prp} in 	print\\
\vspace{0.5em}
When Juz came to the editorial office of Oktjabr' and laid a thick manuscript in from of Panferov, Panferov, without even thumbing through it, wrote on it: ``To print.''\Source{Samuil Ale\v{s}in. \textit{Vstre\v{c}i na gre\v{s}noj zemle} (2001)}



%A possible explanation of such behavior is a suggestion that distributive \textit{pere-} is attached a lot easier in case there is already some iteration in the denotation of the derivational base. In this case what \textit{pere-} does is not introducing the iteration, but imposing an order (and this is the main difference between the distributive \textit{po-} and distributive \textit{pere-}) and delimiting the event (by iterating through all of the members of the set denoted by the direct object).

\subsection{Restrictions on the attachment}
I claim that all the usages discussed above except for the repetitive one (but including the distributive), can be unified using the idea that \textit{pere-} can be only attached to a scale that is closed and non-binary. In other words, the scale that \textit{pere-} selects for must contain at least three distinct points. Along with this strong requirement (in comparison with other prefixes) there are several ways to construct an appropriate scale and this explains the polysemous nature of the prefix. 

Let the two extreme points on the scale $s$ that is provided as an input for the prefixation with \textit{pere-} be $x$ and $z$ and the set $Y$ be the set of all intermediate points $y$ such that $\forall y \subset Y:  x < y < z$. All the intermediate points must be ordered as well. The prefix requires that $Y$ is not empty. This corresponds to a Complex type in \citealt{Beavers:12} (44c).\footnote{In the earlier work, \citealt{Beavers:02} and \citealt{Beavers:08}, the notion of Non-Minimally Complex Object is used.} I propose the following general procedure for acquiring a scale that \textit{pere-} can attach to. 
\begin{enumerate}
\item If the direct object provides a closed scale that is non-binary, $x$ is the minimum of this scale, $z$ is the maximum and $Y$ is the set of all the intermediate points.\footnote{Note that extracting a \textit{path} scale from the direct object that refers to some landmark is also a complex process, as the \textit{path} scale is not present in the semantic structure of the object, but has to be constructed taking into account the position of the subject.}
\item If the direct object (possibly in combination with the context) provides a single point on some scale, this point becomes a member of the set $Y$. The points $x$ and $z$ are chosen arbitrarily in such a way that they are located respectively below and above the marked point on the scale. 
\item If the direct object denotes a set, the scale is constructed by arranging the equivalence classes corresponding to the gradually increasing number of objects. $x$ is 0, $z$ is the cardinality of the set, and $Y$ contains points that represent subevents related to the subsets consisting of a whole number of objects in the set (the first point in Y is an equivalence class of all single objects in the set, the second point is the equivalence class of all pairs of objects, the third point is the equivalence class of all triplets, etc.).
%\item If the direct object provides measure of change information and there is a way the event can proceed along the same scale again, the scale is acquired by assuming that $x$ is some point on the scale that reflects the state of the world before the event, the set $Y$ contains one point that is the zero point of the measure of change scale, and $z$ is the maximum point of the measure of change scale. Note that this new constructed scale is such that on its first interval from from $x$ to $y$ it is the reverse of the measure of change scale, which means that the first transition is the decrease on the measure of change scale of the source event.
\end{enumerate}

%The third step is to convert this scale with one marked point into a closed scale. As the scale we are dealing with does not have two distinguished minimum and maximum points, the reasonable way to select points that will be related to the event start and end is to pick two arbitrary values around the fixed comparison point. 

The motivation behind such scale selection is the idea that when \textit{pere-} is attached to a verb, the action denoted by that verb has to be performed at all the intermediate points on the relevant scale and each point on that scale has to correspond to some subevent. If the scale is dense (first case described above), as with \textit{time} and \textit{path} scales, this will mean performing the action while moving along the scale. If the scale is discrete (third case), as with the \textit{cardinality} type of the scale, the verb prefixed with \textit{pere-} acquires distributive interpretation. 

The mapping that is done by the attachment of \textit{pere-} is then the following. %If $Y$ contains a single point $y$, the event consists of two phases: the preparatory phase and the main phase. The preparatory phase of the event starts when the value on the scale $s$ equals $x$ and ends when it is equal $y$. The event itself starts when the value on the scale $s$ is equal $y$ and ends when it is equal $z$. 
If $Y$ contains multiple points, the event consists of the iteration of the event denoted by the derivational base for each point on the scale until the point $z$ is reached. Each individual event is measured according to the measure of change scale of the corresponding element.

If $Y$ contains a single point or an infinite number of points, the event proceeds along the scale $s$ from $x$ to $z$ through all the points in $Y$. This mapping can be unified with the previous one (for multiple points) if the continuous movement along the scale is represented as iteration of movement through the infinite number of points on the closed scale. I do not think that this is computationally reasonable and prefer to have two separate representations for the implementation.

The process of scale selection I propose does not rely on the semantics of the verbal roots and it is even independent of the scale dimension. For example, usually those verbs that lexicalize \textit{path} and \textit{time} scales acquire the crossing semantics that relies on traversing all the points on the scale (related to the scale of the type 1 in the list above). But they can also acquire the interpretation using the same mechanism as is used for the excess meaning (second procedure in the list above). This happens when the direct object denotes something that is conceptualized as having point-like width or point-like duration. In the case of point-like width, unlike the case of non point-like width, the crossing event has to start in front of the crossed object and end behind it and not on its border.

For example, the phrase \ref{ex:perejti} cannot be uttered in the situation when someone steps over the puddle on their way, they have to step in the puddle at least once and at the same time it is enough that the actor crosses the puddle with the last step on the border of the puddle and not outside it. If the crossed object is conceptualized as being point-like, then the event necessarily starts and ends on the different sides of the object: in this case, stepping over the same puddle can be described by \ref{ex:pereshagnut} and the end point of the motion cannot be in the puddle.

%Remark: in case when motion consists of smaller events, the path consists of those points where a subevent either begins or ends:  in the situation when there are grey and red pavement tiles the event of walking when stepping on the red tiles only can be referred to `idti po krasnym plitkam.'

\ex.\ag.\label{ex:perejti}perejti lu\v{z}u\\
pere.go.\glb{inf} puddle\\
\trans `to cross the paddle'
\bg.\label{ex:pereshagnut}pere\v{s}agnut' lu\v{z}u\\
pere.step.\glb{inf} puddle\\
\trans `to step over the puddle'

This approach accounts for the ambiguity allowed in the analysis of \citet{Kagan:book} simply by the absence of the proper upper inclusion constraint: verbs that acquire \textit{path}- and \textit{time}-related semantics denote events the measure of which can be equal to the measure that is contributed by the direct object or can exceed it. The analysis I offer here allows to disentangle these possibilities a bit further though still maintaining the idea of the underlying uniform semantics of the prefix.

The other two usages, that of excess and comparison, are related to the scale constructed according to the second procedure in the list above. These usages are also guided by the same idea of proceeding through some values on the scale. In these cases, only the marked point is important and it is the only point through which the event has to proceed. The event starts when the value on the scale is below the marked point, proceeds through this point and ends when the value on the scale is above it. This accounts for such examples as \ref{ex:pere:excess}, \ref{ex:pere:comparison}, and \ref{ex:Kagan:perezhit}.

The case of the repetitive meaning of the prefix (`again') is not unified naturally with the other cases. First, it is the only case where a separate preparatory phase has to be created. Second, it is widely available, often simultaneously with other interpretations, and such \textit{pere-}prefixed verbs seem to be disambiguated only by the context. So despite the fact that the repetitive meaning had received a unified account with the other interpretations of the prefix \textit{pere-} in some earlier works \citep{Demjjanow:97, Kagan:book}, I will set is aside.

% the points on the scale are states and the scale contains three elements: the state that is associated with a non-zero point on the \textit{measure of change} scale, the state that corresponds to the zero point on the \textit{measure of change} scale, and the state that corresponds to the maximum point of the \textit{measure of change} scale. This leads to the interpretation that at the start point of the event the state of the world is such that it corresponds to the non-minumum value on the \textit{measure of change} scale (this follows from the inequality of $x$ and $y$). Then the preparatory phase of the event consists of setting this value to the minimum. The main phase of the event proceeds like the regular event that is related to the \textit{measure of change} scale. 

%Th idea of locating a value that corresponds to the non-minimum point of the \textit{measure of change} scale below the point that corresponds to the minimum on that scale is supposed to reflect the fact that despite some effort has been done in order to move along this scale, the situation is not better than it is 

%In sum, what we have shown is that for those cases that require proper upper inclusion instead of just upper inclusion there is an external motivation behind this requirement. This does not entail that we have to accept uniform semantic representations for those usages of the prefix \textit{pere-} and run the reasoning sketched above for each vary with comparison and excess meaning. This explanation may be considered as the reason why one the prefix became polysemous, how this polysemy is motivated. For the synchronous semantic representation the approach offered by \citet{Kagan:book} works fine as various usages of \textit{pere-} receive minimally different semantics and this is just enough to encapsulate the variability. What I have added to this so far is a reason for this slight change of semantics that gave rise to comparison and excess usages of \textit{pere-}. 

The approach presented here allows us to ``move'' most of the differences between the different uses of \textit{pere-} in the domain of scale selection. An important property of such an approach is that various meanings arise as a result of different properties of the scales lexicalized by verbs or contributed by the direct objects. So this formalizes the intuition that the particular meaning of \textit{pere-}prefixed verb can be only determined in the context (and the direct object plays the crucial role in it). 

As we have seen, the prefix \textit{pere-} is both very demanding and very flexible: in order to be attached, it requires a closed not two-point scale on which all the intermediate points can be mapped onto sub-events, but there are various mechanisms that can be used to obtain this scale. Moreover, it does not impose any restrictions on the dimension of the scale: as \citet[151]{Kagan:book} summarizes, \textit{pere-} can apply to ``all scale dimensions that are familiar from the literature on verbal domain''. So depending on the type of the scale available, one or several interpretations are possible for the verbs derived by the attachment of the prefix \textit{pere-} to any derivational base. I will provide various examples in Chapter~\ref{Chapter7}.

\subsection{Subsequent imperfectivization of a verb with the discussed prefix}
Secondary imperfective formation is allowed with all the usages of the prefix \textit{pere-}: crossing, waiting, excess, comparison, distributive, and repetitive semantics.

Examples \ref{ex:pere:imp:space1} and \ref{ex:pere:imp:space2} illustrate the usage of the secondary imperfective verbs \textit{perebegat'}$^{\IPF}$ `to run/be running across' and \textit{perepl\"{e}vyvat'}$^{\IPF}$ `to spit/be spitting over something' formed from the \textit{pere-}prefixed verbs \textit{perebe\v{z}at'}$^{\PF}$ `to run across' (see Section~\ref{subsection:perf:motion} for more details about why I consider the verb \textit{pe\-re\-be\-gat'}$^{\IPF}$ `to run/be running across' to not be derived from the verb \textit{begat'}$^{\IPF}$ `to run' via prefixation) and \textit{perepljunut'}$^{\PF}$ `to spit over something'. This provides evidence for the existence of the secondary imperfective verbs derived from \textit{pere-}prefixed verbs with crossing semantics.

\ex.\ag.\label{ex:pere:imp:space1}I ot ka\v{z}doj pary valenok, kto v lagere gde \v{s}\"{e}l ili perebegal, -- skrip.\\
and from each pair {felt boots} who in camp where go.\glb{pst.sg.m} or pere.run.\glb{pst.sg.m} {} creak\\
\trans `And each pair of boots when someone in the colony went or run somewhere produced a creak.'\\\Source{Aleksandr Sol\v{z}enicyn. \textit{Odin den' Ivana Denisovi\v{c}a} (1961)}
\bg.\label{ex:pere:imp:space2}Byl skup na slova. Ele pereplevyval \v{c}erez vyvoro\v{c}ennye guby.\\
be.\glb{pst.sg.m} stingy on words barely pere.spit.imp.\glb{pst.sg.m} over vy.turned lips\\
\trans `He was stingy on words. Barely spat them over his everted lips.'\\\Source{R. B. Gul'. \textit{Azef} (1958)}

Sentences \ref{ex:pere:imp:wait}, \ref{ex:pere:imp:excess}, \ref{ex:pere:imp:compar}, and \ref{ex:pere:imp:iter} serve as an evidence for the existence of secondary imperfectives formed from \textit{pere-}prefixed verbs with waiting (\textit{pere\v{z}dat'} `to pass time waiting for something to end' $\rightarrow$ \textit{pere\v{z}idat'}  `to pass/be passing time waiting for something to end'), excess (\textit{peregret'} `to overheat' $\rightarrow$ \textit{peregrevat'} `to overheat/be overheating'), comparison (\textit{perepljunut'$^{\PF}$} `to surpass' $\rightarrow$ \textit{perepl\"{e}vyvat'$^{\IPF}$} `to surpass/be surpassing'), distributive(\textit{perepisat''$^{\PF}$} `to list all of' $\rightarrow$ \textit{perepisyvat'$^{\IPF}$} `to be listing all of'), and repetitive (\textit{perepisat'} `to rewrite' $\rightarrow$ \textit{perepisyvat'} `to rewrite/be rewriting') semantics, respectively.

\exg.\label{ex:pere:imp:wait}Pravda, na zimu ona ostanavlivaetsja v roste, no ne obrazuet nastoja\v{s}\v{c}ix po\v{c}ek, a li\v{s}' pere\v{z}idaet zimnee poxolodanie.\\
truth on winter she stop.\glb{pres.sg.3}.refl in growth but not form.\glb{pres.sg.3} real burgeon but only pere.wait.imp.\glb{pres.sg.3} winter cooling\\
\trans `It does, in fact, stop to grow for the winter time, but does not form real burgeons, only waits for the cool winter period to pass.'\\\Source{Ju. N. Karpun. \textit{Priroda rajona So\v{c}i} (1997)}

\exg.\label{ex:pere:imp:excess}Inogda na rynke popadaetsja \v{z}idkij med, kotoryj prodavcy special'no peregrevajut, \v{c}toby ostanovit' bro\v{z}enie.\\
sometimes on market po.fall.\glb{pres.sg.3}.refl liquid honey that seller.\glb{pl.nom} intentionally pere.heat.imp.\glb{pres.pl.3} that stop.\glb{inf} fermentation\\
`Sometimes liquid honey can be found on the market; it is overheated on purpose by the sellers to stop fermentation processes.'\\\Source{Vladimir \v{S}\v{c}erbakov. \textit{``Pravil'nyj'' med }(2002)}

\exg.\label{ex:pere:imp:compar}Da u\v{z}, puskaj lu\v{c}\v{s}e v vese i roste nas mal'\v{c}iki-sentjabriki perepl\"{e}vyvajut.\\
yes well let better in weight and height us boys-september.ik.\glb{pl.nom} pere.spit.\glb{pres.pl.3}\\
\trans `Oh well, I'll better let those September-born boys to overtake us in weight and height.'\Source{Na\v{s}i deti: Maly\v{s}i do goda (forum) (2004)}

\exg.\label{ex:pare:imp:distr}Kogda inspektor Mykomel' perepisyval vsex passa\v{z}irov, ona nazvalas' Melodiej Dz'ujn.\\
when inspector.\glb{sg.nom} Mokomel pere.write.imp.\glb{pst.sg.m} all.\glb{acc} passenger.\glb{pl.gen} she na.name.\glb{pst.sg.f}.refl Melody Dzujn\\
`When Inspector Mukomel was writing down the list of all the passengers, she named herself Melody Dzujn.'\\\Source{Vadim Rossik. \textit{T\"{e}mnyj \v{c}elovek} (2015)}

\exg.\label{ex:pere:imp:iter}Vmesto togo \v{c}toby ka\v{z}dyj raz perepisyvat' istoriju, razumnee prinjat' e\"{e} takoj, kakoj ona vyjasnjaetsja sama.\\
instead that that each time pere.write.imp.\glb{inf} history.\glb{sg.acc} rational.\glb{comp} accept her that as she vy.clear.\glb{pres.sg.3}.refl herself\\
\trans `Instead of rewriting the history each time, it is more rational to accept it as it turns out to be.'\\\Source{\`{E}duard Limonov. \textit{U nas byla Velikaja \`{E}poxa} (1987)}

%The question that remains open is why verbs that contain the prefix \textit{pere-} and have a distributive interpretation cannot be imperfectivized. This seems to contradict the predictions of \citet{Tatevosov:09}, who lists the distributive prefix \textit{pere-} among selectionally limited prefixes for which the derivation of the form \textit{basic imperfective verb} $\rightarrow$ \textit{prefixed perfective verb} $\rightarrow$ \textit{secondary imperfective verb} is not excluded on the syntactic grounds. There seem to be no clear semantic reasons motivating the absence of the secondary imperfective for such verbs as \textit{perele\v{c}it'} `to heal all of'. The semantics that derived imperfective verb could have is conceivable: it could be either habitual healing of all the patients or being in the process of the healing all the patients. 
%
%The only explanation line I can provide at this point is that the iteration that ``wraps'' the individual events is not compatible with the secondary imperfective attachment because the imperfective suffix requires another type of the input. On the computational part, this will fall out automatically from the representations, as I represent the verbs that contain the distributive \textit{pere-} by means of the different, two-layered structure with the outer layer being responsible for the iteration of subevents. Those verbs will then denote events of a type different from the type of events without explicit iteration. The question whether this is a coincidence or it is indeed the iteration that makes distributive verbs incompatible with the imperfective suffix, is left for future research.

\subsection{Summary}
As has been shown by \citet{Kagan:book}, various usages of \textit{pere-} that seem to be unrelated at first sight can be unified under a scalar account for prefixation. We have gone a little bit further and shown that some of the differences between the usages that are present in the account by \citet{Kagan:book} can be motivated by the properties of the input scale. The available scales may be provided by the direct object, world knowledge, context, or the verb itself. I have proposed a mechanism that takes as an input scales of various types and (depending on the properties of a concrete scale) provides as an output a scale suitable as an input to the prefixation by \textit{pere-}. One of the interpretations of the prefix that arises as a result of applying the proposed system is the distributive usage of \textit{pere-}, that has previously not been unified with other interpretations. The scale selection process that leads to various interpretations of the prefix ends up being motivated by the requirement that the prefix has to receive as an input a non-binary scale. The notorious polysemy of the prefix \textit{pere-} arises due to the availability of the different ways to satisfy this requirement. 

On the other hand, I have decided to exclude the repetitive interpretation of the prefix \textit{pere-} from being integrated in the system described above. At the moment, I do not see a natural way of unifying the repetitive meaning of the prefix with the other interpretations, as it has several distinctive properties. First, it includes a preparatory phase (presupposition on the accounts of \citealt{Demjjanow:97}, \citealt{Kagan:book}, more details in Chapter~\ref{Chapter6}), that is not present in other usages. Second, it is compatible with a binary scale as an input for the prefixation. Third, the attachment of the repetitive \textit{pere-} to a non-basic imperfective or biaspectual verb does not lead to the change of aspect (see Section~\ref{section:new:perfectivity} for more details). These facts allow to consider analyzing the repetitive prefix \textit{pere-} and the prefix \textit{pere-} that may acquire all the other meanings described here as being homonyms. This hypothesis, however, needs to be tested further.

Despite all the work towards the unification of the usages of \textit{pere-}, for the computational analysis I propose to allow three different representations that will be responsible for various types of the mapping different scales require. Remember, this mapping is always motivated by the idea of performing the action denoted by the derivational base at all the intermediate points of the scale. 

The basic representation should account for spatial (``crossing''), time (``waiting'') , and distributive usages in cases of closed scales. The prefix in this case establishes the mapping between all the points on the scale and distinct event stages. The second representation serves cases when there is only one marked point on the relevant scale. In this case the event proceeds from some point below the marked point through this point to the point above it. The last representation is needed for the repetitive usage: it takes the event denoted by the source verb, creates a copy of it, and constructs a new event (from the copy) that has the old one as the preparatory phase.

\section{\textit{do-}}\label{subsection:semantics:do}
\subsection{Semantic contribution}
Let us again start by looking up the characterizations of the verbs derived with the prefix in question (now \textit{do-}) in the grammar by \citet[357--358]{Shvedova:82}. Three possible interpretations of the derived verbs are listed there:
\begin{enumerate}
\item to perform the action denoted by the derivational base until the end or until some limit (productive type): \textit{dovarit'} `to finish cooking';
\item to perform the action denoted by the derivational base in addition to something, or in order to reach a certain norm (productive type): \textit{doplatit'} `to pay in addition';
\item to lead to an undesirable condition by performing the action denoted by the derivational base (productive in colloquial speech): \textit{dole\v{c}it'} `to cure incorrectly, causing a serious illness'.
\end{enumerate}

As we see, \textit{do-} is not a highly polysemous prefix. Nevertheless, \textit{do-} is very interesting concerning the prefix stacking phenomena as it is very productive and can lead to the formation of biaspectual verbs, as we have discussed in Section~\ref{section:new:biaspectual}. 

\citet[70]{Kagan:book} characterizes the prefix \textit{do-} as relating ``the standard of comparison to the degree that is achieved at the endpoint of an event''. \citet{Kagan:book} identifies this prefix as delimitative and distinguishes between the terminative and additive  usages. The terminative usage corresponds to the first and the additive usage corresponds to the second usage in the provided above list by \citet{Shvedova:82}. My primary goal is to study the terminative usage. \citet[72]{Kagan:book} describes the semantics of the terminative usage of the prefix \textit{do-} in the following way: ``The prefix introduces the relation of identity between two degrees. It applies to a gradable property an increase along which is entailed by the predicate.''

A simple illustration is provided by \ref{ex:do:varit}. The verb \textit{varit'} `to cook' lexicalizes a scale with the maximum point corresponding to \textit{fully cooked} and the prefix \textit{do-} contributes information that at the end of the event this point is reached. 

\exg.\label{ex:do:varit}Liza dovarila sup.\\
Liza do.cook.\glb{pst.sg.f} soup\\
\trans `Liza finished cooking the soup.'

What is important is that \ref{ex:do:varit} normally refers to an event of cooking the soup that starts not from scratch. It may be the case that the soup was almost ready but Liza had to pause cooking and answer a phone call before finishing cooking. It can also be the case that John was cooking the soup, considered it cooked, and left it for Liza. Liza came later, tasted the soup and realized it is not ready, and then had to do some additional cooking to make the soup really cooked. The second interpretation falls under the additive usage of the prefix. However, it does not represent a special case different from the first usage in terms of scalar semantics: in both cases, the event the \textit{do-}prefixed verb refers to proceeds along the relevant scale from some point $x$ until the scale's maximum. The difference between the prefix \textit{do-} and other prefixes is that $x$ does not have to be the minimum point on the relevant scale. It can also be the case that there is no minimum point on the relevant scale at all. For example, the event of heating the soup proceeds along the temperature scale and the start of the event is associated with some temperature of the soup that cannot be easily reconstructed, but is definitely not equal to the minimum of the scale. From the fact that such sentence as \ref{ex:do:PP} normally refers to the whole event of heating the soup up to the boiling point it follows that the condition I have formulated above seems to work well in such case. A stronger requirement (for the presence of another event associated with the temperature increase) would be superficial.

\exg.\label{ex:do:PP}Liza dovela sup do kipenija.\\
Liza do.lead.\glb{pst.sg.f} soup until boiling\\
\trans `Liza made the soup boil.'

\citet[75]{Kagan:book} claims that the semantics of the terminative \textit{do-} ``can be divided into an entailed and a presupposed part''. The observation provided above seems to speak against such formulation of the additional inference associated with the prefix \textit{do-}. The sentence \ref{ex:do:PP:ne} can be successfully uttered in a situation when Liza did not heat the soup at all. We will discuss this topic further in the next chapter.

\exg.\label{ex:do:PP:ne}Liza ne dovela sup do kipenija.\\
Liza not do.lead.\glb{pst.sg.f} soup until boiling\\
\trans `Liza did not make the soup boil.'

Although the additive \textit{do-} is not in the focus of this thesis, I would like to add some remarks about it, as these remarks contribute to the overall picture of pragmatic competition between the different prefixes. \citet[79]{Kagan:book} points out that the main difference between the terminative and the additive interpretations is that in the first case the presupposed and the entailed events are viewed as constituting one event and in the second case they are viewed as two separate events. What usually comes along with this distinction is that in the first case the degree on the measure of change scale that has to be reached in the end is specified, whereas in the second case what is linguistically supplied is the measure of change of the second event and the cumulative standard that has to be reached in the end can be left implicit. \citet[p 79]{Kagan:book} provides the examples repeated under \ref{ex:Kagan:dospat} to illustrate the differences between these usages.

\ex.\label{ex:Kagan:dospat}\ag.\label{ex:Kagan:dospat:1}(Ivan l\"{e}g pospat'.) On dospal do poluno\v{c}i.\\
Ivan lay po-sleep he do-slept do midnight\\
\trans `Ivan went to bed. He slept till midnight.'
\bg.\label{ex:Kagan:dospat:2}(Ivan za no\v{c} ne vyspalsja.) Potom on dospal paru \v{c}asov.\\
Ivan in night NEG vy-slept-refl then he do-slept couple hours\\
\trans `Ivan hadn't had enough sleep during the night. He then slept for a couple more hours.'\Source{= (12) in \citet[79]{Kagan:book}}

In the first case (example \ref{ex:Kagan:dospat:1}, terminative usage) there is a single event of sleeping that lasts until midnight.\footnote{Note that as the first (bracketed) sentence refers only to the initiating the sleeping situation and does not even require the agent to fall asleep. This is clear due to the possibility to continue with the negation of the falling asleep fact, as in \ref{ex:dospat:no}.
\exg.\label{ex:dospat:no}Ivan l\"{e}g pospat'. On prole\v{z}al 3 \v{c}asa, no tak i ne smog usnut'.\\
Ivan lay po.sleep.\glb{inf} he pro.lay.\glb{pst.sg.m} 3 hours, but so and not able.\glb{pst.sg.m} fall.asleep.\glb{inf}\\
\trans `Ivan went to bed. He stayed in bed for 3 hours but did not manage to fall asleep.'

} In the second case, there was one sleeping event that proved to be not sufficient so there was a second event in course of which Ivan slept for 2 hours and thus cumulatively over two events reached the required amount of sleep. 

As \citet[80]{Kagan:book} points out, the first event in case of the additive usage of the prefix \textit{do-} can be of a different kind, as illustrated by the example \ref{ex:doplatit} that describes a situation when additional payment has to be made not after another payment, but after giving away empty bottles.

\exg.\label{ex:doplatit}Kupili dju\v{z}inu butylok fruktovoj vody, a v obmen sdali 8 pustyx butylok. Skol'ko deneg doplatili?\\
bought dozen bottles fruit water, but in exchange s.give.\glb{pst.pl} 8 empty bottles how.much money do.pay.\glb{pst.pl}\\
\trans `We have bought a dozen bottles of fruit water and gave away 8 empty bottles. How much money did we have to pay in addition?'\Source{\url{vcevce.ru}}

%Such examples allow to come to the conclusion that the interpretation of the prefix depends exclusively on whether the linguistically supplied measure of change is absolute (so if fixes the end point and the verb tends to be interpreted terminatively) or relative (and fixes the difference between the start and the end points of the main event, which leads to the additive interpretation). As for the division into two events, in the first case we know that the border is somewhere on the scale between the zero point (and may be at this point) and the supplied maximum point and in the second case we only know that the state of the world before the event start is such that when the value is augmented by the specified change, some desired standard is reached. 

Another example is provided under \ref{ex:buy:raisins}. The sentence \ref{ex:buy:raisins} does not exclude the state of the world in which the speaker never bought raisins, dried apricots, and/or plums before nor had he possessed any of those, it is just that he needed them in order to make stewed fruit. What the verb \textit{dokupit'} `to buy in addition' means in this case is that he bought the dried fruits but this was not the first step in gathering the ingredients for something he wanted to cook. The ``scale'' in this case includes possession of the necessary amount of raisins, dried apples, apricots, and plums. 

\exg.\label{ex:buy:raisins}Mne test' vydal su\v{s}\"{e}nyx jablok s da\v{c}i, ja dokupil izjuma, kuragi, \v{c}ernosliva i teper' reguljarno vspominaju detstvo -- varju kompot iz suxofruktov.\\
me father-in-law vy.give.\glb{pst.sg.m} dried apples from dacha I do.buy.\glb{pst.sg.m} raisins {dried apricots} {dried plums} and now regularly rememner childhood {} cook {stewed fruit} from {dried fruits}\\
\trans `My father-in-law gave me some dried apples from his dacha, I also\linebreak bought raisins, dried apricots and plums and now I regularly remember the childhood by making myself some stewed dried fruit.'\\\Source{\url{https://murmolka.com}}

Based on this observations, I propose to derive the inference of the event being an addition to something else being drawn in the process of the pragmatic competition between the \textit{do-}prefixed verb and other perfective verbs that can express the same literal meaning (in case of the example \ref{ex:buy:raisins} it would be the verb \textit{kupit'$^{\PF}$} `to buy'). The competition is triggered by the absence of the requirement that the starting point of the event has to be the minimum on the relevant scale in the semantic representation of the prefix \textit{do-} (unless it is overtly specified, as in \ref{ex:do:iz:do}, or the scale is of a measure of change type, as in \ref{ex:do:measure}). A broader pragmatic picture will be provided in the next chapter.

\exg.\label{ex:do:iz:do}Za \v{s}est' \v{c}asov mo\v{z}no doletet' iz N'ju-Jorka do San-Francisko.\\
behind six hours can do.fly.\glb{inf} from {New York} to San-Francisco.\\
\trans `In six hours one can get from New York to San-Francisco by plane.'\\\Source{Boris Levin. \textit{Inorodnoe telo} (1965--1994)}

\exg.\label{ex:do:measure}A na poljax nota bene -- takoj-to ne doplatil tri kopejki, vozmestit togda-to…\\
but on margins nota bene {} such-that not do.pay.\glb{pst.sg.m} three pennies, compensate.\glb{pres.sg.3} then-that\\
\trans `And on the margins there is a nota-bene: mister X failed to pay 3 pennies, will compensate on day Y.'\Source{Jurij Davydov. \textit{Sinie tjul'pany} (1988--1989)}

\subsection{Restrictions on the attachment}
\citet[236]{Kagan:12} points out that the prefix \textit{do-} in its terminative interpretation can apply to a variety of scales. Let me first illustrate this thesis with a poem by Ekaterina Starostina called \textit{Do\v{c}uvstvovat'} `To finish feeling' I found in the internet. This poem contains 13 \textit{do-}prefixed verbs in 12 lines (they are marked with bold font), whereby in 4 verbs \textit{do-} is not the only prefix.

\ex.\label{ex:poem}\a.\label{poem:a}\ag.[]$\ldots$\textbf{Do\v{c}uvstvovat'}. \textbf{Doo\v{s}\v{c}u\v{s}\v{c}at'}.\\
do.feel.\glb{inf} do.sense.\glb{inf}\\
\bg.[]\textbf{Dotronut'sja} ili kosnut'sja$\ldots$\\
do.touch.\glb{inf}.refl or touch.\glb{inf}.refl\\
\bg.[]\textbf{Dobyt'} tebja, \textbf{docelovat}'$\ldots$\\
do.be.glb{inf} you do.kiss.\glb{inf}\\
\bg.[]$\ldots$i polnym serdcem ulybnut'sja$\ldots$\\
and full heart smile.\glb{inf}.refl\\
\trans To finish feeling. To finish sensing.\\
To touch you slightly$\ldots$\\
To get you and finish kissing\\
$\ldots$and smile with the full heart$\ldots$
\z.
\b.\label{poem:b}\ag.[]\textbf{Dogladit'} pal'cy na rukax$\ldots$\\
do.caress.\glb{inf} fingers on hands\\
\bg.[]\textbf{Domno\v{z}it'} s\v{c}ast'e v na\v{s}ix du\v{s}ax.\\
do.multiply.\glb{inf} happiness in our souls\\
\bg.[]\textbf{Dopere\v{z}it'}, \textbf{dopere\v{z}dat'}$\ldots$\\
do.pere.live.\glb{inf}, do.pere.wait.\glb{inf}\\
\bg.[]\textbf{Dorazobrat'} vs\"{e} to, \v{c}to nu\v{z}no$\ldots$ \\
do.raz.take.\glb{inf} all that that needed\\
\trans To finish caressing the fingers$\ldots$\\
To multiply the joy in our souls.\\
To get over it, to wait till the end$\ldots$\\
To disassemble all we need$\ldots$
\z.
\b.\label{poem:c}\ag.[]\textbf{Dorazukra\v{s}ivat'} me\v{c}ty,\\
do.raz.u.paint.imp.\glb{inf} dreams\\
\bg.[]\textbf{Dobit'sja} srazu: vs\"{e} i mnogo$\ldots$\\
do.hit.\glb{inf}.refl {at once} all and {a lot}\\
\bg.[]I dobrym utrom do poroga\\
and kind morning until doorstep\\
\bg.[]\v{C}ut' zabludiv\v{s}ejsja \textbf{dojti}$\ldots$\\
slightly za.wander.\glb{part.act.pst}.refl do.go.\glb{inf}\\
\trans To finish coloring the dreams,\\
To get at once all that I wanted$\ldots$\\
And one good morning to the doorstep\\
To come being slightly strayed$\ldots$\\
\Source{Ekaterina Starostina, \textit{Do\v{c}uvstvovat'} (www.stihi.ru)}

In this poem we evidence the attachment of the prefix \textit{do-} to a scale of stages through which the event develops (e.g., \textit{do\v{c}uvstvovat'} `to finish feeling', \ref{poem:a}), to a path scale (e.g., \textit{dojti} `to get to', \ref{poem:c}), and to the time scale that either comes directly from the semantic structure of the verb (e.g., \textit{doo\v{s}\v{c}u\v{s}\v{c}at'} `to finish sensing', \ref{poem:a}) or is already used in course of the attachment of another prefix (e.g., \textit{dopere\v{z}it'} `to survive something', \ref{poem:b}). \citet{Kagan:book} proposes the following hierarchy of the sources for a scale the prefix \textit{do-} can attach to: 

\begin{itemize}
\item ``If the verbal stem lexicalizes a scale, it is to this scale that \textit{do-} will apply.''
\item ``If the verb itself does not contribute a scale, but it is an incremental
theme verb, then the prefix will apply to the scale introduced by the direct object (a volume/extent scale).''
\item ``If none of these conditions are satisfied, the prefix can apply to the time scale.''
\end{itemize}

\citet{Kagan:12} also notes that \textit{do-} can apply to both upper closed and open scales, but ``[i]f \textit{do-} applies to a scale that is not upper closed, and a \textit{do-}PP is absent, the context has to be sufficiently rich to determine what counts as the standard of comparison.'' I would like to provide one more illustration of this point for the latter of the three cases mentioned above: when \textit{do-} applies to the time scale. As follows from the observations made by \citet{Kagan:12}, the maximum point that is reached has to be specified (at least by the context) in this case as the time scale is an open scale. For example, \ref{ex:dosidel} cannot be uttered if it is not clear from the context until what time the actor was supposed to sit. The situation is different with \ref{ex:posidel} and \ref{ex:peresidel} that can be used without any supportive context, which illustrates that the requirements of these prefixes vary (\textit{po-} can create limits on an open scale and \textit{pere-} gets help from the scale construction mechanism that is able to extract non-linguistic information about the appropriate time for the actor to spend sitting).

\ex.\label{ex:dosidel}\ag.Ja dosidel.\\
I do.sit.\glb{pst.sg.m}\\
\trans `I sat till the end.'
\bg.\label{ex:posidel}Ja posidel.\\
I po.sit.\glb{pst.sg.m}\\
\trans `I sat for a while.'
\bg.\label{ex:peresidel}Ja peresidel.\\
I pere.sit.\glb{pst.sg.m}\\
\trans `I sat for too long.'

What is also important is that in case the time point until which the sitting lasted is explicit, the difference between the literal semantics of the verb \textit{dosidet'} `to sit until some certain time' and \textit{posidet'} `to sit for a while' is lost, as illustrated by the examples \ref{ex:dosidet:do} and \ref{ex:posidet:do}. In this situation the differenc between the \textit{po-} and the \textit{do-}prefixed verbs emerges as a result of a pragmatic competition between them. We obtain the enriched meaning of the \textit{do-}prefixed verb that the sitting event lasted relatively long and the enriched meaning of the \textit{po-}prefixed verb that the sitting event was rather short. 

\ex.\ag.\label{ex:dosidet:do}Ja dosidel do pjati utra, i, tak i ne do\v{z}dav\v{s}is' tebja, usnul.\\
I do.sit.\glb{pst.sg.m} until 5 morning and that and not do.wait.\glb{part.pst}.refl you, fall.asleep.\glb{pst.sg.m}\\
\trans `I sat there waiting for you until 5 a.m. and fell asleep.'\\\Source{\url{https://ficbook.net}}
\bg.\label{ex:posidet:do}Priexal na u\v{c}ebu k 7, posidel do 8:15 -- otpustili domoj.\\
pri.ride.\glb{pst.sg.m} on study to 7, po.sit.\glb{pst.sg.m} until 8:15 -- ot.let.\glb{pst.pl} home\\
\trans `I've arrived for the studies at 7, sat there until 8:15 and then I was free to go home.'\Source{\url{https://twitter.com}}

Another predictable consequence of the bleached difference between the literal semantics of \textit{po-} and \textit{do-}prefixed verbs when these prefixes apply to the time scale is that they cannot be stacked. When the prefix \textit{po-} with its `for a while' meaning is attached to a verb, e.g. \textit{sidet'} `to sit', the event denoted by this verb is conceptualized as being homogeneous and having some limited duration. This verb cannot be further prefixed with \textit{do-}: the verb \textit{*doposidet'} does not exist. The potential semantics of this verb after the attachment of two prefixes would be `to complete sitting for a while', which is equivalent either to `to sit for a while' or `to finish sitting', that can be expressed with morphologically simpler verbs. In case only the time scale is available in the verbal semantic structure, the reverse stacking (\textit{po-} on top of \textit{do-}) is not available for the same reason: the verb \textit{*podosidet'} could mean `to sit for a while finishing sitting', but there is no event falling under this denotation that could not be described by either `to sit for a while' or `to finish sitting'. Note that when \textit{do-} selects some other scale rather than time, the prefix \textit{po-} can be stacked on top of it after the verb is imperfectivized. This is illustrated by the chain \ref{chain:podo}\footnote{Only additive interpretations are provided for the verbs in the chain, but terminative interpretations are also possible. In this case the last derived verb means either `to write the final part for a while' or `to finish writing all of.'} and the example \ref{ex:podopisyval}.

\exg.\label{chain:podo}pisat'$^{\IPF}$ $\rightarrow$ dopisat'$^{\PF}$ $\rightarrow$ dopisyvat'$^{\IPF}$ $\rightarrow$ podopisyvat'$^{\PF}$\\
{to write} $\rightarrow$ {to write in addition} $\rightarrow$ {to (be) writing in addition} $\rightarrow$ {to write in addition in all of/for a while}\\

\exg.\label{ex:podopisyval}Podopisyval noli v isxodnye dannye.\\
po.do.write.imp.\glb{pst.sg.m} zeros in initial data\\
\trans `I added zeros to the initial data.'\Source{\url{www.planetaexcel.ru}}

\citet{Tatevosov:09} lists \textit{do-} as a positionally limited prefix which means that it can be attached only below the secondary imperfective suffix. As we have already discussed in Section~\ref{section:new:biaspectual}, this is not a valid observation. For example, the verb \textit{dovy\v{s}ivat'} `to finish embroidering' is either perfective or biaspectual, depending on whether the individual speaker considers the verb \textit{dovy\v{s}it'} `to finish embroidering' existent or not. What is important is that no speaker I have consulted with responded that this verb can have only imperfective interpretation, as suggested by the theory proposed in \citealt{Tatevosov:09}. In the poem \ref{ex:poem} the verb \textit{dorazukra\v{s}ivat'} `to finish coloring' is also perfective as it is constructed according to the derivation presented in \ref{chain:dorazu1}. The verb containing the same morphemes can also be imperfective if the order of attachment is different, as represented in \ref{chain:dorazu2}.

\ex.\ag.\label{chain:dorazu1}krasit'$^{\IPF}$ $\rightarrow$ ukrasit'$^{\PF}$ $\rightarrow$ razukrasit'$^{\PF}$ $\rightarrow$ razukra\v{s}ivat'$^{\IPF}$ $\rightarrow$ dorazukra\v{s}ivat'$^{\PF}$\\
{to paint} $\rightarrow$ {to decorate} $\rightarrow$ {to color} $\rightarrow$ {to color/be coloring} $\rightarrow$ {to finish coloring}\\
\bg.\label{chain:dorazu2}krasit'$^{\IPF}$ $\rightarrow$ ukrasit'$^{\PF}$ $\rightarrow$ razukrasit'$^{\PF}$ $\rightarrow$ dorazukrasit'$^{\PF}$ $\rightarrow$ dorazukra\v{s}ivat'$^{\IPF}$\\
{to paint} $\rightarrow$ {to decorate} $\rightarrow$ {to color} $\rightarrow$ {to finish coloring} $\rightarrow$ {to finish/be finishing coloring}\\

A couple of other biaspectual verbs are the verbs \textit{doobdumyvat'} `to finish thinking about' (see examples under \ref{ex:doobdumyvat}) and \textit{dozabivat'} `to finish hammering' (see examples under \ref{ex:dozabivat}).

\ex.\label{ex:doobdumyvat}\ag.V processe \v{c}tenija v golove na\v{c}ali oformljat'sja vsjakie xitrye i kovarnye idei, no ix e\v{s}\v{c}\"{e} nu\v{z}no akkuratno doobdumyvat'$^{\PF}$.\\
in process reading in head start.\glb{pst.pl} form.\glb{inf}.refl various tricky and crafty ideas but they also needed accurately do.ob.think.imp.\glb{inf}\\
\trans `While I was reading it some tricky and crafty ideas came to my head, but I need to think them over accurately.'\\\Source{\url{http://nicka-startcev.livejournal.com}}
\bg.Zasim ja idu morozit' nos i doobdumyvat'$^{\IPF}$ v\v{c}era\v{s}njuju ideju, poka ona ne ube\v{z}ala ot menja okonchatel'no.\\
hereupon I go.\glb{pres.sg.1} freeze.\glb{inf} nose and do.ob.think.imp.\glb{inf} yesterday's idea while she not u.run.\glb{pst.sg.f} from me completely\\
\trans `Hereupon I go to freeze my nose and think more about yesterday's idea until it has fled from me completely.'\Source{\url{8794.diary.ru}}


%Tatevosov: *do-(za-b-iva)-t', but:
%naverno ja tebe prishlju ejo v takom nepolnom vide - a ty posmotri i poprobuj kalendar' dozabivat' do konca
%http://rusport.eu/threads/890/page-39
\ex.\label{ex:dozabivat}\ag.Tam e\v{s}\v{c}\"{e}, chut' popoz\v{z}e, krjuk e\v{s}\v{c}\"{e} i dozabivat'$^{\PF}$ v sneg umudrjajutsja, i, pre\v{z}de \v{c}em verjovku rezat', celuju re\v{c}' proiznosjat.\\
there also {a bit} later hook also and do.za.hit.imp.\glb{inf} in snow manage.\glb{inf}.refl and before what.\glb{instr} rope cut.\glb{inf} whole speech pronounce.\glb{pres.pl.3}\\
\trans `In the same video, a bit later, they also manage to hammer the hook in the snow completely and then they pronounce a whole speech before cutting the rope.'\Source{\url{http://yarin-mikhail.livejournal.com}}
\bg.Gvozdi inogda dozabivat'$^{\IPF}$ prixoditsja.\\
nails sometimes do.za.hit.imp.\glb{inf} pri.go.\glb{pres.sg.3}.refl\\
\trans `The nails sometimes have to be additionally hammered.'\\\Source{\url{https://forumhouse.ru}}

It seems that the prefix \textit{do-} is very undemanding with respect to the verb it attaches to. Sometimes the resulting verb seems odd, as \textit{donapisat'} `to finish writing', but such difficulties are of the same kind as with attaching the repetitive prefix \textit{pere-} to some perfective verbs (see Section~\ref{subsection:semantics:pere}) and we do find these verbs in some contexts. Such contexts require exactly the semantics obtained by composing the semantics of the prefix \textit{do-} with the semantics of the prefixed verb (e.g., \textit{napisat} `to write/create something written') and not with the semantics of the unprefixed verb (e.g., \textit{pisat'} `to write'). An example is provided in \ref{ex:donapisat} and the contrast sentence with the replaced verb is given in \ref{ex:donapisat:mod}. As we see, the speaker wants to express the additive semantics, and as the most natural interpretation of the verb \textit{dopisat'} is `to finish writing', they prefer to use the verb \textit{donapisat'} `to write something in addition'. This leads to the question of how the meaning of the prefix is related to the properties of the derivational base.

\ex.\ag.\label{ex:donapisat}Tam ja donapisal pis'ma i novoe stixotvorenie, a tak\v{z}e porabotal s fotografijami.\\
there I do.na.write.\glb{pst.sg.m} letter.\glb{pl.acc} and new poem but also po.work.\glb{pst.sg.m} with photos\\
\trans `There I also wrote letters and a new poem, and also worked a bit with the photos.'\Source{\url{dd.vl.ru}}
\bg.\label{ex:donapisat:mod}Tam ja dopisal pis'ma i novoe stixotvorenie, a tak\v{z}e porabotal s fotografijami.\\
there I do.write.\glb{pst.sg.m} letter.\glb{pl.acc} and new poem but also po.work.\glb{pst.sg.m} with photos\\
\trans `There I finished writing the letters and the new poem, and also worked a bit with the photos.'

What can be noticed is that the aspect of the derivational base matters. In general, if the derivational base is perfective, the interpretation of the derived \textit{do-}prefixed verb tends to be additive (compare \ref{ex:do:kupit} and \ref{ex:do:pokupat}), and if the derivational base is a secondary imperfective verb, the additive interpretation seems to be not available (see example \ref{ex:do:zapisyvat}). In case a \textit{do-}prefixed verb gets imperfectivized, both additive and terminative interpretations become available for the derived imperfective verb (see examples under \ref{ex:zapravit}).

\ex.\ag.\label{ex:do:kupit}Katja dokupila mandarin.\\
Katja do.buy$^{\PF}$.\glb{pst.sg.f} tangerine.\glb{pl.gen}\\
\trans `Katja also bought some tangerines.'/`Katja bought some additional tangerines.'
\bg.\label{ex:do:pokupat}Katja dopokupala mandariny.\\
Katja do.buy$^{\IPF}$.\glb{pst.sg.f} tangerine.\glb{pl.acc}\\
\trans `Katja finished buying tangerines.'

\ex.\ag.\label{ex:do:zapisyvat}Petja dozapisyval$^{\PF}$ dva diska.\\
Petja do.za.write.imp.\glb{pst.sg.m} two CDs\\
\trans `Petja finished recording two CDs.'
\bg.\label{ex:do:zapisat}Petja dozapisal$^{\PF}$ dva diska.\\
Petja do.za.write.\glb{pst.sg.m} two CDs\\
\trans `Petja additionally recorded two CDs'/`Petja finished recording two CDs.'

\ex.\label{ex:zapravit}\ag.\label{ex:zapravit3}Mexanik dozapravil$^{\PF}$ samol\"et (i zakuril sigaretu).\\
mechanic do.fill.\glb{pst.sg.m} plane.\glb{sg.acc} (and za.smoke.\glb{pst.sg.m} cigarette)\\
\trans `The mechanic additionally fueled the plane and lightened a cigarette.'
\bg.\label{ex:zapravit2}Mexanik dozapravljal$^{\PF}$ samol\"et (i zakuril sigaretu).\\
mechanic do.fill.imp.\glb{pst.sg.m} plane.\glb{sg.acc} (and za.smoke.\glb{pst.sg.m} cigarette)\\
\trans `The mechanic finished fueling the plane and lightened a cigarette.'
\bg.\label{ex:zapravit1}Mexanik dozapravljal$^{\IPF}$ samol\"et (i kuril sigaretu).\\
mechanic do.fill.imp.\glb{pst.sg.m} plane.\glb{sg.acc} (and smoke.\glb{pst.sg.m} cigarette)\\
\trans `The mechanic was finishing fueling/additionally fueling the plane and smoking.'

The verbs used in \ref{ex:zapravit} are acquired in course of the following derivations. The perfective verb \textit{zapravit'} `to fuel' can be either directly prefixed with \textit{do-} (as in the chain \ref{chain:dozapravljat1}) or first imperfectivized (as in the chain \ref{chain:dozapravljat2}). In the first case the derived verb is \textit{dozapravit'}$^{\PF}$ `to fuel additionally' (used in the example \ref{ex:zapravit3}) that can be then imperfectivized in order to obtain the verb \textit{dozapravljat'}$^{\IPF}$ that can either mean `to finish/be finishing fueling' or `to fuel/be fueling additionally', as illustrated by the example \ref{ex:zapravit1}. If the morphemes are attached in the different order, as illustrated by the chain \ref{chain:dozapravljat2}, the derived verb \textit{dozapravljat'}$^{\PF}$ `to finish/be finishing fueling' is perfective and acquires terminative semantics (see example \ref{ex:zapravit2}).

\ex.\label{chain:dozapravljat}\ag.\label{chain:dozapravljat1}zapravit'$^{\PF}$ $\rightarrow$ dozapravit'$^{\PF}$ $\rightarrow$ dozapravljat'$^{\IPF}$\\
{to fuel} $\rightarrow$ {to fuel additionally} $\rightarrow$ {to (be) finish(ing) fueling/to (be) fuel(ing) additionally}\\
\bg.\label{chain:dozapravljat2}zapravit'$^{\PF}$ $\rightarrow$ zapravljat'$^{\IPF}$ $\rightarrow$ dozapravljat'$^{\PF}$\\
{to fuel} $\rightarrow$ {to fuel/be fueling} $\rightarrow$ {to finish/be finishing fueling}\\

The chain \ref{chain:dozapravljat1} illustrates that the additive meaning component associated with the \textit{do}-prefixed verb is not inherited and can be replaced by another inference after the imperfectivization step. This speaks in favor of the hypothesis that this kind of the additional inference is not specified in the semantic structure of the verb but arises as a result of the interpretation of the semantic representation followed by a pragmatic competition. For this reason, I will abandon the distinction between the additive and the terminative usages of \textit{do-}. In sum, I claim that it is not only possible to unify the additive and the terminative usages of the prefix \textit{do-}, but that there are no distinct representations for these usages. Instead, there are different ways to interpret the semantic representation of the derived verb that result in different inferences. 

\ex.\ag.\label{ex:do:pere1}Nu, doperepisal, tak-to proizvedenie bylo napisano v 97--98 gg...\\
well do.pere.write.\glb{pst.sg.m} that composition was written in 97--98 years\\
\trans `Well, I finished rewriting it, as the work was actually written in 1997--1998.'\Source{\url{na-ive.diary.ru}}
\bg.\label{ex:do:pere2}Doperepisyval na\v{c}isto, s nekotorymi ispravlenijami, preljudiju do ma\v{z}or.\\
do.pere.write.imp.\glb{pst.sg.m} clean with some corrections prelude C major\\
\trans `Finished rewriting the final version of the C major prelude (with some corrections).'\Source{\url{1001.ru}}

Another observation concerns stacking the prefix \textit{do-} on top of the prefix  \textit{pere-}: when \textit{pere-}prefixed verbs are further prefixed with \textit{do-}, they acquire terminative interpretation independently of the aspect of the derivational base (see examples \ref{ex:do:pere1} and \ref{ex:do:pere2}). Putting it simply, the events referred to by the \textit{pere-}prefixed verbs are conceptualized as proceeding through contiguous stages. The additive interpretation of the prefix \textit{do-} requires (according to the proposal of \citet{Kagan:book}) that there is a break between the event associated with the initial part of the scale and the event associated with the final part of the scale. Such gap is incompatible with the semantics of the derivational base if it contains the prefix \textit{pere-}. 

In sum, I propose to represent the contribution of the prefix \textit{do-} as fixing the final stage of the event and specifying the event denoted by the derived verb as being a part of an event denoted by the derivational base.

%I propose that whenever it is not possible to interpret the derived \textit{do-}prefixed verb additively, the event can be decomposed in two stages. One of these stages will be the preparatory phase (corresponding to the presupposition in the account of \citet{Kagan:book}) and the other will be the current event that the speaker is focusing on. We will see why all this in detail once the formal representations are constructed.

%If we recall the section addressing the prefix \textit{pere-} (section \ref{subsection:semantics:pere}) and also jump ahead and think of the secondary imperfective (when interpreted progressively) as adding an intermediate point while preserving the information about the boundaries (see section \ref{section:imperfective}), than we can see that those situations are unified by the presence of three distinguished points in the semantic structure of the event. As we have discussed, the terminative usage of the prefix \textit{do-} is also based on distinguishing three points in the semantic structure of the event (boundaries plus an intermediate point that divides the preparatory phase from the main phase). Those facts seem to reveal why it is the terminative and not the additive semantics that arises in the discussed cases. What also should fall out from if those observations are on the right track is a non-zero preparatory phase in case the prefix \textit{do-} is attached to a secondary imperfective verb (details will become clear once we discuss formal representations and their combinatorics in chapter~\ref{chapter:formal}}. 
%
%Judging from the introspection, this is indeed so, but as the same inference can arise due to the pragmatic competition between such verbs and perfective verbs (derivational bases for those imperfectives that were then prefixed with \textit{do-}), reliable test contexts have to exclude the possibility of pragmatic reasoning. I leave more detailed experimental investigation of this point for future research. The only note I want to add is that this is something that no other approach is predicting, so it can turn out to be a strong evidence in its favor or a problem.


\subsection{Subsequent imperfectivization of a verb with the discussed prefix}
The existence of a prefix that has transparent semantic contribution and does not block subsequent imperfectivization at all is not predicted by the theory of distinct structural positions for the lexical and superlexical groups of prefixes. However, the possibility of attaching the imperfective suffix to the \textit{do-}prefixed verbs cannot be denied and this prefix has been incorporated in the lexical/superlexical framework, acquiring a different status (e.g., falling in the category of \textit{intermediate} prefixes in the theory of \citealt{Tatevosov:07}). Imperfectivization of the verbs prefixed with \textit{do-} seems to be possible in all the cases when the verbal stem allows the addition of the imperfective suffix. Some examples of secondary imperfective verbs with the prefix \textit{do-} have been provided above: these are the sentences \ref{ex:zapravit1} and \ref{ex:do:pere2}.

The cases when imperfectivization is not possible are those cases when the verbal stem is not compatible with the imperfective suffix at all, as in case of the verb \textit{\v{z}eltet'} `to turn yellow/to be seen as yellow' that we have already discussed in connection with the prefix \textit{za-}. This verb in its `to turn yellow' interpretation can be prefixed with \textit{do-}. The result is the verb \textit{do\v{z}eltet'} `to finish turning yellow' (see example \ref{ex:dozheltet}). This verb cannot be further imperfectivized. 

\exg.\label{ex:dozheltet}Te list'ja do\v{z}elteli i opali.\\
that leaves do.turn.yellow.\glb{pst.pl} and o.fall.\glb{pst.pl}\\
\trans `Those leaves finished turning yellow and fell off.'\\\Source{\url{www.bonsaiforum.ru}}

\subsection{Summary}
Summing up the above discussion, I want to note the following points that have to be observed when the formal representation of the prefix \textit{do-} is constructed.
\begin{enumerate}
\item If the derivational base lexicalizes a scale, \textit{do-} selects this scale. If not, the second choice is the scale contributed by the direct object (it can be a measure of change scale). If both options are unavailable, \textit{do-} can quantify over the time scale.
\item The scale selected by \textit{do-} has to be upper-closed.
\item The end point of the event denoted with the \textit{do-}prefixed verb has to correspond to the maximum point on the scale.
\item If the \textit{do-} attaches to a perfective verb and the start of the event denoted by this verb is related to the minimum on the scale, the event can be decomposed into the preparatory and the focused phases.
\end{enumerate}

%\subsection{pod-}\label{subsection:semantics:pod}
%\subsection{Semantic contribution}
%\citet[pp.365--366]{Shvedova:82}
%
%\begin{enumerate}
%\item to direct the action denoted by the derivational base down, under something (productive type): \textit{podstavit'} `to put something under something else';
%\item to direct the action denoted by the derivational base upward (productive type): \textit{podbrosit'} `to throw something in the air';
%\item to approach or attach to something by performing the action denoted by the derivational base (productive type): \textit{podojti'} `to come closer by walking';
%\item to perform the action denoted by the derivational base with low intensity (productive type, especially in colloquial speech): \textit{podbodrit'} `to cheer someone up';
%\item to perform the action denoted by the derivational base in addition and, usually, with low intensity (productive type, some derivational bases are perfective): \textit{podgladit'} `to iron a bit more,' \textit{podzarabotat'} `to earn a some money';
%\item to perform the action denoted by the derivational base in secret (productive type): \textit{podslu\v{s}at'} `to eavesdrop';
%\item to clear something or remove the rests by performing the action denoted by the derivational base (non productive type): \textit{podjest'} `to eat the rests';
%\item to perform the action denoted by the derivational base in coarse or immediately after another action, adapting to something (productive type): \textit{podygrat'} `to play, adapting to the play of someone else';
%\item to perform the action denoted by the derivational base until the result (non productive type): \textit{podmesti} `to sweep the floor.'
%\end{enumerate}
%\subsection{Restrictions on the attachment}
%\subsection{Subsequent imperfectivization of a verb with the discussed prefix}
%\subsection{Summary}
%Podhmelet' exists (contra Tatevosov, who marks it with ??): maybe it's all about phonetics?
%
%\subsection{ot-}\label{subsection:semantics:ot}
%\subsection{Semantic contribution}
%\citet[pp.362--363]{Shvedova:82}
%\begin{enumerate}
%\item to part or move away from something by performing the action denoted by the derivational base (productive type): \textit{otletet'} `to move away by flying';
%\item to move somewhere by performing the action denoted by the derivational base (this type is productive if the prefix is combined with verb denoting relocation): \textit{otvezti} `to bring something somewhere,' \textit{otvesti} `to carry something somewhere';
%\item intensively, completely, finally perform the action denoted by the derivational base (productive type): \textit{otgladit'} `to iron thoroughly';
%\item to lead to an undesired state or condition as a result of performing the action denoted by the derivational base (non productive type, takes as its input only transitive verbs): \textit{otdavit'} `to step on something';
%\item as a result of performing the action denoted by the derivational base refuse something or make someone else to refuse something (non productive type): \textit{otsovetovat'} `to dissuade';
%\item to perform the action denoted by the derivational base as a response to some other action (non productive type): \textit{otplatit'} `to pay off';
%\item to end the action denoted by the derivational base, that lasted for some time (productive type): \textit{otgremet'} `to stop rattling';
%\item to perform the action denoted by the derivational base until the result is reached (non productive type): \textit{otremontirovat'} `to repair';
%
%\end{enumerate}
%\subsection{Restrictions on the attachment}
%\subsection{Subsequent imperfectivization of a verb with the discussed prefix}
%\subsection{Summary}

%\subsection{pri-}\label{subsection:semantics:pri}
%\paragraph*{Semantic contribution.}
%\citet[pp.366--367]{Shvedova:82}
%\begin{enumerate}
%\item to reach some destination, deliver something or become attached to something by performing the action denoted by the derivational base (productive type): \textit{prinesti'} `to deliver by carrying';
%\item to perform the action denoted by the derivational base with low intensity or not to the end (productive type): \textit{pritormozit'} `to brake slightly,' \textit{prizadumat'sja} `to become slightly thoughtful';
%\item to perform the action denoted by the derivational base in addition to something, add something to something else (productive type): \textit{pririsovat'} `to draw something additional on a drawing;
%\item to perform the action denoted by the derivational base in course or directly after the other action (productive type, derivational bases are perfective verbs denoting single actions): \textit{prixlopnut'} `to clap while doing something else';
%\item to perform the action denoted by the derivational base until the result is reached (non productive type): \textit{primerit'} `to try something.'
%\end{enumerate}
%\paragraph*{Restrictions on the attachment.}
%\paragraph*{Subsequent imperfectivization of a verb with the discussed prefix.}
%\paragraph*{Summary.}

%\subsection{pro-}\label{subsection:semantics:pro}
%\paragraph*{Semantic contribution.}
%\citet[pp.366--367]{Shvedova:82}
%\begin{enumerate}
%\item to direct the action denoted by the derivational base through or into something (productive type, some derivational bases are perfective): \textit{projti'} `to walk through,' \textit{protolknut'} `to push through';
%\item to direct the action denoted by the derivational base past something (productive type): \textit{probe\v{z}at'} `to run past something';
%\item to move forwardor cover some distance by performing the action denoted by the derivational base (productive type): \textit{pronesti} `to carry something for some distance';
%\item to perform the action denoted by the derivational base intensively or thoroughly (productive type): \textit{progladit'} `to iron thoroughly';
%\item to spend or expend something by performing the action denoted by the derivational base through or into something (productive in colloquial language): \textit{propit'} `to spend the money on drinking';
%\item to miss something while performing the action denoted by the derivational base (productive in colloquial language): \textit{proguljat'} `to go strolling instead of going to work or study ';
%\item to perform the action denoted by the source for some (usually long) time (productive type): \textit{pro\v{z}dat'} `to wait for a long time';
%\item to perform the action denoted by the derivational base until the result is reached (productive type): \textit{prozvu\v{c}at'} `to sound.'
%\end{enumerate}
%\paragraph*{Restrictions on the attachment.}
%\paragraph*{Subsequent imperfectivization of a verb with the discussed prefix.}
%\paragraph*{Summary.}

\section{Secondary Imperfective}\label{section:imperfective}
Formally representing the semantics of the imperfective suffix is a task I am not aiming to complete in this thesis. However, it is not possible to construct the desired compositional semantics of complex verbs without a semantic representation of the imperfective suffix. In order to achieve the goal of analyzing prefix stacking (with respect to those prefixes we have discussed here plus verbs that are listed in the dictionaries) I have to construct some formal representation of the semantics of the imperfective suffix. I will do this for two cases: (1) progressive meaning of the imperfective and (2) habitual meaning of the imperfective. This is going to involve some decisions that I am just going to lay out without proper justification.

The first puzzle that has to be solved in some way concerns the general problem with the progressive interpretation of the secondary imperfective that seems to cancel the ``reaching the boundary'' component brought in by the prefix. I claim that when secondary imperfectivization happens, there is no ``reversion" to the initial imperfective semantics. I will account for this in the following way. 

Let us start with a basic imperfective verb. Such verb denotes an activity or a process that is not mapped onto the time scale. If one wants to describe it in terms of telicity, it can be either atelic, as \textit{sidet'} `to sit/be sitting' or telic, as \textit{pisat' pis'mo} `to write/be writing a letter', but in neither case it has endpoints that are mapped onto the time scale. This mapping is what, according to my view, prefixes take care of. As the verb gets prefixed, its semantic structure gets enriched with endpoints that are related to some time points. In case the scale selected by the prefix is the time scale, some points on this scale are directly associated with the start and the end of the event. In case the event proceeds along some other scale, points on that scale are mapped onto the time scale. 

I propose that when the imperfective suffix with the progressive semantics is attached to a perfective verb, the boundaries that are present in the semantic structure of the derivational base do not disappear. Instead, the derived verb denotes an event that is a part of the event denoted by the derivational base and is of type \textit{progression}. It can as well turn out that this partial event coincides with the whole event in case the verb is prefixed further or the imperfective is actually used to describe a completed event.

%another point that represents an intermediate stage on the scale is added. This point has to be located in between the points on the scale corresponding to the start and end points of the event and has to be distinct from the point corresponding to the start of the event. The new point indicates the progress of the event and allows to refer to some intermediate stage. Putting it in a more formal language, if the point on the scale associated with the start of the event is $x$ and the point associated with the end of the event is $z$, then the imperfective adds a third point: a \textit{current} point $y$ such that $x < y \leq z$. 

The second meaning of the secondary imperfective suffix that I will formalize is the repetitive/habitual meaning. This will function similarly to the distributive \textit{pere-} except for the absence of the set that has to be iterated through. In case of the imperfective suffix the iteration is performed without imposing restrictions on when the first event of the iterated series started and when (and whether) the series is going to end. The attachment of the imperfective suffix with an repetitive/habitual interpretation is similar to providing a repetitive context for a telic verb in English: independently of the language, an iteration of a bounded event becomes an unbounded event. For English this means that verbs denoting accomplishments and achievements become compatible with \textit{for}-adverbials. For Russian the consequence of the attachment of the imperfective suffix is an additional layer of verbal structure that makes the event unbounded and thus imperfective and also opens additional prefixation possibilities. 

%Basic imperfective verbs denoting atelic processes/activities do not receive habitual interpretation: \textit{on sidit v t'jurme} cannot be interpreted as `he regularly sits in jail' (there is a possibility of getting `his occupation is to sit in jail' interpretation that leads to regularity, but this is different). Plurality can be contributed by the DO: \textit{on \v{c}itaet knigi} can be interpreted as an event of reading many books simultaneously or multiple events of reading. 

\section{Summary}
In this chapter I have provided an overview of semantic approaches to Russian verbal prefixation and inspected semantic and combinatorial properties of five verbal prefixes: \textit{za-, na-, po-, pere-,} and \textit{do-}. For each prefix I have discussed its semantic contribution, restrictions on the attachment and on further combination with the imperfective suffix. 

As, following \citet{Kagan:book}, I adopt scalar analysis of prefix semantics, I have also provided general information about scales and paid attention to the types of the scales individual prefixes are compatible with and the relations they impose between scalar points and event stages. I have concluded that the prefix \textit{za-} in its inceptive usage requires \textit{time} scale and the initial stage of the event denoted by the derived verb corresponds to the absence of the event denoted by the derivational base while the final stage corresponds to the presence of the event denoted by the derivational base. 

The prefix \textit{na-} accepts a wide range of scales as long as they are provided by the verb and belong to the set of parameters of the object. It maps the initial stage of the event to the minimal point of the scale and the end of the event -- to some point that is at or above the contextually specified standard degree. The prefix \textit{po-} is compatible with any verbal scale and the \textit{cardinality} scale in case of a plural object. It relates the initial and the final stages of the event to some points on the scale. 

The prefix \textit{pere-} has three different interpretations that depend on the type of the scale: in case of a closed scale the event proceeds from the minimum to the maximum on the scale through all its points; in case of a scale with one marked point the event proceeds from the point below the marked point through the marked point to some point above it; in case of a \textit{property} scale the repetitive interpretation of the prefix is also available and the new event is created by copying the event denoted by the derivational base which, in turn, becomes the preparatory phase of the new event. 

The last prefix, \textit{do-}, is compatible with scales provide by the verb and by the object as long as they are upper-closed. It maps the initial stage of the event onto some point on the scale and the final stage of the event onto the maximum of the scale. 

In course of the discussion of the prefix \textit{do-} and the repetitive usage of the prefix \textit{pere-} I have also raised questions concerning possible presuppositional components in the semantic structure of those verbs, as suggested by \citet{Kagan:book}. I will address these questions in the next chapter.

After that, in Chapter~\ref{Chapter7}, I will offer a formalization of the intuitions and observations laid out in this chapter, using the combination of Frame Semantics (\citealt{Fillmore:82}) and LTAG (\citealt{JoshiSchabes:97}) formalized in \citealt{KallmeyerOsswald:13}. 

%Compare the system that emerges with the classification by \citet{Janda:07b}. Her Natural Perfectives correspond to relating of the end points of the verbal scale to the end points of the event and Complex Act Perfectives correspond to other types of connections.
