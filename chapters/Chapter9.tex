% Chapter 9

\chapter{Conclusions, remarks and further questions} % Write in your own chapter title
\label{Chapter9}
%\lhead{Chapter 8. \emph{Conclusions, remarks and further questions}} % Write in your own chapter title to set the page header

In this work I have explored the Russian verbal \isi{prefixation} system and proposed a complex account that models it. In Chapter~\ref{Chapter2} I have presented new data that did not receive an appropriate analysis within the earlier accounts of Russian \isi{prefixation}. I have also developed a method of collecting data that prevents any decisions that may be biased by the theory one proposes. This method was used throughout the entire work to ensure careful data representation.

After considering the data I have discussed the commonly assumed distinction between lexical and \isi{superlexical} prefixes in Chapter~\ref{Chapter4}. I have shown that despite the clear differences between the properties of particular prefixes the proposal of the strict distinction between the classes has to be rejected together with the possibility to restrict \isi{prefix stacking} due to different positions of various prefixes. The division into prefix classes is then substituted with a scale. One end of this scale is occupied by those prefixes that do not have a predictable semantic contribution, can never stack on top of other prefixes, and change the \isi{argument structure} of the verb. On the other end of the scale are located those prefixes that have a transparent semantics, can stack freely and do not change the \isi{argument structure} of the verb. Other prefixes are located in between these extremes without clear class borders. On this basis I have decided to abandon the hypothesis of different structural positions of various prefixes and develop a semantic account that would have at least the same predictive power with respect to possible affix combinations and also explains the data presented in Chapter~\ref{Chapter2}.

In Chapter~\ref{Chapter5}, I went through the first step towards a semantic account of verbal \isi{prefixation} in Russian: I provided an informal analysis of the semantic and combinatorial properties of five prefixes (\Prefix{za-}, \Prefix{na-}, \Prefix{po-}, \Prefix{pere-} and \Prefix{do-}) as well as a brief discussion of the (simplified) treatment of the \isi{imperfective suffix} that I assume. I then continued with the exploration of the pragmatic properties of individual prefixes and of the competition between various prefixed verbs derived from the same base in Chapter~\ref{Chapter6}. I have shown that there is not enough evidence to assume the presuppositional account of the prefixes \Prefix{do-} and \Prefix{pere-} and concluded that the inferences associated with their usage should be treated as entailments and implicatures. In the second part of the chapter I have outlined a preliminary version of the \isi{pragmatic competition} between prefixed verbs. I have shown some examples of how the interpretation of a prefixed verb can be derived using underspecified semantics and basic pragmatic principles. 

Following the theoretical part,  in Chapter~\ref{Chapter7} I have provided a frame \isi{semantic analysis} of the five prefixes which I have explored in this work. I have introduced the formalism, provided frame representations of various prefixes and shown how these frames combine with verbal frames, frames for the direct object, measure phrases, and special \isi{dimension constructors}. To evaluate the predictions of the analysis I have implemented it for a small language fragment using the \isi{metagrammar} description formalism (\isi{XMG}). I have provided the details of the implementation and discussed the difficulties related to it in Chapter~\ref{Chapter8}. I have also implemented the proposal of \citet{Tatevosov:09} and compared the output of the two proposals with respect to the predictive power of available affix combinations for a given verb. 

In sum, I have provided and partially implemented an account that predicts the possibility of prefix attachment (for five prefixes) and in case of a positive answer also the semantics, aspect, and semantic and \isi{syntactic restrictions} on the arguments of the derived verb. 

On the other hand, I have raised a number of questions that could not be answered in course of this work and are worth further investigation. These are, for example, questions about the unexpected behaviour of \isi{loaned} \isi{biaspectual verbs} when they are prefixed with \Prefix{do-} or \Prefix{pod-} and about the status of \isi{loaned} prefixes, such as \Prefix{dis-} or \Prefix{re-}. I also have not examined the behaviour of the \isi{imperfective suffix} in detail and instead used a simplification that has to be replaced with a more thorough description in the \isi{future}. 

Another research direction that I aim to address in my \isi{future} work is the development of the pragmatic part. I hope to implement the proposal concerning the competition of various prefixed verbs using the Rational Speech Act framework. In parallel, I would like to run the experiments to obtain probabilistic predictions for various interpretations of the prefixed verbs.  Of particular interest are cases where, according to my analysis, a particular interpretation is part of the semantics of the verb, but is blocked for pragmatic reasons. I then plan to compare the quantitative output of the implemented system with experimental results that would allow to test the whole theory in an objective way.

The implementation of the proposal I have done so far also needs to be extended. This would be possible as soon as the relevant tools are available (most important of which is a parser that would work with TAG and frame representations) and the contribution of other prefixes is represented in terms of frames. A large-scale implementation would allow to create the \isi{derivational graph}, as proposed in Chapter~\ref{Chapter2}, that would open the way for further research and testing in the domain of Russian \isi{complex verbs}. 
