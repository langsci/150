% Author: Till Tantau
% Source: The PGF/TikZ manual
\documentclass{standalone}
\def\xcolorversion{2.00}
\def\xkeyvalversion{1.8}
\usepackage{graphicx}
\usepackage[version=0.96]{pgf}
\usepackage{tikz}
\usetikzlibrary{arrows,shapes,automata,backgrounds,petri,positioning}
\usetikzlibrary{shapes.multipart}
\usepackage{libertinus-otf}
\newcommand{\type}[1]{\textit{#1}}
\newcommand{\framerel}[1]{\textit{#1}}
\newcommand{\reltype}[1]{\textit{#1}}
\newcommand{\feat}[1]{\textsc{#1}}

\begin{document}
\begin{tikzpicture}[node distance=4.3cm,>=stealth',bend angle=45,auto,every text node part/.style={align=center}]

  \tikzstyle{place}=[circle,thick,draw=black!75,minimum size=5mm]
  \tikzstyle{placeout}=[circle,double, double distance=1mm,draw=black!75,minimum size=5mm]
  \tikzstyle{red place}=[place,draw=red!75,fill=red!20]
  \tikzstyle{transition}=[rectangle,thick,draw=black!75,
  			  fill=black!20,minimum size=4mm]


  \begin{scope}
    % First net
    \node [place, label={[align=center]\type{\v{s}it'}\textsuperscript{IPF}\\`to sew'}] (e1)            {};

  
    \node [place] (e3) [right=3.8cm of e1, label={[align=center]\type{vy\v{s}it'}\textsuperscript{PF}\\`to embroider'}] {}
      edge [pre] node[swap] {\textit{vy-}}                          (e1);
     
     \node [place] (e5) [below=2cm of e3, label={[align=center]below:\type{\textsuperscript{?}dovy\v{s}it'}\textsuperscript{PF}\\`to finish embroidering'}] {?}
      edge [pre] node[swap] {\textit{\textsuperscript{?}do-}}                          (e3);
      
        \node [place] (e4) [right=3.8cm of e3,  label={[align=center]\type{vy\v{s}ivat'}\textsuperscript{IPF}\\`to (be) embroider(ing)'}] {}
      edge [pre, below] node[swap] {\textit{-yva-}}  (e3);
      
      \node [place] (e6) [right=3.8cm of e5,  label={[align=center]below:\type{dovy\v{s}ivat'}\textsuperscript{PF\slash \textsuperscript{?}IPF}\\`to finish/\textsuperscript{?}be finishing\\ embroidering'}] {}
      edge [pre] node[swap] {\textsuperscript{?}\textit{-yva-}}  (e5)
      edge [pre] node[swap] {\textit{do-}}  (e4);
  

  \end{scope}

%,label=left:\type{bounded-event}
\end{tikzpicture}
\end{document}
