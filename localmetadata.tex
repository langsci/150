\title{Russian verbal prefixation}  %look no further, you can change those things right here.
\subtitle{A frame semantic analysis}
% % \BackTitle{} % Change if BackTitle != Title
\BackBody{Russian verbal prefixation system has been extensively studied but yet not explained. Traditionally, different meanings have been investigated and listed in the dictionaries and grammars (ˇSvedova, 1982). More recently, Jakobson (1984), Janda (1985, 1988), Paillard (1997), and Kagan (2012, 2015) attempted to unify various prefix usages under more general descriptions.

The existent semantic approaches, however, do not aim to use semantic representations in order to account for the problems of prefix stacking and aspect determination. This task has been so far undertaken by syntactic approaches to prefixation, such as Ramchand 2004, Romanova 2006, Svenonius 2004b, Tatevosov 2007, 2009, that divide verbal prefixes in classes and limit complex verb formation by restricting structural positions available for the members of each class. I show that these approaches have two major drawbacks: the implicit prediction of the non-existence of complex biaspectual verbs and the absence of uniformly accepted formal criteria for the underlying prefix classification.

In this work, I propose an implementable formal semantic approach to prefixation and cover five prefixes: za-, na-, po-, pere-, and do-. Using the combination of an LTAG and frame semantics (Kallmeyer and Osswald 2013), I predict the existence, semantics, and aspect of a given complex verb. I also model the interaction between the semantics of the verb and that of its arguments. The task of identifying the possible affix combinations is distributed between three modules: syntax that is kept simple (only basic structural assumptions), frame semantics that ensures that the constraints are respected, and pragmatics that rules out some prefixed verbs and restricts the range of available interpretations.

In order to evaluate the predictions of the theory, I provide an implementation of the proposed analysis for a grammar fragment using a metagrammar description. I then show that it delivers more accurate and complete predictions with respect to the existence of complex verbs than the most precise syntactic account (Tatevosov 2009).}
%\dedication{Change dedication in localmetadata.tex}
\typesetter{Yulia Zinova, Felix Kopecky}
\proofreader{Adriana Sabatino, Agnes Kim, Amir Ghorbanpour, Amy Lam, Ana Afonso, Andreas Hölzl, Andrew J. Spencer, Geoffrey R. Sampson, Jean Nitzke, Jeroen van de Weijer, Tania Avgustinova, Teodora Mihoc, Tom Bossuyt}
\author{Yulia Zinova}
\BookDOI{10.5281/zenodo.4446717}%ask coordinator for DOI
\renewcommand{\lsISBNdigital}{978-3-96110-298-3}
\renewcommand{\lsISBNhardcover}{978-3-96110-299-0}
\Series{eotms} % use lowercase acronym, e.g. sidl, eotms, tgdi
\SeriesNumber{??} %will be assigned when the book enters the proofreading stage
\renewcommand{\lsID}{150}
