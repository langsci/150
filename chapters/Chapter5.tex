% Chapter 5
\chapter{Semantics of individual prefixes} % Write in your own chapter title
\label{Chapter5}
%\lhead{Chapter 4. \emph{Semantics of individual prefixes}} % Write in your own chapter title to set the page header
\section{Semantic approach to verbal prefixation}
The main things that we have discussed so far are an efficient way of collecting and verifying the data and the fact that these data cannot be fully accounted for by means of existing syntactic approaches to Russian \isi{prefixation}. Let us now explore what has been done in the domain of prefix semantics.

Semantics-oriented studies of Russian prefixes can be divided in three groups: (i) studies following the Russian tradition that investigate nuances of different prefix usages, (ii) studies following the ``Western'' tradition that aim to find \isi{uniform semantics} (or one function) for all the prefixes (not only in Russian, but in Slavic languages in general), and (iii) studies that try to bridge the gap between the first two approaches. Let me provide a bit more detail about each of these directions of research.

The main question that is addressed in the Russian tradition is nicely formulated by \citet[18]{Boguslawski:63}, who writes that ``the problem of defining all the meanings of `the same prefixes' is first of all a practical problem and is of a great importance for the lexicographic studies". The main purpose of the grammar \citep{Grammar:52, Shvedova:82} and dictionaries \citep{Dict:50, Dict:57}, as well as of many other studies of Russian prefixes (\citealt{Avilova:64, Golovin:59, Lopatin:97, Tixonov:98}, among others) is to examine the data in great detail and provide a full picture of the different usages that a particular prefix may have. As a next step, the type of relation (\isi{polysemy} of \isi{homonymy}) between these usages is analysed \citep{Krongauz:97, Plungyan:01}. This work is necessary, but its focus is on descriptive adequacy and not on finding differences or similarities between different prefixes or explaining why a particular combination of stacked prefixes is available or not.

As for the ``Western'' approach, the main idea they exploit is that Slavic verbal prefixes are markers of perfective aspect (\citealt[see, e.g.,][]{Binnick:91, Krifka:92, Zucchi:99}, among others). Perfective aspect itself then gets analysed in terms of \isi{quantisation} (first proposed in \citealt{Krifka:86, Krifka:92}, and later repeated by \citealt{Pinon:95}), from which it follows that the semantic function of verbal prefixes is to contribute \isi{quantisation}, defined by \citet{Krifka:86} as shown in \defref{def:quant}. 

\theoremstyle{definition}
\begin{definition}{Quantisation}\label{def:quant}
\textit{QUA(P)} $\leftrightarrow \forall x,y[P(x) \wedge P(y) \rightarrow \neg y<x]$\\
A predicate \textit{P} is quantised iff, whenever it applies to \textit{x} and \textit{y}, \textit{y} cannot be a proper part of \textit{x}.
\end{definition}

However, \citet{Filip:92} noticed that matters are more complicated, as there are \isi{perfective verbs} that fail to be quantised according to the \defref{def:quant}. \citet{Filip:92} raised a number of questions in this respect, and proposed that ``the semantic property of the Incremental Theme NPs that is determined by aspect should not be characterised in terms of the ``\isi{cumulative}\slash quantised'' distinction, but rather in terms of the `bounded/unbounded' distinction, which characterises aspect'' \citep[][147]{Filip:92}.

In a next step, \citet{Filip:92} shifted the focus to the contribution of the Slavic linguistic tradition \citep{Wierzbicka:67, Rassudova:75, Merrill:85} and concluded that verbal prefixes must be associated with local quantificational effects\footnote{A-quantification in terms of \citealt{BachPartee:87, BachPartee:95}, which is typically expressed at the sentence level or at the level of VP with sentence adverbs, ``floated'' quantifiers (e.g., \textit{each}), verbal affixes, auxiliaries, and various argument-structure adjusters.} (among other meaning components). \citet{Filip:99} later proposes to analyse Slavic verbal prefixes as \isi{scalar expressions}. This became a departure point for the subsequent analyses \citep{Filip:00, Filip:03, Filip:05, FilipRothstein:05, Kagan:11, Kagan:12, Kagan:13, Kagan:book}. For example, \citet[183]{Filip:99} writes that the prefix \Prefix{na-} ``adds to a verb the meaning of a sufficient or large quantity, or a high degree measured with respect to a certain contextually determined scale and with respect to some standard or subjective \isi{expectation value}.'' Later, \citet{Filip:08} also formulated the general idea that prefixes (at least under certain usages) ``contribute to the specification of the ordering criterion on events'' and proposed to include them in the class of scale inducing expressions. This idea allowed \citeauthor{Kagan:12} (\citeyear{Kagan:12}, \citeyear{Kagan:book}) to further develop the semantic approach to \isi{prefixation} under which ``the major semantic function of a prefix is to impose a certain relation between two \isi{degrees on a scale}''. Various prefixes then differ with respect to the type of the scale they can apply to and the exact relation between the degrees  they establish.\largerpage[2]

Following \citet{Filip:08}, the idea of scalar interpretation of verbal prefixes serves as a bridge between the two traditions: on the one hand, it reveals the common core of the prefixes, and on the other hand, it provides the space for explaining the distinctions between individual prefix usages. 

I propose to use a \isi{scalar approach} to prefix semantics in order to account for another complex issue: \isi{prefix combinatorics}. \citet{Tatevosov:09} correctly notices that descriptive approaches and structuralist theories of the semantics of Russian prefixes, such as \citet{Avilova:64}, \citet{Golovin:59}, \citet{Lopatin:97}, and \citet{Tixonov:98}, did not bring us closer to the understanding of how complex verb formation operates. On this basis, \citet{Tatevosov:09} concluded that a semantic approach is not helpful for predicting the existence and properties of \isi{complex verbs}. This conclusion is, however, not a valid one: an inspiring counterexample is the work by \citet{Filip:03}, who uses the ``\isi{one delimitation per event}'' constraint to motivate the exclusion of some prefix-verb combinations on semantic grounds. This constraint is formulated by \citet[79]{Tenny:94} as ``[t]he event described by a verb may only have one measuring-out and be delimited only once''. It is grounded in the independent restrictions that come from the grammar of measurement in natural languages and it operates across both nominal and verbal domains. 

Taking this as a point of departure, I propose to analyse certain restrictions on the formation of \isi{complex verbs} as \isi{semantic restrictions}. As I have shown in Chapter~\ref{Chapter2} and Chapter~\ref{Chapter4}, a significant amount of data cannot be treated adequately in the syntactic approaches: \isi{biaspectual verbs}, stacking of prefixes, formation of \isi{secondary imperfective} verbs. I propose to look at these processes from a different angle, taking into account the semantics of verbal prefixes. I will show that the scalar semantic approach can be successfully used to motivate stacking of prefixes (as well as the existence of \isi{biaspectual verbs} and certain restrictions on the formation of \isi{secondary imperfective} verbs). A formalism that allows us to restrict derivations on the basis of \isi{semantic constraints} is then required. 

%My intuition here is similar to that of \citet{Kagan:book}: prefixes are related to \isi{scales} and introduce some delimitation of events. Due to these delimitations events may become telic and perfective. 

%This intuition is also related to other ways of thinking about \isi{prefixation} presented in the literature, including syntactic accounts. For example \citet{Tatevosov:09} constraints prefixes in the selectionally limited group in that they cannot be attached to a formally perfective verb. In the account I propose, the impossibility of such an attachment is explained via the incompatibility of the \isi{semantic restrictions} associated with the verb and the prefix. This explains those cases that are exceptional from the syntactic point of view, as it turns out that \isi{semantic restrictions} are compatible. This is, for example, the case of combining two prefixes with the \isi{delimitative} function: the \isi{delimitative} prefix \Prefix{po-} cannot be attached to a perfective verb, unless this verb is already prefixed with another \isi{delimitative} prefix. We will discuss such examples in detail later in this chapter.
 
%The crucial difference between the previous semantic accounts of Russian verbal \isi{prefixation} and this work is the different formulation of the research question and the formalization of the semantics of prefixes, which will be provided in Chapter~\ref{Chapter7}. An important property of the formal part of the account is its capability to grasp by means of the semantic representation not only the semantics the prefix contributes to the derived verb, but also the \isi{semantic restrictions} on its attachment. In this chapter I will prepare the ground for this formalization, providing informal descriptions of the semantics and attachment restrictions of individual prefixes after carefully investigating their properties. 

The goal of this chapter is to motivate intuitions about the behaviour of individual prefixes and provide informal semantic analyses of the prefixes under discussion in such a way that their combinatorial properties derive naturally from their semantic properties. 
This discussion provides the basis for the formalisation of prefix semantics that will follow in Chapter~\ref{Chapter7}. The prefixes that we are going to look at are the following: \Prefix{za-} (\isi{inchoative} usage), \Prefix{na-} (accumulative usage), \Prefix{po-} (\isi{delimitative} and \isi{distributive} usages), \Prefix{pere-} (iterative, \isi{distributive}, and \isi{excessive} usages), and \Prefix{do-} (\isi{completive} usage). I will occasionally mention some of the extra usages of the discussed prefixes, but analysing them, as well as other prefixes, is beyond the scope of this book.

For each prefix, the structure of the respective subsection is the same, covering three important issues and followed by a summary:
\begin{enumerate}
\item semantic contribution;
\item restrictions on the attachment: (in)compatibility of lexical semantics of verbal stems with prefix semantics;
\item subsequent imperfectivisation of a verb with the discussed prefix;
\item summary.
\end{enumerate}

%In the sections \ref{subsection:semantics:za}-\ref{subsection:semantics:do} I discuss contributions of individual prefixes on the pre-formal level. This discussion provides the basis for the formalization of prefix semantics. The goal of the formalization I propose is not only to adequately capture the semantics of the discussed prefixes, to be able to predict whether verbs containing certain combinations of a verbal root and prefixes exist and which aspect and semantics such verbs have. 

Before we proceed, I would like to note that shifting the focus from the \isi{syntactic restrictions} to the semantic ones in the domain of \isi{prefix stacking} does not mean that no syntactic theory of verbal structure is needed. There still remain constraints that are better formulated in (morpho-)syntactic terms. An example of such a constraint is the unavailability of multiple \is{imperfective suffix}imperfective suffixes in Russian. 

Another module that is involved in regulating complex verb formation in Russian is pragmatics. I propose some preliminary pragmatic explanations for the non-existence of certain verbs in this chapter and provide some more details in Chapter~\ref{Chapter6}.

As \isi{scales} are crucial in the analysis of the prefixes, let me provide a brief overview of the concept before discussing the properties of individual prefixes.

\section{Scales}

%Studies that look at the relation between telicity and either measure (Tenny 1994), or scale (Filip 93/99, 2004, Hay, Kennedy, Levin 99, Beavers 06, 07, Wehsler 05), or \isi{incremental theme} (Dowty 91, Krifka 89, 92,98), or quantity criterion (Filip 2005), or ordering criterion (Filip-Rothstein 06)

%Rappaport Hovav: three kinds of \isi{scales} recognized in the literature: property \isi{scales}, path \isi{scales}, volume/extent \isi{scales}
The original area of application of \isi{scales} in linguistics is the domain of \isi{gradable adjectives}. As suggested by \citet{Kennedy:99}, \isi{gradable adjectives} (e.g., \textit{wide, tall, expensive}) denote properties that for different individuals hold to different degrees. This means that they are analysed as denoting a certain relation between an individual-type and a degree argument. One formalisation of this idea is that an adjective lexicalises a \textit{scale} and maps its argument to a certain degree on that scale \citep{Kennedy:01, KennedyLevin:02}. An alternative formalisation \citep[e.g.,][]{Heim:00} represents such adjectives as taking a degree as an argument and providing as its output the set of the individuals for which the lexicalised property holds up to this degree.

 A \textit{scale} is defined as a set of points (degrees, values), %\footnote{The term \textit{degree} is used to refer to either points or intervals on the scale (see, e.g., \citealt{KennedyMcNally:05}), so we will refer to the either points or values on the scale to avoid confusion.}
 totally ordered along some \textit{dimension} (e.g., length, quantity, volume, \isi{duration}). If the scale has maximal and minimal elements, it is a \textit{totally closed} scale (often called just \textit{closed} scale). If the scale has neither a maximal nor a minimal element, it is a \textit{totally open} (or just \textit{open}) scale. Scales that have a minimal and lack the maximal element are \textit{lower closed} and \isi{scales} that lack the minimal and have the maximal element are \textit{upper closed}. These properties play an important role in accounting for the adjectival semantics \citep[see, e.g.,][]{KennedyMcNally:05, RotsteinWinter:04, KaganAlexeyenko:10}.
 
Other central notions in this domain are that of \textit{\isi{comparison} class} and \textit{standard of comparison}. The relevant \isi{comparison class} \citep[see, e.g.][]{Klein:80, KennedyMcNally:05, Kennedy:07} is constituted of objects similar to the individual argument in the relevant respects. The \isi{comparison class} then provides the \isi{standard of comparison} and the sentence like \ref{ex:expensive} is interpreted as asserting that the price of the house is higher then the standard price of a house from the \isi{comparison class} (houses with similar parameters in the same area).
 
 \ex.\label{ex:expensive}This house is expensive.

Comparative adjectives, such as in \ref{ex:expensive:more}, differ in that they overtly specify the \isi{comparison class} and, thus, the \isi{standard of comparison}. 

 \ex.\label{ex:expensive:more}This house is more expensive than the one we saw yesterday.
 
Differential degrees (\citealt{Kennedy:01}, also called difference values in \citealt{KennedyLevin:02}) and the operation of degree addition \citep{KennedyLevin:02} allow to represent the semantics of such sentences as \ref{ex:expensive:more:1000} by explicitly stating how the relevant degrees of the individuals are related.
 
 \ex.\label{ex:expensive:more:1000}This house is five thousand dollars more expensive than the one we saw yesterday.

The \isi{scalar approach} to the semantics of event predicates has proven to have a considerable explanatory power and has been advocated in numerous works on event semantics \citep[see, e.g.,][]{Ramchand:97, Hay:99, KennedyLevin:02, CaudalNicolas:05, FilipRothstein:05, Kearns:07, KennedyLevin:08, Filip:08, Pinon:08, Rappaport:08, Rappaport:11, McNally:11}. Let me provide a very brief overview of the works that adopt a \isi{scalar approach} in order to account for the aspectual properties of event predicates (for a more detailed observation and extra references see \citealt{Arsenijevic:13}).

The first class of verbs that has been explored from the scalar semantics perspective is the class of \isi{degree achievements}, such as \textit{cool}, \textit{grow,} or \textit{widen}. The crucial difference between adjectives and \isi{degree achievement} verbs is that while the former map individuals to degrees, the latter denote a change in degree: the degree to which the argument possesses the property at the end of the event is higher than at the beginning. Therefore, a \isi{temporal argument} has to be introduced \citep{Hay:99, KennedyLevin:02}.

%\ex.\label{ex:Kennedy2}\a.Kim walked from the bank to the store in/??for an hour.
%\b. Kim walked for/??in an hour.\\
%= (4.4) in \citealt{Kennedy:12}, p. 104

In a next step, scalar approaches to \isi{degree achievements} were integrated with earlier approaches to the aspectual composition. The theory of aspectual composition has been developed based on the observations about the behaviour of \textit{incremental theme} verbs. Such verbs are characterised by referring to eventualities that involve an \isi{incremental change} that is related to the \isi{internal argument} \citep[see][]{Garey:57, Wierzbicka:67, Verkuyl:72, Krifka:86, Krifka:92, Filip:92, Filip:99}. An example of a verb of an incremental creation is provided in \ref{ex:Kennedy1}. An important observation is that when the \isi{incremental theme} has some specified quantity, the predicate is telic; when there is no such specification, the predicate is atelic.

%verbs of directed motion (as \textit{ascend}) and  were included in the class of verbs that receive scalar interpretation. Verbs of motion, according to this view, are associated with an increase of a degree on the \textit{path} scale. Incremental theme verbs 

\ex.\label{ex:Kennedy1}\a. Lee wrote a poem in/??for an hour. \hfill telic
\b. Lee wrote poetry for/??in an hour. \hfill atelic\\
\hbox{}\hfill\hbox{= ex.~(4.1) in \citealt[103]{Kennedy:12}}


%The analysis of \isi{incremental theme} verbs has been later generalized to capture other types of verbs that demonstrate similar behaviour: motion verbs, as in \ref{ex:Kennedy2}, and 
%
%\ex.\label{ex:Kennedy3}\a. The canyon widened 30 kilometers in/??for one million years.
%\b. The canyon widened for/??in one million years.\\
%= (4.5) in \citealt{Kennedy:12}, p. 104

Later, \citet{Filip:05} has shown that the basic meaning of an \isi{incremental theme} verb in English does not introduce a scale. This approach has been adopted by \citet{Rappaport:08}, \citet{LevinRappaport:10}, \citet{Kennedy:12}, and \citet{Bochnak:13} who concluded that measure of change functions must be associated with the \isi{incremental theme} arguments. These arguments then supply some value that is used to select an appropriate portion of the scale that has to be covered in course of the event. 

Now let us describe additional kinds of scale types that will be relevant for the following discussion. First of all, I want to distinguish two types of situations involving a change along a scale: for the first type, the absolute value on the scale matters, and for the second type, the absolute values are not important and we are only interested in the difference between the values at the beginning and at the end of the event. For example, if John heated the water up to 40 degrees Celsius, it is the absolute value that matters, and if John gained 2 kilos it is the difference that is relevant. We will say that the first event proceeds along the \textit{temperature} scale (I will call the class of such \isi{scales} \textit{proper scales}) and the linguistic \isi{context} supplies the \textit{maximum value} on that scale. In the second case, we will say that the event proceeds along the \textit{measure of change scale} for \textit{weight} and the direct object provides the \textit{measure of change value}. 

I adopt the notion of the \textit{measure of change scale} from \citet{KennedyLevin:08} and \citet{Kennedy:12}. The \isi{measure of change scale} for \textit{weight} is of course related to the proper \textit{weight} scale, whereby the zero point (which is also the minimum point in this case) on the \isi{measure of change scale} corresponds to the value on the \textit{weight} scale at the beginning of the event. The point that is related to the end of the event may be not straightforwardly related to the measure of change: in the basic case, it can be represented as a sum of the value on the scale at the beginning of the event and the measure of change. If John gained 2 kilos and his weight before this event was 70\,kg, his weight at after the event of gaining weight is 72\,kg. This leads to the idea of keeping only the \isi{proper scale} in the semantic representation and express changes in terms of the difference between the absolute values, as it is done by \citet{Kennedy:01} and \citet{KennedyLevin:02} by means of differential degrees and degree addition. However, there are cases where the connection is not so straightforward.

To illustrate the last point, let us consider a lexicalised example of measure of change/\isi{proper scale} opposition: the \isi{duration}/time pair. Duration can be seen as, but it is not reducible to, a difference between two time points. For example, the event denoted by \ref{ex:John:dance} can consist of ten weekly one-hour classes. In this case the \isi{duration} is the sum of the (approximate) durations of individual events, but not the difference between the time the first class started and the last class ended. 

\ex.\label{ex:John:dance}John took ten hours of dance classes.

\ex.\label{ex:Mary:bike}Mary did two hours of biking on Sunday. 

One can argue that such a case is special as multiple \isi{subevents} are involved. Indeed, in the case of ten hours of dance classes we can represent the whole event as a \isi{series of} events. This solution is not so obvious in case \isi{subevents} that do not naturally form a series: if \ref{ex:Mary:bike} is true, it could have been that Mary did two hours of continuous biking, or that she did one hour in the morning and one hour in the evening, or her whole day was full of small trips that resulted in a \isi{cumulative} biking time of 2 hours (probably calculated by a fitness-tracker that also counted very short trips). I think that the semantic representation of the sentence should be neutral with respect to these scenarios, so I propose to keep distinct representations of time and \isi{duration} as well as other proper and measure of change related attributes. This allows us to leave the relation between the \isi{proper scale} and the \isi{measure of change scale} underspecified.

In case it seems that the discussion above is only relevant to the \isi{duration}/time pair and not to the other types of \isi{scales}, let me provide one more example. A hiking guidebook usually provides information about the elevation gain on the route. If one looks at the description of the circular route, the elevation gain will be positive (theoretically it can be also 0, but it is very improbable). At the same time, the difference between the elevation level at the start and at the end of the event is 0. In such situations, we are dealing with three different \isi{scales}: a proper elevation scale that has heights as its points, the elevation \isi{measure of change scale}, that represents the difference between the elevations of the start and the end points of the path, and the elevation gain scale that represents \isi{cumulative} elevation gain on the route. From the example \ref{ex:elevation} we can conclude that English does not distinguish between the last two situations, as \ref{ex:elevation} can be interpreted as either the net elevation gain or the \isi{cumulative} elevation gain of 1000 metres took place. 

\ex.\label{ex:elevation}The group of tourists went up a thousand metres up today.

\exg.\label{ex:elevation:rus}20 aprelja my podnjalis' na tysja\v{c}u pjat'sot metrov.\\
20 april we ascend.\glb{pst.pl.refl} on thousand five.hundred metres\\
\trans `On April 20 we made a 1500 metres ascent/reached the 1500 metres elevation.' (translation without \isi{context})\\
`By April 20 we had risen to an average level of 1,500 metres.' (English original)~~\hbox{}\hfill\hbox{\textit{Twenty thousand leagues under the sea}, Jules Verne, 1870}

As for Russian, some expressions can be interpreted using all the three \isi{scales}: sentence \ref{ex:elevation:rus} is most naturally interpreted with respect to one of the measure of change \isi{scales}, although it is a translation of the English sentence that refers to reaching a depth of 1500 metres (by ascending). On the basis of such observations, I would like to have the means for both the \isi{underspecification of the scale} and the co-existence of various types of \isi{scales} without hard connections between their points. For example, the semantic representation of \ref{ex:elevation:rus} should only contain the information that the \isi{maximum point} of the scale of the type \textit{elevation} is equal to 1500 metres without specifying whether this is a proper or a \isi{measure of change scale}. If more information is available, as in \ref{ex:elevation:top}, both the measure of change (400 metres) and the elevation scale (with a marked point on 1917\,m) should be visible in the semantic representation.

\ex.\label{ex:elevation:top}We gained another 400 metres and reached the top of Mount Washington.

In sum, the crucial difference between the measure of change and the \isi{proper scale} types is that only the latter type is directly bound to some parameters of the world, whereas for each \isi{measure of change scale} there exist multiple proper \isi{scales} it can correspond to. I claim that some of Russian prefixes are sensitive to this property, so in my analysis I will distinguish not only between open\slash closed\slash \isi{upper-closed} and various dimensions of the scale, but also between proper \isi{scales} (my term) and measure of change \isi{scales} (term borrowed from \citealt{KennedyLevin:08}). It is also possible to relocate this property from the scale level to the level of the event: in this case a \textit{proper scale} event would be an event for which each degree on the scale is mapped to a unique time point, and a \textit{measure of change} event would only require the extreme points to be mapped to different time points. In the proposal presented here I leave the proper/measure of change feature on the level of scale properties, although the event level could be conceptually more appropriate. This issue should be addressed in further research.

%Another important observation is that there is no unique mapping between the portions of event and the non-extreme points on the \isi{measure of change scale}. Consider the sentence \ref{ex:Kennedy1} again. It conveys the following information: at some moment of time the poem did not exist and one hour later it was fully written. It does not tell us how the process was organized in between. There is some freedom in the selection of the dimension: it can be the length of the poem in symbols, words, or lines (a \isi{spatial} extent scale), it can be a completeness scale that goes from 0$\%$ to 100$\%$, it can be a binary scale that represents non-existence as 0 and existence as 1. Even if we choose a scale that is \isi{non-binary}, there is no guarantee that the event proceeds through the scale really incrementally: in the coarse of writing a poem some lines can be written and then erased and the poem can become shorter, maybe even without becoming longer again. The completeness scale is the scale along which the writing proceeds incrementally, but these is no information about how intermediate points on that scale are mapped onto the \isi{subevents}. This leads to the conclusion that from the point of view of the quantity of information such \isi{scales} are equivalent to binary \isi{scales}. 

%A similar observation can be made for \isi{degree achievements}: when clothes are hang outside to dry in the evening and the drying begins, then it rains during the night but they still become dry in the morning, the whole event can be described by \ref{ex:dry}. 

%\ex.\label{ex:dry}The clothes dried in 12 hours. 

%So when we analyze the sentence that involves a direct object like \textit{twenty songs,} we can talk about scalar analysis of the verb already when we assume that at the beginning of the event no songs have been listened and in the end all the twenty songs are listened, but there is no ordering on the parts of the set of songs neither any requirements that \isi{subevents} of listening to individual songs do not overlap: in the extreme case, all the twenty songs could have been listened simultaneously. This situation is similar to a situation of a closed two point scale. This means that the situation when the direct object only provides one value is similar to the situation of a two point scale: there is a transition between those two points, but as we do not have any information about how the event proceeds between those points, we can


%As throughout this chapter I am going to say things like `the direct object contributes a scale' I want to talk a bit about what does it mean. Although it seems to be very natural to think about peeling 5 kilos of potatoes as proceeding gradually along the \textit{amount} scale, the connection between the referent of the direct object and stages of the event is not transparent. On one hand, \isi{scales} themselves are ordered (according to the definition), so for any two points on the scale that we will be using to talk about the event progress we know which of them precedes the other. On the other hand, direct objects do not provide \isi{scales}, but only supply values that can be used to construct a scale. In the potato peeling example there is some amount of the object and it all has to be involved in the event, so we can say that the event proceeds along the \textit{measure of change} scale for \textit{amount}, starting with 0 and ending with the full amount. This does not imply that there is any ordering on the parts of the direct object such that some of them get involved in the event earlier than others. In other words, for any two potatoes among those in the 5 kilo package we do not know which of them will be peeled earlier. So \textit{amount} itself is a scale, but what is contributed by the direct object is only a value on that scale and no mapping of the intermediate points on the scale onto the parts of the direct object is provided. It is also not required that the events proceeds gradually through the parts of the direct object: if Mary listened to 20 songs she could, in principle, have listened them at once. 

%Not all the \isi{incremental theme} verbs are as complex. Eating an apple, for example, is supposed to proceed incrementally along the volume scale. Even this is not so straightforward, though. If one bites an apple, the volume that is eaten should increase at that moment. But what is they take the bite out of the mouth and put it on the table and then actually chew and swallow after the rest of the apple is consumed? What I want to say with this is that there is no linguistic information that can provide a cue about the coarse of the event. It is only the start and the end of the event that we know something about. 


%The situation if different with \textit{path} and \textit{time}. They are \isi{scales} themselves, not points that allow to select an appropriate portion on the \isi{measure of change scale}. 

%from Beavers:12\\
%(56) a. Caesar wiped the table clean (in/?for an hour).
%b. Caesar wiped the table (cleaner and cleaner) (for/??in an hour).
%
%Furthermore, progress towards cleanliness here may allow backtracking and
%stopping — something may get cleaner and dirtier on its way to cleanliness,
%and may spend some time at various levels of cleanliness. Looping, though,
%is not allowed, since the scale is one dimensional (as per fn. 20). These
%interpretations further motivate an MR-type analysis.
%]]
%
%These data also suggest that result XPs like clean, a fierce red, and
%to a nice shine can supply endpoints on \isi{scales} just as goal PPs supply
%endpoints on paths, i.e. result XPs and goal PPs are two manifestations of
%the same notion. Further evidence for this comes from the fact that goal
%and result modifiers have similar effects on durativity. Recall that simplex
%paths give rise to punctual motion predicates, while complex paths give
%rise to durative motion predicates, for an atomic theme. For change-ofstate
%predicates, some result XPs likewise determine punctuality, while
%others determine durativity. This is shown in (57), where shot the sheriff
%is punctual with the result XP dead but durative with to death.
%(57) a. Wyatt shot the sheriff dead in five minutes. (after)
%b. Wyatt shot the sheriff to death in five minutes. (after/during)
%
%--------------

%An obvious exception is time: if the direct object contains an attribute related to the time, the \isi{subevents} are automatically ordered. For example, if John is going to run for twenty minutes, he will first run for the first of those minutes, then for the second, than for the third, etc. This means that in case of the property and extent \isi{scales} 
%
%The \textit{path} \isi{scales} are in between the \textit{time} \isi{scales} and other types of sales. On one hand, there are multiple ways to from the bank to the store that are acceptable to refer to with \ref{ex:Kennedy2}. On the other hand, there is an order of the points of each path, so if a particular path is selected the situations turns out to be similar to that for the \textit{time} scale. Intuitively, if the event of going from the bank to the store described by \ref{ex:Kennedy2} is close to its end, we know that Kim not only covered most of the \isi{distance} between the origin and the destination but is also located at some point that is close to the store and not to the bank. In case of peeling potatoes when the event is close to the termination point we know that the peeled area of potatoes is close to the total area to be peeled but the probability of a certain potato to be peeled at the moment is equal for all potatoes.   there is an \isi{internal} order of points that can be captured by the following construct. Take the length of the shortest existing path (that goes through all the relevant intermediate destinations) between the bank and the store. Mark this length as a \isi{maximum point} on the \isi{measure of change scale} for \textit{distance}. Now any path between the bank and the store (even those that have a different length) can be mapped onto the marked chunk on the \isi{measure of change scale}, providing equivalence classes for the different points in space\footnote{The procedure is trickier in case there are intermediate destinations, e.g. when Kim is the owner of the store and has to pick up a key that she has at home on the way from the bank to the store which results in going through some point $x, y$ (space coordinates) two times: one before Kim picks up the key and one after that. In this case we have to add a dimension (just a simple one that has a $+/- key$ feature only) so that we are able to distinguish the equivalence class of the $(x, y, -key)$ point from that of $x, y, +key$ point.}.
%
%It is not important whether a banana is eaten from the top to the bottom or from the bottom to the top, it can be still described by the same predicate. It is important whether the path from the bank to the store is passed in one direction or in the other, the sentence \ref{ex:Kennedy2} is true only in one case. On the other hand, if there is a \isi{distance} description instead of the path, it behaves in the same way as the information about the quantity of the direct object: the sentence \ref{ex:ran} is true independently of the direction of running. In this case, five kilometers serve not as a scale, but as the \isi{maximum point} on the \isi{measure of change scale}
%
%\ex.\label{ex:ran}John ran 5 kilometers.
%
%The distinction between the situations when the direct object or the \isi{context} provide a point that allows to identify the appropriate interval on the \isi{measure of change scale} and the situations when the direct object supplies directly the scale along which the event proceeds will be 

%A general idea underlying the \isi{scalar approach} to event semantics is that an \isi{open scale} in the event semantic structure leads to atelic interpretation and a closed scale leads to a telic interpretation (see, e.g., \citealt{KennedyLevin:08} or \citealt{Kearns:07}). The correlation is very prominent when one studies the variable behaviour of \isi{degree achievements}.

%Similarly, Piñón (2005, 2008) and Caudal and Nicolas (2005) propose degree-based accounts of aspect, which they apply to different predicates. 

%Levin and Rappaport Hovav (2006, and subsequent work) argue that the notion of telicity can in general be associated with scalar change. Also in this volume, they make a distinction between scalar change associated with particular verbs (their result verbs), which are basically change of \isi{state verbs} such as break or open, and non-scalar change (their manner verbs). In the latter case, however, a scale can be introduced by the \isi{internal argument} (e.g. with \isi{incremental theme} verbs, on which see also Kennedy 2012) or by a path phrase (with motion events). Again, if the scale is bounded, the eventuality is interpreted as telic. Beavers (2008) builds on this system and adds the important observation that \isi{scales} can be simple (a transition between two states with no intermediate states, as in achievements) or complex (as in accomplishments).

%There are a lot of proposals concerning the treatment of \isi{scales} in the literature. I will not go into many details and peculiarities and will not discuss different approaches to formalization of scalar structures. I will only introduce those notions and distinctions that are relevant for the discussion in this chapter.
%
%So what we are going to need: 
%\begin{enumerate}
%\item the concept of the \textit{scale} (definition);
%\item the notion of the \textit{dimension} of the scale;
%\item the distinction between \textit{open} and \textit{closed} \isi{scales};
%\item the notion of a \textit{measure of change} scale.
%\end{enumerate}


\section{\textit{za-}}\label{subsection:semantics:za}
\subsection{Semantic contribution}
There are three main uses of the prefix \Prefix{za-} as described in the dissertation by \citet{Braginsky:08}: \isi{spatial}, \isi{resultative} and \isi{inchoative}. The \isi{resultative} meaning is further subdivided into four categories that Braginsky calls \textit{accumulative, cover, damage} and \textit{get}. He shows that different usages of \Prefix{za-} can and should be analysed in a unified way. Braginsky argues convincingly that it is not the case that these meanings apply to all verbs indiscriminately, nor is it the case that they are distributed across specific verbs. So a particular verb does not have to be compatible with any meaning of \Prefix{za-} nor does it have to have at most one interpretation when prefixed with \Prefix{za-}.

I will, however, limit my remarks to the \isi{inchoative}\footnote{I follow \citet{Braginsky:08} and adopt the term \textit{\isi{inchoative},} that he takes from \citet{Zemskaja:55} and \citet{Zaliznjak:95}. There are alternative terms in the literature, referring to the same usage of \Prefix{za-}, such as \textit{inceptive} or \textit{ingressive}, see also the relevant discussion in \citealt{Maslov:65}.} use of \Prefix{za-}, that is considered \isi{superlexical}. The analysis provided here is extendable to other uses of \Prefix{za-}. For example, \cite{ZinovaOsswald:paper} cover the case of the \isi{spatial} interpretation of the prefix \Prefix{za-}. The extension to the \isi{resultative} uses is also possible, but requires some more work in order to define the procedure of selecting a scale along which the event is measured. Some of the \isi{resultative} usage cases are covered in \citet{Zinova:14}, a paper which deals with the locative alternation in Russian and English. The approach presented there is concerned with the `accumulative' and `cover' subclasses of the \isi{resultative} meaning of \Prefix{za-}, but does not include the `damage' type of meaning (see \citealt{Braginsky:08} for more details about the classification of the \isi{resultative} sub-meanings).

As for the description of the semantics of the \isi{inchoative} \Prefix{za-}, \citet{Braginsky:08} writes \citep[following][]{Sheljakin:69} that ``the function of the \isi{inchoative} ZA- is to ensure that a given process\slash state, denoted by an input verb, has passed from the state of non-existence into existence.'' Importantly, there are no restrictions imposed by \Prefix{za-} on the \isi{duration} of the process or the state that is initiated.

\subsection{Restrictions on the attachment}
There has been a good deal of discussion about the types of verbs that serve as input for \isi{prefixation} with the \isi{inchoative} \Prefix{za-} \citep{Isachenko:60, Zemskaja:55, Sheljakin:69, Zaliznjak:95, Braginsky:08}. Most of the work focuses on listing different types of possible derivational bases, but as this list turns out to be rather long and is still unlikely to be complete, I will try to approach the problem from the other side and concentrate on listing the restrictions on the derivational bases.

When one thinks about the \isi{inchoative} semantics of the prefix \Prefix{za-}, the obvious restriction on the \isi{derivational base} that will be prefixed with it is the presence of a \isi{time scale} in the verbal semantic structure. On one hand, it seems that all verbs are connected to a \isi{time scale}. On the other hand, there are indeed verbs that cannot be combined with the \isi{inchoative} \Prefix{za-} and such verbs seem to be not non-eventive predicates. Let us first explore the literature on this point.

\citet[275]{Braginsky:08}, based on proposals by \citet{Sheljakin:69} and \citet{Paducheva:96}, formulates the following conditions that have to hold in order for the verb to be incompatible with any of the core meanings of \Prefix{za-}:
\begin{enumerate}
\item the verb is not compatible with expressing motion into some location;
\item the verb does not have theme arguments;
\item the verb is not localised in time or the event denoted by the verb holds for extra-long intervals.
\end{enumerate}

The first condition captures verbs that are combined with \Prefix{za-} in its \isi{spatial} meaning and the second condition plays a role if one wants to attach the \isi{resultative} \Prefix{za-} to the \isi{derivational base}. What is interesting for us here is the third condition, as it refers to the \isi{inchoative} usage of the prefix \Prefix{za-}. According to \citet{Paducheva:96}, three classes of verbs are not compatible with the meaning of initiation:

\begin{enumerate}
\item State verbs that denote \isi{atemporal} properties/relations, i.e., cannot be localised at specific time moment or interval: \textit{stoit'}$^{\IPF}$ `to cost', \textit{vesit'}$^{\IPF}$ `to weigh', \textit{zna\v{c}it'}$^{\IPF}$ `to mean', \textit{imet'}$^{\IPF}$ `to have'.
\item State verbs denoting steady situations,  i.e., hold for extra long temporal intervals: \textit{golodat'}$^{\IPF}$ `to hunger', \textit{ljubit'}$^{\IPF}$ `to love', \textit{gorditsja'}$^{\IPF}$ `to feel proud', \textit{znat'}$^{\IPF}$ `to know'.
\item Activity verbs denoting occupation and behaviour: \textit{\v{z}it'}$^{\IPF}$ `to live',\linebreak \textit{pravit'}$^{\IPF}$ `to rule', \textit{u\v{c}itel'stvovat'}$^{\IPF}$ `to work as a teacher', \textit{filosovstvovat'}$^{\IPF}$ `to philosophise'.
\end{enumerate}

\citet{Paducheva:96} also writes that verbs denoting \isi{atemporal} properties do not occur with \isi{punctual time} or \isi{duration modifiers} (e.g., \textit{sej\v{c}as} `now', \textit{vsegda} `always', \textit{X dnej} `for X days'). This seems reasonable if there is no \isi{time scale} made available for these verbs, but it turns out to be an invalid observation: examples in \ref{ex:atemporal:adv} illustrate successful combinations of verbs listed above with such modifiers.


\ex.\label{ex:atemporal:adv}\ag.Moloko sej\v{c}as stoit 60 rublej za litr.\\
milk now cost.\glb{pres.sg.3} 60 rubles for litre\\
\trans `Milk costs 60 rubles per litre now.'
\bg.Takaja formulirovka vsegda zna\v{c}it otkaz.\\
such formulation always mean.\glb{pres.sg.3} rejection\\
\trans `Such a formulation always means a rejection.'
\bg.On vesil 100 kilogramm 5 let.\\
he weigh.\glb{pst.sg.m} 100 kilos 5 years\\
\trans `He weighed 100 kilos for 5 years.'

A similar problem occurs with the observations made by \citet{Paducheva:96} about the verbs denoting steady states. \citet{Paducheva:96} writes that they are incompatible with punctual (as \textit{v X \v{c}asov} `at X hours'), frequency (as \textit{dva\v{z}dy} `twice', \textit{inogda} `sometimes') and intensive \isi{duration}  (as \textit{ves' den'} `all day long') modifiers. The examples in \ref{ex:steady:adv} illustrate that at least some of the verbs belonging to that class are compatible with some of these modifiers.

\ex.\label{ex:steady:adv}\ag.On ljubil dva\v{z}dy: v 18 i v 35.\\
he love.\glb{pst.sg.m} twice: in 18 and in 35\\
\trans `He loved twice: when he was 18 and when he was 35.'
\bg.On gordilsja synom ves' den', poka ve\v{c}erom oni ne porugalis'.\\
he feel.proud.\glb{pst.sg.m} son.\glb{instr} whole day, until evening they not argued\\
\trans `He felt proud of his son for the whole day, until they had an argument in the evening.'

Another observation is that if verbs like \textit{stoit'}$^{\IPF}$ `to cost' or \textit{zna\v{c}it'}$^{\IPF}$ `to mean' were \isi{atemporal} and verbs like \textit{ljubit'$^{\IPF}$} `to love' were not semantically compatible with time descriptions, then the sentences in \ref{ex:steady:time} would not be acceptable.\largerpage

\ex.\label{ex:steady:time}\ag.No vposledstvii my uvidim, kak i pod kakimi vlijanijami \`{e}tot obraz u nego razvilsja i \v{c}to stal zna\v{c}it'.\\
but later we will.see, how and under which influence this image of he.\glb{gen} develop.\glb{pst.sg.m}.refl and what become.\glb{pst.sg.m} mean.\glb{inf}\\
\trans `But we will later see how, and under what influences, this image developed in him, and what meaning it began to acquire.'\\\hbox{}\hfill\hbox{V. F. Xodasevi\v{c}. \textit{Esenin} (1926)}
\bg.Cement-to voob\v{s}\v{c}e be\v{s}enye den'gi stal stoit'!\\
cement-somehow {at all} mad money become.\glb{pst.sg.m} cost.\glb{inf}\\
\trans `Moreover, cement somehow started to cost a crazy amount of\linebreak money!'\hbox{}\hfill\hbox{Roman Sen\v{c}in. \textit{Elty\v{s}evy} (2008)}
\bg.Lida zdravo ob'jasnjala, \v{c}to tak ne byvaet, \v{c}toby v\v{c}era ljubil, a segodnja zabyl.\\
Lida soundly explained, that so not be.imp.\glb{pres.sg.3}, that yesterday loved, but today forget.\glb{pst.sg.m}\\
\trans `Lida sensibly explained that it cannot be that today he forgot the person he loved yesterday.'\\\hbox{}\hfill\hbox{Nina Gorlanova. \textit{Filologi\v{c}eskij amur} (1980)}

In sum, verbs of these three classes are special in the sense of the relation to the \isi{time scale}, but not ``\isi{atemporal}'': they are compatible with time specifications. \citet[132]{Paducheva:96} herself notes that ``[m]nogie glagoly javljajutsja ili ne javljajutsja \isi{atemporal}'nymi v zavisimosti ot tipa subjekta'' (many verbs are or are not \isi{atemporal} depending on the type of the subject). As an example she points to the verb \textit{stojat'} `to stand' that is, according to her, \isi{atemporal}\footnote{According to \citet{Paducheva:96} incompatibility with the adverbial \textit{sej\v{c}as} `now' is diagnostic of \isi{atemporality}.} only when used with non-animate subjects, as in \ref{ex:stojat':xram} and not with animate subjects, as in \ref{ex:stojat':Vasja}.\largerpage

\ex.\label{ex:stojat':xram}\ag.Xram stoit na xolme.\\
church stand.\glb{pres.sg.3} on hill\\
\trans `The church stands on the hill.'
\bg.\label{ex:stojat':xram2}$^?$Xram sej\v{c}as stoit na xolme.\\
church now stand.\glb{pres.sg.3} on hill\\
\trans `The church now stands on the hill.'

\ex.\label{ex:stojat':Vasja}\ag.Vasja stoit na xolme.\\
Vasja stand.\glb{pres.sg.3} on hill\\
\trans `Vasja stands on the hill.'
\bg.Vasja sej\v{c}as stoit na xolme.\\
Vasja now stand.\glb{pres.sg.3} on hill\\
\trans `At the moment, Vasja stands on the hill.'

In fact, the verb \textit{stojat'} `to stand' exhibits some \isi{atemporality} (or, better, it is not compatible with the adverbial \textit{sej\v{c}as} `now') only when it is combined with certain types of subject. Consider the noun \textit{kniga} `book'. Example \ref{ex:stojat':kniga} illustrates that the combination of the verb \textit{stojat'} `to stand' with the non-animate subject \textit{kniga} `book' and an adverbial \textit{sej\v{c}as} `now' is possible.  In my view, this is clear evidence that ``\isi{atemporality}'' is not a property of a verb, but part of world knowledge: it is hard to imagine the church moving around in the normal world, so it does not make sense to utter \ref{ex:stojat':xram2}. The sentence becomes fine if uttered in a world where buildings can disappear and reappear at a nearby location. There are also cases when similar sentences can be uttered to describe a situation in our world: for example, there are some famous houses in Moscow that were moved to allow to widen the road. Another possibility is a change in the landscape: a small island may turn out to be a hill if the water level drops. 

Note that if the word order (and, thus, the \isi{information structure}) is changed in such a way that the hill becomes the focus of the sentence, as in \ref{ex:stojat':xram3}, the initial sentence \ref{ex:stojat':xram2} becomes unmarked also if uttered in the real world in non-exceptional situations. This favours the hypothesis that the problem with  sentence \ref{ex:stojat':xram2}, noticed by \citet{Paducheva:96}, is not due to the semantic properties of the verb \textit{stojat'} `to stand'. It also seems reasonable to suggest that the same applies to similar verbs in other languages. 

\ex.\label{ex:stojat':kniga}\ag.Kniga stoit na polke.\\
book stand.\glb{pres.sg.3} on shelf\\
\trans `The book is on the shelf.'
\bg.Kniga sej\v{c}as stoit na polke.\\
book now stand.\glb{pres.sg.3} on shelf\\
\trans `The book is on the shelf.'

\exg.\label{ex:stojat':xram3}Na xolme sej\v{c}as stoit xram.\\
on hill now stand.\glb{pres.sg.3} church\\
\trans `On the hill there is now a church.'

Let us now examine the incompatibility of the \isi{inchoative} prefix \Prefix{za-} with verbs denoting \isi{atemporal}/steady situations or occupations. At the first glance, verbs like \textit{*zastoit'} (\textit{za+stoit'} `za + to cost'), \textit{*zavesit'} (\textit{za+vesit'} `za + to weigh'), \textit{*zazna\v{c}it'} (\textit{za+zna\v{c}it'} `za + to mean'), \textit{*zagordit'sja} (\textit{za+gordit'sja} `za + to feel proud'), \textit{zau\v{c}itel'stvovat'} (\textit{za+u\v{c}itel'strvovat'} `za + to work as a teacher') seem to be non-existent. However, after a careful consideration it becomes clear that there is no semantic reason why the core meaning of such verbs cannot be combined with that of the \isi{inchoative} \Prefix{za-}. It turns out that these (and similar) verbs can be divided in the following three categories:

\begin{enumerate}
\item Verbs that can be prefixed with the \isi{inchoative} \Prefix{za-}, as \textit{u\v{c}itel'stvovat'} `to work as a teacher'. The derived \isi{inchoative} verbs are not frequent and thus seem odd out of the \isi{context}, but native speakers do occasionally use them, as illustrated by \ref{ex:zateach}.
\item Verbs that can be combined with the \isi{resultative} \Prefix{za-}, as \textit{zagordit'sja} `to become stuck-up', \textit{zavesit'} `to weigh something' (colloquial).
\item Verbs that do not exist in combination with the prefix \Prefix{za-}, as \textit{*zastoit', *zazna\v{c}it'}.
\end{enumerate}

\exg.\label{ex:zateach}Malen'kij Ilja \`{e}to soobra\v{z}al, a bol\v{s}oj vyros -- zava\v{z}ni\v{c}al, zau\v{c}itel'stvoval, {nu i} polu\v{c}il spolna, \v{c}to zarabotal!\\
little Ilja this understand.\glb{pst.sg.m}, but big grow.\glb{pst.sg.m} -- za.showboat.\glb{pst.sg.m}, za.teach.\glb{pst.sg.m}, {and} receive.\glb{pst.sg.m} full, that earn.\glb{pst.sg.m}\\
\trans `When he was little, Ilja understood this, but when he grew up, he started to showboat, to teach others, and got everything he deserved!'\\\hbox{}\hfill\hbox{
\url{positive-lit.ru/novels/gde-konchajutsa-relsy/224}}

The difference between the first group of verbs and the other two that one may see when looking at the lists above (except for the verb \textit{zagordit'sja} `to become stuck-up') is that verbs like \textit{u\v{c}itel'stvovat'} `to work as a teacher' are intransitive.\footnote{The verb \textit{zagordit'sja} `to become stuck-up' it is a reflexive verb, so in some sense the direct object is ``integrated'' in the verb, so we will leave it aside.}

Let us explore this connection. Note that there are verbs that can be combined both with the \isi{resultative} and the \isi{inchoative} \Prefix{za-}. In such cases one can notice that the verb with the \isi{inchoative} \Prefix{za-}, as in \ref{ex:zagovorit:inch}, is intransitive, whereas the verb with the \isi{resultative} \Prefix{za-}, as in \ref{ex:zagovorit:res}, is transitive.

\ex.\ag.\label{ex:zagovorit:inch}On zagovoril.\\
he za.talk.\glb{pst.sg.m}\\
\trans `He started talking.'
\bg.\label{ex:zagovorit:res}On zagovoril menja.\\
he za.talk.\glb{pst.sg.m} me\\
\trans `He made me forget about something by his talking.'
 
An evident exception to this observation are motion verbs. With motion verbs, transitiveness does not hinder the attachment of the \isi{inchoative} \Prefix{za-}, as illustrated by \ref{ex:zacarry:indet}. At the same time, the \isi{resultative} \Prefix{za-} cannot be attached to motion verbs. What can be attached is the \isi{spatial} \Prefix{za-}, but it requires the \textit{path} scale to be presented in the structure of the verb and the path itself has to be provided (more details in \citealt{ZinovaOsswald:paper}). As we have discussed in Section~\ref{subsection:perf:motion}, prefixes acquire \isi{spatial} interpretations only with determinate motion verbs. The derived prefixed verbs (see example \ref{ex:zacarry:indet}) may, in turn, look identical to the corresponding indeterminate motion verbs that are prefixed with the same prefix (see \ref{ex:zacarry:det:pf}) and then imperfectivised (see \ref{ex:zacarry:det:ipf} and compare the examples \ref{ex:zacarry:indet} and \ref{ex:zacarry:det:ipf}).
 \ex.\label{ex:zacarry}\ag.\label{ex:zacarry:indet}Ma\v{s}a zanosila$^{\PF}_{\text{\INDET}}$ posylki.\\
 Ma\v{s}a za.carry.\glb{pst.sg.f} parcel.\glb{pl.acc}\\
 \trans `Masha started carrying parcels.'
\bg.\label{ex:zacarry:det:pf}Ma\v{s}a zanesla$^{\PF}_{\text{\DET}}$ posylku Kate.\\
 Ma\v{s}a za.carry.\glb{pst.sg.f} parcel.\glb{sg.acc} Katja.\glb{dat}\\
 \trans `Masha brought Katja the parcel.'
\bg.\label{ex:zacarry:det:ipf}Ma\v{s}a zanosila$^{\IPF}_{\text{\DET}}$ posylku Kate.\\
 Ma\v{s}a za.carry.\glb{pst.sg.f} parcel.\glb{sg.acc} Katja.\glb{dat}\\
 \trans `Masha was carrying the parcel to Katja.'
 
As is pointed out by \citet[227]{Braginsky:08}, some transitive non-motion verbs can be prefixed with the \isi{inchoative} \Prefix{za-} if the direct object is a bare plural noun (no measure phrases or numeral expressions).

\exg.Ivan za\v{c}ital$^{\PF}$ (*vse) / (*tri) / (*kak minimum tri) knigi.\\
Ivan za.read.\glb{pst.pl.m} all / three / at least three books\\
\trans `Ivan started reading books (in general).'\\\hbox{}\hfill\hbox{= example (17) in \citealt[227]{Braginsky:08}}

The verb \textit{\v{c}itat'} `read' can also be combined with the \isi{resultative} \Prefix{za-}. The output is the verb \textit{za\v{c}itat'} `to damage as a result of prolonged reading' \ref{ex:zachital}. In this case the direct object must be definite, so also a bare plural noun is interpreted as a definite description.

\exg.\label{ex:zachital}Ivan za\v{c}ital$^{\PF}$ vse knigi.\\
Ivan za.read.\glb{pst.sg.m} all books\\
`Ivan damaged all the books by his reading.'\\\hbox{}\hfill\hbox{= example (37a) in \citealt[246]{Braginsky:08}}

The unifying property of all the examples we have just considered is that in cases when the attachment of the \isi{inchoative} \Prefix{za-} is not possible, some scale, except for the \isi{time scale}, is available either due to the verbal semantic structure or due to the direct object. In parallel, when the \isi{inchoative} \Prefix{za-} can be attached, the \isi{time scale} is the only scale available. On the basis of this observation I agree with \citet{Paducheva:96} that the relation to the \isi{time scale} is the crucial property for the attachment of the \isi{inchoative} \Prefix{za-}, but I want to propose a different explanation for this fact. I claim that what prevents these verbs that have been categorised as holding for extra-long intervals of time by \citet{Paducheva:96} from being prefixed with the \isi{inchoative} \Prefix{za-} is that they lexicalise some specific scale: the event of weighing is by default measured in weight units, not in terms of time, as an event of jumping, for example. Time specification is still available for such verbs, but it is not the default domain, which prevents them from being combined with the \isi{inchoative} \Prefix{za-}. This is related to the other pattern we will discuss later in this chapter: verbs that do not lexicalise any other scale, except for the \isi{time scale}, are usually capable of serving as a source for \isi{prefixation} with the \isi{delimitative} prefix \Prefix{po-} (applied to the \isi{time scale}). 

The proposed explanation does not cover the case of the verb \textit{ljubit'} `to love', as there seems to be no other scale except for the \textit{time} in the semantic structure of this verb. I do not have an answer why the verb \textit{ljubit'} `to love' cannot be prefixed by the \isi{inchoative} \Prefix{za-}, but I would like to note that it can be interpreted \isi{inchoatively} when it is prefixed with \Prefix{po-}. The result of the \isi{prefixation} is the verb \textit{poljubit'} `to fall in love with'. If the verb \textit{ljubit'} was \isi{atemporal}, the derivation of a verb with an inceptive interpretation from it would not be possible with any prefix, yet it is possible and also unusual, as the prefix \Prefix{po-} is (except in this case) only interpreted \isi{inchoatively} when attached to determinate motion verbs. So it seems that the verb \textit{ljubit'} `to love' is special and deserves an investigation from a historical perspective. 

Let us now discuss another example, the verb \textit{za\v{z}eltet'} `to become seen as yellow', mentioned by \citet{Braginsky:08} as a verb that contains the \isi{inchoative} \mbox{\Prefix{za-}.} The verb \textit{\v{z}eltet'} has two interpretations: `to become yellow' and `to have yellow colour and be seen'. These two interpretations are connected to different \isi{internal} \isi{scales}: the first one is about colour intensity, whereas the second one is about visibility while the colour remains constant (yellow). The two interpretations also lead to different prefix contributions when \Prefix{za-} is attached: \isi{resultative} semantics of the derived verb in case of `to become yellow' meaning of the \isi{derivational base}, as illustrated by \ref{ex:zazeltet1}, and an \isi{inchoative} interpretation in case the \isi{derivational base} denotes a  situation in which an object of yellow colour becomes visible, as in \ref{ex:zazeltet2}.

%In previous work (1991, 2006, and Rappaport Hovav and Levin 2010) they argue for a particular constraint on what a verb root can lexicalize, which has come to be known as manner/result complementarity. In particular, they propose that a single verb root can lexicalize manner (non-scalar change) or result (scalar change), but not both at the same time. In the contribution to this volume, the authors underline that this complementarity is a constraint rather than a tendency, and they discuss two cases, which have been brought forwards as counterexamples to the manner/result complementarity, namely cut and climb. 

\ex.\label{ex:zazeltet}\ag.\label{ex:zazeltet1}On podros i sdelalsja neprijatno zubastym, glaz za\v{z}eltel, zra\v{c}ki priobreli demoni\v{c}eskuju vertikal'nuju formu.\\
he pod.grow.\glb{pst.sg.m} and s.make.\glb{pst.sg.m.refl} unpleasantly toothy, eye za.become.yellow.\glb{pst.sg.m}, pupils pri.get.\glb{pst.pl} demonic vertical form\\
`He grew up and became unpleasantly toothy, his eye became yellow-coloured and his pupils acquired a demonic vertical form.'\\\hbox{}\hfill\hbox{\url{https://books.google.com/books?isbn=5457040119}}
\bg.\label{ex:zazeltet2}\v{C}erez neskol'ko minut na gorizonte za\v{z}eltel svet far.\\
across several minutes on horizon za.seen.as.yellow.\glb{pst.sg.m} light headlight.\glb{pl.gen}\\
`Several minutes later yellow headlights appeared on the horizon.'\\\hbox{}\hfill\hbox{\url{https://books.google.com/books?isbn=5457264963}}

It is sometimes difficult to distinguish between the \isi{resultative} and the \isi{inchoative} interpretation of the prefix \Prefix{za-}. To do this, the first idea is to use a part of the traditional test for telicity (see Section~\ref{section:new:telicity}): try to modify the verbal phrase with a time measure phrase like \textit{za 3 \v{c}asa} `in 3 hours'. If this is not possible, can only be interpreted inceptively. Unfortunately, there is no reverse implication: if the event described by the \isi{inchoative} verb has a non-instantaneous \isi{preparatory phase}, such a verb is also compatible with the \textit{za 3 \v{c}asa} `in 3 hours' measure phrase. In order to distinguish such verbs from \textit{za}-prefixed verbs that have \isi{resultative} interpretation, I propose to use the \isi{context} schematically represented in \ref{context:za}.

\exg.\label{context:za}On Y-al, Y-al, i za-Y-al.\\
he verb.\glb{pst.sg.m} verb.\glb{pst.sg.m} and za.verb.\glb{pst.sg.m}\\
\trans `He was Y-ing, Y-ing, and finally Y-ed.'

Such contexts can be embedded directly into the original sentence in order to check the interpretation of the given verb in the given \isi{context}. If structure \ref{context:za} can be successfully embedded in the sentence, the usage of the verb prefixed with \Prefix{za-} is \isi{resultative}. If the sentence does not make sense after the insertion of \isi{context} \ref{context:za}, the prefix \Prefix{za-} has \isi{inchoative} semantics.

Let us run the test with the sentences in \ref{ex:zazeltet} in order to illustrate how it works. We substitute the verb \textit{za\v{z}eltel} `became yellow/seen as yellow' with the phrase \textit{\v{z}eltel, \v{z}eltel, i za\v{z}eltel}. If the verb \textit{\v{z}eltet'} is interpreted as `to become yellow', this phrase means `was becoming and becoming more yellow and then became yellow'. The same phrase in the `to have yellow colour and be seen' interpretation of the verb \textit{\v{z}eltet'} can be translated as `it was yellow and was seen and seen and then it appeared and it was yellow'. It is obvious that the second interpretation of this phrase does not make sense, so the whole sentence \ref{ex:zazeltet:test2} can not be interpreted. Sentence \ref{ex:zazeltet:test1} is a perfect Russian sentence (although its English translation is not natural).

\ex.\label{ex:zazeltet:test}\ag.\label{ex:zazeltet:test1}On podros i sdelalsja neprijatno zubastym, glaz \v{z}eltel, \v{z}eltel, i za\v{z}eltel, zra\v{c}ki priobreli demoni\v{c}eskuju vertikal'nuju formu.\\
he pod.grow.\glb{pst.sg.m} and s.make.\glb{pst.sg.m.refl} unpleasantly toothy, eye become.yellow.\glb{pst.sg.m}, become.yellow.\glb{pst.sg.m}, and za.become.yellow.\glb{pst.sg.m}, pupils pri.get.\glb{pst.pl} demonic vertical form\\
\trans `He grew up and became unpleasantly toothy, his eye became more and more yellow and finally it turned completely yellow, and his pupils acquired a demonic vertical form.'\\\hbox{}\hfill\hbox{\url{https://books.google.com/books?isbn=5457040119}}
\bg.$^\#$\v{C}erez neskol'ko minut na gorizonte \v{z}eltel, \v{z}eltel, i za\v{z}eltel svet far.\label{ex:zazeltet:test2}\\
across several minutes on horizon seen.as.yellow.\glb{pst.sg.m}, seen.as.yellow.\glb{pst.sg.m}, and za.seen.as.yellow.\glb{pst.sg.m} light headlight.\glb{pl.gen}\\
\trans $^\#$`After several minutes the yellow light was seen and seen and then appeared on the horizon.'

What these examples show is that in case the verb \textit{za\v{z}eltel} `to become yellow/to be yellow and become seen' has the colour intensity scale in its structure (when interpreted as `to become yellow'), it acquires \isi{resultative} meaning after being prefixed with \Prefix{za-}. If no other scale than the \isi{time scale} is available in the structure of the verb as it is the case with the second interpretation of the verb \textit{\v{z}eltet'} (`to be yellow and become seen'), the attachment of the prefix \Prefix{za-} leads to the \isi{inchoative} interpretation of the derived verb.

Similarly, obligatorily \isi{transitive verbs} are usually not compatible with the \isi{inchoative} interpretation of the prefix \Prefix{za-}, as for these verbs the obligatory direct objects provide \isi{scales} associated with it: the event of reading three books is measured in the \isi{cumulative} length or quantity of the books that are read. As for motion verbs, \textit{katat' tri tele\v{z}ki} `to roll three carts' is not measured by the number of carts rolled, as the action denoted by this phrase is perceived as happening simultaneously with all three carts. So for indeterminate motion verbs the \isi{time scale} is the only scale available. It is different in the case of determinate motion verbs: the phrase \textit{katit' tri tele\v{z}ki} `to push three carts' describes rolling three carts along some path, so the attachment of the prefix \Prefix{za-} leads to the \isi{spatial} interpretation.

Apart from indeterminate motion verbs, there are other cases when the direct object does not contribute a scale to the verb which makes the attachment of the \isi{inchoative} \Prefix{za-} is possible, e.g., the verb \textit{xotet'} `to desire', mentioned by \citet{Braginsky:08}. As desiring three ice creams is not an event progressing along the \textit{quantity} scale but is only related to time, the prefix \Prefix{za-} is interpreted \isi{inchoatively} when attached to the verb \textit{xotet'} `to desire'.

\exg.\label{ex:zaxotet}Ivan zaxotel$^{\PF}$ tri moro\v{z}ennyx srazu.\\
Ivan ZA-wanted three ice-creams {at once}\\
\trans `Ivan began to want three ice-creams at once.'
\\\Source{= example (47b) in \citealt[254]{Braginsky:08}}

The explanation I offer for the (non-)availability of the \isi{inchoative} interpretation of the prefix \Prefix{za-} with particular verbs is in some respect similar to the explanation of \citet{Braginsky:08}, who proposes that \isi{inchoative} interpretations occur in cases where \isi{resultative} interpretations are blocked. The absence of any other scale except for the \isi{time scale} guarantees that the \isi{resultative} interpretation is not available. The advantage of the approach advocated here is that there is no need for a separate explanation for the cases when both \isi{resultative} and \isi{inchoative} interpretations are not possible, which is a part missing in the account of \citet{Braginsky:08}.

Now that we came closer to the understanding of the semantic properties that are required for the attachment of the \isi{inchoative} prefix \Prefix{za-}, let us consider another type of restriction associated with this prefix. \citet{Tatevosov:09} categorises \Prefix{za-} as a selectionally limited prefix, namely, a prefix that can be attached to \isi{imperfective verbs} only. Judging from the available data and introspection, this generalisation seems to be correct. A question one may ask is whether there is some deeper motivation for such a restriction. I claim that the answer to this question is positive and, again, motivated semantically.\largerpage[-1]

Let us consider the semantic structure of a perfective verb and the semantic contribution of the \isi{inchoative} prefix \Prefix{za-}. A perfective verb normally (not always) denotes an event that is maximal with respect to some scale (i.e., the end point of that scale is reached). As we have just discussed, in order for the \isi{inchoative} prefix \Prefix{za-} to be attached, the \isi{time scale} should be the only available scale in the verbal semantic structure. This rules out the possibility of attachment of the \isi{inchoative} \Prefix{za-} to any perfective verb with a prefix that does not select the \isi{time scale}. What is left are verbs that are measured with respect to the \isi{time scale} (as can happen in the case of \isi{perfective verbs} with prefixes \Prefix{po-} and \Prefix{pere-}). The problem is that such events are associated with an endpoint at which the activity (denoted by the \isi{derivational base} verb) stops. 

On the other hand, the \isi{inchoative} \Prefix{za-} contributes the information that, at the end of the event described by the derived verb, the activity denoted by the \isi{derivational base} is being performed. These two pieces of semantic information are incompatible and thus the attachment of \Prefix{za-} is impossible. There is one case when the explanation provided above is not valid, namely when \Prefix{po-} has inceptive semantics. However, the inceptive semantics of \Prefix{po-} results from its attachment to a directed \isi{motion verb} and is associated with the initial segment of the \textit{path} scale. There is one exception to this pattern, as we have seen above: the verb \textit{poljubit'} `to fall in love' contains the prefix \Prefix{po-} with inceptive semantics, even though it is not a \isi{motion verb}. Indeed (and to my personal surprise), the verb \textit{zapoljubit'} `to start loving' is used by some native speakers, as illustrated by example \ref{ex:zapoljubit}. The semantics of this verb is intensified inception, which is not a very clear concept, but the number of examples in the web evidencing this verb is such that its existence (at least in the colloquial language) is beyond doubt.\largerpage

\exg.\label{ex:zapoljubit}ili \v{z}e, naoborot, igral s det'mi, \v{c}to o\v{c}en' v poslednee vremja zapoljubil\\
or again conversely played with children that very in last time za.po.love.\glb{pst.sg.m}\\
\trans `or, on the contrary, he played with children, which he suddenly started to love in the last time'\Source{\url{www.poezia.ru}}

From this follows that the restriction on the aspect of the \isi{derivational base} is motivated by two aspects. First, it is the semantic representations of the verb and prefixes, and second, a principle that tells that two verbs belonging to a \isi{derivational chain} cannot have the exact same semantics. The latter is another way of saying that additional \isi{morphological complexity} has to be avoided if the semantics is not enriched. As \citet{Braginsky:08} formulates it, ``the \isi{economy principle} of the word-formation does not allow grammar to form new words with the
exact lexical meanings as the existing ones.'' This principle will be used repeatedly in the proposed analysis.

%TODO: Perhaps this principle should be put into the \isi{context} of the monotonicity hypothesis advocated by Koontz-Garboden (Fabienne); Horn constraint?

\subsection{Secondary imperfective}\largerpage
It has been observed that suffixing an \isi{inchoative} \Prefix{za-}prefixed verb with the \isi{imperfective suffix} is not always possible. The question when it is possible and when not is discussed in the literature, but the conclusions different authors arrive at are vague. For example, \citet[230]{Svenonius:04b} writes that ``inceptive \Prefix{za-} almost never forms secondary imperfectives in Russian'' and \citet[220]{Braginsky:08} states that ``some \isi{inchoative} ZA-prefixed forms allow secondary \isi{imperfectivisation}.'' \citet[231]{Braginsky:08} also claims that ``[t]hose \isi{inchoative} forms that do undergo secondary \isi{imperfectivisation} acquire a \isi{habitual reading} of imperfective aspect, rather than a \isi{progressive one}.'' In addition, he notes that this may be due to the fact that ``\isi{inchoative} ZA-prefixed verbs are achievements'', but acknowledges that ``[t]he problem is, however, that most inchoatives block even a habitual secondary \isi{imperfectivisation}.'' However, \citet{Tatevosov:09} associates the \isi{inchoative} prefix \Prefix{za-} with a \isi{restriction on its attachment site}, but not with a restriction on the subsequent \isi{imperfectivisation}. With this in mind, let us look at the data. 

As we have already seen in Section~\ref{section:new:imperfectivization}, there are in fact cases when the imperfective verb derived from the \Prefix{za-}prefixed \isi{inchoative} verb receives an \isi{ongoing interpretation}. One example, which we have already seen, is repeated in \ref{ex:zakurival:prog:rep}, another is given i \ref{ex:zakurival:prog:new}.

\exg.\label{ex:zakurival:prog:rep}Arkadij Sergeevi\v{c} kak raz zakurival, po\`{e}tomu ne zametil, kak na poslednej fraze Olafson po\v{c}emu-to vorovato strel'nul glazami.\\
Arkadij Sergeevich as time za.smoke.\glb{imp.pst.sg.m}, {that is why} not notice.\glb{pst.sg.m}, as on last phrase Olafson {because of something} thievishly shoot.sem.\glb{pst.sg.m} eye.\glb{pl.inst}\\
\trans `Arkadij Sergeevich was just lightning the cigarette, so he didn't notice Olafson's thievish glance during the last phrase.'\Source{= example \ref{ex:zakurivat2} here}

\exg.\label{ex:zakurival:prog:new}Ja dal emu sigaretu i, kogda on zakurival, ja zametil, \v{c}to u nego dro\v{z}at ruki.\\
I give.\glb{pst.sg.m} he.\glb{dat} cigarette and, when he za.smoke.imp.\glb{pst.sg.m}, I notice.\glb{pst.sg.m}, that near he.\glb{gen} tremble.\glb{inf} hand.\glb{pl.nom}\\
\trans `I gave him a cigarette and, when he was lightning it, I noticed, that his hands were trembling.'\Source{Charles Bukowski, \textit{Jug bez priznakov severa}}\\\Source{[South of no north] (Russian translation)}

For many other verbs, however, the \isi{progressive interpretation} is indeed impossible. \citet{Braginsky:08} provides the following examples of usages of perfective and \isi{imperfective verbs} that contain the \isi{inchoative} prefix \Prefix{za-}:

\ex.\label{ex:za:imp:Brag}\ag.\label{ex:za:imp:Brag1}Ivan zagovoril$^{\PF}$ / zagovarival$^{\IPF}$ s proxo\v{z}imi.\\
Ivan ZA-talked / {used to ZA-talk} with passers-by\\
\trans `Ivan started talking / used to start talking with passers-by.'
\bg.\label{ex:za:imp:Brag2}Ivan zapel$^{\PF}$ / zapeval$^{\IPF}$ pesnju.\\
Ivan ZA-sang / {used to ZA-sang} song\\
\trans `Ivan started singing / used to start singing a song.'\\\Source{= ex.~(7) in \citealt[221]{Braginsky:08}}

The \isi{imperfective verbs} in the examples \ref{ex:za:imp:Brag1} and \ref{ex:za:imp:Brag2} do not receive a \isi{progressive interpretation}. (At least, searching for progressive usages of these verbs does not provide any result.) I claim that the difference between them and the verbs that allow a \isi{progressive interpretation}, as \textit{zakurivat'} `to start smoking' in examples \ref{ex:zakurival:prog:rep} and \ref{ex:zakurival:prog:new}, is the absence of a \isi{preparatory phase}. 

From the above follows, that whenever a \isi{secondary imperfective} is derived from a \Prefix{za-} prefixed verb with \isi{inchoative} semantics, it can acquire a \isi{progressive interpretation} if the event denoted by the verb has a \isi{preparatory phase} with a non-zero time span. In \ref{ex:zakurival:prog:new} the trembling happens in the period of lightning the cigarette, the end of which is referred to by the perfective verb \textit{zakurit'} `to start smoking'\footnote{While the English translation is ambiguous, the Russian verb refers to the \isi{preparatory phase} and not to the event of smoking itself.}. 

So the idea of \citet{Braginsky:08} seems to be on the right track: many inceptive \Prefix{za-}prefixed verbs do not receive a \isi{progressive interpretation} when imperfectivised because they denote achievements: inception events that are instantaneous and usually lack a \isi{preparatory phase}. What \citet{Braginsky:08} has not described is the possibility of a \isi{progressive interpretation} where the event denoted by the verb can be \isi{coerced into} an event with a \isi{preparatory phase}. Here, the \isi{preparatory phase} is understood as something unambiguously identified as preceding the start of the process/activity described by the \isi{derivational base} verb. E.g., for the verb \textit{zaprygat'} `to start jumping' it is hard to imagine some phase that is unambiguously identified as the preparation for jumping, but is not a part of the jumping event. In the case of \textit{zakurit'} `to start smoking', lighting a cigarette is, one one hand, an obvious preparation for smoking, but is also, on the other hand, not smoking per se. 

The situation with achievements in English is, in a way, similar: the progressive of some verbs denoting achievements is more acceptable than of some others (see examples \ref{ex:achiev:1} and \ref{ex:achiev:2}). As \citet{Rothstein:04} proposes, it is possible to coerce some achievements into accomplishments by adding a \isi{preparatory phase} (for further discussion on this topic, see \citealt{Gyarmathy:15}).

\ex.\a.\label{ex:achiev:1}The train was arriving at the station.
\b.*John was finding his phone.\label{ex:achiev:2}

So the difference between the \isi{resultative} and the \isi{inchoative} interpretations of \Prefix{za-} can be formulated in the following way. Verbs prefixed with the \isi{resultative} \Prefix{za-} focus on the culmination point (and may refer to this point plus a period that precedes it) achieved as a result of performing the action denoted by the \isi{derivational base}, as revealed by \isi{context} \ref{context:za}. Verbs prefixed with the \isi{inchoative} \Prefix{za-} focus on the point after which the action denoted by the \isi{derivational base} is performed (and, again, may also refer to the preceding period), so they fit into \isi{context} \ref{context:za:inch}.

\exg.\label{context:za:inch}On za-Y-al i Y-al 10 minut.\\
he za.verb.\glb{pst.sg.m} and verb.\glb{pst.sg.m} 10 minutes\\
\trans `He started to Y and Y-ed for 10 minutes.'

Note also, that if a time measure phrase can be added to a verbal phrase headed by a \Prefix{za-}prefixed verb with \isi{inchoative} interpretation, this time phrase refers to the \isi{duration} of the \isi{preparatory phase}, rather than to the \isi{duration} of the initiated event. This is illustrated by \ref{ex:zarabotat:rep}. Therefore, \isi{inchoative} \Prefix{za-}prefixed verbs that allow \isi{progressive interpretation} of their imperfective derviate also should allow modification by the time measure phrase headed with the preposition \textit{za}. (There is no implication in the other direction as the completed \isi{preparatory phase} can be identified via the initiated process, while an incomplete one requires other non-linguistic cues.)

\exg.\label{ex:zarabotat:rep}Kompjuter zarabotal za \v{c}etyre \v{c}asa.\\
computer za.work.\glb{pst.sg.m} behind four.\glb{acc} hour.\glb{sg.gen}\\
\trans `The computer started to work in four hours.'

Now we will explore the second point that has been noticed by \citet{Svenonius:04b} and \citet{Braginsky:08}, but not taken into account by \citet{Tatevosov:09}: the absence of secondary imperfectives from many \isi{inchoative} \Prefix{za-}prefixed verbs. 

The first class of such verbs consists of verbs that generally do not form secondary imperfectives after being prefixed, such as \textit{\v{z}eltet'} `to become yellow/to be yellow and become visible'. As it is not possible to construct any \isi{secondary imperfective} form of this verb, the restriction may be a phonological one or related to the fact that the verb is derived from a colour name. In this case, the impossibility of secondary imperfectivisation seems associated with the verbal stem and not with the \isi{inchoative} semantics of the prefix. 

The second class of verbs is more interesting: these are verbs that have secondary imperfectives, but not when prefixed with the \isi{inchoative} \Prefix{za-}. For example, \textit{zatalkivat'} is an imperfective verb formed from \textit{zatolkat'} `to push inside/to start pushing', but it only means `to push/be pushing inside', not `to start/be starting pushing'. A similar behaviour is observed for the verb \textit{zana\v{s}ivat'} that means `to wear/be wearing until the thing is damaged', but not `to start/be starting wearing', although the perfective verb \textit{zanosit'} can mean both `to wear until the thing is damaged' and `to start wearing'.

For this class I offer the following explanation. On the one hand, the \isi{resultative} meaning of such verbs when they are prefixed with \Prefix{za-} is much more common than the \isi{inchoative} meaning. So when the \isi{secondary imperfective} verb is analysed, the more frequent meaning is processed as a candidate meaning for the source perfective verb. And, as we have discussed above, \isi{resultative} and \isi{inchoative} interpretations are produced on the basis of different interpretations of the \isi{derivational base} (one involving only the \isi{time scale}, another including some other scale), so there is no possibility of an easy shift between these interpretations. On the other hand, there is an alternative lexical way to express the \isi{inchoative} meaning: one has to use the combination of the \isi{non-prefixed} verb together with the verb \textit{na\v{c}at'} `to start'. If the imperfective is needed, the `auxiliary' verb \textit{na\v{c}at'} `to start' can be imperfectivised. No comparable standard solution can be offered for the \isi{resultative} interpretation of \Prefix{za-}. These two facts together may have lead to the current state, in which \Prefix{za-}prefixed verbs with \isi{resultative} interpretation, form the \isi{secondary imperfective} only from this interpretation. This explanation is tentative and leaves space for further research.

The third class consists of verbs that seem to have no secondary imperfectives, but can form them, if needed. As an example, consider the verb \textit{zaigrat'} `to start playing'. Out of \isi{context}, the verb \textit{zaigryvat'} is interpreted as `to flirt', but it can also mean `to start/be starting playing', if a supporting \isi{context} is provided. This is the case of the example \ref{ex:zaigryvat}. 

\exg.\label{ex:zaigryvat}$\ldots$[n]o on smejalsja, zeval, preryval e\"{e} vostor\v{z}ennye me\v{c}tanija pros'boju zakazat' k zavtra\v{s}nemu obedu pobol'\v{s}e vet\v{c}iny ili, sosku\v{c}iv\v{s}is' slu\v{s}at' neponjatnye dlja nego zvuki, zaigryval na svoj lad pesenku, kotoraja vozmu\v{s}\v{c}ala vs\"{e} su\v{s}\v{c}estvovanie bednoj Ol'gi.\\
$\ldots$but he laughed, yawned, intervened her enthusiastic dreams request order to tomorrow dinner more ham or, {become.bored} listen {not understandable} for him sounds, za.play.imp.\glb{pst.sg.m} on his mood song.\glb{sg.acc}, that perturbed all existence poor Olga\\
\trans `$\ldots$[b]ut he laughed, yawned, interrupted her enthusiastic dreams with a request to order more ham for the dinner tomorrow or, bored from listening to sounds he could not understand, was starting to play a song in his own way, that perturbed the whole existence of poor Olga.'\\\Source{E.\,A. Gan. \textit{Ideal} (1837)}

Another example is the verb \textit{zasmejat'sja} which can be interpreted both \isi{inchoatively} (`to start laughing') and \isi{resultatively} (`to laugh until reaching some state'). The \isi{resultative} meaning is, however, very uncommon. When this verb is suffixed with the \isi{imperfective suffix} \textit{-iva-}, the resulting verb, \textit{zasmeivat'sja}, receives two interpretations: the habitual interpretation `to regularly start laughing' that stems from the \isi{inchoative} meaning of \textit{zasmejat'sja} `to start laughing', as in \ref{ex:za:laugh1}, and the habitual interpretation `to regularly laugh until reaching some state' that is based on the \isi{resultative} meaning of \textit{zasmejat'sja} `to laugh until reaching some state', as in \ref{ex:za:laugh2}. These examples support the tentative explanation of the behaviour of verbs in the second class: the frequency of different interpretations seems to play a role in the possibility of getting a \isi{secondary imperfective} with a particular interpretation.

\ex.\label{ex:za:laugh}\ag.\label{ex:za:laugh1}Priam vs\"{e} zasmeivalsja s bol'\v{s}im azartom\\
Priam all za.laugh.\glb{pst.sg.m}.refl with greater fervour\\
\trans `Priam started laughing again and again, every time with greater\linebreak fervour.'\Source{\url{https://ficbook.net}}
\bg.\label{ex:za:laugh2}$\ldots$postojanno do sl\"{e}z zasmeivalsja zaklju\v{c}\"{e}nnymi$\ldots$\\
$\ldots$constantly until tears za.laugh.\glb{pst.sg.m}.refl prisoner.\glb{pl.inst}$\ldots$\\
\trans `$\ldots$he always laughed at prisoners until he wept tears$\ldots$'\Source{\url{mobooka.ru}}


\subsection{Summary}
To sum up, the formal representation of the \isi{inchoative} \Prefix{za-} should have the following properties: 
\begin{enumerate}
\item the \isi{inchoative} interpretation of the prefix is only possible when the \isi{derivational base} does not have any explicit \isi{scales} except for the \isi{time scale} in its semantic representation (and the derived verb can only be used in contexts that do not contribute a scale);
\item attaching the prefix \Prefix{za-} relates the starting point of the event to the state of the absence and the end point of the event to the state of the presence of the activity denoted by the \isi{derivational base}.
\end{enumerate}

Other properties that we have discussed should be reflected in the representation of the verbs and the \isi{secondary imperfective} suffix: e.g., verbs that denote events with an extended \isi{preparatory phase} should have information about it in their semantic structure. In turn, the \isi{progressive interpretation} of the \isi{secondary imperfective} and the time measure phrase should be capable of modifying the \isi{preparatory phase} of the event in case the event itself does not have any \isi{duration}. The lexical entries of verbs that do not allow the attachment of the \isi{imperfective suffix} under any circumstances should be marked as such.

What is not possible to formalise within the framework adopted in the current analysis are the restrictions on the attachment of the \isi{imperfective suffix} associated with the frequency (or probability) of a particular interpretation of the given verb. If a probabilistic approach to semantics is integrated in the system, this should become possible, provided the explanation offered above is on the right track.

\section{\textit{na-}}\label{subsection:semantics:na}
\subsection{Semantic contribution}
First let us have a look at the different usages available for the prefix \Prefix{na-}. For this, we consult the grammar by \citet[360]{Shvedova:82}, where the following six types of \Prefix{na-}prefixed verbs are listed:
\begin{enumerate}
\item to direct the action denoted by the \isi{derivational base} onto some surface, to place on or come across something (productive type): \textit{nakleit'} `to paste';
\item to accumulate something by performing the action denoted by the \isi{derivational base} (productive type): \textit{navarit'} `to cook a lot of';
\item to perform the action denoted by the \isi{derivational base} intensively (productive type): \textit{nagladit'} `to iron thoroughly' (colloquial);
\item to perform the action denoted by the \isi{derivational base} weakly, lightly, on the go (non productive type): \textit{naigrat'} `to strum' (colloquial);
\item to learn something or acquire some skill by performing the action denoted by the \isi{derivational base} (productive type): \textit{natrenirovat'} `to train until\linebreak some level', \textit{nabegat'} `to train to run' (only in professional slang);
\item to perform the action denoted by the \isi{derivational base} until the result (productive type): \textit{nagret'} `to heat up', \textit{namo\v{c}it'} `to make wet', \textit{napoit'} `to give something to drink'.
\end{enumerate}

This section investigates the \isi{cumulative} usage (type 2 in the above list) more closely. Note that other productive usages of the prefix \Prefix{na-} are not considered \isi{superlexical} by those linguists who adopt the distinction. At the same time the representation I provide for the prefix \Prefix{na-} in Chapter~\ref{Chapter7} covers not only the second usage, but also the usages listed under three, five, and six.

The \isi{cumulative} prefix \Prefix{na-} and the prefix \Prefix{po-} (in the \isi{delimitative} meaning) that we are going to discuss in Section~\ref{subsection:semantics:po}, share a number of properties. Both prefixes are claimed to denote a vague measure function \citep{Filip:00, Souchkova:04}. \citet{Souchkova:04} formulates two differences between these prefixes: the direction of the relation and the dimensions of the \isi{scales} they select for.

There are two main usages of the \isi{cumulative} prefix \Prefix{na-} in Russian: transitive and reflexive. Transitive usage is exemplified by \ref{ex:na:trans}, where the prefix measures the quantity of the direct object (potatoes) that has been cleaned. Reflexive usage is exemplified by \ref{ex:na:refl}; here, the prefix \Prefix{na-} measures the degree to which the subject (Katja) is full after eating potatoes. The case of the reflexive usage will not be discussed in this thesis, for analyses see \citet{KaganPereltsvaig:11a,KaganPereltsvaig:11b,Souchkova:04,Filip:00,Filip:05}. (In fact, the analysis of \Prefix{na-} would remain the same, what is needed for this case is the interpretation of the \isi{postfix} \textit{-sja} that would provide the appropriate scale.)

\ex.\ag.\label{ex:na:trans}Katja na\v{c}istila karto\v{s}ki.\\
Katja na.clean.\glb{pst.sg.f} potato.\glb{gen}\\
\trans `Katja peeled a lot of potatoes.'
\bg.\label{ex:na:refl}Katja naelas' karto\v{s}ki.\\
Katja na.eat.\glb{pst.sg.f}.refl potato.\glb{gen}\\
\trans `Katja became full by eating potatoes.'

There is another usage of \Prefix{na-} (listed under (6) above) that is closely related to the \isi{cumulative} usage exemplified by \ref{ex:na:trans}. The verb \textit{namo\v{c}il} `wet' in \ref{ex:namochit} denotes an event of wetting something that is non-\isi{cumulative} in every respect: a single actor wet a single object with a single move. Another difference with respect to the verbs such as \textit{na\v{c}istit'} `to peel a lot of' is the source of the scale: in \ref{ex:na:trans} the event is measured along the quantity scale provided by the direct object, while in case of \ref{ex:namochit} the relevant \isi{wetness scale} is encoded by the verb.

\exg.\label{ex:namochit}Petja namo\v{c}il kisto\v{c}ku v stakane vody.\\
Petja na.wet.\glb{pst.sg.m} brush.\glb{sg.acc} in glass.\glb{sg.prp} water.\glb{sg.gen}\\
\trans `Petja wet the brush by putting it into a glass with water.'

%As the \isi{cumulative} usage of \Prefix{na-} is supposed to be productive, let us consider a verb that appeared in the language recently, i.e., \textit{guglit'$^{\IPF}$} `to google.' The \isi{cumulative} \Prefix{na-} can be attached to this verb, producing the derived verb \textit{naguglit'$^{\PF}$} `to google a lot of/to find out by googling.' An example of the utterance is given in \ref{ex:naguglit'}. At the same time, the same verb can be used in the \isi{context} like \ref{ex:naguglit':film} where no \isi{cumulative} reading is possible, as in \ref{ex:naguglit':film}.
%
%\exg.\label{ex:naguglit'}Ona naguglila informaciju po \'{e}toj teme\\
%she google.\glb{pst.sg.f} information on this topic\\
%\vspace{0.5em}
%`She googled information about this topic.'
%
%\exg.\label{ex:naguglit':film}K tomu \v{z}e, poka ja guglila pro nix, ja naguglila fil'm ``Ximera'' ...\\
%to that also, whole I google.\glb{pst.sg.f} about them, I na.google.\glb{pst.sg.f} film ``Chimera''\\
%\vspace{0.5em}
%`In addition, while I googled information about them, I found the film ``Chimera''. '
%\begin{flushright}
%\vspace{-0.5em}
%http://liekenlee.diary.ru/
%\end{flushright}

To account for this, one can either accept the \isi{polysemy} among the productive usages of the prefix \Prefix{na-} or try to unify them. If one considers the list of \Prefix{na-}prefixed verbs that do have clear \isi{cumulative} semantics, one can notice that for verbs in this list there is another way to express the completion of the event denoted by the \isi{derivational base}. For example, instead of \ref{ex:na:trans} the speaker could have uttered \ref{ex:na:po:gen} which would be neutral with respect to the quantity of the potatoes peeled or \ref{ex:na:po:acc} that would mean that Katja peeled all of the potatoes. The same happens in the pair of sentences \ref{ex:navarit} and \ref{ex:svarit}. The sentence with the verb prefixed with \Prefix{na-} refers to an event of cooking involving some quantity of the soup that exceeds the \isi{standard amount}. The sentence with the \Prefix{s-}prefixed verb does not carry any information about the quantity of soup produced.

\ex.\label{ex:na:po}\ag.\label{ex:na:po:gen}Katja po\v{c}istila karto\v{s}ki.\\
Katja po.clean.\glb{pst.sg.f} potato.\glb{gen}\\
\trans `Katja peeled some potatoes.'
\bg.\label{ex:na:po:acc}Katja po\v{c}istila karto\v{s}ku.\\
Katja po.clean.\glb{pst.sg.f} potato.\glb{acc}\\
\trans `Katja peeled the potatoes.'

\ex.\ag.\label{ex:navarit}Liza navarila supa.\\
Liza na.cook.\glb{pst.sg.f} soup.\glb{gen}\\
\trans `Liza cooked a lot of soup.'
\bg.\label{ex:svarit}Liza svarila sup.\\
Liza s.cook.\glb{pst.sg.f} soup.\glb{acc}\\
\trans `Liza cooked soup.'

On the basis of these observations I can offer the following potential explanation of what is happening with the prefix \Prefix{na-}: the core meaning of the \isi{cumulative} prefix \Prefix{na-} is `performing an action until the validation point is reached'. The validation point is, in different cases, either some standard quantity of the direct object or some degree on the scale. When it is reached, the action denoted by the \isi{derivational base} counts as having been performed. For example, the verb \textit{gret'}$^{\IPF}$ means `to warm' and the verb \textit{nagret'}$^{\PF}$ `to heat up' denotes warming until the warm state of the object is reached. Such an approach would unify the second, third, fifth, and sixth usages in the list by \citet{Shvedova:82}, so that the only other productive usage not covered here is associated with the \isi{spatial} scale (first usage in the list above). 

This description is very close to that of \citet{Kagan:book}, who offers the semantic representation of the prefix \Prefix{na-}, as shown in \ref{Kagan:na}. \citet[55]{Kagan:book} proposes that ``\textit{na}- looks for a verbal predicate that takes a degree, an individual and an event argument and imposes the `$\geqslant$' relation between the degree argument and the \isi{contextually provided} \isi{expectation value} d$_c$. As a result, the degree of change is entailed to be no lower than the standard.''
\ex.\label{Kagan:na}$\llbracket \Prefix{na-} \rrbracket = \lambda$P$\lambda$d$\lambda$x$\lambda$e.[P(d)(x)(e) $\wedge$ d $\geqslant$ d$_c$]\\
where d = degree of change \citep{KennedyLevin:02}\\\Source{= (17) in \citealt[55]{Kagan:book}}

The semantic representation proposed by \citet{Kagan:book} allows us to capture the semantics of the \isi{cumulative} and the \isi{resultative} usages of the prefix \Prefix{na-}. What is left unclear are the circumstances, in which the \isi{cumulative} interpretation is obtained. For example, for the verb \textit{nagret'} `to heat up' one does not want to derive the interpretation like `heat more than expected', as this would be the meaning of the verb \textit{peregret'} `to overheat'. A possible solution will be to simplify the semantics of \Prefix{na-} by restricting it to achieving the standard/expected degree on the scale and derive the additional component of exceeding the expectations in some cases in the pragmatic module. For this, one has to look at the competition between different \isi{perfective verbs} derived from the same \isi{derivational base}. If there is an alternative competing verb that is neutral with respect to the quantity of the direct object, uttering the verb prefixed with \Prefix{na-} implies a higher degree on the scale than standard. Similar pragmatic reasoning is not uncommon in the literature: for example, \citet[21]{KennedyLevin:08} use pragmatic reasoning to explain certain preferences in the domain of \isi{degree achievements}. I will provide more details in this respect in Chapter~\ref{Chapter6}.

\subsection{Restrictions on attachment}
As we have discussed in the previous chapter, the \isi{cumulative} prefix \Prefix{na-} is usually attached to \isi{imperfective verbs}. There are, however, exceptions to this generalisation. At least two verbs formed by \isi{prefixation} of \isi{perfective verbs} with the \isi{cumulative} \Prefix{na-} are accepted by all native speakers of Russian. These are \textit{nakupit'}$^{\PF}$ `to buy a lot of something' and \textit{napustit'}$^{\PF}$ `to fill with a lot of something'. In addition, \citet{Tatevosov:13a} notes that there is a group of speakers, seemingly from an older generation (and representing an earlier linguistic norm of the language) who accept a larger class of verbs derived by the \Prefix{na-}\isi{prefixation} of \isi{perfective verbs}, such as \textit{$^?$napridumat'}$^{\PF}$ `to come up with a lot of something', \textit{$^?$narasskazat'}$^{\PF}$ `to tell a lot of something', and \textit{$^?$naso\v{c}init'}$^{\PF}$ `to write/compose a lot of something'.

Starting with the information about the earlier norm of the language, let us take a diachronic perspective in order to explain the behaviour of the \isi{cumulative} \Prefix{na-}. Suppose some time ago the attachment of the \isi{cumulative} \Prefix{na-} to a perfective verb was the norm in the language (for whatever reason). This does not mean that \Prefix{na-} was attached only to \isi{perfective verbs}, but just the absence of the restriction (as is suggested by \citet{Tatevosov:13a} for those speakers who nowadays produce verbs such as \textit{narasskazat'}$^{\PF}$ `to tell a lot of something'). Then in such pairs as \textit{$^?$napridumat' -- napridumyvat'} `to come up with a lot of something', \textit{$^?$naotkryt' -- naotkryvat'} `to open a lot of', \textit{nakupit' -- napokupat'} `to buy a lot of' both verbs were acceptable. As the first members of these pairs are morphologically less complex, they might have been preferred over the second members of the pairs.\footnote{This can be explained by a \isi{pragmatic principle} related to the one we have already discussed: if there are two forms with identical semantics, the less complex form is preferred. In this case forms of different complexity do not belong to one \isi{derivational chain}, so this principle is only about the preference, not about the exclusion of one of the verbs.} 

Note that the difference in \isi{morphological complexity} of the two members of the pair can vary. The \isi{morphological complexity} difference between the competing verbs \textit{naotkryt'} `to open a lot of' and \textit{naotkryvat'} `to open a lot of' is only one morpheme: the \isi{imperfective suffix}, as is clear from the \isi{derivational chains} \ref{chain:naotkryt} and \ref{chain:naotkryvat}. In the pair  \textit{nakupit'} `to buy a lot of' and  \textit{napokupat'} `to buy a lot of' this difference is two morphemes: in order to derive a \isi{cumulative} verb from an imperfective verb, a prefix is added and the suffix changed, as illustrated by the \isi{derivational chains} \ref{chain:nakupit} and \ref{chain:napokupat}.

\ex.\ag.\label{chain:naotkryt}ot-kr-y-t'$^{\PF}$ $\rightarrow$ na-ot-kr-y-t'$^{\PF}$\\
{to open} $\rightarrow$ {to open a lot of}\\
\bg.\label{chain:naotkryvat}ot-kr-y-t'$^{\PF}$ $\rightarrow$ ot-kr-y-va-t'$^{\IPF}$ $\rightarrow$ na-ot-kr-y-t'$^{\PF}$\\
{to open} $\rightarrow$ {to open/be opening} $\rightarrow$ {to open a lot}\\

\ex.\ag.\label{chain:nakupit}kup-i-t'$^{\PF}$ $\rightarrow$ na-kup-i-t'$^{\PF}$\\
{to buy} $\rightarrow$ {to buy a lot}\\
\bg.\label{chain:napokupat}kup-i-t'$^{\PF}$ $\rightarrow$ po-kup-a-t'$^{\IPF}$ $\rightarrow$ na-po-kup-a-t'$^{\PF}$\\
{to buy} $\rightarrow$ {to buy/be buying} $\rightarrow$ {to buy a lot}\\

To provide some evidence in favour of the theory of competition sketched above, let us consider cases where the perfective verb is equally or more morphologically complex than the corresponding imperfective verb. In the first pair of verbs, \textit{o\v{s}\v{c}ut-i-t'}$^{\PF}$\slash\textit{o\v{s}\v{c}u\v{s}\v{c}-a-t'}$^{\IPF}$ `to feel', the imperfective verb is as complex as the perfective one, as the two verbs include the same number of morphemes. In the second pair, \textit{vz-j-a-t'}$^{\PF}$\slash\textit{br-a-t'}$^{\IPF}$ `to take', the perfective verb is morphologically more complex than the corresponding imperfective verb. It turns out that in both pairs the \isi{cumulative} prefix \Prefix{na-} can only be attached to the imperfective verb for all speakers of Russian (see chains in \ref{chain:naoschu} and \ref{chain:nabr} and examples \ref{ex:naoschutit} and \ref{ex:nabrat}). 

\ex.\label{chain:naoschu}\ag.o\v{s}\v{c}ut-i-t'$^{\PF}$ $\nrightarrow$ $^*$na-o\v{s}\v{c}ut-i-t'$^{\PF}$\label{chain:oschutit}\\
{to feel} {} {}\\
\bg.\label{chain:oschuschat}o\v{s}\v{c}u\v{s}\v{c}-a-t'$^{\IPF}$ $\rightarrow$ na-o\v{s}\v{c}u\v{s}\v{c}-a-t'$^{\PF}$\\
{to feel/be feeling} $\rightarrow$ {to feel a lot}\\

\ex.\label{chain:nabr}\ag.vz-j-a-t'$^{\PF}$ $\nrightarrow$ $^*$na-vz-j-a-t'$^{\PF}$\label{chain:navzjat}\\
{to take} {} {}\\
\bg.\label{chain:nabrat}br-a-t'$^{\IPF}$ $\rightarrow$ na-br-a-t'$^{\PF}$\\
{to take/be taking} $\rightarrow$ {to take a lot}\\

\exg.\label{ex:naoschutit}Instinkt \v{z}izni diktuet nao\v{s}\v{c}u\v{s}\v{c}at' kak mo\v{z}no bol'\v{s}e za \v{z}izn'.\\
instinct life.\glb{sg.gen} dictates na.feel.\glb{inf} as possible more for life\\
\trans `The instinct of life dictates that you should feel as much as possible during your life.'\Source{Mixail Veller. \textit{Belyj oslik} (2001)}

\exg.\label{ex:nabrat}On nabral celoe o\v{z}erel'e raku\v{s}ek [$\ldots$]\\
he na.take.\glb{pst.sg.m} whole necklace shell.\glb{pl.gen}\\
\trans `He gathered shells for a whole necklace [$\ldots$]'\\\Source{Aleksandr Dorofeev. \textit{\`{E}le-Fantik} (2003)}

Taking this into account, we can modify the assumption about the absence of a restriction on the attachment of the \isi{cumulative} \Prefix{na-}, saying that the attachment to the \isi{imperfective verbs} was still slightly preferred over the attachment to the perfective verb. Together with the \isi{pragmatic principle} that penalises morphologically more \isi{complex verbs} we then obtain a system that corresponds to the earlier norm. 

Now that we have discussed the competition between different verbs in the situation when the \isi{cumulative} \Prefix{na-} can be attached to both imperfective and \isi{perfective verbs}, let us see what happens when the norm shifts and the attachment of the \isi{cumulative} \Prefix{na-} to a perfective verb becomes significantly dispreferred. At this moment the rules of competition change: increasing the \isi{morphological complexity} of the verb by one morpheme becomes better than violating the \isi{aspectual restriction}. And in such pairs as \textit{napridumat'} vs. \textit{napridumyvat'} `to come up with a lot of something' the second member becomes preferred over the first. If, however, increasing the \isi{morphological complexity} by two is still penalised more than violating the \isi{aspectual restriction}, verbal pairs with greater difference in \isi{morphological complexity} would still allow the attachment of the \isi{cumulative} prefix \Prefix{na-} to the perfective \isi{derivational base}. And this is exactly what we observe in case of \textit{kupit' -- pokupat'} `to buy'.

Another exception is the verb \textit{napustit'}$^{\PF}$ `to fill with a lot of something' that is derived from the perfective verb \textit{pustit'}$^{\PF}$ `to let'. It is not clear what exactly happens with this particular verb, but it is exceptional not only with respect to the combination with the \isi{cumulative} \Prefix{na-}. First of all, a whole range of prefixed verbs that seem to be formed via \isi{prefixation} of the \isi{derivational base} \textit{puskat'}$^{\IPF}$ `to let' turn out to be imperfective: \textit{otpuskat'}$^{\IPF}$ `to let leave', \textit{zapuskat'}$^{\IPF}$ `to start something', \textit{napuskat'}$^{\IPF}$ `to fill with a lot of something', \textit{spuskat'}$^{\IPF}$ `to let out', etc. If we assume that these verbs are indeed derived from the imperfective verb \textit{puskat'}$^{\IPF}$ `to let', as shown in \ref{chain:puskat}, we have to postulate non-perfectivising usages for a number of prefixes. This is an argument in favour of the alternative hypothesis: the assumption that the last step in the derivation of these verbs is \isi{imperfectivisation}, as shown in \ref{chain:pustit}. Such an explanation is not complete as it just reduces the problem to the puzzle about a concrete verb, not about the \isi{prefixation} system, but I have no solution for this new puzzle at the moment. I believe that the answer might be given from a historical perspective and may have similar roots as the answer to the puzzle of the motion verbs. I leave this question open for future research.

\exg.puskat'$^{\IPF}$ $\rightarrow$ zapuskat'$^{\IPF}$ / napuskat'$^{\IPF}$ \label{chain:puskat}\\
{to let} $\rightarrow$ {to (be) starting something} / {to (be) fill(ing) with a lot of}\\

\exg.pustit'$^{\PF}$ $\rightarrow$ zapustit'$^{\PF}$ / napustit'$^{\PF}$ $\rightarrow$ zapuskat'$^{\IPF}$ / napuskat'$^{\IPF}$ \label{chain:pustit}\\
{to let} $\rightarrow$ {to start something} / {to fill with a lot of} $\rightarrow$ {to (be) starting something} / {to (be) fill(ing) with a lot of}\\

\subsection{Subsequent imperfectivisation}
The attachment of the \isi{imperfective suffix} to verbs prefixed with \Prefix{na-} is treated in the literature similarly to the case of the \isi{inchoative} prefix \Prefix{za-}: \citet[230]{Svenonius:04b} classifies the \isi{cumulative} \Prefix{na-} as a prefix that sometimes allows the formation of the \isi{secondary imperfective}, whereas \citet{Tatevosov:09} does not pose any specific restrictions (if fact, such restrictions are absent in his account at all).

An illustrative example is provided by \citet[233]{Svenonius:04b} and repeated here as \ref{ex:na:Sven}. In \ref{ex:na:Sven:1} we see a perfective verb with a literal interpretation of the \isi{derivational base}, whereas in \ref{ex:na:Sven:2} and \ref{ex:na:Sven:3} we observe that the \isi{secondary imperfective} can not be interpreted literally. \citet[233]{Svenonius:04b} attributes this asymetry of the \isi{secondary imperfective} formation to the difference in the structural positions. I claim that the verb \textit{nakalyvat'}$^{\IPF}$ `to pin/be pinning/to cheat/be cheating' is usually not interpreted as `to crack/be cracking a lot' not because of the position of the prefix in the structure of the verb \textit{nakolot'}$^{\PF}$ `to crack a lot', but because the latter verb also has the other meaning `to pin', derived from the \isi{spatial} interpretation of the prefix \Prefix{na-}. 

\ex.\label{ex:na:Sven}\ag.\label{ex:na:Sven:1}On na-kolol orexov.\\
he cmlt-cracked$^P$ nuts\\
\trans `He cracked a sufficiently large quantity of nuts'
\bg.\label{ex:na:Sven:2}*On na-kalyval orexov.\\
he cmlt-cracked$^I$ nuts\\
\vspace{0.5em}
(`He was cracking a sufficiently large quantity of nuts')
\bg.\label{ex:na:Sven:3}On na-kalyval klijentov.\\
he on-cracked$^I$ clients\\
\trans `He was cheating the clients'\\\Source{= example (63) in \citealt[230]{Svenonius:04b}}

So the situation turns out to be similar to that of the \isi{inchoative} prefix \Prefix{za-}: when a \Prefix{na-}prefixed verb has two interpretations, one (more frequent) of them involving \isi{spatial} and the other involving \isi{cumulative} meaning, the \isi{secondary imperfective} of this verb will be normally interpreted as formed on the basis of the \isi{spatial} interpretation. The reason is also similar: there is a regular lexical way to express the meaning that a \isi{secondary imperfective} verb with the \isi{cumulative} interpretation of the prefix \Prefix{na-} would have (use the \isi{non-prefixed} imperfective and the adverb \textit{mnogo} `a lot'). For the lexical meaning of the prefix, no such regular replacement of the \isi{secondary imperfective} is available. Indeed, if we search for the examples of the usage of the verb \textit{nakalyvat'}, we mostly find sentences like \ref{ex:nakalyvat}, involving the \isi{spatial} usage of the prefix \Prefix{na-}. 

\ex.\label{ex:nakalyvat}\ag.Izvestny slu\v{c}ai, kogda e\v{z}i podbirali i nakalyvali na svoi igly okurki ili pytalis' ``vyvaljat'sja'' v kofejnyx zernax.\\
known cases when hedgehogs pod.take.imp.\glb{pst.pl} and na.prick.imp.\glb{pst.pl} on their needles {cigarette stubs} or try.\glb{pst.pl} vy.waalow.imp.\glb{inf.refl} in coffee beans\\
\trans `We know about cases when hedgehogs picked up and pinned on their needles cigarette stubs or tried to roll in and get covered with coffee beans.'\Source{\url{http://www.ogoniok.com}}
\bg.O\v{c}i\v{s}\v{c}ennye orexi nu\v{z}no nakolot', ja nakalyvala vilkoj - tak bystree, \v{c}em zubo\v{c}istkoj.\\
peeled nuts necessary na.pin.\glb{inf}, I na.pin.imp.\glb{pst.sg.f} fork - so faster, then toothpick\\
\trans `You have to make holes in the peeled nuts, I pierced them with a fork, this is faster than using a toothpick.'\Source{\url{www.carina-forum.com}}

At the same time if we consult the dictionary, it turns out that the first interpretation provided for the verb \textit{nakalyvat'} is `to crack something in some (normally big) quantity' \citep{Efremova:00}, which is exactly the interpretation of the \isi{secondary imperfective} verb derived from the verb \textit{nakolot'} `to crack a lot of', that, according to \citet{Svenonius:04b} does not exist and, according to the internet data, is at least very uncommon, if used at all. As dictionaries tend to represent an outdated norm, this phenomenon can be related to the norm shift we have discussed above.

I want to emphasise that the \isi{imperfectivisation} of verbs prefixed with the \isi{cumulative} \Prefix{na-} is available in a larger number of cases than seems at first sight. I have sketched a possible explanation why its formation is dispreferred in case a \isi{spatial} interpretation of the \isi{derivational base} is available, but this explanation is about preference, not complete unavailability and uses information about the relative frequency of different interpretations. Consider the verb \textit{navarit'}$^{\PF}$ `to cook a lot/to weld something'. For the perfective verb, the \isi{cumulative} interpretation is the default one, but the \isi{spatial} interpretation is accessible in the relevant \isi{context}. After the attachment of the \isi{imperfective suffix}, the \isi{spatial} interpretation (see example \ref{ex:navarivat1}) is the default. The \isi{cumulative} interpretation is dispreferred, but possible and easy to identify, as illustrated by \ref{ex:navarivat2}.

\ex.\ag.\label{ex:navarivat2}Ona navarivala sebe bol'\v{s}ie kastrjuli kompotu i s''edala ego s serym xlebom, v odino\v{c}ku.\\
she na.cook.imp.\glb{pst.sg.f} yourself big pots compot and s.eat.imp.\glb{pst.sg.f} him with grey bread, in singleton\\
\trans `She regularly cooked herself large pots of compote and ate it on her own together with rye bread.'\Source{\url{http://gatchina3000.ru/}}
\bg.\label{ex:navarivat1}V ob\v{s}\v{c}em, vse vyxodnye brigada mestnyx svar\v{s}\v{c}ikov latala im nos, navarivala listy ob\v{s}ivki prjamo poverx izmjatyx.\\
in general, all weekends team local welders patch.\glb{pst.sg.f} them bow, na.weld.imp.\glb{pst.sg.f} sheet.\glb{pl.acc} sheathing directly {on top} wrinkled\\
\trans `In sum, the whole weekend the team of local welders patched their bow, welding the sheathing sheets directly on top of the wrinkled ones.'\Source{\url{http://kamafleetforum.ru/}}

It turns out that the formation of \isi{secondary imperfective} verbs from verbs prefixed with \isi{cumulative} \Prefix{na-} is in general available, although the derived \isi{imperfective verbs} may not sound acceptable without a \isi{context}. To provide another example, let us try to imperfectivise the verb \textit{naguglit'} `to find something by googling'. The derived verb \textit{naguglivat'} `to find something by googling occasionally' is used, as evidenced by the examples one can find in the internet, such as \ref{ex:naguglivat}. This verb is interpreted exclusively habitually which can be explained by using the principle based on the Horn's division of labour (see \citealt{Horn:84}): if there are two verbs that express the same meaning, the simpler one should be used. Indeed, the potential \isi{progressive interpretation} of the verb \textit{naguglivat'} is `to google something', exactly the same as the interpretation of the verb \textit{guglit'} `to google' when it is used transitively. As for the habitual interpretation, there is a clear difference between the semantics of the basic imperfective verb \textit{guglit'} `to google' and the semantics of the derived \isi{secondary imperfective} verb \textit{naguglivat'} `to find something by googling occasionally', as the latter includes the \isi{resultative} component for every event of googling. 

\exg.\label{ex:naguglivat}Spaseniem dejstvitel'no byli sovremennye stat'i, blogi, sajty, kotorye ja naguglivala na plan\v{s}ete, v kotorom \v{z}e borolas' so ``Star\v{s}ej \`{E}ddoj''.\\
salvation really were contemporary articles, blogs, pages, that.\glb{pl.nom} I na.google.imp.\glb{pst.sg.f} on tablet, in that.\glb{m.sg.prp} again fought with ``older Edda''\\
\trans `My salvation was in contemporary articles, blogs and web pages that I googled on my tablet, that I also used to fight with ``Older Edda''.'\\\Source{\url{http://www.livelib.ru/review/259836}}

Based on what we have observed so far, one can hypothesise that the \isi{progressive interpretation} of \isi{secondary imperfective} verbs that include the \isi{cumulative} prefix \Prefix{na-} should be possible in cases when the \isi{derivational base} is interpreted not just \isi{resultatively}, but also carries the `a lot' component (which happens due to competition with other verbs). This is confirmed by the data. As an example, consider the verb \textit{nagotovit'} `to cook/prepare a lot'.\footnote{I consider it instead of the verb \textit{navarit'} `to cook' here, as there are no other interpretations involving spacial \Prefix{na-} available for it and thus the \isi{secondary imperfective} is in general easily accessible. The \isi{neutral perfective} derived from the verb \textit{gotovit'} `to prepare/be preparing' is the verb \textit{prigotovit'} `to cook/prepare something'.}  The derived \isi{secondary imperfective} verb \textit{nagotavlivat'} `to prepare/be preparing a lot' can be interpreted progressively \ref{ex:nagotovit1} as well as habitually \ref{ex:nagotovit2}.

\ex.\label{ex:nagotovit}\ag.\label{ex:nagotovit1}s 5 \v{c}asov u\v{z}e ne spitsja, nagotavlivaju detjam\\
from 5 hours already not sleep.\glb{pres.sg.3.refl}, na.prepare.imp.\glb{pres.sg.1} child.\glb{pl.dat}\\
\trans `I can't sleep since 5 a.m., so I am preparing food for the children'\\\Source{\url{www.plastic-club.ru}}
\bg.\label{ex:nagotovit2}Vprok nikogda ne nagotavlivaju, ljubim vse sve\v{z}ee.\\
{in store} never not na.prepare.imp.\glb{pres.sg.1}, love.\glb{pres.pl.1} all fresh\\
\trans `I never cook food for the next several days, we prefer to eat fresh.'\\\Source{\url{forum.bel.ru}}

\subsection{Summary} 
To sum up, the formal representation of the \isi{cumulative} prefix \Prefix{na-} should have the following properties: 
\begin{enumerate}
\item the prefix requires an \isi{open scale} that is provided by the verb and a parameter of the object;
\item when the prefix is attached, it specifies the starting point of the event being at the starting point of the scale and the end of the event being at (or, possibly, at or above, see the discussion in the beginning of the section) the \isi{standard degree} on the same scale.
\end{enumerate}

Similarly to the analysis of \Prefix{za-}, I am not going to restrict the attachment of the \isi{secondary imperfective} to verbs prefixed with the \isi{cumulative} \Prefix{na-} in the semantic module.

\section{\textit{po-}}\label{subsection:semantics:po}
\subsection{Semantic contribution} To begin with, let us again look at the Russian grammar by \citet{Shvedova:82}, who provides a list of possible usages of the prefix \Prefix{po-} and their productivity. \citet[364--365]{Shvedova:82} names the following five types of situations the verbs prefixed with \Prefix{po-} can refer to:
\begin{enumerate}
\item to do the action that is denoted by the \isi{derivational base} with low intensity, sometimes also gradually: \textit{poprivyknut'} `to get somehow used', \textit{po\-izno\-sit'sja} `to get somewhat worn out', \textit{pomaslit'} `to put some butter on something'  (productive, especially in spoken language);
\item to do the action that is denoted by the \isi{derivational base} repeatedly, with many or all of the objects or by many or all of the subjects: \textit{povyvezti} `to take out many/all of something' (productive, especially in spoken language);
\item to do the action that is denoted by the \isi{derivational base} for some (often short) time: \textit{pobesedovat'} `to spend some time talking' (productive);
\item to start the action that is denoted by the \isi{derivational base}: \textit{pobe\v{z}at'} `to start running' (productive);
\item to complete the action denoted by the \isi{derivational base}: \textit{poblagodarit'} `to thank' (productive).
\end{enumerate}

We are going to look at the usages of the prefix \Prefix{po-} that are traditionally called \isi{delimitative} and \isi{distributive}. The \isi{delimitative} usage covers both the first and the third class of \Prefix{po-}prefixed verbs listed by \citet{Shvedova:82}, and the \isi{distributive} usage corresponds to the second type of outcome in the list above. The fourth usage (inceptive) is encountered when the prefix \Prefix{po-} is attached to a \isi{motion verb}; this usage is discussed in \citealt{ZinovaOsswald:paper}. As for the last usage from the list by \citet{Shvedova:82}, I will show that it can be unified with the \isi{delimitative} usage of \Prefix{po-}. In sum, I will provide a unified underspecified semantics for the prefix \Prefix{po-}.

\subsubsection{Delimitative \textit{po-}}
Traditionally, the \isi{delimitative} meaning of \Prefix{po-} is associated with some characteristic of an event being lower than the expected value: for example, an event lasting for a short period of time, a small quantity of the theme consumed, etc. This usage of \Prefix{po-} is also called \isi{attenuative} by some authors \citep[e.g.][]{Svenonius:04b}. According to \citet[47--48]{Filip:00}, who compares it with accumulative \Prefix{na-}, ``[t]he prefix \Prefix{po-} contributes to the verb the opposite meaning of a small quantity or a low degree relative to some \isi{expectation value}, which is comparable to vague quantifiers like \textit{a little, a few} and vague measure expressions like \textit{a (relatively) small quantity\slash piece\slash extent of}.''

\citet[183]{Braginsky:08} applies a neat test in order to show the difference between the verbs prefixed with the \isi{resultative} \Prefix{za-} and the verbs prefixed with \Prefix{po-}. The idea of this test is to continue the given sentence with `but it is hard to call it X' where X is the \isi{result state} corresponding to the \isi{derivational base}. Such a continuation is only possible if there is no restriction on the degree reached on the relevant scale by the end of the event. \citet[183]{Braginsky:08} provides two examples repeated as \ref{ex:Brag:pogustet} and \ref{ex:Brag:porzhavet} here. What these examples show is that, indeed, when sentences are headed by the \Prefix{po-}prefixed verb, the \isi{result state} must not be reached, which is not the case with the \Prefix{za-}prefixed \isi{resultative} verbs.

\ex.\label{ex:Brag:pogustet}\ag.Varen'je pogustelo$^{\PF}$, no ego e\v{s}\v{c}e trudno nazvat' gustym.\\
Jam PO-thickened but it yet hard {to call} thick\\
\trans `The jam thickened a bit, but it is hard to define it as thick yet.'
\bg.*Varen'je zagustelo$^{\PF}$, no ego e\v{s}\v{c}e trudno nazvat' gustym.\\
Jam ZA-thickened but it yet hard {to call} thick\\\Source{= example (49) in \citealt[183]{Braginsky:08}}

\ex.\label{ex:Brag:porzhavet}\ag.Gvozd' por\v{z}avel$^{\PF}$, no ego e\v{s}\v{c}e trudno nazvat' r\v{z}avym.\\
Nail {PO-became rusty} but it yet hard {to call} rusty\\
\trans `The nail became a bit rusty, but it is hard to define it as rusty yet.'
\bg.*Gvozd' zar\v{z}avel$^{\PF}$, no ego e\v{s}\v{c}e trudno nazvat' r\v{z}avym.\\
Nail {ZA-became rusty} but it yet hard {to call} rusty\\
\Source{= example (50) in \citealt[183]{Braginsky:08}}

\citet{Souchkova:04}, analysing Czech prefixes, shows that \Prefix{po-} can quantify over different dimensions: \isi{duration}, \isi{distance}, or \isi{degree of the property} attained by the \isi{internal argument}. \citeauthor{Souchkova:04} argues that despite the existance of different domains of quantification there is one single \isi{delimitative} \Prefix{po-} and its semantic contribution is sensitive to the content of the VP. This is true also for Russian and allows us to unify the first and the third usage listed by \citet{Shvedova:82}: the unified semantic representation later combines with a scale provided either by the verb or by the direct object, leading to different relevant interpretations.\largerpage[-1]

Examples of the \isi{delimitative} usage of the prefix \Prefix{po-} include such sentences as \ref{ex:po:delim}, taken from \citet{Filip:00} and \citet{Souchkova:04} and also used by \citet{Kagan:book}, whereby the sentence \ref{ex:po:delim1} expresses that the walk around the city was short, and \ref{ex:po:delim2} that the quantity of the apples eaten was relatively small.

\ex.\label{ex:po:delim}\ag.\label{ex:po:delim1}Ivan poguljal po gorodu.\\
Ivan po.walk.\glb{pst.sg.m} around town\\
\trans `Ivan took a (short) walk around the town.'\\\Source{= example (9c) in \citealt{Filip:00}}
\bg.\label{ex:po:delim2}Ivan poel jablok.\\
Ivan po.eat.\glb{pst.sg.m} apple.\glb{pl.gen}\\
\trans `Ivan ate some (not many) apples.'
\Source{= example (3) in \citealt[46]{Kagan:book}}

%\ex.\label{ex:po}\ag.On nemnogo po-razmy\v{s}lyal ob \'{e}tom.\\
%He {a little bit} {ATTN-thought} about this\\
%`He spent a little bit of time thinking about this.
%\bg.\label{ex:po2}Marina {za vremya} bolezni po-xudela.\\
%Marina during sickness {PO-lost weight}\\
%`Marina lost weight during the sickness.

Although the observations about the \isi{low degree on} some scale, associated with the discussed usage of the prefix \Prefix{po-}, are commonly accepted and seem to be well established, the assumption that this degree has to be low in any case prevents us from accounting for some of the prefix usage cases one can find. As an illustration, let me provide some examples from the corpora.

\ex.\label{ex:po:alot}\ag.\label{ex:po:alot1}Znat', mnogo po svetu pobrodil, vsjakogo raznogo uspel {naslu\v{s}at'sja-} {nasmotret'sja.}\\
know {a lot} on world po.wander.\glb{pst.sg.m} all different {have time} na.hear.\glb{inf}.refl na.look.\glb{inf}.refl\\
\trans `You know, he wandered a lot around the world, he had time to see and hear all kinds of different things.'\\\Source{Marija Semenova. \textit{Volkodav: Znamenie puti} (2003)}
\bg.\label{ex:po:alot2}Kogda do stolicy ostavalos' tridcat' kilometrov, na\v{s}\"{e}l stolovuju i o\v{c}en' plotno poel, poskol'ku do sleduju\v{s}\v{c}ego pri\"{e}ma pi\v{s}\v{c}i neizvestno skol'ko vremeni.\\
when before capital stay.\glb{pst.sg.n.}refl thirty kilometres found canteen and very full po.eat.\glb{pst.sg.m} because before next reception food unknown {how much} time\\
\trans `When I was about 30 km away from the capital, I found a canteen and had a very square meal, as I didn't know how long it would take until my next chance to eat something.'\\\Source{Anatolij Azol'skij. \textit{Lopu\v{s}ok} (1998)}

In \ref{ex:po:alot1} the verb \textit{pobrodil} `wandered', that presumably contains the \isi{delimitative} prefix \Prefix{po-}, refers to a lot of wandering, and in \ref{ex:po:alot2} the verb \textit{poel} `he ate' refers to a situation of eating a lot. If the semantics of the \isi{delimitative} prefix \Prefix{po-} included the semantic component `the degree is lower than the expected value', such sentences would be unacceptable or would trigger an additional pragmatic \isi{inference}, i.e., be interpreted sarcastically. This is not the case: both \ref{ex:po:alot1} and \ref{ex:po:alot2} are unmarked. What is also important is that some verbs can also be used in combination with adverbials denoting a small quantity (such as \textit{nemnogo} `a bit'), as in the examples \ref{ex:po:abit}.

\ex.\label{ex:po:abit}\ag.On pobrodit nemnogo i sej\v{c}as \v{z}e ujdet.\\
he po.wander.\glb{pres.sg.3} {a bit} and now same u.go.\glb{pres.sg.3}\\
\trans `He will wander around a little bit and immediately leave.'\\\Source{Anna Berseneva. \textit{Vozrast tret'ej ljubvi} (2005)}
\bg.My kupim pti\v{c}kam kormu i sami poedim nemnogo.\\
we buy.\glb{pres.pl.1} birds food and ourselves po.eat.\glb{pres.pl.1} {a bit}\\
\trans `We will buy food for the birds and we'll have a bite to eat ourselves.'\\\Source{V. P. Kataev. \textit{Bezdel'nik \`{E}duard} (1920)}

A possible solution would be to say that we are dealing with two different usages of \Prefix{po-}: a \isi{delimitative} in the examples \ref{ex:po:delim1} and \ref{ex:po:delim2} and some other in the examples \ref{ex:po:alot1} and \ref{ex:po:alot2}, probably corresponding to the last, \isi{resultative}, usage of \Prefix{po-} in the list provided by \citet{Shvedova:82}. This solution does not seem right to me: the verb \textit{poel} `he ate' in \ref{ex:po:delim2} and the verb \textit{poel} `he ate' in \ref{ex:po:alot2} seem to have the same meaning. If one consults dictionaries, one will find just one meaning of the verb \textit{poest'} `to eat' that reflects the meaning of the verbs \textit{poel} `he ate' in the examples \ref{ex:po:delim2} and \ref{ex:po:alot2}. This can be either `to eat not much' \citep{Ushakov:50} or `to eat' \citep{Efremova:00}. Further evidence in favour of the single meaning is that the verbal phrase in example \ref{ex:po:delim2} can also be modified with an adverbial denoting sufficient quantity, as evidenced by example \ref{ex:po:vdovol}, taken from the corpora.\largerpage

\exg.\label{ex:po:vdovol}Togda on poel jablok vdovol'.\\
then he po.eat.\glb{pst.sg.m} apple.\glb{pl.gen} enough\\
\trans `Then he ate apples to his heart's content.'\\\Source{Aleksandr Ili\v{c}evskij. \textit{Matiss} (2007)}

So again I propose to apply the same technique as in the case of the \isi{cumulative} \Prefix{na-}. We can define the semantics of the \isi{delimitative} usage{\interfootnotelinepenalty=10000\footnote{I will use the term \textit{delimitative} to refer to this in order to differentiate it from the \isi{distributive} and \isi{inchoative} usages, but I will not imply attenuativity.}} of \Prefix{po-} in such a way that the verb prefixed with it can either denote the unmarked completion of the event or include the semantic component `quantity/degree is lower than some \isi{expectation value}'. 

\citet[48]{Kagan:book}, following the analyses proposed by \citet{Filip:00} and \citet{Souchkova:04}, proposes that ``\textit{po}- looks for a predicate that takes a degree, and individual and an event argument and imposes the `$\leqslant$' relation between the degree argument and the \isi{contextually provided} \isi{expectation value} d$_c$.''

\ex.\label{Kagan:po}$\llbracket \Prefix{po-} \rrbracket = \lambda$P$\lambda$d$\lambda$x$\lambda$e.[P(d)(x)(e) $\wedge$ d $\leqslant$ d$_c$]\\
where d = degree of change \citep{KennedyLevin:02}

This approach captures the semantics of the prefix in the examples discussed here as it includes the possibility that $d = d_c$ and thus both completion and delimitation can be expressed by the same prefix. What needs to be added here is some elaboration on discussion of the conditions under which the verb prefixed with \Prefix{po-} tends to be interpreted delimitatively when used out of the \isi{context} or in the neutral \isi{context}.

Let me sketch how the \isi{pragmatic competition} mechanism can be used in order to evoke such conditions. Consider sentence \ref{ex:po:delim2}. For this sentence, there are alternative ways of denoting a completed eating event, such as \ref{ex:sjest}. So if the speaker wants to describe an event of eating all of the apples, they can utter \ref{ex:sjest}. The most appropriate description of the situation of eating the apples until becoming full is \ref{ex:najestsja3}. Given this competition when sentence \ref{ex:po:delim2} (that literally means that some apples were eaten) is uttered, it gets enriched with an additional \isi{inference} that the quantity of the apples eaten is lower than the number of apples available and the amount of apples necessary for the actor to become full. I will provide some additional details on this kind of \isi{pragmatic competition} in Chapter~\ref{Chapter6}.

\ex.\ag.\label{ex:sjest}Ivan s''el jabloki.\\
Ivan s.eat.\glb{pst.sg.m} apple.\glb{pl.acc}\\
\trans `Ivan ate the apples.'
\bg.\label{ex:najestsja3}Ivan naelsja jablok.\\
Ivan na.eat.\glb{pst.sg.m}.refl apple.\glb{pl.gen}\\
\trans `Ivan ate the apples until becoming full.'

From the proposed competition between different \isi{perfective verbs}, it also follows that if \Prefix{po-} is not the first prefix that is attached to the verb, it often tends to be interpreted as referring to a partial event because it competes with the perfective verb without the prefix \Prefix{po-}.

%TODO: provide an explanation when and what happens. Preliminary: \isi{delimitative} effect occurs if in some sense the scale cannot be changed. So if the direct object is singular (zapisat' disk), then `pozapisyvat' disk' won't mean `record the disk completely'. If the direct object is plural, po will have \isi{distributive} meaning, so we don't consider this case. If the direct object is singular but the \isi{imperfectivisation} with habitual intrpretation allows to `get rid of this scale' (proletet' mimo okna princessy -- poproletat' mimo okna princessy)

%Note that in any case two components of the semantic contribution have to be separated: the end of the event has to be linked to achieving the some given value on the scale and this value can meet the standard or be below it. This has to be not confused with the situation when the end of the event is linked directly to some value on the scale that is either at the standard or below it, as this will mean that the event can be not a completed one (what we need is a completed event possibly with some limitation). To provide an example, let us consider again the sentence \ref{ex:po:delim2}. The desired semantics of the sentence is `Ivan ate some of the apples (and the quantity of the apples eaten is lower than expected),' not 

\subsubsection{Distributive \textit{po-}}
%NOTE from Filip: Participant-based individuation of \isi{subevents} yields readings involving notionsl ike individually, each separatelv. Individuation of \isi{subevents} based on separate running times results in adverbial temporal meanings of successfully, consecutively, one at a time ( e.g.,pozamykat' to lock X part by part, one (group) at a time, after another'). Individuation of \isi{subevents} based on separate locations yields readings like here and there, all over. With base verbs describing some action of applying or attaching something onto something else or creating marks on something, \isi{po-} generates the totality meaning of `to cover x with V-ing

Another usage of \Prefix{po-} we discuss in detail is the \isi{distributive} (second meaning in the list taken from the grammar by \citealt{Shvedova:82}). The \isi{distributive} interpretation of the prefix \Prefix{po-} seems to be the least studied prefix usage among all the prefix usages that are classified as \isi{superlexical} by those linguists that adopt the distinction. \citet{Tatevosov:09}, for example, identifies it as a \isi{left periphery prefix} (the only one in this category) and suggests the reader to look in the other paper of the same author for discussion, but this paper is a 2009 manuscript and not available in any form. In the book by \citet{Kagan:book} the \isi{distributive} usage of \Prefix{po-} is not discussed either. 

What one can find are a few descriptive notes in Russian studies of verbal \isi{prefixation}. For example, \citet[289--290]{Isachenko:60} compares \Prefix{po-}prefixed and \Prefix{pere-}prefixed verbs with \isi{distributive} semantics and concludes that \isi{distributive} verbs containing the prefix \Prefix{po-} ``obozna\v{c}ajut distributivnost' dejstvija, no bez ottenka poo\v{c}erednosti otdel'nyx aktov, svojstvennogo glagolam na pere-... Semanti\v{c}eskaja raznica, odnako, o\v{c}en' tonkaja i ne\v{c}etkaja'' [denote the distributivity of the action, but without the semantics of the succession of the separate acts, that is characteristic for the verbs prefixed with \Prefix{pere-}... The difference in the semantics between the classes of verbs is, however, very slight and fuzzy].

So for the moment let us assume that the \isi{distributive} usage of the prefix \Prefix{po-} can be characterised as `performing the action denoted by the \isi{derivational base} with all of the objects or by all of the subjects specified in the sentence, without the individualisation of the \isi{subevents}.' We will compare the \isi{distributive} usage of the prefix \Prefix{po-} with the \isi{distributive} usage of the prefix \Prefix{pere-} in Section~\ref{subsection:semantics:pere}.

\subsection{Restrictions on attachment} 
Let us start by considering the \isi{delimitative} usage of the prefix \Prefix{po-}. \citet{Tatevosov:09} classifies the \isi{delimitative} prefix \Prefix{po-} as a selectionally limited prefix. As we have already discussed in Section~\ref{section:Tat09}, there are exceptions to this observation. For example, the verb \textit{popriotkryt'} `to open very slightly' in sentence \ref{ex:popriotkryl:rep} is derived by prefixing the perfective verb \textit{priotkryt'} `to open slightly' with the \isi{delimitative} prefix \Prefix{po-}.\largerpage

\exg. \label{ex:popriotkryl:rep}A na e\v{s}elone on nemno\v{z}ko \v{c}ut' popriotkryl oko\v{s}ko.\\
But at {flight level} he {a little bit} {slightly} po.pri.open.\glb{pst.sg.m} window.\glb{sg.acc}\\
\trans `And at the flight level he just a little bit opened the window.'\\\Source{= ex. \ref{ex:popriotkryl} in Chapter~\ref{Chapter4}}

If one consults the list of usages of the prefix \Prefix{po-} provided by \citet{Shvedova:82}, one will find that the list of examples for the first usage contains verbs with two prefixes and no \isi{imperfective suffix}, such as \textit{poprivyknut'} `to get somehow used' and \textit{poiznosit'sja} `to get somewhat worn out'. 

%\exg. \label{ex:popriotkryval:rep}A na e\v{s}elone on nemno\v{z}ko chut' popriotkryval$^{\IPF}$ oko\v{s}ko.\\
%but at {flight level} he {a little bit} {slightly} po.pri.open.imp\glb{pst.sg.m} window.\glb{sg.acc}\\
%\vspace{0.5em}
%`And at the flight level he used to open the window just a little bit.'
%\begin{flushright}
%\vspace{-1em}
%= example \ref{ex:popriotkryval} here
%\end{flushright}

A possible informal explanation of the observed facts is the following: the \isi{delimitative} prefix \Prefix{po-} normally cannot be attached to a perfective verb, because such a verb already denotes a completed\footnote{``Completed'' here means that the \isi{maximum point} or the contextually determined standard point on the scale is reached. Punctual events can be considered a marginal case when the maximum and the minimum points are identical.} event. The semantic contribution of the prefix \Prefix{po-} is weaker than the semantic contribution of prefixes that demand the culmination of the event to correspond to the maximum on the scale or be higher than some expected value. Consequently, combining \isi{perfective verbs} that contain such prefixes with the \isi{delimitative} \Prefix{po-} will not enrich their semantics. The only possible change is removing the completeness (reaching the \isi{maximum point} on the scale) component from the source event semantics, but this is not possible if one accepts the \isi{Monotonicity Hypothesis} \citep{Kiparsky:82}.\largerpage[2]

Let us consider again example \ref{ex:po:Tat2} from Chapter~\ref{Chapter4}, repeated here as\ref{ex:po:Tat2:rep} \citet{Tatevosov:09}. The verb \textit{zapisat'} `to write down/to record' refers to a completed event of writing something down or recording. The relevant scale in this case is provided by the direct object, so the event is considered completed when the whole object is written down/recorded. If the verb \textit{zapisat'} `to write down/record' could be combined with the \isi{delimitative} prefix \Prefix{po-}, the semantics of the derived verb would remain unchanged: the \isi{derivational base} includes the information that the \isi{maximum point} of the relevant scale has been reached whereas the prefix contributes the information that some point on the scale has been reached. In this case the attachment of the prefix violates the \isi{pragmatic principle} introduced above, as it leads to a \isi{derivational chain} in which two subsequent verbs have exactly the same semantics.\footnote{This is the case when semantic representations would be literally the same, as the information contributed by the prefix is already contained in the semantics of the \isi{derivational base}.}

\exg.\label{ex:po:Tat2:rep}Po\`{e}tomu zapustil programmu, zapisyvaju\v{s}\v{c}uju dejstvija na \`{e}krane, otkryl PSP, i nemnogo $^\#$po-zapisal ($^{\textit{OK}}$po-zapisyval$^{\PF}$), \v{c}to i kak.\\
{because of it} za.let.\glb{pst.sg.m} program.\glb{sg.acc}, za.write.\glb{PAP.sg.f.acc} action.\glb{pl.acc} on screen.\glb{sg.prep} open.\glb{pst.sg.m} PSP and {a bit} $^\#$po.write.\glb{pst.sg.m} ($^{\textit{OK}}$po.write.imp\glb{pst.sg.m}) what and how\\
\trans `For this reason I ran the program that records the actions on the screen and recorded for some time, what was happening and how.'\\\Source{= ex.~(63b) in \citealt{Tatevosov:09} and \ref{ex:po:Tat2} in Chapter~\ref{Chapter4}} 

Why is the proposed preliminary semantic explanation preferable to the syntactic one? Exactly because, according to this explanation, there is no reason why the verb \textit{popriotkryt'} `to open very slightly' could not exist. The semantic explanation why \Prefix{po-} does not usually combine with \isi{perfective verbs} hinges on the fact that most of them denote events such that the end point of the event corresponds to one fixed point on the scale. If a perfective verb denotes an event such that its end point is not bound to the maximum (or contextually determined standard) point on the scale, but can be any point from a range of points, then it should be possible to prefix it with the \isi{delimitative} \Prefix{po-}. The meaning of the resulting verb would be the intensified (which in our case means further limitation) meaning of the \isi{derivational base}. This is exactly the case of \ref{ex:popriotkryl:rep}.

Another example is provided in \ref{ex:popod-:rep}. In accordance with the intuition we are describing, the \isi{delimitative} prefix \Prefix{po-} is redundant when it is attached to a perfective verb, as its semantic contribution is already present in the semantic representation of the \isi{derivational base}. This explains why such verbs are awkward without a good \isi{context} that motivates the need to emphasise the \isi{low degree on} the relevant scale. In \ref{ex:popriotkryl:rep}, the usage of the verb is motivated by the speaker's intention to report the actor's idea that a tiny opening cannot harm. In the other example, \ref{ex:popod-:rep}, that we have already discussed in Chapter~\ref{Chapter4}, it would be very harsh to use the frequent verb \textit{podsoxnut'} `to dry to some extent' with respect to one's brains, so the author of this comment chooses to soften the description by adding another \isi{delimitative} prefix, \Prefix{po-}. 

\exg.\label{ex:popod-:rep}Za sorok let despotizma mozgi popodsoxli.\\
after forty year.\glb{pl.gen} despotism brain.\glb{nom} po.pod.dry.\glb{pst.pl}\\
\trans `During forty years of despotism his brain kind of dried up a bit.'\\\Source{= ex.~\ref{ex:popod-} in Chapter~\ref{Chapter4}}

Let us go back to the discussion of example \ref{ex:popriotkryl:rep}. It turns out that there also exists a perfective verb \textit{popriotkryvat'}$^{\PF}$ `to slightly open \isi{multiple times}', that is formed with an additional \isi{imperfectivisation} before the attachment of the prefix \Prefix{po-}. This verb denotes multiple events of opening within a short time period. 

Consider the examples \ref{ex:popriotkryval:pf1} and \ref{ex:popriotkryval:pf2}. In \ref{ex:popriotkryval:pf1} the verb \textit{popriotkryvala}$^{\PF}$ `she slightly opened' denotes a short \isi{series of} occurrences of slight opening of the mouth, so the prefix \Prefix{po-} temporally limits the \isi{series of} openings. This series, in turn, is denoted by the \isi{derivational base} \textit{priotkryvat'}$^{\IPF}$ `to open/be opening slightly'. In the example \ref{ex:popriotkryval:pf2} the verb  \textit{popriotkryval}$^{\PF}$ `he slightly opened all of' also refers to a \isi{series of} opening events. The difference between \ref{ex:popriotkryval:pf1} and \ref{ex:popriotkryval:pf2} is that in the latter case each opening event takes place with a different object (all the pots where there were no saplings to see), so according to descriptions of Russian \isi{prefixation} this \Prefix{po-} is not \isi{delimitative}, but \isi{distributive}.\footnote{One can say that the verb \textit{popriotkryvala} `she slightly opened \isi{multiple times}' is \isi{distributive} as well, if distribution over time is allowed.}

\exg.\label{ex:popriotkryval:pf1}Poprobovali dat' im krevetku, Oskar ne otreagiroval, a Matil'da nemnogo rot popriotkryvala$^{\PF}$, no tak i ne poela.\\
po.try.\glb{pst.pl} give.\glb{inf} they.\glb{dat} shrimp.\glb{sg.acc} Oskar.\glb{nom} not ot.react.\glb{pst.sg.m} but Matilda {a bit} mouth po.pri.open.imp.\glb{pst.sg.f} but so and not po.eat.\glb{pst.sg.f}\\
\trans `We have tried to give them a shrimp, Oskar didn't react at all and Matilda slightly opened her mouth several times but didn't eat it.'\\\Source{\url{http://cherepahi.ru}}

\exg.\label{ex:popriotkryval:pf2}Daby izbe\v{z}at' podobnogo, slegka popriotkryval$^{\PF}$ vatu vo vsex gor\v{s}o\v{c}kax, gde net vsxodov.\\
for iz.run.\glb{imp} similar.\glb{sg.m.gen} slightly po.pri.open.imp.\glb{pst.sg.m} {cotton wool} in all.\glb{prep} pot.\glb{pl.prep} where no sapling.\glb{pl.gen}\\
\trans `To avoid a similar situation, I slightly opened the cotton wool coverage on all the pots where there were no saplings to see.'\\\Source{\url{http://ganja-forum.com}}

In some cases it is not clear which meaning the prefix contributes. Even the number of the relevant noun does not always help. Consider example \ref{ex:popod-?}. It can be interpreted as a statement about the generation as a whole growing up a little bit and it can also mean that each person from this generation grew up. This example is useful to illustrate the intuition of \citet{Isachenko:60} that there is no object-by-object iteration when the verb contains the \isi{distributive} prefix \Prefix{po-}.

\exg.\label{ex:popod-?}...a nyn\v{c}e \v{z} – novoe pokolenie, kak-nikak, popodroslo, a ono \v{z}, \`{e}to pokolenie, -- ogo-go!\\
...but nowadays well -- new generation.\glb{sg.nom}, {after all}, po.pod.grow.\glb{sg.pst.n}, but it.\glb{nom} {} this generation.\glb{sg.nom}, -- wow
\\
\trans `...but now, after all, the new generation grew up a bit, and it is quite a generation!'\Source{\url{http://ergos-paragogis.livejournal.com/37099.html}}
 
The conclusion one can arrive at after considering the examples above and in particular \ref{ex:popod-?} is that the \isi{delimitative} and the \isi{distributive} meanings of \Prefix{po-}, despite being very distinct at first sight, are instances of the same underlying semantic representation. As we have seen, it is sometimes difficult to determine which of the two usages of prefixes we are looking at in any given example. This is an argument if favour of abandoning the hypothesis of a strict boundary between the \isi{delimitative} \Prefix{po-} and the \isi{distributive} \Prefix{po-}.

It turns out that the \isi{scalar approach} to \isi{prefixation} allow us to provide a single representation that can result in either interpretation depending on the type of scale selected to measure the event progress. As we have seen, a \isi{distributive} interpretation occurs only in cases when there is a plural direct object that is interpreted definitely. This means that in the representation of this object there is an attribute such that its value can be used as the \isi{maximum point} on the \isi{measure of change scale}. (The minimum point on the \isi{measure of change scale} is always~0.) The maximum and minimum points then become linked to the start and the end points of the event, respectively. This is interpreted as the event taking place until the action denoted by the verb has been applied to all of the members in the set denoted by the direct object. If the amount of the direct object is indefinite, no value that can serve as a maximum on the \isi{measure of change scale} is available, so the end point of the event will correspond to an arbitrary point of this scale, leading (through an additional step of \isi{pragmatic strengthening}) to the \isi{delimitative} interpretation of the event. More details about the pragmatic level and the formal representation of the prefix will be provided together in Chapter~\ref{Chapter6} and Chapter~\ref{Chapter7}.

\subsection{Subsequent imperfectivisation of a verb with the discussed prefix}
As the prefix \Prefix{po-} in its \isi{distributive} usage does not have any puzzling restrictions on its attachment, the intriguing part turns out to be located in the \isi{imperfectivisation} domain. \citet[365]{Shvedova:82} notes that many of the verbs prefixed with the \isi{distributive} \Prefix{po-} are derived from \isi{perfective verbs} (and at the same time are colloquial) and are synonymous with the verbs that are motivated by the imperfective counterparts of the derivational bases (some of these verbs are also colloquial, but their percentage is much lower), as in the pair \textit{povybit'}$^{\PF}$ -- \textit{povybivat'}$^{\PF}$ `to knock out many/all of'.

For the account presented here, such data poses a certain challenge, i.e.\ it has to be explained why, e.g., in the pair \textit{povybit'}$^{\PF}$ -- \textit{povybivat'}$^{\PF}$ `to knock out many/all of' the second verb could not be derived from the first one or, if it could, why it is perfective despite the fact that adding the \isi{imperfective suffix} is the last step of the derivation. I propose to take the first path and to explain why \isi{imperfectivisation} is not possible after attaching the \isi{distributive} \Prefix{po-} (or, adjusting to the merge of the two usages proposed above, why in the situation where attachment of the prefix \Prefix{po-} leads to the \isi{distributive} interpretation of the derived verb, this verb is not compatible with further \isi{imperfectivisation}. It turns out that if the semantics of the \isi{imperfective suffix} is added to the semantics of the verb prefixed with \isi{distributive} \Prefix{po-}, the semantics of the resultant verb is similar to that of an imperfective verb that is not prefixed with \Prefix{po-}. For this reason, the derivation of a more complex form to express the same meaning is blocked.

To provide more details, let us consider the pair of verbs \textit{povybe\v{z}at'}$^{\PF}$ -- \textit{po\-vy\-be\-gat'}$^{\PF}$ `to run out'. The sentence \ref{ex:povybegat} illustrates the usage of the second verb in this pair. The first verb, formed from the perfective \isi{derivational base} \textit{vybe\v{z}at'} `to run out', can also be used in the same sentence (the verb itself is colloquial) which is illustrated by \ref{ex:povybezhat}.

\ex.\label{ex:povy}\ag.\label{ex:povybegat}I povybegali$^{\PF}$ na ulicu, i stali smotret' v zv\"{e}zdnoe nebo i slu\v{s}at' goluboj zvon.\\
and po.vy.run.\glb{pst.sg.m} on street, and begin.\glb{pst.sg.m} look.\glb{inf} in starry sky and listen.\glb{inf} blue ringing\\
\trans `And they all ran out onto the street and started staring at the starry sky and listening to the blue ringing.'\\\Source{Sergej Kozlov. \textit{Pravda, my budem vsegda?}}
\bg.\label{ex:povybezhat}I povybe\v{z}ali$^{\PF}$ na ulicu, i stali smotret' v zv\"{e}zdnoe nebo i slu\v{s}at' goluboj zvon.\\
and po.vy.run.\glb{pst.sg.m} on street, and begin.\glb{pst.sg.m} look.\glb{inf} in starry sky and listen.\glb{inf} blue ringing\\
\trans `And they all ran out onto the street and started staring at the starry sky and listening to the blue ringing.'

If it were possible to imperfectivise the verb \textit{povybe\v{z}at'}$^{\PF}$ `to run out'  by \isi{suffixation}, that \isi{secondary imperfective} verb would have two interpretations: progressive and habitual. A \isi{progressive interpretation} in the above \isi{context} would mean that people are in the process of running out to the street. This meaning can be conveyed with the imperfective verb \textit{vybegat'}$^{\IPF}$ `to run/be running out', as exemplified by \ref{ex:vybegat} (the verb in the second clause has to be changed in order to satisfy discourse restrictions on the aspect of the verbs in the \isi{narrative sequence}, see Section~\ref{sec:tests:new} for more details). The second possible interpretation of a potential imperfective verb formed by suffixing the verb \textit{povybe\v{z}at'}$^{\PF}$ `to run out' is habitual: each time after a certain other event, people run out onto the street and stare at the sky. This interpretation is also a possible interpretation of sentence \ref{ex:vybegat}. So if we accept that there is competition between different verbs such that when the semantics of the two verbs is effectively the same,\footnote{As I provide a compositional account, it cannot be exactly the same in this case as the representation of the \isi{derivational base} gets updated after the \isi{prefixation} with \Prefix{po-}. The semantics being effectively the same means that when the formal representation is interpreted, there is no semantic difference between the two verbs.} only the verb that is \isi{morphologically simpler} can be used, the absence of the \isi{secondary imperfective} verbs derived from \Prefix{po-}prefixed verbs with the \isi{distributive} interpretation is expected.

\exg.\label{ex:vybegat}I vybegali$^{\IPF}$ na ulicu, i na\v{c}inali smotret' v zv\"{e}zdnoe nebo i slu\v{s}at' goluboj zvon.\\
and vy.run.\glb{pst.sg.m} on street and begin.\glb{pst.sg.m} look.\glb{inf} in starry sky and listen.\glb{inf} blue ringing\\
\trans `And they were running out onto the street and starting to stare at the starry sky and to listen to the blue ringing.'

This explanation is valid in case the only meaning that is contributed by the prefix is \isi{distributive}. Now let us explore what happens if there is a \isi{delimitative} component in the semantic contribution of \Prefix{po-}. Consider the verb \textit{poest'}$^{\PF}$ `to eat/to eat up', that we have already discussed. It can be suffixed with the \isi{imperfective suffix} and yield the imperfective verb \textit{poedat'}$^{\IPF}$ `to eat up/be eating up'. Examples \ref{ex:poedat1} and \ref{ex:poedat2} show how the habitual and the progressive interpretations of this verb can be expressed. Note that it is the submeaning `to eat up/destroy by eating' that is relevant in these contexts.\largerpage[2]

\ex.\ag.\label{ex:poedat1}V dikoj prirode tak u\v{z} zavedeno: milye i trogatel'nye zveru\v{s}ki poedajut drug druga.\\
in wild nature so well organised cute and touching beast.dim.\glb{pl.nom} po.eat.imp.\glb{pres.pl.3} friend.\glb{sg.nom} friend.\glb{sg.acc}\\
\trans `It is just like this in the wild: cute and touching animals eat each other up.'\Source{\url{mixstuff.ru}}
\bg.\label{ex:poedat2}Ja s\v{c}itaju, \v{c}to \v{c}inovniki -- \`{e}to takoe sugubo nadstroe\v{c}noe soslovie, kotoroe sej\v{c}as prosto poedaet stranu.\\
I consider.\glb{pres.sg.1} that official.\glb{pl.nom} {} this such especially superstructural estate that now simply po.eat.imp.\glb{pres.sg.3} country.\glb{sg.acc}\\
\trans `I think that officials are just a superstructural estate, that now is simply eating up the country.'\\\Source{Elena Semenova. \textit{Oligarx bez galstuka} (2003)}

Let us try to see why in this case the formation of the imperfective is not blocked. Consider sentences \ref{ex:est1} and \ref{ex:est2}, obtained by replacing the verb \textit{poedat'}$^{\IPF}$ `to eat up/be eating up' with the verb \textit{est'}$^{\IPF}$ `to eat' in the sentences \ref{ex:poedat1} and \ref{ex:poedat2}, respectively.
 
\ex.\ag.\label{ex:est1}V dikoj prirode tak u\v{z} zavedeno: milye i trogatel'nye zveru\v{s}ki edjat drug druga.\\
in wild nature so well organised: cute and touching beast.dim.\glb{pl.nom} po.eat.imp.\glb{pres.pl.3} friend.\glb{sg.nom} friend.\glb{sg.acc}\\
\trans `It is just like this in the wild nature: cute and touching animals eat each other.'
\bg.$^?$Ja s\v{c}itaju, \v{c}to \v{c}inovniki -- \`{e}to takoe sugubo nadstroe\v{c}noe soslovie, kotoroe sej\v{c}as prosto est stranu.\label{ex:est2}\\
\hspaceThis{$^?$}I count.\glb{pres.sg.1} that official.\glb{pl.nom} -- this such especially superstructural estate, that now simply po.eat.imp.\glb{pres.sg.3} country.\glb{sg.acc}\\
\trans `I think that officials are just a superstructural estate, that now is simply eating the country.'

The English translations of the sentence pairs \ref{ex:poedat1}/\ref{ex:est1} and \ref{ex:poedat2}/\ref{ex:est2} show that  the meaning changes when the verb \textit{poedat'} `to eat up/be eating up' is replaced by the verb \textit{est'} `to eat'. Sentence~\ref{ex:est1} lacks the destruction meaning component and is naturally interpreted as referring to a situation of two animals sitting and chewing each others' parts simultaneously. So the sentence \ref{ex:est1} can be uttered instead of \ref{ex:poedat1}, but it does not convey the same meaning.

The difference between the sentences \ref{ex:poedat2} and \ref{ex:est2} is even bigger: while sentence \ref{ex:poedat2} has the meaning that the country is being destroyed and in the end will be destroyed (`eaten up') completely by the officials, sentence \ref{ex:est2} sounds strange, as the verb \textit{est} `eats' lacks the figurative meaning of destroying and is interpreted literally as officials nourishing on the country. It also lacks the component of the intention to eat the whole country. In sum, the verb \textit{est'} `to eat' refers to a situation of eating literally, whereas the verb \textit{poest'} `to eat/to eat up' can have both the literal and the figurative meaning and the verb \textit{poedat'} `to eat up/be eating up' retains only the figurative part of the meaning. This is summarised in Table~\ref{table:eat}. For discussion of a similar phenomenon in English and Italian see \citet{FolliHarley:05}.\largerpage

\begin{table}
\caption{Distribution of literal and figurative meanings of \textit{est'} `to eat' and its derivatives \label{table:eat}}
\begin{tabular}{lll}
\lsptoprule
& literal & figurative \\ \midrule
IPF & est' & poedat' \\
PF & poest' & poest' \\ \lspbottomrule
\end{tabular}
\end{table}

The verb \textit{popriotkryvat'} `to open slightly' provides another illustration of the same phenomenon. As we have seen, it can have both \isi{distributive} and \isi{delimitative} interpretations. The \isi{derivational chains} in \ref{chain:popriotkryvat} show two ways in which the verb \textit{popriotkryvat'} `to open slightly' can be derived, each of which leads to a different aspect and a different interpretation of the verb: if the prefix \Prefix{po-} is attached in the last step of the derivation (chain \ref{chain:popriotkryvat:1}), the derived verb denotes a \isi{series of} slight opening events. If the \isi{imperfective suffix} is attached in the last step of the derivation (chain \ref{chain:popriotkryvat:2}), the derived verb is imperfective and denotes a set of very slight opening events. 

\ex.\label{chain:popriotkryvat}\ag.\label{chain:popriotkryvat:1}otkryt'$^{\PF}$ $\rightarrow$ priotkryt'$^{\PF}$ $\rightarrow$ priotkryvat'$^{\IPF}$ $\rightarrow$ popriotkryvat'$^{\PF}$\\
{to open} {} {to open slightly} {} {to (be) slightly open(ing)} {} {to slightly open multiple times}\\
\bg.\label{chain:popriotkryvat:2}otkryt'$^{\PF}$ $\rightarrow$ priotkryt'$^{\PF}$ $\rightarrow$ popriotkryt'$^{\PF}$ $\rightarrow$ popriotkryvat'$^{\IPF}$\\
{to open} {} {to open slightly} {} {to open very slightly} {} {to (be) open(ing) very slightly}\\

The imperfective aspect of the verb \textit{popriotkryvat'} `to open slightly' may be hard to access, but it is attested, as evidenced by example \ref{ex:popriotkryvat:ipf}. 

\exg.\label{ex:popriotkryvat:ipf}A e\v{s}\v{c}e pojavljaetsja prikol'naja, \v{c}isto pontovaja, vozmo\v{z}nost' poprikryvat' {$\backslash$} popriotkryvat' kry\v{s}ku v ljuboj moment.\\
but also po.apear.\glb{pres.sg.3}.refl neat pure {show off} possibility po.pri.close.\glb{inf} {$\backslash$} po.pri.open.\glb{inf} lid in any moment\\
`And you also have the neat, purely exhibitionistic, ability to very slightly close and open the lid at any moment.'\Source{\url{www.chevrolet-cruze-club.ru}}

Let us now consider example \ref{ex:popisyvat} where the imperfective verb \textit{popisyval} `wrote' seems to be interpreted distributively. This sentence means that the actor wrote his articles without devoting much time to it, non-seriously. So the prefix in this case delimits the time spent during each writing session, but not the length of the article: the sentence is interpreted in a way that the articles were probably completed and it is also possible that during each writing session a whole article was written. On the other hand, this does not have to be the case and can be explicitly denied, as is illustrated by \ref{ex:popisyvat:none}. The holistic implication is also lost if the direct object is singular \ref{ex:popisyvat:single}, as in this case occasional writing is only possible if the article is not completed. 

\exg.\label{ex:popisyvat}V svobodnoe vremja on popisyval statji.\\
in spare time he po.write.imp.\glb{pst.sg.m} article.\glb{pl.acc}\\
\trans `In his spare time he wrote articles.'

\exg.\label{ex:popisyvat:none}V svobodnoe vremja on popisyval staji, no ni odnu ne zakon\v{c}il.\\
in spare time he po.write.imp.\glb{pst.sg.m} article.\glb{pl.acc} but nor one not za.complete.\glb{pst.sg.m}\\
\trans `In his spare time he wrote articles, but never finished any of them.'

\exg.\label{ex:popisyvat:single}V svobodnoe vremja on popisyval statju.\\
in spare time he po.write.imp.\glb{pst.sg.m} article.\glb{sg.acc}\\
\trans `In his spare time he was writing an article.'

\exg.\label{ex:pisatstatji}V svobodnoe vremja on pisal statji.\\
in spare time he write.\glb{pst.sg.m} article.\glb{pl.acc}\\
\trans `In his spare time he wrote articles.'

This serves as evidence that the \isi{delimitative} interpretation of the prefix \Prefix{po-} only arises when the progression of the event is not related to the scale contributed by the direct object. The plural object creates the distributivity effect, which is also present in case of the \isi{non-prefixed} verb: sentence \ref{ex:pisatstatji} lacks the component of `non-serious occupation that does not take much time', but still refers to the situation of multiple articles being written on multiple occasions. 
%This is also related to what has been said above about the difficulty of separating the \isi{delimitative} and the \isi{distributive} usages of \Prefix{po-}. 

\subsection{Summary}
I propose to provide a unified formal representation for the \isi{delimitative}, \isi{resultative}, and \isi{distributive} usages of the prefix \Prefix{po-}, thereby covering all the interpretations provided by \citet{Shvedova:82}. The following observations are crucial for the construction of the desired semantic representation:

\begin{itemize}
\item \Prefix{po-} can be attached to different \isi{scales}; in the default case, the scale is one of the verbal \isi{scales}; if an event denoted by the \isi{derivational base} is an iteration, a \textit{cardinality} scale provided by the direct object can be used as well;
\item if the scale selected by \Prefix{po-} is of type \textit{cardinality}, then the start point of the event gets linked to the minimum point on the scale and the end point of the event gets linked to the \isi{maximum point} on the scale; if the scale is a verbal scale, an arbitrary point on (the open end of) the scale is linked to the respective endpoint of the event;
\item in case the endpoint of the event results in being linked to an arbitrary point of the scale, \isi{pragmatic strengthening} can take place if there are other verbs capable of denoting events corresponding to some definite portions of the scale (for more details see Chapter~\ref{Chapter6}).
\end{itemize}


%\begin{avm}
%      \[\asort{event}
%             \feat{dim\_spec} & \[\@x\]\\
%             \feat{measure} & \[
%             	\asort{scale $\wedge$ \@x}
%             	\feat{min} & \@y\\
%             	\feat{max} & \@z \]\\
%             \feat{startp} & \@y\\
%             \feat{endp} & \@z
%        \]
%\end{avm}\\

\section{\textit{pere-}}\label{subsection:semantics:pere}
\subsection{Semantic contribution}
The prefix \Prefix{pere-} is notoriously polysemous. To start, we will consult \citet[pp. 363--364]{Shvedova:82}, who distinguishes the following ten meanings that the prefix may contribute to the semantics of the derived verb:
\begin{enumerate}
\item to direct the action denoted by the \isi{derivational base} from one place to another through space or over another object: \textit{perenesti'} `to carry something over something', \textit{perebrosit'} `to throw over' (productive usage, some derivational bases are perfective); 
\item place something between other objects or parts of other objects by performing an action denoted by the \isi{derivational base}: \textit{peresypat'} `to pour something between something else' (non-productive); 
\item to perform the action denoted by the \isi{derivational base} again or \isi{anew}: \textit{peredelat'} `\isi{to redo}', \textit{pereizbrat'} `to reelect', \textit{pereproektirovat'} `to redesign', \textit{pereoborudovat'} `to reequip' (productive usage, some derivational bases are perfective or biaspectual, some derived verbs are biaspectual);
\item to perform the action \isi{multiple times} with different objects of the same kind or by different subjects: \textit{pereglotat'} `to swallow all of something one by one', \textit{perezarazit'} `to infect all of', \textit{pereranit'} `to wound all of' (productive usage, some derivational bases are perfective or biaspectual);
\item to perform the action denoted by the \isi{derivational base} with too much intensity or for too long a time: \textit{peregret'} `to overheat' (productive); 
\item to perform the action denoted by the \isi{derivational base} intensively: \textit{perepugat'} `to scare a lot' (non-productive); 
\item to overcome someone else, performing an action denoted by the \isi{derivational base}: \textit{peresporit'} `to win the argument' (productive, derived verbs are obligatory transitive); 
\item to perform the action denoted by the \isi{derivational base} for a \isi{predefined time}: \textit{pere\v{z}dat'} `to pass the necessary time waiting' (productive in colloquial speech);
\item to stop the state, process or activity denoted by the \isi{derivational base} after a long period: \textit{perebolet'} `to recover from illness' (productive); 
\item a short, non-intense action, performed during a pause of another action: \textit{perekurit'} `to smoke, taking a break' (non-productive).
\end{enumerate} 

This is a detailed list of \Prefix{pere-} usages, some of which can be merged. For example, \citet[119--125]{Kagan:book} provides a unified account covering the following five different meanings of \Prefix{pere-}: 
\begin{enumerate}
\item `\isi{to cross}' (corresponds to the first usage in the list above, see example \ref{ex:pere:cross});
\item `\isi{to redo}' (corresponds to the third usage in the list above, see example \ref{ex:pere:redo});
\item \isi{excess} (corresponds to the fifth usage in the list above, see example \ref{ex:pere:excess});
\item \isi{comparison} (corresponds to the seventh usage in the list above, see example \ref{ex:pere:comparison});
\item spending time (corresponds to the usages eight, nine, and ten in the list above, see example \ref{ex:pere:time});
\end{enumerate}

\ex.\label{ex:pere}\ag.\label{ex:pere:cross}Vasja pereplyl reku.\\
Vasja pere.swim.\glb{pst.sg.m} river.\glb{sg.acc}\\
\trans `Vasja swam across the river.'
\bg.\label{ex:pere:redo}Vasja perepisal examen.\\
Vasja pere.write.\glb{pst.sg.m} exam.\glb{sg.acc}\\
\trans `Vasja rewrote the exam.'
\bg.\label{ex:pere:excess}Vasja peregrel sup.\\
Vasja pere.warm.\glb{pst.sg.m} soup.\glb{sg.acc}\\
\trans `Vasja overheated the soup.'
\bg.\label{ex:pere:comparison}Vasja pereigral Ma\v{s}u.\\
Vasja pere.play.\glb{pst.sg.m} Masha.\glb{acc}\\
\trans `Vasja outplayed Masha.'
\bg.\label{ex:pere:time}Vasja pere\v{z}dal do\v{z}d'.\\
Vasja pere.wait.\glb{pst.sg.m} rain.\glb{sg.acc}\\
\trans `Vasja waited for the rain to stop.'

Let me show how \citet{Kagan:book} unifies different usages of the prefix \Prefix{pere-}. For the base meaning, \citet[120--121]{Kagan:book}, following \citet{Janda:88}, takes the \isi{spatial} interpretation `\isi{to cross}'. Here is the characterisation that \citet[121]{Kagan:book} gives for the underlying meaning of \Prefix{pere-}: ``[t]here is a certain \isi{spatial} location, and the individual that undergoes motion moves through this location, eventually getting to `the other side'.'' Based on this, \citet[122]{Kagan:book} proposes that the ``prefix imposes a relation of inclusion between two intervals on a scale''. This is formalised in \ref{Kagan:pere}, where d$_s$ refers to the \isi{contextually provided} \isi{standard degree}.

\ex.\label{Kagan:pere}$\llbracket \Prefix{pere-} \rrbracket = \lambda$P$\lambda$d$_s\lambda$d$\lambda$x$\lambda$e.[P(d)(x)(e) $\wedge$ d$_s \subseteq _U$ d]\\
where d = degree of change \citep{KennedyLevin:02} and $\subseteq _U$ is defined as\\
$\forall$d$\forall$d' [d $\supset$ d' $\leftrightarrow$ (d $\supset$ d' $\wedge$ max \{p : p $\in$ d\} $>$ max \{p: p $\in$ d'\})]\\\Source{\citep[from][123]{Kagan:book}}

The formal semantics in \ref{Kagan:pere} give rise to the \isi{spatial} meaning of \Prefix{pere-} when applied to the \textit{path} scale. When the same is applied to the \textit{time} scale, the meaning `to spend some particular time' arises. So the event of swimming described in \ref{ex:pere:cross} is terminated when the path covered in course of swimming includes the width of the (deep part of the) river. As for \ref{ex:pere:time}, the time of the waiting event is determined by the time of the rain: the waiting started when the rain started (or shortly after) and the waiting stopped when the rain was over (or became insignificant).

\subsubsection{Excessive and comparison usages}
In order to derive meanings of \isi{excess} and \isi{comparison}, \citet[133]{Kagan:book} additionally strengthens the representation in \ref{Kagan:pere} by replacing the \isi{upper inclusion} ($\subseteq _U$) relation with the proper \isi{upper inclusion} ($\subset _U$). This is motivated by the fact that a sentence such as \ref{ex:pere:excess} refers to a situation when Vasja heated the soup more than the soup should be heated. (Note that \ref{ex:pere:excess} cannot be uttered in a situation when Vasja heated (and thus immediately started to overheat) the soup that was already hot at the moment Vasja started to heat it.) Similarly, sentence \ref{ex:pere:comparison} refers to a situation where Vasja played better or longer than Masha, not equally good or long.

Two meanings are related to two different sources of \isi{scales}. Consider the example \ref{ex:pere:comparison}. The only scale that is present in the semantic representation of the verb \textit{igrat'} `to play' is the \isi{time scale}. If \Prefix{pere-} is attached to it, we find ourselves in the \textit{excess} situation: the verb \textit{pereigrat'} `to play for too long' refers to exceeding the time of playing appropriate for the subject. Again, the verb \textit{pereigrat'} `to play for too long' cannot refer to a situation where any time of playing would be too long (in other words, when the playing starts at the point that marks the appropriate time for the subject to play). Together with the verbs \textit{poigrat'} `to play for some time' and \textit{proigrat' (3 \v{c}asa)} `to play continuously (for 3 hours)' the verb \textit{pereigrat'} `to play for too long' covers the domain of possible time-related meanings the speaker may want to express with respect to the playing event.

To acquire the \isi{comparison} meaning, the verb has to become transitive, as noted by \citet{Shvedova:82}. The reason for this is that when it becomes transitive, the direct object becomes another, \isi{external}, source of \isi{scales}. The process of obtaining a scale may not be straightforward, though. An individual (e.g., \textit{Masha} in example \ref{ex:pere:comparison}) is not a scale. So, in order to interpret the sentence, the scale has to be constructed. I propose to describe the scale construction process as proceeding along the following lines. First, one of the \isi{scales} that are relevant in the situation described by the verb is picked (this can be playing quality or playing length in our example); second, one point that corresponds to the performance of the individual that is denoted by the direct object (how well or how long has Masha played) is \isi{marked on this scale}. When this is done, the situation is no longer different from that of playing too much, where a point that represents the appropriate time of playing for the subject is marked on the \isi{time scale}.

Before we proceed, I would like to make two observations that concern the \isi{comparison} meaning and reveal some details about the structure of this meaning. First, note that verbs of \isi{comparison} illustrated in \ref{ex:pere} are only used in situations where the initial stage of the event favours the patient, not the actor (when they do not refer to the \isi{time scale}). This means that for sentence \ref{ex:pere:comparison} to be true it has to not only be the case that Vasja ended up outplaying Masha, but also that when Vasja started to play he had a weaker position than Masha. If this is not the case and they simultaneously start to play without expectations who will be playing better, another verb, \textit{obygrat' X} `to win from X' will be used, as in example \ref{ex:comparison:obygrat}. 

\exg.\label{ex:comparison:obygrat}Vasja obygral Ma\v{s}u.\\
Vasja ob.play.\glb{pst.sg.m} Masha.\glb{acc}\\
\trans `Vasja won against Masha.'

Another illustrative pair of examples is given in \ref{ex:comparison:peregnat} and \ref{ex:comparison:obognat}, where the verb prefixed with \Prefix{pere-} (\textit{peregnat'} `to overtake') is used in the situation when the actor was located behind the patient (in the literal or metaphorical sense) at the beginning of the event, whereas the verb prefixed with {\textit{ob-},} \textit{obognat'} `to overtake' lacks this requirement: sentence \ref{ex:comparison:obognat} can be used in a situation when the height of the trunks has been exactly the same all the time. If we try to modify the sentence, replacing the verb \textit{obognat'} `to overtake' with the verb \textit{peregnat'} `to overtake', the resulting sentence in \ref{ex:peregnat} is suitable to use in a situation when the periods of the `height leadership' of one trunk are followed by the periods of the `height leadership' of the other.

\exg.\label{ex:comparison:peregnat}Dognal, kone\v{c}no, i peregnal, potom sbavil skorost' i poravnjalsja.\\
do.race.\glb{pst.sg.m} {of course} and pere.race.\glb{pst.sg.m} then reduce.\glb{pst.sg.m} speed and po.equal.\glb{pst.sg.m}.refl\\
\trans `I caught up, of course, and overtook, then reduced speed and came alongside.' \Source{I. Grekova. \textit{Na ispytanijax} (1967)}

\exg.\label{ex:comparison:obognat}Ix korni s maloletstva splelis', ix stvoly tjanulis' vverx rjadom k svetu, starajas' obognat' drug druga.\\
their roots from childhood weave.\glb{pst.pl}.refl their trunks stretch.\glb{pst.pl}.refl up near to light trying ob.race.\glb{inf} one another\\
\trans `Their roots got woven together from their childhood, their trunks stretch\-ed up to the sun, trying to overtake each other.'\\\Source{M. M. Pri\v{s}vin. \textit{Kladovaja solnca} (1945)}

\exg.\label{ex:peregnat}Ix korni s maloletstva splelis', ix stvoly tjanulis' vverx rjadom k svetu, starajas' peregnat' drug druga.\\
their roots from childhood weave.\glb{pst.pl}.refl their trunks stretch.\glb{pst.pl}.refl up near to light trying pere.race.\glb{inf} one another\\
\trans `Their roots got woven together from their childhood, their trunks stretch\-ed up to the sun, trying to overtake each other.'

The second observation concerns with cases where the \isi{time scale} is used for the \isi{comparison}. Let us consider an example provided by \citet[142]{Kagan:book} and repeated here in \ref{ex:Kagan:perezhit}. Sentence \ref{ex:Kagan:perezhit} refers to a situation when the lifespans of Dima and Masha overlap and there is an interval following Dima's death when Masha is still alive. This sentence can be uttered also in case Masha and Dima are conjoined twins and were born simultaneously, as is illustrated by the example \ref{ex:siam}.

\exg.\label{ex:Kagan:perezhit}Ma\v{s}a pere\v{z}ila Dimu.\\
Masha pere-lived Dima\\
\trans `Masha outlived Dima.'\Source{= example (50) in \citet{Kagan:book}}

\exg.\label{ex:siam}V Londone umerli razdelennye siamskie bliznecy: odna sestra pere\v{z}ila druguju na 4 nedeli.\\
in London die.\glb{pst.pl} separated conjoined twins: one sister pere.live.\glb{pst.sg.f} other on 4 weeks\\
\trans `In London, separated conjoined twins have died: one sister outlived the other for 4 weeks.'\Source{\url{http://newsru.com/arch/world/26dec2008/twins.html}}

Examples \ref{ex:Kagan:perezhit} and \ref{ex:siam} show that the only point on the scale that is taken from the information about the direct object is the date and time of death. The time when Dima was born does not matter for the truth conditions of \ref{ex:Kagan:perezhit}. So only the point of Dima's death becomes the fixed point on the scale and the information conveyed by sentence \ref{ex:Kagan:perezhit} is that Masha started to live at some time before the death of Dima, lived at the moment of the death of Dima, and stopped living at some time after the death of Dima. This is exactly what \citet{Kagan:book} considers this sentence to mean. 

The difference between the approach I offer and that of \citet{Kagan:book} is that \citet{Kagan:book} operates with a time interval (corresponding to Dima's lifespan in the discussed example),\footnote{\citet[143--144]{Kagan:book} has to deal with additional difficulties related to the elimination of the condition that Masha started to live not later than Dima. She proposes to use \textit{an upper part} of the time interval of Dima's life.} whereas I propose to use only one point (that of Dima's death). The value on the scale has to change from some value below this point to some value above it in the course of the event. As follows both from the explanations provided by \citet{Kagan:book} and from what we have just discussed, the information about the birth of Dima is of no importance for the interpretation of sentence \ref{ex:Kagan:perezhit}. So the proposal of \citet{Kagan:book} can be simplified by replacing the interval with the relevant point, as is done here. I will show how this works in Chapter~\ref{Chapter7}.

\subsubsection{Repetitive usage}
Now let us discuss how the analysis proposed by \citet{Kagan:book} can be extended to the \isi{repetitive} usage of the prefix \Prefix{pere-}, as this extension seems to be more tricky. \citet[149]{Kagan:book} provides a number of valuable observations in this respect, arriving at the conclusion that ``\isi{repetitive} \Prefix{pere-} is only possible with those predicates that contribute closed \isi{scales}''
such that ``an increase along the same scale can be repeated''. She also emphasises the importance of the event and its iteration being connected to each other. \citet[148]{Kagan:book} arrives at the following description of the important properties of the \isi{repetitive} meaning of \Prefix{pere-} (conditions (2) and (3) come together in the original proposal): 
\begin{enumerate}
\item ``An event that falls under the denotation of the VP (or brings about the same kind of \isi{result state}) is presupposed to have taken place before event time.'' 
\item ``The event predicate is interpreted as telic. Both the \isi{presupposed event} and the entailed one are associated with a \isi{natural endpoint}.'' 
\item ``In the course of the \isi{presupposed event}, this point [the \isi{natural endpoint}] has been reached.''
\item ``Typically, the entailed and the \isi{presupposed event} are interrelated and can be conceptually unified.''
\end{enumerate}

I agree with the second point about the telicity of the events and also with the last point about the two events being interrelated. We will discuss the first point in detail in the next chapter (Chapter~\ref{Chapter6}). 

As for the third point, there seems to be some confusion with respect to the identification of natural endpoints. \citet{Kagan:book} provides example \ref{ex:perestirat} to support her claim. She notices that \ref{ex:perestirat} cannot be uttered in the situation when the dress was first washed, then worn, became dirty and was washed again. A possible scenario would be one where the dress was washed but did not become clean and thus it had to be washed again. In this case the first event of washing terminates but it does not reach the \isi{natural endpoint} which corresponds to the clean state of the dress.

\exg.\label{ex:perestirat}Lena perestirala plat'e.\\
Lena pere-washed dress\\
\trans `Lena rewashed the dress.'\Source{= example (56) in \citet{Kagan:book}}

In fact it is even possible that the first washing was not complete: for example, the power could have gone off, the washing machine stopped without finishing its cycle and because of this the whole washing of the dress had to be redone. So it turns out that exactly the fact that the event did not reach the \isi{natural endpoint} motivates why the whole process must be repeated.

Another example \ref{ex:peresdat} describes a situation where a girl did not have a chance to finish the exam (which is the \isi{natural endpoint} of writing it) because she was expelled. Nevertheless, a new attempt to pass the same exam can be referred to by either the perfective verb \textit{peresdat'} `to retake' or the imperfective verb \textit{peresdavat'} `to retake/be retaking'. This situation is not compatible with one of the conclusions of \citet{Kagan:book}.

\exg.\label{ex:peresdat}Sud ne razre\v{s}il peresdat' EG\`{E} \v{s}kol'nice, kotoruju vygnali s \`{e}kzamena za spisyvanie.\\
court not allow.\glb{pst.sg.m} pere.s.give.\glb{inf} EGE schoolgirl.\glb{sg.dat}, that vy.chase.\glb{pst.pl} from exam for cheating\\
\trans `The court did not allow the schoolgirl to retake the EGE exam she was expelled from for cheating.'\Source{\url{http://www.newsmsk.com/}}

One more example to consider is provided in \ref{ex:perestelit}. The event of redoing the bed (changing the linen) does not require the bed to be done inappropriately. Sentence \ref{ex:perestelit} can be used in the situation when Katja did the bed, someone slept in it, it became dirty and she changed it. What I consider crucial here is that Katja had to undo the bed before doing it again. This is revealed in \isi{comparison} with sentence \ref{ex:postelit} where the verb prefixed with \Prefix{po-} denotes an event of doing the bed but does not require the bed to be undone as a preparatory step for the main event. 

\exg.\label{ex:perestelit}Katja perestelila postel'.\\
Katja pere.lay.\glb{pst.sg.f} bed\\
\trans `Katja changed the bedlinen.'

\exg.\label{ex:postelit}Katja postelila postel'.\\
Katja po.lay.\glb{pst.sg.f} bed\\
\trans `Katja made the bed.'

I think that the semantics of the \Prefix{pere-}prefixed verbs in examples \ref{ex:perestirat}, \ref{ex:peresdat}, and \ref{ex:perestelit} can be unified by imposing a requirement for the \isi{preparatory phase} of the event denoted by a \Prefix{pere-}prefixed verb. The \isi{preparatory phase} has to include the annulation of the result of the previous event. This can be represented as moving from the point on the scale that has been reached earlier back to the start point. In the case of \ref{ex:perestirat} an event of washing a dress after it has been washed and became dirty again is excluded due to the result of the washing being already annulled by the wearing of the dress. In case of the exam, the result of the previous attempt is annulled when the new attempt begins. If we are talking about redoing the bed, it still has bedlinen at the beginning of the redoing event and the fact it is dirty does not affect its presence. Thus we obtain the desired asymmetry between the examples \ref{ex:perestirat} and \ref{ex:perestelit}. This approach also works in other cases discussed in \citealt{Kagan:book} with respect to the \isi{repetitive} usage of the prefix \Prefix{pere-}.

In sum, I propose to weaken the condition formulated by \citet{Kagan:book} that the first event must reach the \isi{natural endpoint} and make the last condition about the two events being interrelated more precise. This is done by introducing the \isi{preparatory phase} that includes an event that proceeded along the same scale and had some final stage associated with a certain point on this scale. The transition from the \isi{preparatory phase} to the main event then necessarily includes annuling the result of the preparatory event, as this corresponds to the transition to the minimum point of the scale (that is, in turn, the initial stage of the main event).

%From what I have just proposed it follows that the only requirement on the state of the world at the beginning of the \isi{preparatory phase} of the event denoted by the verb prefixed with the \isi{repetitive} \Prefix{pere-} is that this state corresponds to the non-zero point on the relevant scale. This means that it is not necessarily the case that the event itself has to be repeated. It proves to be the correct prediction due to the presence of the examples like \ref{ex:perekrasit}
%
%\exg.\label{ex:perekrasit}Ja pervyj raz perekrasila volosy v 14 let, do six por svoj cvet vosstanovit' ne mogu, a krasit'sja snova i snova - volosy \v{z}al'.\\
%I first time pere.colour.\glb{pst.sg.f} hair in 14 years until these time my colour regain not can but colour.\glb{inf.refl} again and again {} hair sorry\\
%\vspace{0.5em}
%`When I dyed my hair for the first time I was 14; I still cannot regain my natural colour and dying the hair again and again is a pity.'\\
%\vspace{-0.5em}
%\begin{flushright}
%chatic.net
%\end{flushright}
%
%
%In the example \ref{ex:perekrasit} the verb \textit{perekrasila} `recoloured' refers to the first time the actor ever dyed her hair. Obviously, her hair had some colour before the event of dying (which satisfies the requirement of the \isi{repetitive} prefix \Prefix{pere-}) but no dying ever occurred before. 

There is a certain flexibility with respect to the scale selection that leads to various possible interpretations of the same \isi{repetitive} verb. For example, the verb \textit{pere\v{s}it'} `to resew' often refers to changing a piece of clothing to fit the size of the other person without changing its kind, as in example \ref{ex:pereshit:same}.\largerpage[-1]

\ex.\label{ex:pereshit}\ag.\label{ex:pereshit:same}ona s udovol'stviem pere\v{s}ila na devo\v{c}ek svoi svetlye, v melkij cveto\v{c}ek, v veno\v{c}ek, v buketik plat'ja\\
she with pleasure pere.sew.\glb{pst.sg.f} on girls her light in little flower.dim in wreath.dim in bouquet.of.flowers.dim dresses\\
\trans `she took pleasure in resewing her light dresses with prints of little flowers, wreathes and bouquets for girls'\\\Source{Ljudmila Ulickaja. \textit{Kazus Kukockogo} (2000)}
\bg.\label{ex:pereshit:other}A barin-to byl v pot\"{e}rtom pal'ti\v{s}ke, pere\v{s}itom iz soldatskoj \v{s}ineli\\
but barin-\glb{particle} was in shabby coat.dim pere.sew.\glb{part.pst.sg.m.prp} from soldier greatcoat\\
\trans `And the barin himself was in a shabby coat resewn from a military greatcoat'\Source{V. P. Kataev. \textit{Almaznyj moj venec} (1975--1977)}

It is also possible to utter the verb \textit{pere\v{s}it'} `to resew' when a piece of clothing is transformed into another, as in example \ref{ex:pereshit:other}, where the coat that comes into existence as a result of the resewing event is no longer the greatcoat it used to be. This points to the fact that the scale is not necessarily bound to the type of object sewn in case of the verb \textit{\v{s}it'} `to sew'. In such cases, however, the mismatch has to be explicitly specified. E.g., it is not possible to understand sentence \ref{ex:pereshit:same} as an event after which some other clothes, not dresses, come into existence.  If the type of clothing changes in the process of resewing, the material used has to remain the same. This means that the scale of completeness associated with the sewn piece of clothing is also related to the material used in the sewing.

%It has to include information about the event X that follows some other event Y. If it is denied that the event X that follows the event Y took place, there is no denial nor assertion that the event Y took place. 

One more remark that I want to add before we proceed to the \isi{distributive} usage of the prefix \Prefix{pere-} is that the \isi{repetitive} usage is more frequent and flexible than it may seem. Even if the attachment of the \isi{repetitive} \Prefix{pere-} seems impossible, as with the verb \textit{napisat'} `to write down', it is occasionally produced by native speakers when they need to express the relevant meaning, as illustrated by \ref{ex:perenapisat}. 

\exg.\label{ex:perenapisat}Mog by kto-to perenapisat' \`{e}tu programmu, no tol'ko v si?\\
can would someone pere.na.write.\glb{inf} this program but only in C\\
\trans `Could someone reprogram this in C?'\Source{\url{www.cyberforum.ru}}

Usually the verb \textit{perepisat'} `to copy/rewrite' can be used to refer to rewriting, but it means either copying or rewriting and correcting something that already exists. The semantics of the verb \textit{perepisat'} `to copy/rewrite' includes limiting the activity denoted by the verb \textit{pisat'} `to write' and relating it to another writing event that proceeds along the same scale. Now if we consider the attachment of the \Prefix{pere-} prefix in its \isi{repetitive} usage to the verb \textit{napisat'} `to write down', the derived verb would be able to denote not only copying and rewriting something that turned out to be not good enough (for this, there is a \isi{morphologically simpler} alternative -- the verb \textit{perepisat'} `to copy/rewrite'), but also creating something written again. This meaning is derived from `to create something written' interpretation of the verb \textit{napisat'}. This interpretation cannot be obtained by simply bounding the activity denoted by the verb \textit{pisat'} `to write'. Thus the verb \textit{perepisat'} `to copy/rewrite' cannot be used in contexts like \ref{ex:perenapisat}, where not only the writing per se has to be performed, but also the thinking and creating the structure of the code has to be redone to make the program function in the other language.

One more aspect that is related to the \isi{repetitive} usage of the prefix \Prefix{pere-} is the realisation of the requirement for the presence of a closed scale in the event structure. If \Prefix{pere-} is attached to a perfective verb or to a \isi{secondary imperfective} verb, this requirement is automatically satisfied. Complications occur when the \isi{derivational base} is a basic imperfective verb, such as \textit{\v{c}itat'} `to read'. As long as the \isi{derivational base} refers to an unbounded event, the mechanism of constructing the \isi{repetitive} meaning, described above, cannot be applied: there is no \isi{result state} that can be annulled to license the \isi{repetitive} interpretation as neither the final nor the initial stage of the event is defined. A way out in this case is to allow coercion that will select a scale using the \isi{context} (e.g., a scale associated with the direct object) and map the beginning of the event onto the minimum point of this scale and the end of the event onto some other point on the same scale. (Note that a possible way to do this is to leave the scale underspecified by using a variable to identify it and provide the mapping that will be supplied with values later when the semantic representations of the arguments of the verb become available.)

\subsubsection{Distributive usage}
The last usage of the prefix \Prefix{pere-} that we are going to explore is \isi{distributive}. We have already discussed the \isi{distributive} usage of the prefix \Prefix{po-} in Section~\ref{subsection:semantics:po}, so let us compare them, considering the examples \ref{ex:distr:pere} and \ref{ex:distr:po}.

\ex.\ag.\label{ex:distr:pere}Ira pere\v{c}itala vse knigi v biblioteke.\\
Ira pere.read.\glb{pst.sg.f} all books in library\\
\trans `Ira read all the books in the library.'
\bg.\label{ex:distr:po}Ira po\v{c}itala vse knigi v biblioteke.\\
Ira po.read.\glb{pst.sg.f} all books in library\\
\trans `Ira read from all the books in the library.'

Two main differences can be spotted between the situations that the sentences \ref{ex:distr:pere} and \ref{ex:distr:po} can refer to:\largerpage
\begin{enumerate}
\item when the reading event is referred to by the verb \textit{pere\v{c}itat'} `to read all of', events of reading single books are clearly individualised;
\item \ref{ex:distr:pere} denotes an event of reading all the books through, whereas \ref{ex:distr:po} is compatible with the situation of reading only certain portions of every book.
\end{enumerate}

The first difference can be addressed by saying that the prefix \Prefix{pere-} requires a proper \isi{cardinality scale} as an input, whereas the prefix \Prefix{po-} does not impose such a requirement. Let me explain this in more detail. A natural form of representation of plural individualised objects is a set. When we deal with a \Prefix{po-}prefixed verb, we describe the event as happening with all the objects in this set by starting the event when zero objects have been affected and ending when all the objects have been affected. This is achieved by using the \isi{measure of change scale} on which the cardinality of the set corresponds to the \isi{maximum point} but there is no mapping between the subsets of the objects and the intermediate points on the scale.

If we choose to describe the event using the \Prefix{pere-}prefixed verb, such a structure is not sufficient and a \isi{proper scale} that fixes not only the extreme points, but also all the intermediate points on the scale, is needed. It is important that the \isi{subevents} do not overlap when the situation is described with the \Prefix{pere-}prefixed verb. For example, if Misha had five balloons and made them burst one by one, both \ref{ex:balloon:pere} and \ref{ex:balloon:po} can be used. If he was jumping on the balloons and each landing made some balloons burst (e.g., with his first jump he destroyed two balloons, then one, and then another two), then only description \ref{ex:balloon:po} is suitable.\largerpage

\ex.\ag.\label{ex:balloon:pere}Mi\v{s}a perelopal vse \v{s}ary.\\
Mi\v{s}a pere.burst.\glb{pst.sg.m} all ballon.\glb{pl.acc}\\
\trans `Misha bursted all the ballons (one by one).'
\bg.\label{ex:balloon:po}Mi\v{s}a polopal vse \v{s}ary.\\
Mi\v{s}a po.burst.\glb{pst.sg.m} all balloon.\glb{pl.acc}\\
\trans `Misha bursted all the ballons.'

The difference in the requirements of the \Prefix{pere-} and \Prefix{po-}prefixed verbs is also revealed when the direct object is a mass noun: in such a case, only \Prefix{po-}prefixed verbs can be interpreted distributively, as \ref{ex:po:merz}, and \Prefix{pere-}prefixed verbs need to acquire some other interpretation, as in \ref{ex:pere:merz}, where the verb \textit{perem\"{e}rz} `he froze' is interpreted excessively. I explain this by a lack of a mechanism that would extract a \isi{proper scale} from a \isi{cumulative} description.

\ex.\ag.\label{ex:po:merz}Pom\"{e}rzla karto\v{s}ka-to u nas none, vsja pom\"{e}rzla.\\
po.freeze.\glb{pst.sg.f} potato-\glb{particle} at our now all po.freeze.\glb{pst.sg.f}\\
\trans `Our potato plants got frozen now, all of them.'\\\Source{V. G. Korolenko. \textit{\v{C}udnaja} (1880)}
\bg.\label{ex:pere:merz}Minuv\v{s}aja zima byla o\v{c}en' surovoj, i u mnogix uro\v{z}aj perem\"{e}rz v ovo\v{s}\v{c}exranili\v{s}\v{c}ax.\\
last winter was very severe and at many harvest pere.freeze.\glb{pst.sg.m} in vegetable.store\\
\trans `Last winter was very severe and many people lost their harvest in the vegetable stores as it was frozen.'\Source{\url{www.molsib.info}}

Another condition that has to be observed in order to obtain the \isi{distributive} interpretation is that performing the action denoted by the \isi{derivational base} with all the objects that are ordered to form a scale is only possible if every subevent (performing the action with a particular object) is somehow limited. (This is similar to what we have discussed with respect to the \isi{repetitive} usage of the prefix \Prefix{pere-}.) In other words, in order to map the whole event denoted by the \isi{distributive} \Prefix{pere-}prefixed verb onto the \isi{time scale} and ensure that the \isi{subevents} do not overlap, we need to know not only the order of \isi{subevents} (determined according to the order acquired when a \isi{proper scale} is constructed), but also the \isi{duration} of each subevent. I propose to use the coersion mechanism in this case to delimit individual \isi{subevents} if the \isi{derivational base} is a simplex imperfective verb.

Another point that has to be mentioned with respect to the \isi{distributive} usage of the prefix \Prefix{pere-} is that it cannot arise when the prefix is attached to a perfective verb. This has been noticed by \citet{Tatevosov:09}, who identifies this usage of the prefix as selectionally limited. Indeed, when we try to attach the prefix \Prefix{pere-} to a perfective verb, we obtain a verb with \isi{repetitive} and not \isi{distributive} interpretation: prefixing the verb \textit{otkryt'} `to open' provides us with the verb \textit{pereotkryt'} `to open again', prefixing the verb \textit{zapisat'} `to write down/to record' leads to the verb \textit{perezapisat'} `to write down \isi{anew}/rerecord', but not `to write down/record all of'. This naturally follows from the semantic structure of \isi{perfective verbs} according to the view I propose.

Let us consider the verb \textit{zapisat'} `to write down/to record'. In its semantic structure this verb carries information that the start of the writing event is related to the minimum point of the scale contributed by the direct object. The end of the event is related to the \isi{maximum point} on the same scale. It is a scale of the measure of change type and the maximum of this scale is either the length of the direct object, if it is singular, or the number of objects, if the direct object is plural. What it cannot be is the length of one object belonging to the set denoted by the plural direct object. And if the \isi{distributive} \Prefix{pere-} was added to the verb, this is exactly what had to be denoted by the \isi{embedded event}. This is easier to see by looking at the formal representations (see Chapter~\ref{Chapter6}).

Another approach is offered by \citet{Demjjanow:97} who suggests that the \isi{distributive} interpretation of the prefix \Prefix{pere-} should share the prefix schema with the \isi{repetitive} interpretation. This is motivated by the idea that verbs prefixed with the \isi{distributive} \Prefix{pere-} trigger presuppositions (similarly to the verbs prefixed with the \isi{repetitive} \Prefix{pere-}). As an example, \citet{Demjjanow:97} provides sentence \ref{ex:demj} that she claims means that some of the candles were blown out.

\exg.\label{ex:demj}On ne peretu\v{s}il vse sve\v{c}i.\\
he not pere.blow.out.\glb{pst.sg.m} all candles\\
\trans `He did not blow out all the candles.'\\\Source{= example (153) in \citealt[120]{Demjjanow:97}}

Here I only want show that it is not required that any part of the action denoted by the \isi{distributive} \Prefix{pere-}prefixed verb was performed if such verb is uttered under negation. The presuppositional view on the \isi{repetitive} usage of the prefix \Prefix{pere-} will be discussed in Chapter~\ref{Chapter6}. Indeed, the most natural interpretation of \ref{ex:perelistat} is that the editor (Panferov) did not look through any part of the manuscript.

\exg.\label{ex:perelistat}Pridja v redakciju ``Oktjabrja'', Juz polo\v{z}il pered Panferovym tolstuju rukopis', i tot, da\v{z}e ne perelistav, napisal na nej: ``V nabor''.\\
come.\glb{part} in {editorial office} Oktjabr'.\glb{gen} Juz po.lay.\glb{pst.sg.m} {in front} Panferov.\glb{inst} thick manuscript and that even not pere.thumb.\glb{part.pst}, na.write.\glb{pst.sg.m} on she.\glb{prp} in 	print\\
\vspace{0.5em}
When Juz came to the editorial office of Oktjabr' and laid a thick manuscript in front of Panferov, Panferov, without even thumbing through it, wrote on it: ``Publish.''\Source{Samuil Ale\v{s}in. \textit{Vstre\v{c}i na gre\v{s}noj zemle} (2001)}



%A possible explanation of such behaviour is a suggestion that \isi{distributive} \Prefix{pere-} is attached a lot easier in case there is already some iteration in the denotation of the \isi{derivational base}. In this case what \Prefix{pere-} does is not introducing the iteration, but imposing an order (and this is the main difference between the \isi{distributive} \Prefix{po-} and \isi{distributive} \Prefix{pere-}) and delimiting the event (by iterating through all of the members of the set denoted by the direct object).

\subsection{Restrictions on attachment}
I claim that all the usages discussed above except for the \isi{repetitive} one (but including the \isi{distributive}), can be unified using the idea that \Prefix{pere-} can be only attached to a scale that is closed and \isi{non-binary}. In other words, the scale that \Prefix{pere-} selects for must contain at least three distinct points. Along with this strong requirement (in \isi{comparison} with other prefixes) there are several ways to construct an appropriate scale and this explains the polysemous nature of the prefix. 

Let the two extreme points on the scale $s$ that is provided as an input for the \isi{prefixation} with \Prefix{pere-} be $x$ and $z$ and the set $Y$ be the set of all intermediate points $y$ such that $\forall y \subset Y~:~x~<~y~<~z$. All the intermediate points must be ordered as well. The prefix requires that $Y$ is not empty. This corresponds to a Complex type in terms of \citealt{Beavers:12} (44c).\footnote{In earlier work, \citealt{Beavers:02} and \citealt{Beavers:08}, the notion of Non-Minimally Complex Object is used.} I propose the following general procedure for acquiring a scale that \Prefix{pere-} can attach to. 
\begin{enumerate}
\item If the direct object provides a closed scale that is \isi{non-binary}, $x$ is the minimum of this scale, $z$ is the maximum and $Y$ is the set of all the intermediate points.\footnote{Note that extracting a \textit{path} scale from the direct object that refers to some landmark is also a complex process, as the \textit{path} scale is not present in the semantic structure of the object, but has to be constructed taking into account the position of the subject.}
\item If the direct object (possibly in combination with the \isi{context}) provides a single point on some scale, this point becomes a member of the set $Y$. The points $x$ and $z$ are chosen arbitrarily in such a way that they are located below and above the marked point on the scale, respectively. 
\item If the direct object denotes a set, the scale is constructed by arranging the equivalence classes corresponding to the gradually increasing number of objects: $x$ is 0, $z$ is the cardinality of the set, and $Y$ contains points that represent \isi{subevents} related to the subsets consisting of a whole number of objects in the set (the first point in $Y$ is an equivalence class of all single objects in the set, the second point is the equivalence class of all pairs of objects, the third point is the equivalence class of all triplets, etc.).
%\item If the direct object provides measure of change information and there is a way the event can proceed along the same scale again, the scale is acquired by assuming that $x$ is some point on the scale that reflects the state of the world before the event, the set $Y$ contains one point that is the zero point of the \isi{measure of change scale}, and $z$ is the \isi{maximum point} of the \isi{measure of change scale}. Note that this new constructed scale is such that on its first interval from from $x$ to $y$ it is the reverse of the \isi{measure of change scale}, which means that the first transition is the decrease on the \isi{measure of change scale} of the source event.
\end{enumerate}

%The third step is to convert this scale with one marked point into a closed scale. As the scale we are dealing with does not have two distinguished minimum and maximum points, the reasonable way to select points that will be related to the event start and end is to pick two arbitrary values around the fixed \isi{comparison} point. 

This scale selection is motivated by the idea that when \Prefix{pere-} is attached to a verb, the action denoted by that verb has to be performed at all the intermediate points on the relevant scale and each point on that scale has to correspond to some subevent. If the scale is dense (first case described above), as with \textit{time} and \textit{path} \isi{scales}, this will mean performing the action while moving along the scale. If the scale is discrete (third case), as with the \textit{cardinality} type of \isi{scales}, the verb prefixed with \Prefix{pere-} acquires a \isi{distributive} interpretation. 

The attachment of \Prefix{pere-} results in the following types of mappings: %If $Y$ contains a single point $y$, the event consists of two phases: the \isi{preparatory phase} and the main phase. The \isi{preparatory phase} of the event starts when the value on the scale $s$ equals $x$ and ends when it is equal $y$. The event itself starts when the value on the scale $s$ is equal $y$ and ends when it is equal $z$. 
if $Y$ contains multiple points, the event consists of the iteration of the event denoted by the \isi{derivational base} for each point on the scale until the point $z$ is reached. Each individual event is measured according to the \isi{measure of change scale} of the corresponding element.\largerpage

If $Y$ contains a single point or an infinite number of points, the event proceeds along the scale $s$ from $x$ to $z$ through all the points in $Y$. This mapping can be unified with the previous one (for multiple points) if the continuous movement along the scale is represented as an iterated movement through the infinite number of points on the closed scale. I do not think that this is computationally reasonable and prefer to have two separate representations for the implementation.

The process of scale selection I propose does not rely on the semantics of the verbal roots and it is even independent of the scale dimension. For example, usually those verbs that lexicalise \textit{path} and \textit{time} \isi{scales} acquire the crossing semantics that relies on traversing all the points on the scale (related to the scale of the type 1 in the list above). But they can also acquire the interpretation using the same mechanism as is used for the \isi{excess} meaning (second procedure in the list above). This happens when the direct object denotes something that is conceptualised as having point-like width or point-like \isi{duration}. In the case of point-like width, unlike the case of non point-like width, the crossing event has to start in front of the crossed object and end behind it and not on its border.

For example, the phrase \ref{ex:perejti} cannot be uttered in a situation when someone steps over the puddle on their way. The actor has to step into the puddle at least once and at the same time it is enough that the actor crosses the puddle with the last step on the border of the puddle and not outside it. If the crossed object is conceptualised as being point-like, then the event necessarily starts and ends on the different sides of the object: in this case, stepping over the same puddle can be described by \ref{ex:pereshagnut} and the end point of the motion cannot be in the puddle.

%Remark: in case when motion consists of smaller events, the path consists of those points where a subevent either begins or ends:  in the situation when there are grey and red pavement tiles the event of walking when stepping on the red tiles only can be referred to `idti po krasnym plitkam.'

\ex.\ag.\label{ex:perejti}perejti lu\v{z}u\\
pere.go.\glb{inf} puddle\\
\trans `\isi{to cross} the puddle'
\bg.\label{ex:pereshagnut}pere\v{s}agnut' lu\v{z}u\\
pere.step.\glb{inf} puddle\\
\trans `to step over the puddle'

This approach accounts for the ambiguity allowed in the analysis of \citet{Kagan:book} by the absence of the proper \isi{upper inclusion} constraint: verbs that acquire \textit{path}- and \textit{time}-related semantics denote events events with a measure that is either equal to or exceeds the measure contributed by the direct object. The analysis I offer here allows us disentangeling these possibilities while maintaining the idea of the underlying \isi{uniform semantics} of the prefix.

The other two usages, that of \isi{excess} and \isi{comparison}, are related to the scale constructed according to the second procedure in the list above. These usages are also guided by the same idea of proceeding through some values on the scale. In these cases, only the marked point is important and it is the only point through which the event has to proceed. The event starts when the value on the scale is below the marked point, proceeds through this point and ends when the value on the scale is above it. This accounts for examples such as \ref{ex:pere:excess}, \ref{ex:pere:comparison}, and \ref{ex:Kagan:perezhit}.

The case of the \isi{repetitive} meaning of the prefix (`again') is not unified naturally with the other cases. First, it is the only case where a separate \isi{preparatory phase} has to be created. Second, it is widely available, often simultaneously with other interpretations, and such \Prefix{pere-}prefixed verbs seem to be disambiguated only by the \isi{context}. So despite the fact that the \isi{repetitive} meaning has received a unified account with the other interpretations of the prefix \Prefix{pere-} in some earlier works \citep{Demjjanow:97, Kagan:book}, I will set is aside.

% the points on the scale are states and the scale contains three elements: the state that is associated with a non-zero point on the \textit{measure of change} scale, the state that corresponds to the zero point on the \textit{measure of change} scale, and the state that corresponds to the \isi{maximum point} of the \textit{measure of change} scale. This leads to the interpretation that at the start point of the event the state of the world is such that it corresponds to the non-minumum value on the \textit{measure of change} scale (this follows from the inequality of $x$ and $y$). Then the \isi{preparatory phase} of the event consists of setting this value to the minimum. The main phase of the event proceeds like the regular event that is related to the \textit{measure of change} scale. 

%Th idea of locating a value that corresponds to the non-minimum point of the \textit{measure of change} scale below the point that corresponds to the minimum on that scale is supposed to reflect the fact that despite some effort has been done in order to move along this scale, the situation is not better than it is 

%In sum, what we have shown is that for those cases that require proper \isi{upper inclusion} instead of just \isi{upper inclusion} there is an \isi{external} motivation behind this requirement. This does not entail that we have to accept uniform semantic representations for those usages of the prefix \Prefix{pere-} and run the reasoning sketched above for each vary with \isi{comparison} and \isi{excess} meaning. This explanation may be considered as the reason why one the prefix became polysemous, how this \isi{polysemy} is motivated. For the synchronous semantic representation the approach offered by \citet{Kagan:book} works fine as various usages of \Prefix{pere-} receive minimally different semantics and this is just enough to encapsulate the variability. What I have added to this so far is a reason for this slight change of semantics that gave rise to \isi{comparison} and \isi{excess} usages of \Prefix{pere-}. 

The approach presented here allows us to treat most of the differences between the different uses of \Prefix{pere-} as a matter of scale selection. An important property of such an approach is that various meanings arise as a result of different properties of the \isi{scales} lexicalised by verbs or contributed by the direct objects. So this formalises the intuition that the particular meaning of \Prefix{pere-}prefixed verb can only be determined in the \isi{context} (and the direct object plays a crucial role). 

As we have seen, the prefix \Prefix{pere-} is both very demanding and very flexible: in order to be attached, it requires a closed not-two point scale on which all the intermediate points can be mapped onto sub-events, but there are various mechanisms that can be used to obtain this scale. Moreover, it does not impose any restrictions on the dimension of the scale: as \citet[151]{Kagan:book} summarises, \Prefix{pere-} can apply to ``all scale dimensions that are familiar from the literature on verbal domain''. So depending on the type of the scale available, one or several interpretations are possible for the verbs derived through the attachment of the prefix \Prefix{pere-} to any \isi{derivational base}. I will provide various examples in Chapter~\ref{Chapter7}.

\subsection{Subsequent imperfectivisation of a verb with the discussed prefix}
Secondary imperfective formation is possible with all the usages of the prefix \Prefix{pere-}: crossing, waiting, \isi{excess}, \isi{comparison}, \isi{distributive}, and \isi{repetitive} semantics.

Examples \ref{ex:pere:imp:space1} and \ref{ex:pere:imp:space2} illustrate the usage of the \isi{secondary imperfective} verbs \textit{perebegat'}$^{\IPF}$ `to run/be running across' and \textit{perepl\"{e}vyvat'}$^{\IPF}$ `to spit/be spitting over something' formed from the \Prefix{pere-}prefixed verbs \textit{perebe\v{z}at'}$^{\PF}$ `to run across' (see Section~\ref{subsection:perf:motion} for more details about why I consider the verb \textit{pe\-re\-be\-gat'}$^{\IPF}$ `to run/be running across' to not be derived from the verb \textit{begat'}$^{\IPF}$ `to run' via \isi{prefixation}) and \textit{perepljunut'}$^{\PF}$ `to spit over something'. This provides evidence for the existence of the \isi{secondary imperfective} verbs derived from \Prefix{pere-}prefixed verbs with crossing semantics.\largerpage[2]

\ex.\ag.\label{ex:pere:imp:space1}I ot ka\v{z}doj pary valenok, kto v lagere gde \v{s}\"{e}l ili perebegal, -- skrip.\\
and from each pair {felt boots} who in camp where go.\glb{pst.sg.m} or pere.run.\glb{pst.sg.m} {} creak\\
\trans `And each pair of boots when someone in the colony went or ran somewhere produced a creak.'\\\Source{Aleksandr Sol\v{z}enicyn. \textit{Odin den' Ivana Denisovi\v{c}a} (1961)}
\bg.\label{ex:pere:imp:space2}Byl skup na slova. Ele perepl\"{e}vyval \v{c}erez vyvoro\v{c}ennye guby.\\
be.\glb{pst.sg.m} stingy on words barely pere.spit.imp.\glb{pst.sg.m} over vy.turned lips\\
\trans `He was stingy with his words. Barely spat them over his everted lips.'\Source{R. B. Gul'. \textit{Azef} (1958)}

Sentences \ref{ex:pere:imp:wait}, \ref{ex:pere:imp:excess}, \ref{ex:pere:imp:compar}, and \ref{ex:pere:imp:iter} serve as evidence for the existence of secondary imperfectives formed from \Prefix{pere-}prefixed verbs with waiting (\textit{pere\v{z}dat'} `to pass time waiting for something to end' $\rightarrow$ \textit{pere\v{z}idat'}  `to pass/be passing time waiting for something to end'), \isi{excess} (\textit{peregret'} `to overheat' $\rightarrow$ \textit{peregrevat'} `to overheat/be overheating'), \isi{comparison} (\textit{perepljunut'$^{\PF}$} `to surpass' $\rightarrow$ \textit{perepl\"{e}vy\-vat'$^{\IPF}$} `to surpass/be surpassing'), \isi{distributive} (\textit{perepisat'$^{\PF}$} `to list all of' $\rightarrow$ \textit{pere\-pisyvat'$^{\IPF}$} `to be listing all of'), and \isi{repetitive} (\textit{perepisat'} `to rewrite' $\rightarrow$ \textit{perepisyvat'} `to rewrite/be rewriting') semantics, respectively.\largerpage[2]

\exg.\label{ex:pere:imp:wait}Pravda, na zimu ona ostanavlivaetsja v roste, no ne obrazuet nastoja\v{s}\v{c}ix po\v{c}ek, a li\v{s}' pere\v{z}idaet zimnee poxolodanie.\\
truth on winter she stop.\glb{pres.sg.3}.refl in growth but not form.\glb{pres.sg.3} real burgeon but only pere.wait.imp.\glb{pres.sg.3} winter cooling\\
\trans `It does, in fact, stop to grow for the winter time, but does not form real burgeons, it only waits for the cool winter period to pass.'\\\Source{Ju. N. Karpun. \textit{Priroda rajona So\v{c}i} (1997)}

\exg.\label{ex:pere:imp:excess}Inogda na rynke popadaetsja \v{z}idkij m\"{e}d, kotoryj prodavcy special'no peregrevajut, \v{c}toby ostanovit' bro\v{z}enie.\\
sometimes on market po.fall.\glb{pres.sg.3}.refl liquid honey that seller.\glb{pl.nom} intentionally pere.heat.imp.\glb{pres.pl.3} that stop.\glb{inf} fermentation\\
`Sometimes liquid honey can be found on the market; it is overheated by the sellers on purpose to stop fermentation processes.'\\\Source{Vladimir \v{S}\v{c}erbakov. \textit{``Pravil'nyj'' m\"{e}d }(2002)}

\exg.\label{ex:pere:imp:compar}Da u\v{z}, puskaj lu\v{c}\v{s}e v vese i roste nas mal'\v{c}iki-sentjabriki perepl\"{e}vyvajut.\\
yes well let better in weight and height us boys-september.ik.\glb{pl.nom} pere.spit.\glb{pres.pl.3}\\
\trans `Oh well, I'd better let those September-born boys overtake us in weight and height.'\Source{Na\v{s}i deti: Maly\v{s}i do goda (forum) (2004)}

\exg.\label{ex:pare:imp:distr}Kogda inspektor Mykomel' perepisyval vsex passa\v{z}irov, ona nazvalas' Melodiej Dz'ujn.\\
when inspector.\glb{sg.nom} Mukomel pere.write.imp.\glb{pst.sg.m} all.\glb{acc} passenger.\glb{pl.gen} she na.name.\glb{pst.sg.f}.refl Melody Dzujn\\
`When Inspector Mukomel was writing down the list of all the passengers, she named herself Melody Dzujn.'\\\Source{Vadim Rossik. \textit{T\"{e}mnyj \v{c}elovek} (2015)}

\exg.\label{ex:pere:imp:iter}Vmesto togo \v{c}toby ka\v{z}dyj raz perepisyvat' istoriju, razumnee prinjat' e\"{e} takoj, kakoj ona vyjasnjaetsja sama.\\
instead that that each time pere.write.imp.\glb{inf} history.\glb{sg.acc} rational.\glb{comp} accept her that as she vy.clear.\glb{pres.sg.3}.refl herself\\
\trans `Instead of rewriting history each time, it is more rational to accept it as it turns out to be.'\Source{\`{E}duard Limonov. \textit{U nas byla Velikaja \`{E}poxa} (1987)}

%The question that remains open is why verbs that contain the prefix \Prefix{pere-} and have a \isi{distributive} interpretation cannot be imperfectivized. This seems to contradict the predictions of \citet{Tatevosov:09}, who lists the \isi{distributive} prefix \Prefix{pere-} among \isi{selectionally limited prefixes} for which the derivation of the form \textit{basic imperfective verb} $\rightarrow$ \textit{prefixed perfective verb} $\rightarrow$ \textit{\isi{secondary imperfective} verb} is not excluded on the syntactic grounds. There seem to be no clear semantic reasons motivating the absence of the \isi{secondary imperfective} for such verbs as \textit{perele\v{c}it'} `to heal all of'. The semantics that derived imperfective verb could have is conceivable: it could be either habitual healing of all the patients or being in the process of the healing all the patients. 
%
%The only explanation line I can provide at this point is that the iteration that ``wraps'' the individual events is not compatible with the \isi{secondary imperfective} attachment because the \isi{imperfective suffix} requires another type of the input. On the computational part, this will fall out automatically from the representations, as I represent the verbs that contain the \isi{distributive} \Prefix{pere-} by means of the different, two-layered structure with the outer layer being responsible for the iteration of \isi{subevents}. Those verbs will then denote events of a type different from the type of events without explicit iteration. The question whether this is a coincidence or it is indeed the iteration that makes \isi{distributive} verbs incompatible with the \isi{imperfective suffix}, is left for \isi{future} research.

\subsection{Summary}
As has been shown by \citet{Kagan:book}, various usages of \Prefix{pere-} that seem to be unrelated at first sight can be unified under a scalar account for \isi{prefixation}. We have gone somewhat further and shown that some of the differences between the usages that are present in the account by \citet{Kagan:book} can be motivated by the properties of the input scale. The available \isi{scales} may be provided by the direct object, world knowledge, \isi{context}, or the verb itself. I have proposed a mechanism that uses \isi{scales} of various types as input and (depending on the properties of a concrete scale) provides a scale as its output, which is suitable as an input to \isi{prefixation} by \Prefix{pere-}. One of the interpretations of the prefix that arises as a result of applying the proposed system is the \isi{distributive} usage of \Prefix{pere-}, that has previously not been unified with other interpretations. The scale selection process that leads to various interpretations of the prefix ends up being motivated by the requirement that the prefix has to receive a \isi{non-binary} scale as its input. The notorious \isi{polysemy} of the prefix \Prefix{pere-} arises due to the availability of different ways to satisfy this requirement. 

On the other hand, I have decided to exclude the \isi{repetitive} interpretation of the prefix \Prefix{pere-} from being integrated in the system described above. At the moment, I do not see a natural way of unifying the \isi{repetitive} meaning of the prefix with the other interpretations, as it has several distinctive properties. First, it includes a \isi{preparatory phase} (\isi{presupposition} on the accounts of \citealt{Demjjanow:97}, \citealt{Kagan:book}, more details in Chapter~\ref{Chapter6}), that is not present in other usages. Second, it is compatible with a binary scale as an input for \isi{prefixation}. Third, the attachment of the \isi{repetitive} \Prefix{pere-} to a non-basic imperfective or biaspectual verb does not lead to a change of aspect (see Section~\ref{section:new:perfectivity} for more details). These facts allow us to treat the \isi{repetitive} prefix \Prefix{pere-} and the prefix \Prefix{pere-} that may acquire all the other meanings described here as homonyms. This hypothesis, however, requires further scrutiny.

Despite all the work towards the \isi{unification} of the usages of \Prefix{pere-}, for the computational analysis I propose to allow three different representations, which account for the various mapping types required by different \isi{scales}. Remember, this mapping is always motivated by the idea of performing the action denoted by the \isi{derivational base} at all the intermediate points of the scale. 

The basic representation should account for \isi{spatial} (`crossing'), time (`waiting'), and \isi{distributive} usages in cases of closed \isi{scales}. In these, the prefix establishes the mapping between all the points on the scale and distinct event stages. The second representation accounts for cases where there is only one marked point on the relevant scale. In this case the event proceeds from some point below the marked point through this point to the point above it. The last representation is needed for the \isi{repetitive} usage: it takes the event denoted by the source verb, creates a copy of it, and constructs a new event (from the copy) that has the old one as the \isi{preparatory phase}.

\section{\textit{do-}}\label{subsection:semantics:do}
\subsection{Semantic contribution}
Let us again start by looking up the characterisations of the verbs derived with the prefix in question (now \Prefix{do-}) in the grammar by \citet[357--358]{Shvedova:82}. Three possible interpretations of the derived verbs are listed there:
\begin{enumerate}
\item to perform the action denoted by the \isi{derivational base} until the end or until some limit (productive type): \textit{dovarit'} `to finish cooking';
\item to perform the action denoted by the \isi{derivational base} in addition to something, or in order to reach a certain norm (productive type): \textit{doplatit'} `to pay in addition';
\item to lead to an undesirable condition by performing the action denoted by the \isi{derivational base} (productive in colloquial speech): \textit{dole\v{c}it'} `to cure incorrectly, causing a serious illness'.
\end{enumerate}

As we see, \Prefix{do-} is not a highly polysemous prefix. Nevertheless, \Prefix{do-} is very interesting concerning phenomena of \isi{prefix stacking}, as it is very productive and can lead to the formation of \isi{biaspectual verbs}, as we have discussed in Section~\ref{section:new:biaspectual}. 

\citet[70]{Kagan:book} characterises the prefix \Prefix{do-} as relating ``the \isi{standard of comparison} to the degree that is achieved at the endpoint of an event''. She identifies this prefix as \isi{delimitative} and distinguishes between the \isi{terminative} and additive  usages. The \isi{terminative} usage corresponds to the first and the additive usage corresponds to the second usage in the list by \citet{Shvedova:82} provided above. My primary goal is to study the \isi{terminative} usage. \citet[72]{Kagan:book} describes the semantics of the \isi{terminative} usage of the prefix \Prefix{do-} in the following way: ``The prefix introduces the relation of identity between two degrees. It applies to a gradable property an increase along which is entailed by the predicate.''

A simple illustration is provided by \ref{ex:do:varit}. The verb \textit{varit'}$^{\IPF}$ `to cook' lexicalises a scale with the \isi{maximum point} corresponding to \textit{fully cooked} and the prefix \Prefix{do-} contributes information that at the end of the event this point is reached.\largerpage[-2]

\exg.\label{ex:do:varit}Liza dovarila sup.\\
Liza do.cook.\glb{pst.sg.f} soup\\
\trans `Liza finished cooking the soup.'

What is important is that \ref{ex:do:varit} normally refers to an event of cooking the soup that does not start not from scratch. It may be the case that the soup was almost ready but Liza had to pause cooking and answer a phone call before finishing cooking. It can also be the case that John was cooking the soup, considered it cooked, and left it for Liza. Liza came later, tasted the soup and realised it is not ready, and then had to do some additional cooking to make the soup ediable. The second interpretation corresponds to the additive usage of the prefix. However, it does not represent a special case different from the first usage in terms of scalar semantics: in both cases, the event that the \Prefix{do-}prefixed verb refers to proceeds along the relevant scale from some point $x$ until the scale's maximum. The difference between the prefix \Prefix{do-} and other prefixes is that $x$ does not have to be the minimum point on the relevant scale. It can also be the case that there is no minimum point on the relevant scale at all. For example, the event of heating the soup proceeds along the temperature scale and the start of the event is associated with some temperature of the soup that cannot be easily reconstructed, but is definitely not equal to the \isi{minimum of the scale}. From the fact that a sentence such as \ref{ex:do:PP} normally refers to the whole event of heating the soup up to the boiling point it follows that the condition I have formulated above seems to work well. A stronger requirement (for the presence of another event associated with the temperature increase) would be superfluous.

\exg.\label{ex:do:PP}Liza dovela sup do kipenija.\\
Liza do.lead.\glb{pst.sg.f} soup until boiling\\
\trans `Liza made the soup boil.'

\citet[75]{Kagan:book} claims that the semantics of the \isi{terminative} \Prefix{do-} ``can be divided into an entailed and a presupposed part''. The observation provided above seems to speak against such an additional \isi{inference} associated with the prefix \Prefix{do-}. The sentence \ref{ex:do:PP:ne} can be successfully uttered in a situation when Liza did not heat the soup at all. We will discuss this topic further in Chapter~\ref{Chapter6}.\largerpage[-2]

\exg.\label{ex:do:PP:ne}Liza ne dovela sup do kipenija.\\
Liza not do.lead.\glb{pst.sg.f} soup until boiling\\
\trans `Liza did not make the soup boil.'

Although additive \Prefix{do-} is not in the focus of this book, I would like to add some remarks about it, as these remarks contribute to the overall picture of \isi{pragmatic competition} between different prefixes. \citet[79]{Kagan:book} points out that the main difference between the \isi{terminative} and the additive interpretations is that in the first case the presupposed and the entailed events are viewed as constituting one event and in the second case they are viewed as two separate events. What usually comes along with this distinction is that in the first case the degree on the \isi{measure of change scale} that has to be reached in the end is specified. In the second case the measure of change of the second event is linguistically supplied, whereas the \isi{cumulative} standard that has to be reached in the end can be left implicit. \citet[79]{Kagan:book} provides the examples repeated in \ref{ex:Kagan:dospat} to illustrate the differences between these usages.

\ex.\label{ex:Kagan:dospat}\ag.\label{ex:Kagan:dospat:1}(Ivan l\"{e}g pospat'.) On dospal do poluno\v{c}i.\\
Ivan lay po-sleep he do-slept till midnight\\
\trans `Ivan went to bed. He slept till midnight.'
\bg.\label{ex:Kagan:dospat:2}(Ivan za no\v{c} ne vyspalsja.) Potom on dospal paru \v{c}asov.\\
Ivan in night NEG vy-slept-refl then he do-slept couple hours\\
\trans `Ivan hadn't had enough sleep during the night. He then slept for a couple more hours.'\Source{= (12) in \citet[79]{Kagan:book}}

In the first case (example \ref{ex:Kagan:dospat:1}, \isi{terminative} usage) there is a single event of sleeping that lasts until midnight.\footnote{Note that as the first (bracketed) sentence refers only to the initiation of the sleeping situation and does not even require the agent to fall asleep. This is clear from that fact that it is possible to cancel the \isi{inference} that Ivan slept, as in \ref{ex:dospat:no}.
\exg.\label{ex:dospat:no}Ivan l\"{e}g pospat'. On prole\v{z}al 3 \v{c}asa, no tak i ne smog usnut'.\\
Ivan lay po.sleep.\glb{inf} he pro.lay.\glb{pst.sg.m} 3 hours, but so and not able.\glb{pst.sg.m} fall.asleep.\glb{inf}\\
\trans `Ivan went to bed. He stayed in bed for 3 hours but did not manage to fall asleep.'

} In the second case, there was one sleeping event that proved to be insufficient so there was a second event in the course of which Ivan slept for several hours and thus cumulatively over two events reached the required amount of sleep. 

As \citet[80]{Kagan:book} points out, in case of the additive usage of the prefix \Prefix{do-} the first event can be of a different kind, as illustrated by example \ref{ex:doplatit} that describes a situation when additional payment has to be made not after another payment, but after giving away empty bottles.

\exg.\label{ex:doplatit}Kupili dju\v{z}inu butylok fruktovoj vody, a v obmen sdali 8 pustyx butylok. Skol'ko deneg doplatili?\\
bought dozen bottles fruit water, but in exchange s.give.\glb{pst.pl} 8 empty bottles how.much money do.pay.\glb{pst.pl}\\
\trans `We bought a dozen bottles of fruit water and handed back 8 empty bottles. How much money did we have to pay in addition?'\Source{\url{vcevce.ru}}

%Such examples allow to come to the conclusion that the interpretation of the prefix depends exclusively on whether the linguistically supplied measure of change is absolute (so if fixes the end point and the verb tends to be interpreted terminatively) or relative (and fixes the difference between the start and the end points of the main event, which leads to the additive interpretation). As for the division into two events, in the first case we know that the border is somewhere on the scale between the zero point (and may be at this point) and the supplied \isi{maximum point} and in the second case we only know that the state of the world before the event start is such that when the value is augmented by the specified change, some desired standard is reached. 

Another example is provided in \ref{ex:buy:raisins}. Sentence \ref{ex:buy:raisins} does not exclude that the speaker never bought raisins, dried apricots, and/or plums before or that he had ever possessed any. It only implys that the needed them in order to make stewed fruit. What the verb \textit{dokupit'}$^{\PF}$ `to buy in addition' means in this case is that he bought the dried fruits but this was not the first step in gathering the ingredients for something he wanted to cook. The ``scale'' in this case includes possession of the necessary amount of raisins, dried apples, apricots, and plums. 

\exg.\label{ex:buy:raisins}Mne test' vydal su\v{s}\"{e}nyx jablok s da\v{c}i, ja dokupil izjuma, kuragi, \v{c}ernosliva i teper' reguljarno vspominaju detstvo -- varju kompot iz suxofruktov.\\
me father-in-law vy.give.\glb{pst.sg.m} dried apples from dacha I do.buy.\glb{pst.sg.m} raisins {dried apricots} {dried plums} and now regularly remember childhood {} cook {stewed fruit} from {dried fruits}\\
\trans `My father-in-law gave me some dried apples from his dacha, I also\linebreak bought raisins, dried apricots and plums and now I regularly invoke childhood memories by making myself some stewed dried fruit.'\\\Source{\url{https://murmolka.com}}

Based on these observations, I propose that the \isi{inference} of the event being an addition to something else is drawn in the process of the \isi{pragmatic competition} between the \Prefix{do-}prefixed verb and other \isi{perfective verbs} that can express the same literal meaning (in case of the example \ref{ex:buy:raisins} it would be the verb \textit{kupit'$^{\PF}$} `to buy'). The competition is triggered by the absence of the requirement that the starting point of the event has to be the minimum on the relevant scale in the semantic representation of the prefix \Prefix{do-} (unless it is overtly specified, as in \ref{ex:do:iz:do}, or the scale is of a measure of change type, as in \ref{ex:do:measure}). A broader pragmatic picture will be provided in the next chapter.

\exg.\label{ex:do:iz:do}Za \v{s}est' \v{c}asov mo\v{z}no doletet' iz N'ju-Jorka do San-Francisko.\\
behind six hours can do.fly.\glb{inf} from {New York} to {San Francisco}.\\
\trans `In six hours one can get from New York to San Francisco by plane.'\\\Source{Boris Levin. \textit{Inorodnoe telo} (1965--1994)}

\exg.\label{ex:do:measure}A na poljax nota bene -- takoj-to ne doplatil tri kopejki, vozmestit togda-to…\\
but on margins nota bene {} such-\glb{particle} not do.pay.\glb{pst.sg.m} three copecks, compensate.\glb{pres.sg.3} then-\glb{particle}\\
\trans `And on the margins there is a note: he failed to pay 3 copecks, which he will compensate for on day Y.'\\\Source{Jurij Davydov. \textit{Sinie tjul'pany} (1988--1989)}

\subsection{Restrictions on attachment}
\citet[236]{Kagan:12} points out that the prefix \Prefix{do-} in its \isi{terminative} interpretation can apply to a variety of \isi{scales}. Let me first illustrate this thesis with a poem by Ekaterina Starostina called \textit{Do\v{c}uvstvovat'} `To finish feeling'. This poem contains 13 \Prefix{do-}prefixed verbs in 12 lines (they are marked with bold font), and in 4 verbs \Prefix{do-} is not the only prefix.\pagebreak

\ex.\label{ex:poem}\a.\label{poem:a}\ag.[]$\ldots$\textbf{Do\v{c}uvstvovat'}. \textbf{Doo\v{s}\v{c}u\v{s}\v{c}at'}.\\
do.feel.\glb{inf} do.sense.\glb{inf}\\
\bg.[]\textbf{Dotronut'sja} ili kosnut'sja$\ldots$\\
do.touch.\glb{inf}.refl or touch.\glb{inf}.refl\\
\bg.[]\textbf{Dobyt'} tebja, \textbf{docelovat}'$\ldots$\\
do.be.\glb{inf} you do.kiss.\glb{inf}\\
\bg.[]$\ldots$i polnym serdcem ulybnut'sja$\ldots$\\
and full heart smile.\glb{inf}.refl\\
\trans To finish feeling. To finish sensing.\\
To touch you slightly$\ldots$\\
To get you and finish kissing\\
$\ldots$and smile with a full heart$\ldots$
\z.
\b.\label{poem:b}\ag.[]\textbf{Dogladit'} pal'cy na rukax$\ldots$\\
do.caress.\glb{inf} fingers on hands\\
\bg.[]\textbf{Domno\v{z}it'} s\v{c}ast'e v na\v{s}ix du\v{s}ax.\\
do.multiply.\glb{inf} happiness in our souls\\
\bg.[]\textbf{Dopere\v{z}it'}, \textbf{dopere\v{z}dat'}$\ldots$\\
do.pere.live.\glb{inf}, do.pere.wait.\glb{inf}\\
\bg.[]\textbf{Dorazobrat'} vs\"{e} to, \v{c}to nu\v{z}no$\ldots$ \\
do.raz.take.\glb{inf} all that that needed\\
\trans To finish caressing the fingers$\ldots$\\
To multiply the joy in our souls.\\
To live,  to wait till the end of our lives$\ldots$\\
To disassemble all we need$\ldots$
\z.
\b.\label{poem:c}\ag.[]\textbf{Dorazukra\v{s}ivat'} me\v{c}ty,\\
do.raz.u.paint.imp.\glb{inf} dreams\\
\bg.[]\textbf{Dobit'sja} srazu: vs\"{e} i mnogo$\ldots$\\
do.hit.\glb{inf}.refl {at once} all and {a lot}\\
\bg.[]I dobrym utrom do poroga\\
and kind morning until doorstep\\
\bg.[]\v{C}ut' zabludiv\v{s}ejsja \textbf{dojti}$\ldots$\\
slightly za.wander.\glb{part.act.pst}.refl do.go.\glb{inf}\\
\trans To finish colouring my dreams,\\
To get at once all that I wanted$\ldots$\\
And one fine morning, having slightly strayed\\
To reach the doorstep$\ldots$\\
\Source{Ekaterina Starostina, \textit{Do\v{c}uvstvovat'} (www.stihi.ru)}

In this poem the prefix \Prefix{do-} is attached to a scale of stages through which the event develops (e.g., \textit{do\v{c}uvstvovat'} `to finish feeling', \ref{poem:a}), to a \isi{path scale} (e.g., \textit{dojti} `to get to', \ref{poem:c}), as well as to the \isi{time scale} that either derives directly from the semantic structure of the verb (e.g., \textit{doo\v{s}\v{c}u\v{s}\v{c}at'} `to finish sensing', \ref{poem:a}) or is already used in course of the attachment of another prefix (e.g., \textit{dopere\v{z}it'} `to survive something', \ref{poem:b}). \citet{Kagan:book} proposes the following hierarchy of sources for a scale the prefix \Prefix{do-} can attach to: 

\begin{itemize}
\item ``If the verbal stem lexicalizes a scale, it is to this scale that \Prefix{do-} will apply.''
\item ``If the verb itself does not contribute a scale, but it is an incremental
theme verb, then the prefix will apply to the scale introduced by the direct object (a volume/extent scale).''
\item ``If none of these conditions is satisfied, the prefix can apply to the \isi{time scale}.''
\end{itemize}

\citet{Kagan:12} also notes that \Prefix{do-} can apply to both upper closed and open \isi{scales}, but ``[i]f \Prefix{do-} applies to a scale that is not upper closed, and a \Prefix{do-}PP is absent, the \isi{context} has to be sufficiently rich to determine what counts as the \isi{standard of comparison}.'' I would like to provide one more illustration of this point for cases when \Prefix{do-} applies to the \isi{time scale}. As follows from the observations made by \citet{Kagan:12}, the \isi{maximum point} that is reached has to be specified (at least by the \isi{context}) because the \isi{time scale} is an \isi{open scale}. For example, \ref{ex:dosidel} cannot be uttered if it is not clear from the \isi{context} until what time the actor was supposed to sit. The situation is different with \ref{ex:posidel} and \ref{ex:peresidel}. These can be used without any supportive \isi{context}. This illustrates that the requirements of these prefixes vary (\Prefix{po-} can create limits on an \isi{open scale} and \Prefix{pere-} is supported by the scale construction mechanism that is able to extract non-linguistic information about the appropriate time for the actor to spend sitting).

\ex.\ag.\label{ex:dosidel}Ja dosidel.\\
I do.sit.\glb{pst.sg.m}\\
\trans `I sat till the end.'
\bg.\label{ex:posidel}Ja posidel.\\
I po.sit.\glb{pst.sg.m}\\
\trans `I sat for a while.'
\bg.\label{ex:peresidel}Ja peresidel.\\
I pere.sit.\glb{pst.sg.m}\\
\trans `I sat for too long.'

It is also important that in case the time point until which the sitting lasted is explicit, the difference between the literal semantics of the verb \textit{dosidet'} `to sit until a certain time' and \textit{posidet'} `to sit for a while' is lost, as illustrated by \ref{ex:dosidet:do} and \ref{ex:posidet:do}. In this situation the difference between the \Prefix{po-} and the \Prefix{do-}prefixed verbs results from their \isi{pragmatic competition}. We obtain the enriched meaning of the \Prefix{do-}prefixed verb that the sitting event lasted relatively long and the enriched meaning of the \Prefix{po-}prefixed verb that the sitting event was rather short.\largerpage

\ex.\ag.\label{ex:dosidet:do}Ja dosidel do pjati utra, i, tak i ne do\v{z}dav\v{s}is' tebja, usnul.\\
I do.sit.\glb{pst.sg.m} until 5 morning and that and not do.wait.\glb{part.pst}.refl you, fall.asleep.\glb{pst.sg.m}\\
\trans `I sat there waiting for you until 5 a.m. and fell asleep (before you arrive).'\Source{\url{https://ficbook.net}}
\bg.\label{ex:posidet:do}Priexal na u\v{c}ebu k 7, posidel do 8:15 -- otpustili domoj.\\
pri.ride.\glb{pst.sg.m} on study to 7, po.sit.\glb{pst.sg.m} until 8:15 -- ot.let.\glb{pst.pl} home\\
\trans `I arrived for the class at 7, sat there until 8:15 and then I was free to go home.'\Source{\url{https://twitter.com}}

From the bleached difference between the literal semantics of \Prefix{po-} and \Prefix{do-}prefixed verbs when these prefixes apply to the \isi{time scale} follows that they cannot be stacked. When the prefix \Prefix{po-} with its `for a while' meaning is attached to a verb, e.g. \textit{sidet'}$^{\IPF}$ `to sit', the event denoted by this verb is conceptualised as being homogeneous and having some limited \isi{duration}. This verb cannot be further prefixed with \Prefix{do-}: the verb \textit{*doposidet'} does not exist. The potential semantics of this verb after the attachment of two prefixes would be `to complete sitting for a while', which is equivalent to either to `to sit for a while' or `to finish sitting', that can both be expressed with \isi{morphologically simpler} verbs. In case only the \isi{time scale} is available in the verbal semantic structure, the reverse stacking (\Prefix{po-} on top of \Prefix{do-}) is not available for the same reason: the verb \textit{*podosidet'} could mean `to sit for a while finishing sitting', but there is no event falling under this denotation that could not be described by either `to sit for a while' or `to finish sitting'. Note that when \Prefix{do-} selects some other scale than the \isi{time scale}, the prefix \Prefix{po-} can be stacked on top of it after the verb is imperfectivised. This is illustrated by chain \ref{chain:podo}{\interfootnotelinepenalty=10000\footnote{Only additive interpretations are provided for the verbs in the chain, but \isi{terminative} interpretations are also possible. In this case the last derived verb means either `to write the final part for a while' or `to finish writing all of.'}} and example \ref{ex:podopisyval}.

\exg.\label{chain:podo}pisat'$^{\IPF}$ $\rightarrow$ dopisat'$^{\PF}$ $\rightarrow$ dopisyvat'$^{\IPF}$ $\rightarrow$ podopisyvat'$^{\PF}$\\
{to write} $\rightarrow$ {to write in addition} $\rightarrow$ {to (be) writing in addition} $\rightarrow$ {to write in addition in all of/for a while}\\

\exg.\label{ex:podopisyval}Podopisyval noli v isxodnye dannye.\\
po.do.write.imp.\glb{pst.sg.m} zeros in initial data\\
\trans `I added zeros to the initial data.'\Source{\url{www.planetaexcel.ru}}

\citet{Tatevosov:09} lists \Prefix{do-} as a positionally limited prefix which means that it can be attached only below the \isi{secondary imperfective} suffix. As we have already discussed in Section~\ref{section:new:biaspectual}, this is not a valid observation. For example, the verb \textit{dovy\v{s}ivat'} `to finish embroidering' is either perfective or biaspectual, depending on whether the individual speaker considers whether or not the verb \textit{dovy\v{s}it'} `to finish embroidering' exists. What is important is that no speaker I have consulted responded that this verb can have only the imperfective interpretation, as suggested by the theory proposed in \citealt{Tatevosov:09}. In the poem \ref{ex:poem} the verb \textit{dorazukra\v{s}ivat'} `to finish colouring' is also perfective as it is constructed according to the derivation presented in \ref{chain:dorazu1}. The verb containing the same morphemes can also be imperfective if the order of attachment is different, as represented in \ref{chain:dorazu2}.

\ex.\ag.\label{chain:dorazu1}krasit'$^{\IPF}$ $\rightarrow$ ukrasit'$^{\PF}$ $\rightarrow$ razukrasit'$^{\PF}$ $\rightarrow$ razukra\v{s}ivat'$^{\IPF}$ $\rightarrow$ dorazukra\v{s}ivat'$^{\PF}$\\
{to paint} $\rightarrow$ {to decorate} $\rightarrow$ {to colour} $\rightarrow$ {to colour/be colouring} $\rightarrow$ {to finish colouring}\\
\bg.\label{chain:dorazu2}krasit'$^{\IPF}$ $\rightarrow$ ukrasit'$^{\PF}$ $\rightarrow$ razukrasit'$^{\PF}$ $\rightarrow$ dorazukrasit'$^{\PF}$ $\rightarrow$ dorazukra\v{s}ivat'$^{\IPF}$\\
{to paint} $\rightarrow$ {to decorate} $\rightarrow$ {to colour} $\rightarrow$ {to finish colouring} $\rightarrow$ {to finish/be finishing colouring}\\

A couple of other \isi{biaspectual verbs} are the verbs \textit{doobdumyvat'} `to finish thinking about' (see examples in \ref{ex:doobdumyvat}) and \textit{dozabivat'} `to finish hammering' (see examples in \ref{ex:dozabivat}).

\ex.\label{ex:doobdumyvat}\ag.V processe \v{c}tenija v golove na\v{c}ali oformljat'sja vsjakie xitrye i kovarnye idei, no ix e\v{s}\v{c}\"{e} nu\v{z}no akkuratno doobdumyvat'$^{\PF}$.\\
in process reading in head start.\glb{pst.pl} form.\glb{inf}.refl various tricky and crafty ideas but they also needed carefully do.ob.think.\glb{imp}.\glb{inf}\\
\trans `While I was reading it some tricky and crafty ideas came into my head, but I need to think them over accurately.'\\\Source{\url{http://nicka-startcev.livejournal.com}}
\bg.Zasim ja idu morozit' nos i doobdumyvat'$^{\IPF}$ v\v{c}era\v{s}njuju ideju, poka ona ne ube\v{z}ala ot menja okon\v{c}atel'no.\\
hereupon I go.\glb{pres.sg.1} freeze.\glb{inf} nose and do.ob.think.imp.\glb{inf} yesterday's idea while she not u.run.\glb{pst.sg.f} from me completely\\
\trans `Hereupon I go to freeze my nose and think more about yesterday's idea until it has fled from me completely.'\Source{\url{8794.diary.ru}}


%Tatevosov: *\isi{do-}(za-b-iva)-t', but:
%naverno ja tebe prishlju ejo v takom nepolnom vide - a ty posmotri i poprobuj kalendar' dozabivat' do konca
%http://rusport.eu/threads/890/page-39
\ex.\label{ex:dozabivat}\ag.Tam e\v{s}\v{c}\"{e}, \v{c}ut' popoz\v{z}e, krjuk e\v{s}\v{c}\"{e} i dozabivat'$^{\PF}$ v sneg umudrjajutsja, i, pre\v{z}de \v{c}em verjovku rezat', celuju re\v{c}' proiznosjat.\\
there also {a bit} later hook also and do.za.hit.imp.\glb{inf} in snow manage.\glb{inf}.refl and before what.\glb{instr} rope cut.\glb{inf} whole speech pronounce.\glb{pres.pl.3}\\
\trans `In the same video, a bit later, they also manage to hammer the hook in the snow completely and then they deliver a whole speech before cutting the rope.'\Source{\url{http://yarin-mikhail.livejournal.com}}
\bg.Gvozdi inogda dozabivat'$^{\IPF}$ prixoditsja.\\
nails sometimes do.za.hit.imp.\glb{inf} pri.go.\glb{pres.sg.3}.refl\\
\trans `The nails sometimes have to be additionally hammered.'\\\Source{\url{https://forumhouse.ru}}

It seems that the prefix \Prefix{do-} is very undemanding with respect to the verb it attaches to. Sometimes the resulting verb seems odd, as \textit{donapisat'} `to finish writing', but such difficulties are of the same kind as with attaching the \isi{repetitive} prefix \Prefix{pere-} to some \isi{perfective verbs} (see Section~\ref{subsection:semantics:pere}) and we do find these verbs in some contexts. Such contexts require exactly the semantics obtained by the semantic composition of the prefix \Prefix{do-} with the prefixed verb (e.g., \textit{napisat} `to write/create something written') and not with the unprefixed verb (e.g., \textit{pisat'} `to write'). An example is provided in \ref{ex:donapisat} to be contrasted with \ref{ex:donapisat:mod} in which the verb is replaced. As we see, the speaker wants to express the additive semantics, and as the most natural interpretation of the verb \textit{dopisat'} is `to finish writing', he prefers to use the verb \textit{donapisat'} `to write something in addition'. This leads to the question of how the meaning of the prefix is related to the properties of the \isi{derivational base}.

\ex.\ag.\label{ex:donapisat}Tam ja donapisal pis'ma i novoe stixotvorenie, a tak\v{z}e porabotal s fotografijami.\\
there I do.na.write.\glb{pst.sg.m} letter.\glb{pl.acc} and new poem but also po.work.\glb{pst.sg.m} with photos\\
\trans `There I also wrote letters and a new poem, and also worked a bit with the photos.'\Source{\url{dd.vl.ru}}
\bg.\label{ex:donapisat:mod}Tam ja dopisal pis'ma i novoe stixotvorenie, a tak\v{z}e porabotal s fotografijami.\\
there I do.write.\glb{pst.sg.m} letter.\glb{pl.acc} and new poem but also po.work.\glb{pst.sg.m} with photos\\
\trans `There I finished writing the letters and the new poem, and also worked a bit with the photos.'

Note that the aspect of the \isi{derivational base} matters. In general, if the \isi{derivational base} is perfective, the interpretation of the derived \Prefix{do-}prefixed verb tends to be additive (compare \ref{ex:do:kupit} and \ref{ex:do:pokupat}), and if the \isi{derivational base} is a \isi{secondary imperfective} verb, the additive interpretation seems to be not available (see example \ref{ex:do:zapisyvat}). In case a \Prefix{do-}prefixed verb gets imperfectivised, both additive and \isi{terminative} interpretations become available for the derived imperfective verb (see examples under \ref{ex:zapravit}).

\ex.\ag.\label{ex:do:kupit}Katja dokupila mandarin.\\
Katja do.buy$^{\PF}$.\glb{pst.sg.f} tangerine.\glb{pl.gen}\\
\trans `Katja also bought some tangerines.'/`Katja bought some additional tangerines.'
\bg.\label{ex:do:pokupat}Katja dopokupala mandariny.\\
Katja do.buy$^{\IPF}$.\glb{pst.sg.f} tangerine.\glb{pl.acc}\\
\trans `Katja finished buying tangerines.'

\ex.\ag.\label{ex:do:zapisyvat}Petja dozapisyval$^{\PF}$ dva diska.\\
Petja do.za.write.imp.\glb{pst.sg.m} two CDs\\
\trans `Petja finished recording two CDs.'
\bg.\label{ex:do:zapisat}Petja dozapisal$^{\PF}$ dva diska.\\
Petja do.za.write.\glb{pst.sg.m} two CDs\\
\trans `Petja additionally recorded two CDs'/`Petja finished recording two CDs.'

\ex.\label{ex:zapravit}\ag.\label{ex:zapravit3}Mexanik dozapravil$^{\PF}$ samol\"et (i zakuril sigaretu).\\
mechanic do.fill.\glb{pst.sg.m} plane.\glb{sg.acc} (and za.smoke.\glb{pst.sg.m} cigarette)\\
\trans `The mechanic additionally fueled the plane and lit a cigarette.'
\bg.\label{ex:zapravit2}Mexanik dozapravljal$^{\PF}$ samol\"et (i zakuril sigaretu).\\
mechanic do.fill.imp.\glb{pst.sg.m} plane.\glb{sg.acc} (and za.smoke.\glb{pst.sg.m} cigarette)\\
\trans `The mechanic finished fueling the plane and lit a cigarette.'
\bg.\label{ex:zapravit1}Mexanik dozapravljal$^{\IPF}$ samol\"et (i kuril sigaretu).\\
mechanic do.fill.imp.\glb{pst.sg.m} plane.\glb{sg.acc} (and smoke.\glb{pst.sg.m} cigarette)\\
\trans `The mechanic was finishing fueling/additionally fueling the plane and smoking.'

The verbs used in \ref{ex:zapravit} result from the following derivations. The perfective verb \textit{zapravit'} `to fuel' can be either directly prefixed with \Prefix{do-} (as in the chain \ref{chain:dozapravljat1}) or imperfectivised before (as in the chain \ref{chain:dozapravljat2}). In the first case the derived verb is \textit{dozapravit'}$^{\PF}$ `to fuel additionally' (used in example \ref{ex:zapravit3}) that can then be imperfectivised in order to obtain the verb \textit{dozapravljat'}$^{\IPF}$ that can either mean `to finish/be finishing fueling' or `to fuel/be fueling additionally', as illustrated by example \ref{ex:zapravit1}. If the morphemes are attached in the different order, as illustrated by chain \ref{chain:dozapravljat2}, the derived verb \textit{dozapravljat'}$^{\PF}$ `to finish/be finishing fueling' is perfective and acquires \isi{terminative} semantics (see example \ref{ex:zapravit2}).

\ex.\label{chain:dozapravljat}\ag.\label{chain:dozapravljat1}zapravit'$^{\PF}$ $\rightarrow$ dozapravit'$^{\PF}$ $\rightarrow$ dozapravljat'$^{\IPF}$\\
{to fuel} $\rightarrow$ {to fuel additionally} $\rightarrow$ {to (be) finish(ing) fueling/to (be) fuel(ing) additionally}\\
\bg.\label{chain:dozapravljat2}zapravit'$^{\PF}$ $\rightarrow$ zapravljat'$^{\IPF}$ $\rightarrow$ dozapravljat'$^{\PF}$\\
{to fuel} $\rightarrow$ {to fuel/be fueling} $\rightarrow$ {to finish/be finishing fueling}\\

Chain \ref{chain:dozapravljat1} illustrates that the additive meaning component associated with a \textit{do}-prefixed verb is not inherited and can be replaced by another \isi{inference} after the \isi{imperfectivisation} step. This speaks in favour of the hypothesis that this kind of additional \isi{inference} is not specified in the semantic structure of the verb but arises as a result of the interpretation of the semantic representation followed by a \isi{pragmatic competition}. For this reason, I will abandon the distinction between the additive and the \isi{terminative} usages of \Prefix{do-}. In sum, I claim that it is not only possible to unify the additive and the \isi{terminative} usages of the prefix \Prefix{do-}, but that there are no distinct representations for these usages. Instead, there are different ways to interpret the semantic representation of the derived verb that result in different inferences. 

\ex.\ag.\label{ex:do:pere1}Nu, doperepisal, tak-to proizvedenie bylo napisano v 97--98 gg...\\
well do.pere.write.\glb{pst.sg.m} that composition was written in 97--98 years\\
\trans `Well, I finished rewriting it, as the work was actually written in 1997--1998.'\Source{\url{na-ive.diary.ru}}
\bg.\label{ex:do:pere2}Doperepisyval na\v{c}isto, s nekotorymi ispravlenijami, preljudiju do ma\v{z}or.\\
do.pere.write.imp.\glb{pst.sg.m} clean with some corrections prelude C major\\
\trans `Finished rewriting the final version of the C major prelude (with some corrections).'\Source{\url{1001.ru}}

Another observation concerns stacking the prefix \Prefix{do-} on top of the prefix  \Prefix{pere-}: when \Prefix{pere-}prefixed verbs are further prefixed with \Prefix{do-}, they acquire a \isi{terminative} interpretation independently of derviational base's aspect (see examples \ref{ex:do:pere1} and \ref{ex:do:pere2}). Putting it simply, the events referred to by the \Prefix{pere-}prefixed verbs are conceptualised as proceeding through contiguous stages. The additive interpretation of the prefix \Prefix{do-} requires (according to the proposal of \citealt{Kagan:book}) that there is a break between the event associated with the initial part of the scale and the event associated with the final part of the scale. Such a gap is incompatible with the semantics of the \isi{derivational base} if it contains the prefix \Prefix{pere-}. 

In sum, I propose to represent the contribution of the prefix \Prefix{do-} as fixing the final stage of the event and specifying the event denoted by the derived verb as being part of an event denoted by the \isi{derivational base}.

%I propose that whenever it is not possible to interpret the derived \Prefix{do-}prefixed verb additively, the event can be decomposed in two stages. One of these stages will be the \isi{preparatory phase} (corresponding to the \isi{presupposition} in the account of \citet{Kagan:book}) and the other will be the current event that the speaker is focusing on. We will see why all this in detail once the formal representations are constructed.

%If we recall the section addressing the prefix \Prefix{pere-} (section \ref{subsection:semantics:pere}) and also jump ahead and think of the \isi{secondary imperfective} (when interpreted progressively) as adding an intermediate point while preserving the information about the boundaries (see section \ref{section:imperfective}), than we can see that those situations are unified by the presence of three distinguished points in the semantic structure of the event. As we have discussed, the \isi{terminative} usage of the prefix \Prefix{do-} is also based on distinguishing three points in the semantic structure of the event (boundaries plus an intermediate point that divides the \isi{preparatory phase} from the main phase). Those facts seem to reveal why it is the \isi{terminative} and not the additive semantics that arises in the discussed cases. What also should fall out from if those observations are on the right track is a non-zero \isi{preparatory phase} in case the prefix \Prefix{do-} is attached to a \isi{secondary imperfective} verb (details will become clear once we discuss formal representations and their combinatorics in chapter~\ref{chapter:formal}}. 
%
%Judging from the introspection, this is indeed so, but as the same \isi{inference} can arise due to the \isi{pragmatic competition} between such verbs and \isi{perfective verbs} (derivational bases for those imperfectives that were then prefixed with \Prefix{do-}), reliable test contexts have to exclude the possibility of pragmatic reasoning. I leave more detailed experimental investigation of this point for \isi{future} research. The only note I want to add is that this is something that no other approach is predicting, so it can turn out to be a strong evidence in its favour or a problem.


\subsection{Subsequent imperfectivisation of a verb with the \textit{do-} prefix}
The existence of a prefix that has a transparent semantic contribution and does not block subsequent \isi{imperfectivisation} at all is not predicted by the theory of distinct structural positions for the lexical and \isi{superlexical} groups of prefixes. However, the possibility of attaching the \isi{imperfective suffix} to \Prefix{do-}prefixed verbs cannot be denied and this prefix has been incorporated in the lexical/\isi{superlexical} framework, acquiring a different status (e.g., falling in the category of \textit{intermediate} prefixes in the theory of \citealt{Tatevosov:07}). Imperfectivisation of verbs prefixed with \Prefix{do-} seems to be possible in all cases when the verbal stem allows the addition of the \isi{imperfective suffix}. Some examples of \isi{secondary imperfective} verbs with the prefix \Prefix{do-} have been provided above: these are the sentences \ref{ex:zapravit1} and \ref{ex:do:pere2}.

The cases when \isi{imperfectivisation} is not possible are those cases when the verbal stem is not compatible with the \isi{imperfective suffix} at all, as in the case of the verb \textit{\v{z}eltet'} `to turn yellow/to be seen as yellow' that we have already discussed in connection with the prefix \Prefix{za-}. This verb in its `to turn yellow' interpretation can be prefixed with \Prefix{do-}. The result is the verb \textit{do\v{z}eltet'} `to finish turning yellow' (see example \ref{ex:dozheltet}). This verb cannot be further imperfectivised. 

\exg.\label{ex:dozheltet}Te list'ja do\v{z}elteli i opali.\\
that leaves do.turn.yellow.\glb{pst.pl} and o.fall.\glb{pst.pl}\\
\trans `Those leaves finished turning yellow and fell off.'\\\Source{\url{www.bonsaiforum.ru}}

\subsection{Summary}
Summing up the above discussion, we need to bear in mind the following observations when the formal representation of the prefix \Prefix{do-} is constructed.
\begin{enumerate}
\item If the \isi{derivational base} lexicalises a scale, \Prefix{do-} selects this scale. If not, the second choice is the scale contributed by the direct object (which can be a \isi{measure of change scale}). If both options are unavailable, \Prefix{do-} can quantify over the \isi{time scale}.
\item The scale selected by \Prefix{do-} has to be upper closed.
\item The end point of the event denoted with the \Prefix{do-}prefixed verb has to correspond to the \isi{maximum point} on the scale.
\item If \Prefix{do-} attaches to a perfective verb and the start of the event denoted by this verb is related to the minimum on the scale, the event can be decomposed into a preparatory and a focused phase.
\end{enumerate}

%\subsection{pod-}\label{subsection:semantics:pod}
%\subsection{Semantic contribution}
%\citet[pp.365--366]{Shvedova:82}
%
%\begin{enumerate}
%\item to direct the action denoted by the \isi{derivational base} down, under something (productive type): \textit{podstavit'} `to put something under something else';
%\item to direct the action denoted by the \isi{derivational base} upward (productive type): \textit{podbrosit'} `to throw something in the air';
%\item to approach or attach to something by performing the action denoted by the \isi{derivational base} (productive type): \textit{podojti'} `to come closer by walking';
%\item to perform the action denoted by the \isi{derivational base} with low intensity (productive type, especially in colloquial speech): \textit{podbodrit'} `to cheer someone up';
%\item to perform the action denoted by the \isi{derivational base} in addition and, usually, with low intensity (productive type, some derivational bases are perfective): \textit{podgladit'} `to iron a bit more,' \textit{podzarabotat'} `to earn a some money';
%\item to perform the action denoted by the \isi{derivational base} in secret (productive type): \textit{podslu\v{s}at'} `to eavesdrop';
%\item to clear something or remove the rests by performing the action denoted by the \isi{derivational base} (non productive type): \textit{podjest'} `to eat the rests';
%\item to perform the action denoted by the \isi{derivational base} in coarse or immediately after another action, adapting to something (productive type): \textit{podygrat'} `to play, adapting to the play of someone else';
%\item to perform the action denoted by the \isi{derivational base} until the result (non productive type): \textit{podmesti} `to sweep the floor.'
%\end{enumerate}
%\subsection{Restrictions on the attachment}
%\subsection{Subsequent imperfectivization of a verb with the discussed prefix}
%\subsection{Summary}
%Podhmelet' exists (contra Tatevosov, who marks it with ??): maybe it's all about phonetics?
%
%\subsection{ot-}\label{subsection:semantics:ot}
%\subsection{Semantic contribution}
%\citet[pp.362--363]{Shvedova:82}
%\begin{enumerate}
%\item to part or move away from something by performing the action denoted by the \isi{derivational base} (productive type): \textit{otletet'} `to move away by flying';
%\item to move somewhere by performing the action denoted by the \isi{derivational base} (this type is productive if the prefix is combined with verb denoting relocation): \textit{otvezti} `to bring something somewhere,' \textit{otvesti} `to carry something somewhere';
%\item intensively, completely, finally perform the action denoted by the \isi{derivational base} (productive type): \textit{otgladit'} `to iron thoroughly';
%\item to lead to an undesired state or condition as a result of performing the action denoted by the \isi{derivational base} (non productive type, takes as its input only \isi{transitive verbs}): \textit{otdavit'} `to step on something';
%\item as a result of performing the action denoted by the \isi{derivational base} refuse something or make someone else to refuse something (non productive type): \textit{otsovetovat'} `to dissuade';
%\item to perform the action denoted by the \isi{derivational base} as a response to some other action (non productive type): \textit{otplatit'} `to pay off';
%\item to end the action denoted by the \isi{derivational base}, that lasted for some time (productive type): \textit{otgremet'} `to stop rattling';
%\item to perform the action denoted by the \isi{derivational base} until the result is reached (non productive type): \textit{otremontirovat'} `to repair';
%
%\end{enumerate}
%\subsection{Restrictions on the attachment}
%\subsection{Subsequent imperfectivization of a verb with the discussed prefix}
%\subsection{Summary}

%\subsection{pri-}\label{subsection:semantics:pri}
%\paragraph*{Semantic contribution.}
%\citet[pp.366--367]{Shvedova:82}
%\begin{enumerate}
%\item to reach some destination, deliver something or become attached to something by performing the action denoted by the \isi{derivational base} (productive type): \textit{prinesti'} `to deliver by carrying';
%\item to perform the action denoted by the \isi{derivational base} with low intensity or not to the end (productive type): \textit{pritormozit'} `to brake slightly,' \textit{prizadumat'sja} `to become slightly thoughtful';
%\item to perform the action denoted by the \isi{derivational base} in addition to something, add something to something else (productive type): \textit{pririsovat'} `to draw something additional on a drawing;
%\item to perform the action denoted by the \isi{derivational base} in course or directly after the other action (productive type, derivational bases are \isi{perfective verbs} denoting single actions): \textit{prixlopnut'} `to clap while doing something else';
%\item to perform the action denoted by the \isi{derivational base} until the result is reached (non productive type): \textit{primerit'} `to try something.'
%\end{enumerate}
%\paragraph*{Restrictions on the attachment.}
%\paragraph*{Subsequent \isi{imperfectivisation} of a verb with the discussed prefix.}
%\paragraph*{Summary.}

%\subsection{pro-}\label{subsection:semantics:pro}
%\paragraph*{Semantic contribution.}
%\citet[pp.366--367]{Shvedova:82}
%\begin{enumerate}
%\item to direct the action denoted by the \isi{derivational base} through or into something (productive type, some derivational bases are perfective): \textit{projti'} `to walk through,' \textit{protolknut'} `to push through';
%\item to direct the action denoted by the \isi{derivational base} \isi{past} something (productive type): \textit{probe\v{z}at'} `to run \isi{past} something';
%\item to move forwardor cover some \isi{distance} by performing the action denoted by the \isi{derivational base} (productive type): \textit{pronesti} `to carry something for some \isi{distance}';
%\item to perform the action denoted by the \isi{derivational base} intensively or thoroughly (productive type): \textit{progladit'} `to iron thoroughly';
%\item to spend or expend something by performing the action denoted by the \isi{derivational base} through or into something (productive in colloquial language): \textit{propit'} `to spend the money on drinking';
%\item to miss something while performing the action denoted by the \isi{derivational base} (productive in colloquial language): \textit{proguljat'} `to go strolling instead of going to work or study ';
%\item to perform the action denoted by the source for some (usually long) time (productive type): \textit{pro\v{z}dat'} `to wait for a long time';
%\item to perform the action denoted by the \isi{derivational base} until the result is reached (productive type): \textit{prozvu\v{c}at'} `to sound.'
%\end{enumerate}
%\paragraph*{Restrictions on the attachment.}
%\paragraph*{Subsequent \isi{imperfectivisation} of a verb with the discussed prefix.}
%\paragraph*{Summary.}

\section{Secondary imperfective}\label{section:imperfective}
Formally representing the semantics of the \isi{imperfective suffix} is a task I am not aiming to complete in this book. However, it is not possible to construct the desired \isi{compositional semantics} of \isi{complex verbs} without a semantic representation of the \isi{imperfective suffix}. In order to achieve the goal of analyzing \isi{prefix stacking} (with respect to the prefixes discussed above plus verbs that are listed in the dictionaries) I have to construct some formal representation of the semantics of the \isi{imperfective suffix}. I will do this for two cases: (1) progressive meaning of the imperfective and (2) habitual meaning of the imperfective. This involves a number of decisions that I will present without proper justification.

The first puzzle that has to be solved in some way concerns the general problem with the \isi{progressive interpretation} of the \isi{secondary imperfective} that seems to cancel the ``reaching the boundary'' component added by the prefix. I claim that when secondary \isi{imperfectivisation} happens, there is no ``reversion'' to the initial imperfective semantics. I will account for this in the following way. 

Let us start with a basic imperfective verb. Such a verb denotes an activity or a process that is not mapped onto the \isi{time scale}. If one wants to describe it in terms of telicity, it can be either atelic, as \textit{sidet'}$^{\IPF}$ `to sit/be sitting' or telic, as \textit{pisat' pis'mo} `to write/be writing a letter', but in neither case does it have endpoints that are mapped onto the \isi{time scale}. According to my view, this mapping is added by prefixes. As the verb gets prefixed, its semantic structure gets enriched with endpoints that are related to time points. In case the scale selected by the prefix is the \isi{time scale}, some points on this scale are directly associated with the start and the end of the event. In case the event proceeds along another scale, points on that scale are mapped onto the \isi{time scale}. 

I propose that when the \isi{imperfective suffix} with progressive semantics is attached to a perfective verb, the boundaries that are present in the semantic structure of the \isi{derivational base} do not disappear. Instead, the derived verb denotes an event that is part of the event denoted by the \isi{derivational base} and is of type \textit{progression}. It can as well turn out that this partial event coincides with the whole event in case the verb is prefixed further or the imperfective is actually used to describe a completed event.

%another point that represents an intermediate stage on the scale is added. This point has to be located in between the points on the scale corresponding to the start and end points of the event and has to be distinct from the point corresponding to the start of the event. The new point indicates the progress of the event and allows to refer to some intermediate stage. Putting it in a more formal language, if the point on the scale associated with the start of the event is $x$ and the point associated with the end of the event is $z$, then the imperfective adds a third point: a \textit{current} point $y$ such that $x < y \leq z$. 

The second meaning of the \isi{secondary imperfective} suffix that I will formalise is the \isi{repetitive}/habitual meaning. My solution resembles that for the \isi{distributive} \Prefix{pere-} except for the absence of the set that has to be iterated through. In the case of the \isi{imperfective suffix} the iteration is performed without imposing restrictions on when the first event of the iterated series started and when (and whether) the series is going to end. Attachment of the \isi{imperfective suffix} with a \isi{repetitive}/habitual interpretation is similar to providing a \isi{repetitive} \isi{context} for a \isi{telic verb} in English: independently of the language, the iteration of a bounded event results in an unbounded event. For English this means that verbs denoting accomplishments and achievements become compatible with \textit{for}-adverbials. For Russian the consequence of the attachment of the \isi{imperfective suffix} is an additional layer of verbal structure that makes the event unbounded and thus imperfective and also opens additional \isi{prefixation} possibilities. 

%Basic \isi{imperfective verbs} denoting atelic processes/activities do not receive habitual interpretation: \textit{on sidit v t'jurme} cannot be interpreted as `he regularly sits in jail' (there is a possibility of getting `his occupation is to sit in jail' interpretation that leads to regularity, but this is different). Plurality can be contributed by the DO: \textit{on \v{c}itaet knigi} can be interpreted as an event of reading many books simultaneously or multiple events of reading. 

\section{Summary}
In this chapter, I have provided an overview of semantic approaches to Russian verbal \isi{prefixation} and examined the semantic and combinatorial properties of five verbal prefixes: \Prefix{za-}, \Prefix{na-}, \Prefix{po-}, \Prefix{pere-}, and \Prefix{do-}. For each prefix I have discussed its semantic contribution, restrictions on its attachment and on further combination with the \isi{imperfective suffix}. 

As, following \citet{Kagan:book}, I adopt a scalar analysis of prefix semantics, I have also provided general information about \isi{scales} and drawn attention to the types of \isi{scales} individual prefixes are compatible with and the relations they impose between scalar points and event stages. 

I have concluded that the prefix \Prefix{za-} in its inceptive usage requires the \isi{time scale} and that the initial stage of the event denoted by the derived verb corresponds to the absence of the event denoted by the \isi{derivational base} while the final stage corresponds to the presence of the event denoted by the \isi{derivational base}. 

The prefix \Prefix{na-} accepts a wide range of \isi{scales} as long as they are provided by the verb and belong to the set of parameters of the object. It maps the initial stage of the event to the minimal point of the scale and the end of the event to some point that is at or above the contextually specified \isi{standard degree}. The prefix \Prefix{po-} is compatible with any verbal scale and the \textit{cardinality} scale in case of a plural object. It relates the initial and the final stages of the event to points on the scale. 

The prefix \Prefix{pere-} has three different interpretations that depend on the type of scale: in case of a closed scale the event proceeds from the minimum to the maximum on the scale through all its points; in case of a scale with one marked point the event proceeds from the point below the marked point and further through the marked point to some point above it; in case of a \textit{property} scale the \isi{repetitive} interpretation of the prefix is delete available and the new event is created by copying the event denoted by the \isi{derivational base} which, in turn, becomes the \isi{preparatory phase} of the new event. 

The last prefix, \Prefix{do-}, is compatible with \isi{scales} provided by the verb and by the object as long as they are upper closed. It maps the initial stage of the event onto some point on the scale and the final stage of the event onto the maximum of the scale. 

In course of the discussion of the prefix \Prefix{do-} and the \isi{repetitive} usage of the prefix \Prefix{pere-} I have also raised questions concerning possible presuppositional components in the semantic structure of these verbs, as suggested by \citet{Kagan:book}. I will address these questions in the next chapter.

After that, in Chapter~\ref{Chapter7}, I will offer a formalization of the intuitions and observations laid out in this chapter, using the combination of  (\citealt{Fillmore:82}) and \isi{LTAG} (\citealt{JoshiSchabes:97}) formalised in \citealt{KallmeyerOsswald:13}. 

%Compare the system that emerges with the classification by \citet{Janda:07b}. Her Natural Perfectives correspond to relating of the end points of the verbal scale to the end points of the event and Complex Act Perfectives correspond to other types of connections.
